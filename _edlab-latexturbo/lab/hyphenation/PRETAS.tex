\textbf{Júlia Lopes de Almeida} \textls[5]{nasceu no Rio de Janeiro, em 24 de setembro de 1862. Considerada um verdadeiro fenômeno literário, escreveu romances, contos, novelas, peças teatrais, crônicas, ensaios, livros didáticos e infantis. Estreou como escritora em 1881, incentivada pelo pai, com a crônica sobre a atriz italiana Gemma Cuniberti, publicada na \textit{Gazeta de Campinas}, e atuou como cronista nos mais importantes jornais do país. Entusiasta da modernidade e das mentalidades daquele período de efervescência cultural e intenso otimismo, formou em seus textos um amplo painel da Belle Époque carioca. Seu primeiro romance, \textit{Memórias de Marta}, foi publicado em folhetim, na \textit{Tribuna Liberal}, do Rio de Janeiro, de 1888 a 1889. Nele, registrou as condições desumanas vivenciadas pelos moradores de cortiços. Depois, publicou \textit{A família Medeiros}, \textit{A viúva Simões}, \textit{A falência}, \textit{A intrusa}, \textit{Cruel amor}, \textit{Correio da roça}, \textit{A Silveirinha}, \textit{Pássaro tonto} e \textit{O funil do diabo}. Em parceria com o marido, o poeta português Filinto de Almeida, publicou \textit{A casa verde}, sob o pseudônimo A.\,Julinto. Com a irmã Adelina Lopes Vieira, escreveu \textit{Contos infantis}, livro destinado ao uso nas escolas primárias. Em seu casarão no bairro de Santa Teresa, oferecia celebrados saraus nos jardins, então conhecidos como \textit{Salão Verde}. Atuou no meio literário, jornalístico e intelectual brasileiro, e foi uma das idealizadoras da Academia Brasileira de Letras, porém não integrou a lista oficial por ser mulher. Lutou pela emancipação feminina, aconselhou mulheres a trabalharem e terem sua própria fonte de renda para não dependerem dos homens, criticou filósofos misóginos, contestou severamente a falta de educação para as mulheres, mas, sobretudo, o tipo de educação que recebiam em casa, destinada apenas ao casamento e à futilidade. Morreu em 1934 e, desde então, foi gradativa e injustamente alijada da memória e história literárias. Recentemente, sua obra tem sido recuperada por uma nova geração de pesquisadoras e pesquisadores.}

\textbf{A família Medeiros} \textls[15]{(1892) é o segundo romance de Júlia Lopes de Almeida, publicado em folhetim em 1891 e em livro no ano seguinte. Ambientado em Campinas, no estado de São Paulo, retrata os costumes e conflitos entre as gerações da família do Comendador Medeiros: enquanto esse cafeicultor resiste à emancipação dos escravizados e à valorização do trabalho assalariado, Eva, sua sobrinha, e Otávio, seu filho, enfrentam-lhe o conservadorismo e defendem abertamente os ideais abolicionistas e republicanos. Esses embates correspondem, no conjunto do romance, à resistência dos escravizados da Fazenda Genoveva, que articulam um levante pela própria libertação, e ao projeto inovador de Eva na administração dos negócios da Fazenda Mangueiral, herdada ao pai, cujos negócios são conduzidos com respeito à dignidade humana por meio da partilha dos lucros. Com o propósito de sensibilizar o público da época quanto à brutalidade da escravidão, Júlia Lopes de Almeida registrou o ambiente social e político paulista dos últimos anos do século \textsc{xix}, descrevendo o sofrimento dos escravizados e suas formas de resistência, como as revoltas contra os proprietários e os quilombos.}

\pagebreak
\thispagestyle{empty}

%\textbf{Júlia Lopes de Almeida} (Rio de Janeiro, 1862--\textit{id.}, 1934) é uma das escritoras brasileiras mais importantes da virada do século \textsc{xix} para o \textsc{xx}. Romancista, contista, cronista e dramaturga, publicou seus primeiros textos aos dezenove anos em jornais cariocas. Em 1886 mudou-se para Lisboa, cidade de seus pais, onde efetivamente inciou sua carreira de escritora. Seu primeiro romance, \textit{Memórias de Marta}, foi publicado em Portugal em 1888. Um dos principais nomes da \textit{Belle Époque} carioca, Júlia Lopes publicou dez romances --- dentre os quais o famoso \textit{A falência}, de 1901---, cinco livros de contos e sete peças teatrais, muitas das quais escritas durante sua estadia na França. Apesar de sua importância, foi pouco lida se comparada aos escritores, em face da invisibilidade sofrida pelas escritoras. Esteve também entre os idealizadores da Academia Brasileira de Letras, mas foi preterida a assumir uma das cadeiras entre os fundadores por ser mulher.

\textbf{Anna Faedrich} \textls[25]{é doutora em Letras, com especialização em Teoria da Literatura (\textsc{pucrs}), professora de literatura brasileira na Universidade Federal Fluminense (\textsc{uff}) e coordenadora do projeto de pesquisa \textit{Literatura de autoria feminina na belle époque brasileira: memória, esquecimento e repertórios de exclusão}. É autora de \textit{Teorias da autoficção} (\textsc{e}d\textsc{uerj}, 2022) e \textit{Escritoras silenciadas} (Macabéa/\,Fundação Biblioteca Nacional, 2022).}

\textbf{Rafael Balseiro Zin} é sociólogo e doutor em Ciências Sociais, pela \textsc{puc--sp}, onde atua como pesquisador no Núcleo de Estudos em Arte, Mídia e Política (Neamp/\,\textsc{cnp}q). Nos últimos anos, entre outros temas, tem se dedicado a investigar a trajetória intelectual das escritoras abolicionistas no Brasil, com especial atenção ao legado de Maria Firmina dos Reis e Júlia Lopes de Almeida.

\textbf{Norma Telles} graduou-se em História pela \textsc{usp} e é mestre e doutora em Ciências Sociais pela \textsc{puc--sp}, onde lecionou entre 1978 e 2006. Dedicou-se à pesquisa sobre escritoras brasileiras do século \textsc{xix} e ao resgate de suas obras, afastadas da historiografia tradicional da literatura brasileira. 

\textbf{Metabiblioteca} \textls[20]{foi pensada para edições anotadas ou obras completas de cânones da literatura em língua portuguesa. As edições propõem desde estabelecimento de textos até novas hipóteses de leitura.}\par

