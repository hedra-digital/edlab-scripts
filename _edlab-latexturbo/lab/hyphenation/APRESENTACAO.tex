%!TEX root=LIVRO.tex
\chapter*{Apresentação\medskip\subtitulo{«A família Medeiros» em meio ao declínio\\\vspace*{-.015\textheight} da ordem escravocrata no Brasil}}
\addcontentsline{toc}{chapter}{Apresentação, \textit{por Rafael Balseiro Zin}}

\begin{flushright}
\textsc{rafael balseiro zin}
\end{flushright}

%\footnote{As reflexões aqui apresentadas são um desdobramento da minha tese de Doutorado em Ciências Sociais, intitulada \textit{Escritoras abolicionistas no Brasil-Império: Maria Firmina dos Reis e Júlia Lopes de Almeida na luta contra a escravidão}, defendida em março de 2022 na Pontifícia Universidade Católica de São Paulo.}

\noindent{}\textls[10]{Escrito entre os anos de 1886 e de 1888, e veiculado pela primeira vez
no Rio de Janeiro em formato de folhetim, entre as edições de 16 de
outubro e 27 de dezembro de 1891 do jornal \emph{Gazeta de Notícias},
\emph{A família Medeiros} é o primeiro romance escrito e o segundo
publicado por Júlia Lopes de Almeida (1862--1934). No ano seguinte, como
era de costume na época, após obter uma recepção entusiasmada por parte
da crítica e do público leitor, a obra foi editada em formato de livro e
publicada pela Companhia Editora Fluminense, novamente no Rio de
Janeiro. Enquanto a autora ainda era viva, o romance recebeu mais duas
edições: a primeira, em 1894, editada por Horácio Belfort Sabino e
impressa nas oficinas da Tipografia Paulista; e a segunda, cujo texto
foi revisto pela própria Júlia, foi publicada em 1919, no Rio de
Janeiro, pela Empresa Nacional de Publicidade --- cujo prefácio abre
este volume. Desde então, passados noventa anos fora de circulação, o
livro recebeu mais três edições realizadas recentemente: a primeira, de
2009, publicada em Florianópolis pela Editora Mulheres; a segunda, de
2019, publicada no Rio de Janeiro pela Editora Vermelho Marinho; e a
última, de 2021, publicada em São Paulo pela Editora Carambaia,
totalizando sete edições desde a primeira publicação. Visando celebrar
os 160 anos de nascimento da romancista e iniciando um movimento inédito
de reedição das obras completas de Júlia Lopes de Almeida, a Editora
Hedra inaugura a sua coleção justamente com o lançamento de \emph{A
família Medeiros}, obra que angariou expressivo sucesso, dando início à
longa e profícua carreira daquela que seria reconhecida, em vida, como a
escritora brasileira de maior prestígio no início do século \textsc{xx}.}

\textls[-10]{Ambientado na cidade de Campinas, no interior de São Paulo, onde a
autora viveu durante a infância e em seus primeiros anos de juventude, o
enredo discorre sobre os bastidores de uma rica família constituída pelo
comendador de sobrenome Medeiros, um escravocrata inescrupuloso
pertencente à elite cafeicultora paulista de fins de século,
proprietário de terras na região e de inúmeros negros escravizados. Após
a morte do pai, a personagem Eva, sobrinha de Medeiros, se muda para a
casa do tio e passa a morar com sua família. Otávio, seu primo
recém-chegado da Europa, que se convertera ao abolicionismo e ao
republicanismo durante os tempos em que viveu por lá, se apaixona por
Eva, mas segredos obscuros marcam o passado da órfã. A trama se
desenrola em torno de um misterioso assassinato ocorrido anos antes, que
envolve a participação do patriarca da família e de seu irmão, em meio a
um levante e à fuga de escravizados pela Serra do Mar. A história
acontece no momento de transição entre o trabalho forçado e o trabalho
livre assalariado nas plantações de café paulistas, onde a opulência dos
senhores convive com a árdua condição de vida imposta aos cativos e aos
colonos recém-chegados ao país.}\looseness=-1

\textls[15]{Manifestando no romance os seus ideais antiescravistas, Júlia Lopes de
Almeida reconstrói os horrores do sistema, além de apresentar em
detalhes o tratamento e os castigos físicos infligidos aos escravizados.
Antes disso, porém, estabelece importantes diálogos políticos entre suas
personagens, que dão o tom da narrativa. Logo no
primeiro capítulo do livro, por exemplo, entram em cena Otávio Medeiros,
que acabara de retornar do exterior, e o dr.\,Morton,
amigo da família, entusiasta silencioso da abolição e seu antigo tutor,
que travam a seguinte conversa:}

\begin{quote}
\hspace{1.3em}--- {[}\ldots{]}Há uma única coisa que me repugna e entristece aqui, escuso de lhe dizer qual
seja, o meu amigo adivinha-a, mas essa mesma extinguir-se-á em breve,
porque, Otávio, agora o Brasil não dorme, trabalha.

\textls[15]{--- Tenho acompanhado com júbilo o movimento abolicionista do Brasil;
procurei avidamente nas correspondências dos jornais, sempre, tudo o que
se referia a essa grande obra; contudo, de tão longe, não se pode fazer
ideia perfeita das coisas que os jornais exageram e as traduções
adulteram. (p.\,\pageref{Ref2} desta edição)}
\end{quote}

\textls[-5]{Através desse primeiro posicionamento político manifestado pela autora
por meio dessas duas personagens, Júlia Lopes de Almeida marca o seu
patriotismo em relação ao Brasil e deixa explícito desde o início o seu
alinhamento à causa antiescravista. E aproveita o ensejo para render
homenagens a Luiz Gama (1830--1882), Antônio Bento (1843--1898) e José
do Patrocínio (1853--1905), líderes do movimento abolicionista que eram
abominados pelos proprietários de escravos, mas vistos pela romancista
como homens gloriosos. E isso fica evidente logo no segundo capítulo do
livro, em um acalorado diálogo estabelecido entre Otávio e o comendador
Medeiros, que discutem sobre a lavoura. O jovem ouve com desgosto o pai
expor ideias carregadas de rancor e de ressentimento, ao mesmo tempo em
que tenta lhe mostrar as benesses dos novos processos de plantio e de
exploração da mão de obra assalariada. Medeiros, no entanto, repele com
veemência as propostas do filho, vociferando contra os reformadores,
contra as modernas teorias, contra tudo e contra todos:}\looseness=-1

\begin{quote}
\hspace{1.3em}--- A vida agora no Brasil é um inferno. Em São Paulo, um tal Luiz Gama
e outro que tal Antônio Bento especulam com os pobres dos lavradores,
tirando-lhes os escravos. Os jornalistas do Rio são a mesma corja. Eles
acoitam os pretos fugidos para os alugarem por sua conta e irem fazer
conferências públicas nos teatros, pregando a emancipação! É por isso
que a gente séria os chama de ``pescadores de águas turvas''.

\textls[-10]{José do Patrocínio é o chefe dessa bandalheira, que, se o país tivesse
governo, já teria acabado. É por isso mesmo que muitos liberais e muitos
conservadores estão se passando para o partido republicano {[}\ldots{]}}\looseness=-1

--- Se eles se lembrarem de vir a Santa Genoveva --- exclamava ---, os
bandidos dos abolicionistas, eu sei como os hei de receber: a tiro!
Defendo a minha propriedade, estou no meu direito. A culpa é também das
autoridades, que não amoldaçam esses cachorros dos jornais que
latem, latem para os outros morderem! (p.\,\pageref{Ref3})
\end{quote}

\textls[15]{``Pobres dos lavradores!'' É com esse tom de ironia que Júlia Lopes de
Almeida eleva em seu romance os nomes dessas importantes lideranças
políticas, responsáveis diretas pela ampliação e pela radicalização da
campanha abolicionista nos anos de 1880, ao introduzirem novas formas de
agir pela libertação dos cativos, organizando fugas em massa,
orquestrando processos judiciais com base nas leis antiescravistas
vigentes, inserindo os recém-libertos no mercado de trabalho e
expandindo os alcances e os apelos da propaganda a favor da causa.}

\textls[-10]{Alinhada aos valores republicanos e democráticos, a autora incorpora à
narrativa a decadência lenta e gradual da ordem escravocrata no país,
mostrando que abandonar esse sistema significava modernizar o Brasil,
que passaria a ser calcado não mais na lei dos mais fortes, mas nos
princípios da razão, da justiça e do direito.}\looseness=-1

Apesar dessa homenagem às avessas contida nas entrelinhas do discurso de
Medeiros, é a personagem Eva quem se sobressai na trama, através das
propostas inovadoras implementadas por ela na fazenda-modelo Mangueiral.
Por lá, o autoritarismo patriarcal é repudiado, a dignidade humana é
defendida e o lucro é repartido igualmente entre os empregados.
Funcionando como uma espécie de \emph{alter ego} da autora, entre o
discurso conservador das velhas oligarquias e o processo de modernização
das relações de trabalho colocado em marcha no país, a protagonista
assume destaque ao corresponder aos ideais progressistas defendidos pela
própria Júlia Lopes de Almeida, como a emancipação das mulheres e o fim
do sistema escravista. E logo em sua primeira aparição, Eva surge na
trama em meio à conversa de Otávio com o comendador, pedindo clemência
ao tio pelas eventuais faltas cometidas por um cativo de nome Manuel
Sabino:

\begin{quote}
\hspace{1.3em}--- Que temos? --- indagou o tio.

--- Venho pedir-lhe que perdoe ao Manuel Sabino; ele promete ser
obediente daqui por diante. Mande tirar-lhe os ferros, sim?

\textls[-10]{--- Asneira! Deixe-se disso, que não é da competência das moças. Se não
quiser ver o negro com os ferros, não olhe para ele. Era o que faltava!}\looseness=-1

--- Não olho, mas nem assim deixo de saber que os traz. (p.\,\pageref{Ref5})
\end{quote}

\textls[10]{Após definir minimamente, nos capítulos iniciais do livro, o caráter das principais personagens brancas que integram, como membros ou
agregados, o núcleo da família Medeiros, Júlia Lopes de Almeida passa a
descrever com mais detalhes as primeiras personagens negras
escravizadas, em uma cena em que Otávio se põe a caminhar pela
propriedade do pai, tecendo reminiscências sobre a infância vivida
por lá. Em dado momento, o jovem acaba se encontrando por acaso com a
prima, que lhe faz um convite inusitado: rever sua ama de
leite, agora impossibilitada de movimentar-se.}

\begin{quote}
\textls[15]{Continuaram em silêncio até perto de uma casa de barro, sem janelas,
onde a ama, sentada no chão ao pé da porta, escolhia arroz numa peneira,
cantando, com um fio trêmulo e finíssimo de voz. A doença envelhecera-a
muito; estava mirrada, com a carapinha branca e as faces engelhadas que
nem erva seca. Assim mesmo trabalhava e cantava, ocupando sempre as mãos
emagrecidas e a vozinha débil como a de uma criança. Não que lhe dessem
tarefa; pedia-a ela às outras companheiras para aliviá-las do trabalho.}

--- Mamã --- disse Otávio, chegando-se para ela.

\textls[10]{A velha estremeceu, fixou no moço os seus olhos fulgurantes, onde toda a
sua vida parecia concentrada, a peneira caiu-lhe das mãos e ela agitou
os braços, sufocada, a chamá-lo para bem perto de si.}

Otávio aproximou-se, ela abraçou-lhe os joelhos, chorando.

\textls[15]{Era daquela velhinha paralítica das pernas, imprestável, atirada como um
caco velho para um canto imundo, que saía a maior manifestação de júbilo
pelo seu regresso. Todos os outros o haviam recebido com um sorriso
apenas; ela acolhia-o com lágrimas! (p.\pageref{Ref6})}
\end{quote}

\textls[15]{Nota-se, portanto, o tratamento particular que Júlia Lopes de Almeida dá
à negra Joana, a mamã de Otávio, essa primeira personagem negra
escravizada que, a despeito de uma existência inteira vivida sob os
grilhões do cativeiro e da paralisia que contraíra já em idade avançada,
é representada de forma humanizada, como uma mulher carinhosa, solidária
e que ``assim mesmo trabalhava e cantava, ocupando sempre as mãos
emagrecidas e a vozinha débil como a de uma criança''.}

E é preciso notar, também, a habilidade da autora em descrever de maneira
realista as duras condições enfrentadas pelos cativos, como
podemos verificar em uma cena em que Otávio, ao se perder em um matagal
durante uma cavalgada noturna nos arredores da fazenda do pai, acaba
deparando involuntariamente com um grupo de aquilombados. Imprimindo um
olhar que se aproxima dos estudos de cunho etnográfico, devido ao
cuidado em apresentar os modos de vida e os costumes da população negra
escravizada em pleno século \textsc{xix}, a autora aproveita o ensejo
para, uma vez mais, condenar as dinâmicas da escravidão:

\begin{quote}
{[}Otávio{]} Compreendeu. Fora parar a um quilombo! {[}\ldots {]}

\textls[15]{Contrafeitos, apertados, fugindo ao sono, estremecendo ao menor sopro do
vento, ouvindo um passo em cada folha caída, um grito humano em cada pio
de ave, temendo sempre a perseguição do branco, mergulhando nos rios ao
vê-lo de longe, saindo ao sentirem-se morrer, para rolar na terra
ingrata, correndo com o corpo despedaçado de cansaço, sempre arrepiado
com as alucinações do medo, sempre atrás da quimera-liberdade, e sempre
agrilhoados, sob a pressão de um pesadelo medonho, esses desgraçados
tinham um aspecto de fantasmas lúgubres, carbonizados nos braseiros do
inferno.}

\textls[-33]{A escravidão selara-lhes nos lábios o sorriso. O medo pusera-lhes no
olhar a desconfiança e o ódio. A fome, o frio e a insônia
descarnaram-lhes os esqueletos, tornando-os numas verdadeiras múmias
aterrorizadoras. (p.\,\pageref{Ref7})}\looseness=-1
\end{quote}

Outro aspecto importante a ser observado em \emph{A família Medeiros} é
o tratamento que Júlia Lopes de Almeida atribui às demais personagens
negras escravizadas, que aparecem na trama confabulando sobre um levante
planejado por elas próprias, previsto para acontecer na propriedade do
comendador Medeiros durante a noite de celebração do casamento de uma
das irmãs de Otávio. Buscando retratar determinadas características da
linguagem coloquial empregada pelos cativos, ainda que de modo pitoresco
ou caricatural, e novamente por meio de um diálogo estabelecido entre
eles, a conversa se desenrola da seguinte forma:

\begin{quote}
\hspace{1.3em}--- Tá tudo pronto? --- perguntava um deles.

--- Tudo\ldots{} --- respondia outro.

--- Os negro da casa tão no curro?\footnote{\textls[-15]{Na província de São Paulo do
  século \textsc{xix}, ``curro'' era o nome que se dava ao ``quadrado'',
  a moradia coletiva dos cativos, que, em conjunto, formavam as
  senzalas.}\looseness=-1} Carece de sabê bem isso\ldots{}

--- Decerto, estão.

--- Arrepare, Joaquim! É perciso fazê as coisa cum jeito\ldots{}

\textls[-20]{--- Tá dereito, nhô Furtuoso! Eu sou um decidido! Logo pego, arrombo num
baque as porta do quadrado e ponhamo-nos a fazê baruio. Sai cinza!}\looseness=-1

--- Que confusão danada! Mecês tudo não se esqueça de dizê qui foi d.\,Eva 
qui mandou nois cá\ldots{} sinão, os negro nos arresiste!

--- Entonces, quem se havéra de si esquecê disso?!

\textls[15]{--- Si arguém si quisé fazê de tolo, já se sabe, tiro pra frente! As
almas}\footnote{``Almas'', aqui, empregado no sentido de ``armas''.}
\textls[15]{vieram carregada por mó de isso.}

--- Tá bom, gentes\ldots{} (p.\,\pageref{Ref8})
\end{quote}

\textls[-20]{Considerando que boa parte da historiografia desenvolvida ao longo do
século \textsc{xx} no Brasil sobre o abolicionismo não tratou da
participação e das formas de resistência dos escravizados em meio a esse
processo, é de se admirar a preocupação de Júlia Lopes de Almeida em
inserir em sua narrativa uma passagem como essa. Até porque,
diferentemente da forma como a literatura pertinente ao tema tratou da
questão, aqui encontramos sujeitos conscientes de sua condição de vida e
dispostos a exercer a ação em prol de sua própria liberdade. Daí,
também, o uso estratégico por parte da autora do termo ``revolução'', em
detrimento de ``levante'' ou ``revolta'', uma vez que a intenção dos
cativos, em companhia dos abolicionistas, era realizar transformações
profundas nas bases da sociedade brasileira, impactando os contextos
político, econômico e cultural, e estabelecendo uma nova ordem social
que valesse para todos, indistintamente. Mas é na passagem seguinte,
quando Júlia Lopes de Almeida apresenta aos leitores as cenas da
insurgência escrava, que encontramos um dos pontos altos do romance:}\looseness=-1

\begin{quote}
\textls[10]{Passou-lhe pelo espírito uma multidão de alvitres, cada qual mais
confuso e mais disparatado. Felizmente, os capangas estavam do outro
lado da casa e ele {[}Otávio{]} poderia voltar à frente sem ser visto e
chamar em seu auxílio o feitor. Assim fez; caminhou com precaução,
amparando-se à parede, até perto da torre do sino; aí parou. No terreiro
da frente caminhava também um grupo de homens, falando em segredo;
passaram a pouca distância de Otávio, num rumor soturno de passos e de
vozes que a pouco e pouco se foi perdendo à distância\ldots{} Não os
ouvindo já, Otávio galgou de um salto os seis degraus da escada de pedra
e chegou ao patamar, bateu com força na janela do feitor, dizendo alto:
``Sou eu, Otávio, abra depressa!''. Fora sentido fora e dentro da casa
ao mesmo tempo: mas, no momento em que o feitor, pálido e trêmulo, lhe
abria a porta, um tiro disparado de entre a escuridão da noite
cravou-lhe nas costas uma bala.}

\textls[15]{Otávio caiu de bruços, para dentro, indo bater com o peito no chão; a
mulher do feitor puxou-o, arrastando-o para o corredor, enquanto o
marido fechava violentamente a porta; depois, levaram-no para o leito,
onde ele ficou estendido, inerte, manchado de sangue.}

A revolta rebentara lá fora; ouviam-se gritos, imprecações e tiros,
estalos de portas arrombadas e vozes aflitas. Sem cuidarem de Otávio, na
alucinação do medo, sem mesmo saberem se estava vivo ou morto, correram
o feitor e a mulher a colocar móveis junto às janelas e à porta.

\textls[-15]{Que iria fazer lá fora o feitor? Que era um homem para uma revolta de
cem? Como o tratariam tantos e tão encarniçados inimigos? Agora não eram
os mesmos cordeiros, aos quais ele distribuía a esmo chicotadas; não
eram brutos irracionais, sem dignidade e sem coragem: eram homens
enraivecidos e capazes de tudo para satisfazerem os seus fins. (p.\,\pageref{Ref9})}\looseness=-1
\end{quote}

\textls[10]{O leitor do século \textsc{xxi} talvez esperasse da autora uma cena em
que os cativos revoltosos ateassem fogo à residência do feitor,
vingando-se por todos os anos de humilhação, por terem sido
vítimas de tamanha violência. Mas era preciso preservar a vida de
Otávio; afinal, seu ferimento à bala serviria para que Júlia Lopes de
Almeida recuperasse na trama a presença da negra Joana. Contrapondo-a à
imagem da mãe biológica do jovem Medeiros, que não tem nome, não
participa dos diálogos e tampouco das ações do romance, como
bem observou a professora Norma Telles em sua introdução à edição de
2009, que também fizemos questão de incorporar a este volume, a mamã é
nomeada, tem fala e se movimenta, ainda que com dificuldade. Ao ficar
sabendo por sua comadre, a escrava Joaquina, que Otávio estava ferido e
que seu filho de sangue havia sido preso e seria duramente castigado por
ter se envolvido no motim, a velha quase imobilizada sobe com
dificuldade em um carrinho de mão que estava encostado no jardim e
solicita à sua companheira de infortúnio que a leve até a casa-grande,
para poder ver de perto o estado de saúde de seu filho de criação.
Embora longa, a cena é assim apresentada:}

\begin{quote}
\textls[10]{{[}\ldots{]} era a mamã de Otávio, a pobre negra velha, que ia como um
réptil, com o ventre encostado no chão, as pernas mirradas pela
paralisia, pondo nas mãos engelhadas toda a sua atividade, estendendo
ora um braço, ora outro, encarquilhando os dedos nas gretas das tábuas,
quebrando as unhas, rasgando as roupas, arranhando as carnes, arfando de
cansaço, mas prosseguindo no esforço até a porta do quarto do ferido,
que ela abriu com a cabeça, num desespero angustioso.}

\textls[20]{Mas o corpo do moço estava numa cama alta e a desgraçada não se podia
erguer para beijá-lo e vê-lo; extenuada, pôs-se a chamá-lo com uma voz
dolorida e débil, levantando o rosto lacrimoso para o leito a que não
conseguia chegar. {[}\ldots{]}}

Mas não era ao filho preto, das suas próprias entranhas, a quem a
paralítica queria mais. Entre os dois, preferia o branco, nascido dos
seus senhores, a quem amamentara e cuja importância e beleza a enchiam
de orgulho. E nessa preferência não demonstrava Joana nenhuma
originalidade. Não sentiam todas as outras mamãs do mesmo modo?

\textls[10]{A razão dessa absurda anomalia residia talvez no próprio horror da
escravidão. Vendo nos filhos entes destinados a uma condição
indefensável, as pretas fugiam inconscientemente de lhes dedicar muita
ternura. Olhavam para os filhos pequenos como para futuros parceiros,
sujeitos a todos os vilipêndios e que o destino afastaria para longe,
quando bem lhe aprouvesse\ldots{} (p.\,\pageref{ref10})}
\end{quote}
\enlargethispage{\baselineskip}
Nessa dolorida passagem, portanto, Júlia Lopes de Almeida recupera a
famosa ``alegoria da mãe preta'', uma figura pouco explorada na
literatura brasileira oitocentista e também na criação literária de
cunho abolicionista produzida por aqui e em outros países no mesmo
período. Essa alegoria serviu por muito tempo aos proprietários de
escravos e a seus descendentes como uma forma de atenuar as crueldades
que envolviam a prática da ``maternidade transferida'', em meio à
opinião pública. E a própria autora deixa explícita na narrativa a sua
percepção política desse fenômeno, ao definir a situação vivida
pela negra Joana como uma ``absurda anomalia'', que residia ``no próprio
horror da escravidão''. Compreendendo desde cedo que Jacinto, seu filho
biológico, estaria fadado à mesma condição existencial que a sua,
mamã acaba preterindo-o em detrimento de Otávio, transferindo seu
afeto maternal à imagem da criança branca, destinada a um futuro mais
promissor. Vale lembrar que o tema da maternidade foi bastante
trabalhado por Júlia Lopes de Almeida, ao longo de sua trajetória, que o
explorou em obras como \emph{Livro das noivas},\footnote{A primeira
  edição do \emph{Livro das noivas} foi publicada em Lisboa, Portugal,
  em 1896. Após o lançamento internacional, no mesmo ano, ele foi
  editado no Brasil pela Tipografia da Companhia Nacional, sediada no
  Rio de Janeiro.} de 1896, e em um ensaio intitulado
\emph{Maternidade}, publicado originalmente no Rio de Janeiro, entre
agosto de 1924 e agosto de 1925, nas páginas do \emph{Jornal do
Comércio}.

\textls[-7]{Buscando retratar os modos de vida e os costumes da elite escravocrata
paulista, mais adiante, sobretudo no que diz respeito às mulheres, que
se vestiam com roupas de luxo e ostentavam uma vida bastante particular
se comparadas às demais localidades do país, a autora aproveita o ensejo
para denunciar aquilo que o sociólogo brasileiro Jessé de Souza
denominou, já no século \textsc{xxi}, como ``a elite do
atraso''.}\looseness=-1\footnote{\textsc{souza}, Jessé. \emph{A elite do atraso}: da
  escravidão a Bolsonaro. Rio de Janeiro: Estação Brasil, 2019.} \textls[-7]{Isso
  porque, mesmo tendo recursos para importar de Paris ornamentos dos mais
  diversos e esbanjando uma riqueza proveniente da exploração da mão de
  obra escrava, trata-se de um grupo formado majoritariamente por pessoas
  incultas e ignorantes, que, a despeito de todo o seu poderio e da
  autoridade que exerciam sobre os escravizados, sequer sabiam ler e
  tampouco apreciavam as manifestações artísticas:}\looseness=-1

\begin{quote}
\textls[10]{Geralmente as senhoras paulistas apresentavam-se nos bailes num requinte
de elegância e de luxo como se não vê facilmente no Rio de Janeiro. As
suas toaletes,}\footnote{\textls[-20]{No contexto do século \textsc{xix}, o
  termo francês \emph{toilette}, cujas ocorrências foram aportuguesadas neste volume, 
se referia ao traje completo utilizado
  pelas senhoras.}\looseness=-1} \textls[10]{importadas quase sempre diretamente de Paris, tinham,
  a par da riqueza, graça e originalidade.}

\textls[10]{Famílias cujas casas não demonstravam o mínimo vislumbre de gosto,
tendo os interiores sem conforto e sem poesia, quase vazios de móveis e
completamente faltos de objetos artísticos, apresentavam-se nos bailes
de uma maneira verdadeiramente sedutora. (p.\,\pageref{Ref12})}
\end{quote}

\textls[6]{Dito isso, de volta ao levante que ocorrera na fazenda de Santa Genoveva
e aos seus desdobramentos, Júlia Lopes de Almeida constrói um dos
diálogos mais interessantes de seu romance, por conter justamente
problematizações que ainda hoje estão postas para a sociedade
brasileira. Ao denunciar as falhas do que poderíamos chamar de sistema
de justiça no Brasil do século \textsc{xix}, a autora mostra como a
lógica do justiçamento, ao arrepio das leis e também do direito, se
sobrepunha à própria noção de justiça. Para tanto, discorre sobre os
métodos preferidos pelos antigos fazendeiros para punir os escravos
acusados de assassinato, ao mesmo tempo em que deixa explícita a
complacência do Estado para com crimes como esse, representado pelas
figuras tanto do delegado de polícia quanto do jovem escrivão:}

\begin{quote}
Tinha-se espalhado por toda a cidade a notícia do ferimento
de Otávio; e sabendo do fato, a polícia não deixou de se apresentar, sem
ter, contudo, tido para isso aviso especial.

\textls[-25]{O delegado, homem experiente e idoso, não estranhou essa falta; conhecia
de sobejo o sistema preferido dos antigos fazendeiros para a punição de
escravos assassinos\ldots{} Durante toda a viagem no trole fora ele
explicando ao escrivão, ainda novo no ofício, como se fazia geralmente
essa história:}\looseness=-1

\textls[-5]{--- Há um assassinato numa fazenda --- dizia. --- Bem! Que se faz do
assassino? Entregam-no à prisão para que siga os processos legais da
justiça? Qual! Querem vingança mais completa. Apoderam-se da presa,
escondem-na no quarto lúgubre do tronco, e, conforme a categoria da
vítima, castigam-na severamente, procurando aproveitar-lhe o trabalho,
mas carregando-a de ferros. Assim, matam-na a fogo lento.}\looseness=-1

\textls[-20]{Dá-la à decisão de um júri seria mandá-la às galés, e as galés para um
cativo são a liberdade. Não seria punição, bem vê; seria um prêmio! (p.\,\pageref{Ref13})}\looseness=-1
\end{quote}

\textls[15]{Passados os cuidados mais urgentes com Otávio, que se encontrava
combalido pelo ferimento à bala, Medeiros passa a inquirir uma parte de
seus escravizados, todos amarrados ao tronco. Sob inúmeras ameaças de
morte e por meio dos castigos físicos desferidos por seus capangas, os
cativos acabam confessando o nome do mentor do levante: era o de sua
sobrinha, Eva. Nessa altura dos acontecimentos, dirigindo-se sobretudo
aos leitores das zonas urbanas, que viviam distanciados da intimidade
das fazendas e demais propriedades agrícolas, Júlia Lopes de Almeida
constrói em detalhes as cenas de tortura e de sofrimento,
estrategicamente pensadas para sensibilizar o público quanto aos
horrores da escravidão:}\looseness=-1

\begin{quote}
\textls[15]{O feitor designou então para o suplício um outro escravo e para carrasco
o próprio pai da vítima. No quarto sombrio, onde gotejava o sangue dos
corpos quentes, estirados e nus, houve uma cena dolorosa de angústia. O
filho avançou rápido para o lugar da tortura, despiu-se ele mesmo num
movimento febril e, pondo as mãos para trás, encruzou-as para que as
amarrassem; mas o pai continuou no seu canto, com a cabeça baixa, os
olhos reluzentes, as pernas trêmulas e a respiração ofegante; foi
novamente intimidado, mas não deu um passo; um silêncio de admiração e
de angústia sucedeu às imprecações e gemidos. Depois de alguns
instantes, o feitor repetiu a ordem outra vez, num tom forte e áspero, e
o algoz caiu de joelhos, elevando para o teto escuro, como a procurar
através dele o Deus clemente, um olhar cheio de lágrimas e as mãos
súplices. (p.\,\pageref{Ref14})}
\end{quote}

Mesmo tendo apresentado o comendador Medeiros, desde o início da trama, como um
escravocrata cruel e inescrupuloso, que agia sempre sem dó nem piedade
para com os cativos, Júlia Lopes de Almeida passa a explorar, através de um certo
sentimento de culpa e até mesmo de remorso por parte dele, a
irracionalidade dos castigos físicos infligidos aos escravizados. Ao
discorrer sobre as angústias vividas pelo velho Medeiros, que
``sentia-se cansado daquela vida de sustos e de cuidados'' e que sequer
conseguia dormir sem se perceber fora de perigo, a autora aproveita a
deixa para desnaturalizar aos olhos dos leitores essa triste
``tradição'':

\begin{quote}
\textls[10]{Medeiros saiu para o terreiro, sufocava; não compreendia bem o que se
debatia dentro do seu coração. Lá no quarto do tronco que se terminasse
a abominável cena, contanto que seus olhos não a vissem; sentia-se
fatigado, mordido de remorsos. Desgraçada condição a que se sujeitava!
Para ele o castigo era a necessidade absoluta, rígida, indispensável.
Sem ele não haveria disciplina. Os negros são como os soldados, que lá
têm ainda o calabouço e os castigos corporais, para compreensão dos seus
deveres. Por isso, muitas vezes, calcando a piedade, reprimindo o
impulso da natureza, contrafazendo a alma, ordenava a execução das mais
absurdas e intoleráveis penas. Medeiros sentia amargamente essa verdade.
Doíam-lhe agora os golpes que mandava vibrar nos servos e desviava a
vista, impressionado, dos instrumentos de tortura: a grande palmatória,
o relho, a gargalheira de alto gancho, as algemas encorrentadas, todo o
arsenal de instrumentos inventados pela ferocidade humana, pendurados em
uma linha na parede sombria do quarto do tronco. Sentia-se cansado
daquela vida de sustos e de cuidados; não podia conciliar o sono, sem
ter a certeza de que os escravos estavam fechados por fora, no quadrado,
e que o feitor guardava a chave debaixo do travesseiro\ldots{} (p.\,\pageref{Ref15})}\looseness=-1
\end{quote}

\textls[10]{Insistindo na estratégia de sensibilizar os leitores quanto aos horrores
da escravidão, por meio de cenas cruéis e grotescas que causam arrepios
até mesmo nos mais insensíveis, Júlia Lopes de Almeida passa então a
discorrer sobre outras formas de tratamento dadas aos cativos por demais
proprietários, que extrapolavam inclusive os métodos empregados pelo
comendador Medeiros em sua fazenda. Através das reminiscências do
próprio patriarca da família, a autora realiza uma nova investida contra
as dinâmicas do cativeiro, desta vez mostrando que havia senhores ``de
imaginação fértil no ódio'', cuja barbaridade era tida como muito mais
criminosa do que disciplinar:}

\begin{quote}
\textls[-17]{Lembrava-se de sentir-se arrepiado com a narração de um fato ocorrido em
uma fazenda da vizinhança. O senhor, de imaginação fértil no ódio, tinha
mandado surrar um escravo; até aí nada de extraordinário\ldots{} mas
depois, sobre as grandes bocas dos golpes e por todo o corpo derramara
em camadas grossas mel de abelhas, e assim ungido fora o infeliz
amarrado a um poste, no terreiro, exposto às grandes nuvens de
marimbondos que desciam zumbindo, deliciados por tão opíparo quão
inesperado banquete. O negro, sem poder defender-se, via descerem os
insetos e cobrirem-lhe o corpo, movendo voluptuosamente as asas
transparentes, a sugar com avidez o seu sangue envenenado pelo desespero
e pela raiva. (p.\,\pageref{Ref16})}\looseness=-1
\end{quote}

\textls[-5]{Como forma de contrapor-se à ideologia escravocrata e às práticas de
dominação oriundas daquilo que tanto Eva quanto Otávio entendiam como o
velho mundo, finalmente, entre tantas outras passagens interessantes
do romance, mas que não caberiam aqui, sobressaem na
trama os novos métodos de trabalho utilizados pela sobrinha do
comendador Medeiros em sua fazenda Mangueiral. Desde os tempos em que a
propriedade era administrada pelo pai de Eva, um sujeito com visões
progressistas, não se viam mais escravos por lá. Após sua
morte, a jovem decide investir parte de sua herança na racionalização da produção, garantindo o lucro necessário para a
manutenção das atividades e para a subsistência das famílias. Fazendo
uso da mão de obra assalariada, formada por trabalhadores livres e por
imigrantes vindos principalmente do continente europeu, a fazenda
tornara-se um modelo a ser seguido por todos aqueles que tivessem visão
de futuro. Diferenciando-se das demais propriedades que mantinham o
elemento servil como fator determinante de sua produção, assim o local é
descrito:}\looseness=-1

\begin{quote}
\textls[10]{Como se destacava de todas elas o Mangueiral, onde os trabalhadores eram
livres, a cultura feita sem a rega das lágrimas, as habitações dos
colonos claras, arejadas e limpas, a casa da proprietária atufada em
verduras, coberta de perfumes e de sombras doces! Seria menos rendosa,
talvez, mas era com certeza muito mais agradável. Era a fazenda futura,
salientando-se entre todas as outras, rotineiras e estúpidas. Tinha,
como poucas, abundância de frutas, de hortaliças e de água, e,
sobretudo, muita paz e muito contentamento! (p.\,\pageref{Ref17})}
\end{quote}

\textls[10]{Na propriedade moderna idealizada por Júlia Lopes de Almeida, portanto,
além da modificação das relações de trabalho que emergem no bojo das
transformações oriundas do próprio sistema capitalista, encontramos
também alterações profundas na ordem patriarcal. Antes vista como frágil
e fútil, como moeda de troca entre as famílias mais abastadas e sem
poder exercer quaisquer tipos de protagonismo, na fazenda-modelo
Mangueiral a imagem da dona de casa que sequer tem nome e que cumpre
suas tarefas sempre em silêncio, como a esposa do comendador Medeiros, é
substituída por Eva, uma jovem de fibra e caráter, que, mesmo tendo se
tornado órfã, consegue tocar a propriedade herdada do pai, garantindo o
sustento tanto de sua família quanto dos trabalhadores, que exercem suas
atividades em liberdade, devidamente remunerados para isso.
Somando esses fatores aos demais elementos apresentados ao longo desta
breve análise, o que se constata é que \emph{A família Medeiros} pode
ser compreendido como um dos trabalhos mais instigantes de Júlia Lopes
de Almeida, além de um marco para a literatura abolicionista produzida
no Brasil do século \textsc{xix}.}
