%!TEX root=LIVRO.tex
\part{A família Medeiros}

\chapter*{}

\section{I}

%\begin{linenumbers}
\noindent{}\textls[15]{Oeste de S.\,Paulo.
O comboio estremecia ainda nas últimas trepidações e já Otávio
Medeiros, sacudindo-se do pó, atirava pela janela para as mãos de um
carregador a sua maleta de viagem. Chegava, enfim, à sua cidade natal.}

Mal desceu do vagão, olhou à roda à procura de uma cara amiga; não
reconhecendo nenhuma, recomendou ao carregador que lhe guardasse a mala
e saiu para a praça batida de sol.

\textls[10]{Rente à calçada da estação, estendia-se uma fila de carruagens e alguns
troles, à espera dos passageiros. Rejeitando as solicitações dos
cocheiros, Otávio atravessou o largo em diagonal e meteu-se por uma
pequena rua arborizada até uma casa baixa de tijolo vermelho e persianas
verdes. Aí chegando, encostou-se ao peitoril de uma janela aberta e disse
jovialmente a um velhote que se entretinha a ler no interior da sala:}

--- Bons dias, dr.\,Morton!

\textls[15]{O dr.\,Morton voltou-se e fixou no recém-chegado os seus olhos azuis,
empapuçados.}

--- Quem lhe fala é o Otávio Medeiros.

--- Oh! Meu caro, entre, entre, que surpresa!

E o velho foi à porta receber o amigo e antigo discípulo.

--- Sabe que é sua a minha primeira visita?

\textls[-30]{--- Deveras? Quer isso dizer que a sua família não está
prevenida\ldots{}}\looseness=-1

\textls[-10]{--- Não está. Meu pai aconselhou-me que, acabados os estudos, eu fizesse
uma excursão pelos principais países da Europa; mas mal concluí o curso
resolvi partir e chego inesperadamente. Valho-me do senhor para uma
informação: meus pais estão na fazenda?}\looseness=-1

--- Estão. Compreendo agora o motivo de ter sido minha a sua primeira
visita. É justo. Vou mandar preparar-lhe o meu cavalo. Entretanto
conversaremos.

E o dr.\,Morton, depois de ter ido dentro dar ordens, tirou da cabeça o
gorro de seda preta e, alisando com a mão o rosto sem barba, começou:

--- Vai encontrar alguma mudança em casa. Sua irmã mais velha está para
casar. É citada por aí como uma das moças mais bonitas de todo o
município. E realmente ela é, como se costuma dizer por aí: um peixão! A
mais nova saiu do colégio de Itu e tem uma preceptora, que por sinal foi
inculcada por mim\ldots{}

--- Diga-me: minha mãe está muito acabada?

\textls[10]{--- Qual, sua mãe é sempre a mesma, resignada nos momentos tristes,
tranquila nos felizes. Em geral, as senhoras têm muito cuidado em não
demonstrar os seus sentimentos; e sua mãe parece levar esse escrúpulo ao
mais alto ponto.}

\textls[-30]{--- Mas o que eu quero saber é se ela está com saúde, bem-disposta.}\looseness=-1

--- Sim, está\ldots{}

Houve um curto silêncio, depois o velho indagou:

\textls[15]{--- E os seus companheiros de viagem: o João Nunes, o Caio Penteado e
creio que o Rodrigo Costa?}

--- O Costa foi depois.

--- Bem. E que tal?

--- Lá ficaram.

--- A estudar?

--- E a gastar.

--- Tempo?

--- E dinheiro.

\textls[15]{--- Para isso, não precisavam sair daqui. É extraordinário! A maior
parte dos rapazes que vão estudar na Europa voltam de lá na mesma,
quando não vêm piores!}

--- Que quer? Há muitos modos de viver nas grandes capitais, e quase
sempre o que mais seduz é o que menos resultado deixa. Eu mesmo, que fui
entre os seis companheiros o único que completou o curso, poderia ter
vindo mais cedo se não tivesse perdido o primeiro ano na fascinação da
novidade! A uma circunstância desagradável devo a minha
reabilitação\ldots{}

--- Há males que vêm para bem\ldots{}

--- É o caso.

--- E qual foi essa circunstância? Desculpe a curiosidade de um velho amigo\ldots{}

--- Um grande prejuízo financeiro de meu pai. O ano de 1880 correu-lhe
mal. Uma grande geada inutilizou a colheita; além disso, o Elias
Brandão, que era o seu correspondente e comissário em Santos, faliu,
arrastando a nossa casa quase à ruína.

--- Lembro-me bem.

--- Pois foi isso que me abriu os olhos e a vontade de estudar.
Pintaram-me com cores negras, a que a distância deu ainda mais horror, a
nossa situação; a mesada foi reduzida à terça parte e tive de
restringir-me e mudar de hábitos. Conheci nesse meio tempo um estudante
campineiro, rapaz talentoso e de poucos meios; morávamos juntos numa
hospedaria de um bairro modesto e barato. Deixei-me penetrar da sua
influência, afastando-nos ambos dos nossos compatriotas e dos parasitas
que os exploravam\ldots{} E agora, sabe do que eu estou convencido? É de que
com boa vontade ou necessidade, aprende-se igualmente em qualquer país.

\textls[-10]{--- Engana-se, meu amigo. E creia que em poucas terras do mundo se
estuda tão bem como em França, embora o meu amigo Edward, um sábio,
velho viajante e ávido observador, costumasse dizer: ``Na Inglaterra
trabalham, na Itália sonham, em França riem, na Rússia conspiram, na
Espanha falam e na Alemanha estudam''\ldots{}}\looseness=-1

--- E deveria acrescentar: no Brasil dormem.

\textls[20]{--- Ele não conhecia o Brasil; mas diga-me antes: com que ideias veio?}

--- Se quer que lhe fale com franqueza, ainda não sei\ldots{}

O dr.\,Morton arregalou os olhos, mas logo disfarçou o espanto na
placidez de um bom sorriso.

--- Seu pai deve ter pensado nisso.

\textls[-10]{--- Não. Meu pai não pensou. Eu venho resolvido a trabalhar, somente não
sei ainda como nem onde. Está claro que não poderei ficar na fazenda,
nem talvez mesmo aqui. É uma questão a resolver. Por enquanto, confesso,
não tenho grande confiança em mim\ldots{}}\looseness=-1

--- Ora essa! Por quê?

\textls[15]{--- Porque vivo num eterno balanço de ideia para ideia, sem me fixar em
nenhuma. Chego mesmo a recear ter perdido tempo na Europa, há de crer?}

\textls[15]{--- Não diga isso. O senhor é novo, é inteligente, teve o bom tino de
escolher a carreira da engenharia, que eu considero a mais proveitosa ao
seu país, onde há muito a construir, e terá depressa ocasião de exercer
com vantagem a sua atividade. Compreenderá então o enormíssimo proveito
que lhe ficou dos admiráveis modelos que viu e das boas lições que por
certo recebeu. Na verdade, também no Brasil há bons mestres e bons
estudantes, mas falta-lhes o ambiente favorável, que na Europa souberam
criar tantíssimas gerações de estudiosos. O senhor não pôde com
sinceridade julgar improfícuas as coisas que viu no velho mundo. A
impressão deixada por elas rasgar-lhe-á um largo horizonte diante dos
olhos. Se ao princípio todas as suas belezas lhe tumultuarem
desordenadamente no espírito agora, amadurecido e acalmado o entusiasmo,
hão de ser admiradas em todas as suas verdadeiras dimensões. Olhe, um
pintor adianta-se indubitavelmente mais visitando durante um mês o
Louvre do que em trabalhar durante um ano, numa cidade sem
museus. Eu, apesar de velho, tenho na alma pruridos de ambição: os de
correr mundo, estudando deleitosamente, e se não viajo nem vejo é porque
há uma razão simplicíssima que me obriga a isso\ldots{} é fácil de
compreender qual seja: a pobreza.}

\textls[-33]{O doutor mostrava, falando, os seus brancos dentes muito sãos, num
sorriso manso; pousava as mãos curtas e gordas nos braços da cadeira,
todo iluminado pela luz do dia, que entrava pela janela fronteira.}\looseness=-1

\textls[-15]{--- Eu, sinceramente, admiro-me de que o doutor, instruído e observador
como é, se resigne a viver neste canto do mundo, onde com certeza não
sobejam distrações para um espírito como o seu\ldots{}}\looseness=-1

\textls[15]{--- Engana-se, meu amigo; vivo perfeitamente. A índole do povo
brasileiro é de uma bondade cativante; a sua franqueza, encantadora. A
não se viver no país em que se haja nascido, não há pátria mais bela nem
onde tanto à vontade a gente se ache. Aqui estou há muitos anos e nunca
pensei em retirar-me que não sentisse antecipadamente saudades. \label{Ref2} Há uma
única coisa que me repugna e entristece aqui, escuso de lhe dizer qual
seja, o meu amigo adivinha-a; mas essa mesma extinguir-se-á em breve,
porque, Otávio, agora o Brasil não dorme, trabalha.}

\textls[10]{--- Tenho acompanhado com júbilo o movimento abolicionista do Brasil;
procurei avidamente nas correspondências dos jornais, sempre, tudo o que
se referia a essa grande obra; contudo, de tão longe, não se pode fazer
ideia perfeita das coisas que os jornais exageram e as traduções
adulteram.}

\textls[15]{--- Cuidado! Não vá expender ideias de progresso e humanidade no seio
da sua fazenda. A abolição, mais dia menos dia, faz-se; a amizade na
família é que uma vez quebrada nunca mais torna a ser verdadeiramente
sólida.}

--- Por que diz isso?

--- Porque seu pai é um dos maiores inimigos da abolição. Compreende agora?

\textls[-30]{--- Perfeitamente; serei discreto, se me convencer de que o meu braço
não é em nada preciso à santa causa, como dizem os periódicos.}\looseness=-1

O doutor ia começar uma resposta, quando um criado veio dizer que o
cavalo estava à porta.

--- Bem, Otávio, não quero ter o egoísmo de o reter aqui: o meu amigo
deve estar ansioso por chegar a Santa Genoveva, não é assim? Olhe,
talvez queira um companheiro que o guie\ldots{}

\textls[30]{--- Não me faça a injustiça de supor-me esquecido do caminho de casa!}


--- Tem razão! Pudesse eu hoje chegar à minha aldeia, que iria de olhos
fechados à velha casa de meus pais! Note que este é um sentimento que se
apura quando se está exilado, quer voluntária, quer involuntariamente: o
amor da família e do país natal.

Depois, tirando o relógio, calculou:

--- Daqui a hora e meia estará entre os seus\ldots{} boa viagem.

--- Obrigado, doutor.

--- Apareça para uma prosa!

--- Sim, senhor, adeus!

Batendo nas pedras da calçada, o animal seguiu.

\textls[-20]{O doutor entrou de novo para o seu gabinete, estendeu-se no sofá
de palhinha, retomou o livro e, cobrindo a calva com o boné de seda
preta, recomeçou em toada monótona a sua interrompida leitura.}\looseness=-1

%\end{linenumbers}

\section{II}

%\begin{linenumbers}

\textls[-10]{Otávio olhava carinhosamente para tudo, ao passar. Revia com ternura
aquelas casas baixas com porta ao centro e igual número de janelas de
cada lado; a matriz, a botica do Cunha, a loja do Vidigão com peças de
baeta e de algodão grosso mineiro apinhadas na porta. Reconheceu com
alegria o teatro, o Hotel do Comércio com a sua grande tabuleta, o
casarão fechado do fazendeiro Bastos e logo adiante a marcenaria do
Tibúrcio, mulato, ex-escravo do avô e casado com a Sinhana das Taipas,
caboclinha dengosa com quem ele brincara em pequeno, no sítio dos avós à
margem do Atibaia.}\looseness=-1

\textls[3]{O cavalo do dr.\,Morton parecia conhecer bem o caminho. Da terra
muito seca levantava-se uma nuvem de poeira avermelhada e fina. O sol
estava quente. Pouco a pouco as habitações rareavam, já quase fora da
cidade viu dois chalés novos nos terrenos do antigo
chefe do partido conservador, o major Caetano, cuja casa em decadência
aparecia além, cercada pelos muros, onde as chuvas tinham posto grandes
laivos verdes. A estrada precipitava-se agora em uma rampa viva para um
vale estreito, em que pretas de saias curtas e em mangas de camisa
lavavam roupa num riacho pedregoso. Seguia-se depois, na ladeira oposta,
a propriedade do conselheiro Betencourt, com sacadas sobre o jardim e um
pombal verde rumorejante de asas. Além, montes verdes tachonados de
pedras claras e até eles um campo vasto, em ligeiras ondulações cobertas
de barba-de-bode. De vez em quando passava um trole, fazendo nos seus
solavancos tremerem os corpos das passageiras, cobertas com guarda-pós
de linho e chapéus de palha com véus de cor. Depois de um largo espaço
de solidão, cruzou-se com uma família de roceiros que vinha a cavalo e em
fila, um a um, homens na frente, as mulheres atrás, com saiotes de ganga
amarela e chapelinhos guarnecidos de fitas. Deviam vir de longe. Traziam
o ar sério e fatigado.}

\textls[10]{O pitoresco daqueles quadros despertava-lhe na memória a lembrança do
tempo passado, a que a nota plangente de um carro de bois ainda distante
aumentava a poesia.}

E dentro de pouco ele tinha de encostar o seu animal a uma barreira
erriçada de gravatás para dar passagem a esse mesmo carro, puxado por
bois suados, de olhar triste. Um negro de calças arregaçadas e camisa
aberta no peito seguia a pé ao lado, e em cima, sobre lenha acumulada
até uma grande altura, um moleque chupava uma laranja, deitado de bruços
e firme nos cotovelos. Otávio reconhecia aqueles tipos; toda a sua
infância fora passada ali, tinha recordações vivas de tudo. Estava
contente.

\textls[10]{Transposta uma aguada onde o animal bebeu demoradamente, a estrada
bifurcava-se; a da direita seguia em linha reta, a da esquerda subia em
zigue-zague, e, por ser mais estreita, era mais ensombrada. Por ali passava
menos gente, era quase um caminho particular, comum unicamente a dois ou
três fazendeiros. Otávio tomou esta estrada; daí por diante teve um
único encontro, o de um sujeito fiel aos antigos costumes de viajor
paulista: \label{viajor}grande pala cor de café com leite, riscado de branco, que lhe
caía dos ombros arredondando-se sobre as ancas do cavalo, botas até o
joelho, esporas de prata, chapéu desabado e chicote de grosso cabo de
couro; atrás dele, em respeitosa distância, galopava o pajem, com uma
forte manta listrada de vermelho, enrolada como um travesseiro e posta
na frente, sobre o selim. Depois, mais ninguém. Árvores grandes, cipós
entrançados, cantos agudos de pássaros, rumorejar de águas em abismos
perfumados de baunilha, abismos de uma frescura deliciosa, todos
vestidos de folhas claras e florinhas coloridas.}\looseness=-1

\textls[15]{Decorrida uma hora de marcha, Otávio viu ao longe, do lado esquerdo do
vale, sobre a outra colina fronteira, a torre do sino de Santa Genoveva
e as paredes brancas da casa de seus pais. O coração bateu-lhe
violentamente.}

\textls[-15]{Por uma dessas circunstâncias que fazem às vezes com que coisas diversas
se apresentem ao mesmo tempo na memória, Otávio reviu mentalmente os
dias da sua meninice e aqueles em que, já adolescente, voltava das
caçadas com o pai, cheio de fome. Haveria ainda muitas pacas e codornas
na selva penhascosa da Lapa Negra?}\looseness=-1

\textls[-20]{Passada a curva do jequitibá grande, topou com um negro que vencia o
morro a largas passadas e que o saudou com um soturno:}\looseness=-1

--- Sum Cristo!

\textls[20]{--- Sabe-me dizer se há algum atalho novo para a Fazenda de Santa Genoveva?}

--- Pra o sito do coroné Medero? Não há não, sinhô. Eu
também tô indo pra lá.

E atentando em Otávio:

--- A modo qui tô conhecendo mecê\ldots{}

--- Está, sim. E você, como se chama?

\textls[-10]{--- Me chamo Antonho; fio de Luzia pernambucana, sim sinhô.}

--- Foi a algum recado à cidade?

--- Fui na vila buscá remédio pro fio do feitô, qui
foi moldido de cobra, sim sinhô\ldots{}

\textls[15]{--- Ah, então apresse-se --- disse Otávio, para dizer alguma coisa, e
tocou o animal para diante.}

O dia transformava-se subitamente; o sol escondia-se e uma aragem quase
fresca substituía o calor de há pouco. Ouviam-se já as vozes dos
escravos, vindas de além, num ritmo monótono.

\textls[10]{Depois de ter percorrido um extenso túnel de arvoredo, Otávio deparou
enfim com a porteira da fazenda, que transpôs com alvoroço. Achou-se no
cafezal. Percorreu uma grande extensão de caminho no mais profundo
silêncio, entre os pés de café, cujos galhos vergavam cobertos de fruto
ainda verde; ladeou depois o pasto velho, em que à sombra de um
pau-d'alho vacas lambiam amorosamente os novilhos, e desceu até à chapada
do açude, que margeou para subir depois do outro lado, rente ao muro da
horta engrinaldada por um virente pé de buchas. Ao vê-lo aproximar-se,
um grupo de crioulinhos sujos, que andava a brincar na frente da casa,
correu a abrir-lhe a cancela e ele entrou assim alegremente no terreiro
da sua residência.}

\textls[20]{Logo à esquerda, à porta do galinheiro, viu, de costas para ele,
uma mulher vestida de claro, com os cabelos negros entrançados na nuca.}

\textls[20]{Ela atirava mancheias de milho às aves, de um balaio que trazia
no braço.}

\textls[20]{``Noêmia é uma criança e Nicota é loura. Quem será aquela rapariga?''}

\textls[15]{Nesse instante mesmo ouviu uma exclamação de alegria, e
levantando os olhos viu, no patamar alpendrado da escada, a irmã mais
velha, que o tinha reconhecido.}

--- Otávio!

--- Nicota!

\textls[25]{O moço apeou-se à pressa e, tendo abraçado a irmã, seguiu com ela
até à sala de jantar, muito extensa, iluminada pelas janelas das
extremidades.}

\textls[15]{A mãe, sentada na rede, com as pernas cruzadas, escolhia ervas
para o jantar, a tenra cambuquira}\footnote{Grelos de abóbora, prato
  muito usado na província.} \textls[15]{e o agrião, separando-as de um
  cestinho para outro. O filho correu a abraçá-la e a pobre senhora,
  sufocada de susto e de alegria, desatou a chorar.}

\textls[15]{A saleta de costura era contígua e, ouvindo a bulha, Noêmia, a
irmã mais nova, abandonou a lição e, deixando a mestra só, veio cair
risonha nos braços de Otávio.}

\textls[15]{Era uma criaturinha delicada, sem ser bonita, de olhos garços,
cabelos castanhos, rosada, ativa e baixinha.}

Nicota era loura, alta, cheia, um todo grave e lindo, a mais
formosa da família, e a quem até os pais parecia respeitarem.

--- E pai?\footnote{Conserva-se fielmente a linguagem
  provinciana; em São Paulo, pelo menos, no interior não se costuma
  dizer ``papai''.} --- perguntou o recém-chegado à irmã mais
velha.

--- Está lá dentro; eu mando chamar.\footnote{Esta frase, como
  outras, pronunciadas pelas personagens, não é correta, mas é
  verdadeira.}

--- Não! Prefiro ir lá ter com ele.

\textls[20]{Acompanhado da mãe e irmãs, Otávio atravessou o longo corredor
sombrio até a uma das salas da frente. Impelida a porta, entraram.}\looseness=-1

Nenhuma alteração havia ali. Como há sete anos, a mesma mobília
tinha a mesma disposição: o sofá e as cadeiras de braços em frente às
janelas, o piano coberto de chita de ramagens grandes, duas redes ao
fundo, uma de cada lado, entre ambas o consolo antigo, absolutamente
despido de enfeites.

O comendador Medeiros dormia numa das redes, com a barriga para
o ar, a boca entreaberta, o chapéu de feltro caído sobre os olhos e o
chicote de tatu estirado ao comprido no chão.

Despertando ao ruído dos passos e das risadinhas agudas de
Noêmia, deparou atônito com o filho.

--- Que diabo! --- exclamou contendo a sua alegria. --- Então
vindes\footnote{É muito usado no interior de São Paulo o tratamento na
  segunda pessoa do plural.} assim, sem avisar a gente!

E abraçaram-se.

--- Supus que isso não o contrariasse\ldots{}

--- Decerto que não. Já estais formado, agora é tratar da vida.
Isto aqui pode ser que não seja tão bonito, mas sempre é melhor que a
Europa\ldots{} Viestes por Santos?

--- Sim, senhor.

--- Quem vos deu condução para vir para o sítio?

\textls[-30]{--- Dr.\,Morton. Vim a cavalo e deixei na estação a minha bagagem.}\looseness=-1

O fazendeiro espreguiçou-se disfarçando a comoção, depois disse,
voltando-se para a mulher:

--- Mandai um portador buscar os trens dele na cidade.

\textls[10]{Entretanto a mucama Leopoldina entrava com uma bandeja cheia de
canequinhas de café, que Otávio saudou com uma exclamação de
prazer.}\looseness=-1

O fazendeiro continuou:

--- Olhai, Nicota, que o vosso noivo vem jantar cá. Vem também o
dr.\,Azevedo, mais o compadre Antunes.

--- Já sei que Nicota está para casar. Deu-me o dr.\,Morton essa
novidade\ldots{}

--- O dr.\,Morton bem que podia ter avisado a gente pelo telefone ---
notou a irmã Nicota com amuo. --- Não quis ter trabalho!

--- Eta velho duro, aquele! --- observou o dono da casa com a admiração
a luzir-lhe nos olhinhos pardos.

--- E que te disse ele de Eva? --- indagou Noêmia.

--- Não me disse nada. Quem é?

--- É a nossa prima, filha do tio Gabriel; mora agora conosco!

--- Tio Gabriel!\ldots{} Ele fez as pazes com pai? --- perguntou Otávio
voltando-se para o comendador.

--- Antes não fizesse --- respondeu-lhe este.

--- Ora essa! Por quê?

--- Deixou-me a filha, que é\ldots{}

--- Um anjo! --- acudiu Noêmia.

--- Qual anjo! Um diabo levado de seiscentos! --- concluiu enraivecido o
fazendeiro. Nicota sorriu, Noêmia baixou entristecida a cabeça, a mãe
voltou impassível os olhos para fora, e Otávio achou mais prudente
mudar de assunto.

\textls[-10]{As horas avançavam e as senhoras retiraram-se uma para dar ordens, as
outras para fazerem a sua toalete e escreverem às
amigas participando a chegada do irmão. O comendador e o filho ficaram
sós e puseram-se a discutir assuntos de lavoura. Otávio ouvia com
desgosto o pai expender ideias antigas, pejadas de rancor e de fastio; de
vez em quando aventurava um aparte ou um elogio aos processos novos,
fugindo de o molestar, como se tratasse com uma criança. O fazendeiro
repelia indignado as ideias do filho e, firmando-se nelas, seguia
disparatadamente contra os reformadores, contra as modernas teorias,
contra tudo e todos.}\looseness=-1

\label{Ref3}
--- A vida agora no Brasil é um inferno. Em São Paulo, um tal Luiz Gama e
outro que tal Antônio Bento especulam com os pobres dos lavradores,
tirando-lhes os escravos. Os jornalistas do Rio são a mesma corja. Eles
acoitam os pretos fugidos para os alugarem por sua conta e irem fazer
conferências públicas, nos teatros, pregando a emancipação! É por isso
que a gente séria, os chama de ``pescadores de águas turvas''. José do 
Patrocínio é o chefe dessa bandalheira, que, se o país tivesse
governo, já teria acabado. É por isso mesmo que muitos liberais e muitos
conservadores estão se passando para o partido republicano\ldots{ 

\textls[10]{Otávio estremeceu, mas absteve-se de falar. Deixaria passar a onda
amarga em silêncio. Reservava-se para depois.}

\textls[10]{Supunha poder demolir pouco a pouco o brônzeo egoísmo do pai e vê-lo 
enfim cooperar na grande obra de humanidade e patriotismo. Precisava
procurar com cuidado as ocasiões propícias para o completo
desenvolvimento da sua ideia. Naquele momento tudo seria inútil; o
comendador, muito exaltado, não o escutaria, e ele era incapaz nesse dia
de sustentar com o \emph{velho}, para cujos braços voltava
cheio de alegria, uma questão qualquer. Susteve-se, enquanto o pai
continuava amaldiçoando o tempo dos abusos e dos ataques à propriedade
alheia!}

--- Se eles se lembrarem de vir a Santa Genoveva --- exclamava ---, os bandidos
dos abolicionistas, eu sei como os hei de receber: a tiro! Defendo a
minha propriedade, estou no meu direito. A culpa é também das
autoridades, que não amoldaçam esses cachorros dos
jornais que latem, latem para os outros morderem!

\textls[15]{Nesse ponto, bateram de manso à porta, e uma voz de mulher perguntou de
fora:}

--- Dá licença, meu tio?

--- Mau, lá vem a lambisgoia!\ldots{} Entre!

Otávio levantou-se, e, recuando um pouco, encostou-se ao piano; a porta,
impelida docemente, deu passagem à mesma pessoa que ele vira de costas,
dando milho às aves.

\textls[20]{--- Você chegou em bem má ocasião\ldots{} --- disse o comendador secamente.}

--- Demoro-me pouco.

\textls[-35]{Otávio não fora notado e observava com atenção a recém-chegada.}

\textls[10]{Era uma mulher nova, esbelta, morena, de fartos cabelos negros, rosto
oval, olhos franjados por longas pestanas, feições regulares sem serem
belas, andar firme, cabeça erguida sem afetação. Tinha a voz grave, a
atitude serena. Vestia com simplicidade o seu vestido de percal,
escrupulosamente ajustado.}

--- Que temos? --- indagou o tio. \label{Ref5}

--- Venho pedir-lhe que perdoe ao Manuel Sabino; ele promete
ser obediente daqui por diante. Mande tirar-lhe os ferros, sim?

\textls[15]{--- Asneira! Deixe-se disso, que não é da competência das moças. Se não
quiser ver o negro com os ferros, não olhe para ele. Era o que faltava!}

--- Não olho, mas nem assim deixo de saber que os traz.

\textls[15]{O comendador deu uma gargalhada. Pelos olhos de Eva passou um relâmpago
de indignação, mas conteve-se e um sorriso de desdém arqueou-lhe os
lábios.}

\textls[-10]{--- Já não sei quantas vezes tenho, a seu pedido, perdoado faltas dos
escravos! Olhe, é melhor que se vá preparar para o jantar; aqui está meu
filho, que chegou hoje, e espero amigos nesta meia hora\ldots{}}\looseness=-1

\textls[20]{Eva voltou os olhos para Otávio, a quem cumprimentou friamente, sem
avançar um passo; depois, num tom de quem se desculpa, disse:}\looseness=-1

--- Eu não sabia da sua chegada; venho neste momento\ldots{}

--- De alguma senzala --- interrompeu com ironia o tio.

\textls[15]{--- É verdade --- confirmou ela. --- De uma senzala. Fui ver a Josefa,
que está doente. À saída encontrei o Manuel, que me pediu que o
apadrinhasse; prometi vir em seu socorro e atravessei logo para
aqui\ldots{}}

--- Não deve prometer o que não pode cumprir.

\textls[15]{Eva olhou para o primo, como a pedir-lhe auxílio. Otávio, aproximando-se
do fazendeiro, disse, comovido:}

\textls[10]{--- A minha chegada justificará a clemência que tiver para com ele. Em
nome da grande alegria de nos tornarmos a ver, peço-lhe, meu pai, que
atenda aos rogos da prima Eva.}

O comendador fingiu refletir um momento, e, voltando-se para a
sobrinha, disse:

--- Está bom! Por hoje perdoo, mas não torne a fazer semelhantes pedidos; não torne a fazer!

--- Obrigada. --- E Eva saiu da sala sem precipitação.

\textls[15]{Otávio sentiu avivar-se-lhe a curiosidade a respeito da história daquela
prima, que não conhecera nunca, e que vinha encontrar debaixo do teto
paterno, tratada por uns como um anjo, e por outros como um demônio.
Avaliou um momento a triste posição de Eva, recebendo por
caridade a sombra de um telhado e o pão de um velho e encarniçado
inimigo de seu pai. Absteve-se, contudo, de qualquer pergunta naquela
ocasião em que via o comendador excitado contra ela; pensou sensatamente
que qualquer informação seria apaixonada, e reservou-se para mais tarde,
quando o visse de ânimo tranquilo. E no fundo do seu espírito havia já a
convicção de que a opinião de Noêmia era a justa: ``Eva é um anjo!'',
dissera ela, e ele compreendia-a depois de ter presenciado aquela cena.
Só os anjos arrostam com a má vontade dos poderosos a favor dos fracos e
dos oprimidos; só os anjos suportam injúrias com humildade quando a
causa que advogam é a dos desgraçados.}

Sim, Noêmia tinha razão\ldots{} Eva é um anjo!

%\end{linenumbers}

\section{III}

%\begin{linenumbers}

\textls[5]{A dona da casa tivera o cuidado de mandar buscar com urgência as malas
do filho. Durante as horas de palestra familiar, de descrições da
viagem, dos exames e de vários episódios com que se entretiveram,
seguira o pajem a toda a brida, no desempenho dessa previdente missão.
Às quatro horas, Otávio reformava com satisfação a sua
toalete, no seu quarto de outrora, um quarto branco,
pequeno, com uma janela de peitoril sobre o campo; às cinco, reunia-se
ao pai, na mesma sala da frente, onde já era esperado com impaciência e
para onde o haviam chamado, quando ele ainda começava a abotoar o
colarinho em frente ao espelho. Chegando ao corredor, viu que paravam no
terreiro os troles dos visitantes. O pai fez-lhe sinal de que se
aproximasse.}

\textls[-20]{O comendador Medeiros esperava de pé, no patamar, radiante de alegria, os
amigos que ia sucessivamente apresentando ao filho.}\looseness=-1

--- Major Trigueiros, futuro sogro de Nicota\ldots{}

O major Trigueiros era um velho alto e magro, de grandes bigodes e pera
branca, cara curta, engelhada, olhos acastanhados e redondos, movimentos
esquisitos e angulosos, trazendo à ideia de quem atentasse nele o
todo extravagante da cegonha. Otávio cumprimentou-o afavelmente.
Seguiu-se o noivo de Nicota, Álvaro Trigueiros, um rapaz baixo, moreno,
de barba rala, rente ao rosto inexpressivo, beiços finos e rasgados,
cabelo caído na testa, numa pasta luzidia e chata. Agora era o Azevedo,
promotor público, rapaz de estatura mediana, claro e louro, com os olhos
muito azuis a brilhar através das lunetas, barba em ponta, pele bem
tratada. Por fim, subiu o compadre Antunes, o único já conhecido de
Otávio, homem gordo, grisalho, de suíças curtas e nariz pequeno
enterrado entre as bochechas carnudas; colete desabotoado, casaco a
luzir nas costuras, lábios grossos, unhas rentes, ex-feitor da fazenda
de Santa Genoveva, cargo que exercera durante anos e de que se despedira
para tomar conta da lavoura de um filho, que lhe morrera vítima dos
escravos.

\textls[10]{Entraram todos para a sala dos homens.}\footnote{\textls[15]{Em geral nas
  velhas casas paulistas havia uma sala destinada exclusivamente aos
  homens.}} \textls[10]{Depois de meia dúzia de perguntas banais, sobre a
  viagem, e dos parabéns pela volta de Otávio, distribuíram-se aos
  grupos, conversando descansadamente sobre as últimas eleições e futuras
  colheitas, alforrias e corridas de cavalos. O major Trigueiros elevava a
  voz áspera e cortante, acima de todas as mais, desfechando raios de
  cólera sobre os conservadores, que tinham feito uma grande entrada na
  urna eleitoral, enquanto o compadre Antunes perguntava ao Azevedo se
  havia gostado da última corrida no hipódromo de Campinas. --- Que sim,
  afirmava o promotor, e que, só para ver as moças bonitas de Campinas,
  valia a pena ir lá. --- E a égua do Bernardino Aranha, hein? Que bom
  animal! É inglesa e legítima! Ganhei nela há três meses; aquilo é que
  é!}

\textls[-25]{Otávio aproximou-se do noivo da irmã; Trigueirinhos remexeu-se
e falou-lhe de um modo constrangido, trocando amiúde o \emph{l} pelo \emph{r}.
Cansado da viagem e da conversa, Otávio foi encostar-se a um janela.}\looseness=-1

\textls[-10]{O terreiro de tijolo, para a seca do café, estendia-se muito
limpo e largo diante da casa. Ao pé da escada de pedra, dormiam dois
cães, estiradamente ao sol; lá embaixo no grande tanque}\looseness=-1\footnote{Os
paulistas chamam ``tanque'' à represa, que acidental ou
proposital, há em quase todas as fazendas, para aguada da
criação e serviço da casa.} \textls[-10]{havia cintilações
douradas de luz na água serena, e os pombos voavam aos bandos de entre
uma cerrada touceira de bambus. Os troles, sem cavalos,
inclinados para a frente sobre os varais, alinhavam-se à sombra, e do
lado oposto à cancela, entre as palhas de milho espalhadas, fossavam os
porcos. Muito além, fechando o horizonte, a floresta unia com uma linha
lutuosa e reta a terra ao céu.}\looseness=-1

\textls[15]{Aquela paisagem entretinha-o mais do que tudo que se dissesse lá
dentro. Otávio deixou-se ali, longamente, até que o foram chamar para o
jantar.}

\textls[-10]{Na grande sala, a mesa, coberta de cristais, oferecia um aspecto
brilhante. Ao fundo, as senhoras conversavam. Otávio foi apresentado à
mestra de Noêmia, mme.\,Grüber. Os servos e mesmo a dona da casa
simplificavam-lhe o nome, chamando-a Madama, simplesmente. Era uma
senhora de quarenta anos, alta, magra, muito loura, vestida de castanho,
com um colarinho de homem e um alfinete redondo, de marfim,
segurando-lhe a gola do vestido.}\looseness=-1

Nicota e Noêmia trajavam irmãmente de azul, com lacinhos de veludo preto
nos punhos e no pescoço. A mãe ia e vinha, falando baixo com as mucamas,
fazendo tilintar as chaves dos armários, atenta e cuidadosa para que não
faltasse coisa alguma.

\textls[-5]{Sentaram-se à mesa; de um lado ficaram os homens, do outro lado as
senhoras, no sistema paulista; só lá para o fim da mesa se alterara a
ordem por falta de espaço e passou o promotor para junto das senhoras.
Otávio percorreu a vista pelo recinto, admirado de não ver a prima;
quando ela apareceu, sentou-se entre a professora e o dr.\,Azevedo, que,
ao vê-la, se levantou corado. À cabeceira o dono da casa falava muito
alto aos convivas. Otávio, ao pé, assistia às baterias de perguntas e
respostas. Era uma bulha de vir tudo abaixo! O moço de vez em quando
olhava em redor. O Trigueirinhos comia com a faca, olhando para o prato;
não bebia vinho, molhava de vez em quando os beiços num copo de água, e
continuava depois muito sério a encher a boca de feijão, ervas e pastéis
folhados; em frente dele, a noiva, menos preocupada com o prato,
lançava-lhe de muito em muito longe um olhar rápido; Noêmia ria alto,
dando às vezes uns guinchichos agudos, ao ouvir o compadre Antunes
contar velhas anedotas colhidas nos almanaques; a mãe apontava aos
pajens os copos a encher e os pratos a renovar, e, lá no fundo, mme.\,
Grüber comia sem interrupção e o dr.\,Azevedo curvava-se falando para
Eva, que o escutava distraída, empurrando com o pão a comida para o
garfo.}\looseness=-1

\textls[10]{Travara-se uma discussão entre o major Trigueiros e o dono da casa; as
vozes foram subindo, altercavam gritando. O comendador Medeiros
retrucava com desdém às afirmações do outro: ``Capaz! Capaz!''. E o
cavanhaque do major salientava-se, vinha para a frente, ia para
trás, num movimento contínuo. Entretanto, as outras pessoas tratavam de
falar mais alto entre si, para que as suas vozes não fossem abafadas. De
repente a questão acabou. O comendador bebeu um copinho de vinho do
Porto com água, e o major, esquecendo momentaneamente que não estava em
sua casa, fincou o calcanhar esquerdo no banco em que se assentava,
elevando o joelho pontudo à altura da barba.}

\textls[10]{Aproveitando o momento de menos bulha, o Azevedo levantou-se, ergueu o
copo e brindou à família Medeiros, felicitando-a pelo regresso de um de
seus membros: arranjou um discurso florido, num estilo guindado, onde de
vez em quando aparecia, como um espantalho, um nome histórico. Aquilo
durou. Ele lançava a voz em inflexões de efeito, arrastando-a dos sons
mais graves ao mais agudo falsete.}

\textls[-10]{As senhoras ouviam-no, paradas, com os olhos fitos nele. Como isso fosse
pelas alturas da sobremesa, o major Trigueiros mergulhava no seu grande
prato, transbordante de leite, um grosso naco de abóbora açucarada, e o
compadre Antunes ia devastando os cálices de doce de batata-roxa, coisa
muito da sua predileção.}\looseness=-1

Quando se levantaram da mesa, Otávio suspirou de alívio: precisava de
ar fresco e de descansar os ouvidos aturdidos na distração de um
passeio ao jardim com as irmãs e a prima.

O comendador, porém, arrastou-o com os amigos a ver a nova casa da
máquina, extensa e assente no baixio de uma colina, por onde desceram à
sombra de limoeiros floridos; dali passaram ao açude, ao paiol, ao
moinho e à estrada nova, ladeada pelos canaviais de um verde macio e
fresco, até ao cafezal, onde os carreadores muito limpos se estendiam a
perder de vista. O comendador, à frente, guiava a comitiva, orgulhoso da
sua propriedade. Os outros comentavam alto o que iam vendo.

\textls[15]{--- Olhe, meu amigo --- clamava o major Trigueiros ---, as suas terras parece
que já estão cansadas\ldots{}}

--- Qual! --- protestava o fazendeiro --- Nunca deram tanto como agora!

--- Isso não quer dizer nada. Desconfio muito desta secura e amarelidão!
--- E indicava com o beiço inferior o terreno. --- A minha é roxa, que lá as
terras do sertão valem muito mais\ldots{}

\textls[20]{O Azevedo ia ao lado do Trigueirinhos conversando num tom discreto, e o
compadre Antunes, que ia atrás, reteve Otávio e perguntou-lhe:}\looseness=-1

--- Que tal lhe pareceu sua prima?

\textls[20]{Como não recebesse em resposta senão um olhar de estranheza, continuou:}

\textls[15]{--- Eu explico. Aquela moça é perigosa. Foi bom que o senhor chegasse
para tomar sentido em certas coisas\ldots{}}

\textls[30]{A um sorriso de incredulidade de Otávio, Antunes afirmou com mais
veemência:}

\textls[15]{--- É perigosa, sim senhor. Detesta seu pai e gosta do Azevedo, que lhe
está com o olho no dote\ldots{}}

--- E então? --- perguntou Otávio, sem atinar com a razão daquela
advertência.

\textls[15]{--- Um tio do Azevedo é dos tais abolicionistas da capital, que espalha
agentes por toda a parte, entendeu?}

\textls[15]{--- Parece-me que sim. É uma história parecida com aquela do Castelo de Chochurumelo.}

--- Isso é que eu não afirmo, porque não conheço essa história!

\textls[15]{--- Pois, meu amigo, é grave. Ora, imagine que é a de um cão, que mordeu
um gato, que matou o rato, que roeu a correia, que atava a chave do
Castelo de Chochurumelo!}

O Antunes indignou-se, disfarçando a cólera com um riso amarelo.

\textls[10]{--- Não cuide que eu brinco --- continuou. --- Tomai sentido! Seu pai já sabe
de muitas coisas, e por ser prudente, cala-se; está convencido, e tem
provas, de que a sobrinha quer arruiná-lo, vingando assim o finado
Gabriel\ldots{} Ora, o tempo provará se eu tenho razão em recomendar-lhe
vigilância! E demais, aquele procedimento com o Azevedo é
vergonhoso\ldots{} é preciso decidir-se o casamento, e\ldots{}
que se ponham a andar!}

Otávio respondeu com altivez ao Antunes, notando-lhe que lhe não
competia intervenção em tão sérios negócios, e concluiu:

--- Eva é livre; fará o que entender.

\textls[25]{--- Lembre-se que, afinal, seu pai é que é o responsável pelas asneiras
dela!}

--- Não lhe dê isso cuidado\ldots{}

O tom seco e decisivo de Otávio emudeceu o ex-feitor de Santa Genoveva,
que mastigou umas palavras ininteligíveis.

Na frente o grupo ia descuidado e falador.

Antunes apertou o passo e foi ter com o major Trigueiros. Otávio olhava
para as costas estreitas do Azevedo, para o seu corpo efeminado e bem
cuidado; o cabelo louro carapinhoso, onde o sol punha reflexos
avermelhados, cor de cobre, a mão muito branca com que alisava a barba,
aparecendo acima do ombro esquerdo nuns movimentos vagarosos, de
carícia.

\textls[-20]{De repente o noivo de Nicota voltou-se; vendo o futuro cunhado
sozinho, propôs que o esperassem, e continuaram depois os três, indo
Otávio no meio. Então Trigueirinhos, já mais familiarizado,
perguntou-lhe se as árvores na Europa não eram raquíticas, se bebiam por
lá bom café, se havia luxo nas casas e gentileza no povo\ldots{}}\looseness=-1

\textls[15]{O Azevedo escondia sob o bigodinho louro um fino sorriso de ironia.
Otávio ia respondendo sempre com frases curtas, como se as palavras lhe
custassem um grande esforço.}

Trigueirinhos concluiu a palestra convidando-o para uma caçada às pacas
e às perdizes no seu sítio; era louco por esse gênero de divertimento.
Um mal de família, explicava ele. A mãe saía todos os dias para o mato,
com a sua espingarda; era uma destemida! O avô materno, apesar dos seus
oitenta anos, galopava pelos campos do Jaú, dias e dias, farejando
caça\ldots{}

\textls[15]{E aquele homenzinho baixo, magro, amarelo, de dentes postiços e
pastinhas lustrosas, empapadas de Oriza, falava nas correrias; nas
esperas no bosque, mostrando-se adestrado e ágil como um herói das
matas!}\looseness=-1

\textls[15]{Quando voltaram, os troles esperavam já prontos no terreiro. Os
visitantes partiram: Nicota e Trigueirinhos cumprimentaram-se friamente;
era a última vez que se veriam antes do casamento, marcado para desse a
quinze dias. Azevedo murmurou alguma coisa baixo a Eva, que lhe
respondeu disfarçadamente, chegando Otávio a perceber estas palavras:}

--- Eu escrevo amanhã.

--- Obrigado.

\textls[-25]{Eva retirara-se e entrara o feitor, que vinha prestar contas ao dono da
casa, censurando-o por ter mandado tirar os ferros ao Manuel.}\looseness=-1

\textls[-20]{O comendador explicou-lhe o fato e o feitor encolheu os ombros,
aborrecido. Começaram as indagações e narrações do trabalho.}\looseness=-1

\textls[10]{Otávio aproximou-se de Noêmia, que se debruçava na janela. A noite
estava fresca e escura; no céu profundo luziam as estrelas, e ele,
atraindo docemente a irmã, perguntou-lhe:}

--- Como passam vocês aqui as noites?

--- Mal\ldots{}

--- Sim?

\textls[-15]{--- Imagine: mãe acorda de madrugada e, como lida muito, logo à noitinha
tem sono. Pai também. Nicota balança-se na rede ou faz
crochê; eu não gosto de rede nem de
crochê, por isso brinco com o Vinagre ou com o Júpiter,
que fazem muitas sortes; quer ver?}\looseness=-1

--- Amanhã. E que janela é aquela, que está com luz?

--- É de mme.\,Grüber. Eva faz serão no quarto dela, leem e trabalham juntas.

--- Por que não fazem serão antes aqui, na sala, perto de todos?

\textls[10]{--- Porque o barulho não permite que entendam o que leem, e pai, bem
sabe, não gosta de livros em mãos de mulher\ldots{}}

--- Você nunca assiste aos serões?

--- Nunca!

--- Faz mal.

--- Por quê?

\textls[25]{--- Teria ocasião de adiantar-se e ocuparia o espírito durante algumas horas.}

\textls[20]{--- Eu não sou inteligente, não gosto de falar francês, e à noite estou cansada!}

--- Eva fala francês?

--- Fala. E um pouco de alemão.

--- Imagino\ldots{} aposto que não sabe coser!

--- Costura até muito bem\ldots{} Eva sabe tudo.

--- Vejo que você é muito amiga dela.

--- Sou.

--- E Nicota?

--- Essa não!

--- Por quê?

--- Acha Eva muito pretensiosa\ldots{}

--- E não será?

--- Qual!

\textls[20]{--- O dr.\,Azevedo parecia encantado com a rosa que ela tinha no peito\ldots{}}\looseness=-1

--- Ela gosta muito de flores.

--- E ele muito dela, não é assim?

--- Pode ser.

--- Nunca se falou nisso cá em casa?

--- Nunca.

--- Pois admira. Que idade tem Eva?

--- Vinte anos.

--- Só?

--- Nem parece ter mais.

--- É mais velha do que Nicota.

--- Um ano.

--- Mas qual será o motivo da antipatia de ambas?

\textls[20]{--- De ambas não. Eva gosta de Nicota; Nicota é que não gosta de Eva!}\looseness=-1

--- Ciúmes?

\textls[15]{--- De quem? Não; julgo que é porque Eva disse um dia que não se casaria
nunca sem amor, assim como\ldots{}}

--- Nicota.

--- É exato.

--- A quem disse ela isso?

--- A nós.

--- É romântica, mas afinal tem razão\ldots{}

\textls[20]{--- Pois foi esse o motivo; Nicota contou logo tudo a mãe, mãe contou a
pai e\ldots{}}

--- Pai zangou-se, está claro.

\textls[15]{--- E não quer que a gente esteja muito tempo perto dela, com medo que
nos pegue a sua maneira de pensar\ldots{}}

--- Como veio ela para cá?

--- Hein?

\textls[-10]{--- Quem a trouxe, qual o pretexto, como se arranjou tudo isso?}

\textls[15]{--- Eu estava no colégio, não sei de nada; mas quem a trouxe parece que
foi o dr.\,Morton.}

--- O dr.\,Morton!

\textls[15]{--- Sim. Ele era amigo do tio Gabriel e foi sempre o mestre de Eva; estimam-se muito.}

--- An\ldots{} Então você já a encontrou cá?

\textls[10]{--- Já. Havia cinco meses que ela morava em Santa Genoveva. Estava ainda
de luto fechado pela morte do pai; logo que me viu, desceu aquela escada
e foi-me abraçar. Eu não sabia quem era e fiquei admirada; depois
perguntei a Nicota, e, quando ela me respondeu que era nossa prima,
fiquei muito contente.}

--- Eva fala muito no pai?

\textls[15]{--- Pouco, mas chora muitas vezes e lê papéis dele. Quando vai
ao Mangueiral não tem ânimo de voltar de lá!}

\textls[20]{--- E que é o Mangueiral? Não me lembro que houvesse aqui uma única mangueira!}

--- Efetivamente aqui não há.

--- Então, onde é o Mangueiral?

--- É o sítio de Eva que se chama assim.

--- Quê! Eva tem fazenda?!

--- De que se admira? Tio Gabriel era rico.

--- Não me lembrei disso, julguei-a pobre. É perto daqui?

--- É.

--- Costuma ir com pai?

--- Não, com mme.\,Grüber, só; pai não gosta de a acompanhar.

--- Mas se Eva ficou rica, por que mora aqui?

\textls[15]{--- Porque foi a última vontade do tio Gabriel. Olhe, quem
vos pode explicar tudo é o dr.\,Morton.}

E Noêmia afastou-se, chamando Júpiter para as sortes.

\textls[3]{Na grande sala de jantar, a varanda,}\looseness=-1\footnote{Os
  paulistas costumam chamar ``varanda'' à sala de jantar, qualquer que
  seja a sua colocação na casa.} \textls[3]{um lampião preso à parede
    localizava a sua pouca luz num limitado espaço; todo o resto da casa, na
    sombra, tinha um aspecto sonolento e fúnebre. As redes rangiam nas
    argolas de ferro seguras aos umbrais; ouviam-se bocejos. O feitor
    continuava a descrever o seu dia, repisando as faltas do Manuel.
    Otávio não os quis perturbar e deixou-se à janela, olhando para o
    ponto da casa onde brilhava a luz de um candeeiro com abajur
    de porcelana; depois desceu a escada e seguiu até debaixo da janela
    iluminada, sentando-se no banco de pedra que parecia terem
    propositadamente posto ali para ele. As rãs coaxavam nos charcos, e na
    solidão sombria do céu as estrelas punham luminosos botões trêmulos. Lá
    de dentro, do quarto iluminado, vinha um murmúrio de vozes e um
    farfalhar de papéis.}\looseness=-1

Durante todo o dia ele não tivera ocasião de conversar com
nenhuma dessas mulheres, que eram indubitavelmente as de mais espírito
ali e cuja convivência mais distrações lhe ofereceria.

A passividade da mãe; a maneira de pensar do pai, tão contrária
à sua; a frieza da irmã mais velha e a ingenuidade da mais nova,
obrigá-lo-iam a uma vida concentrada, para que não sentia disposição.
Cansado da sua vida de rapaz, sozinho, sem lar, sem alegrias íntimas,
correra ansioso para a família e, logo no primeiro dia em que a viu e em
que a abraçou, convenceu-se de que não seria nunca completamente
compreendido por ela.

\textls[10]{Desiludido, lembrou-se das impertinentes insinuações do Antunes
e, como para confirmação delas, das palavras da prima ao Azevedo, na
despedida. Não tinha tempo para formar de Eva um juízo definitivo, e
via, com estranheza, julgá-la cada qual a seu modo. A sua figura altiva,
a sua fronte erguida sempre, como a desafiar o perigo, o seu olhar
sereno e andar firme, demonstravam-lhe uma natureza fria, orgulhosa,
inacessível; lembrava-se, porém, da sua voz clara, penetrante e meiga,
da sua intervenção pelo miserável escravo e vacilava entre a candura e a
compaixão ou o sentimento calculado e hipócrita. Viera encontrar em casa
um problema, um ser com asas para uns, com patas para outros, misterioso
e atraente por isso mesmo. Eva não se mostrara nem surpresa nem contente
pela chegada do primo; recebera-o como a um estranho de quem não tivesse
ouvido nunca o nome. Otávio refletiu nisso e tornava a lembrar-se das
palavras ditas ao Azevedo na penumbra do corredor. Daí, quem sabe?
Talvez o Antunes tivesse razão\ldots{}}\looseness=-1

A antipatia do pai pela sobrinha deveria ter um ponto de apoio que a
justificasse, Eva metera-se em casa de um velho inimigo da sua família,
com o fito de uma vingança qualquer\ldots{} era o que lhe queria ter
feito acreditar o Antunes, que afinal era um homem prático.
Repugnava-lhe, porém, imaginar que ela se prevalecesse da sua situação
para exercer ali uma espécie de espionagem, de acordo com os
abolicionistas. Uma impressão toda instintiva o fazia julgar a prima uma
rapariga leal, mas não podia duvidar de que ela era uma rapariga de
recursos e tendências muito diversas das de todos que a rodeavam. Fosse
como fosse, ele se sentia alegre em encontrar naquela solidão um motivo
de luta.

\textls[15]{Ia levantar-se, quando ouviu a voz de Eva, dizendo versos, perto da
janela; deixou-se então ficar no banco de pedra a ver sem atenção os
pirilampos cintilando aqui e além\ldots{}}

%\end{linenumbers}

\section{IV}

%\begin{linenumbers}

\textls[-15]{Na manhã seguinte Otávio acordou de madrugada. Adormecera tarde, apesar
do cansaço da viagem e das grandes comoções da véspera. Estranhara a
cama, sentira-se febril; mas quando, de manhã, abriu a janela e olhou
para os vastos campos iluminados pela luz violácea e doce do alvorecer,
sentiu-se reanimado e alegre. Lembrou-se do seu tempo de infância, quando
àquela hora fugia para o terreiro, com um copo na mão, pronto para o
leite espumante e ainda morno da vaca. Vestiu-se, desejoso de caminhar
pelo mato, por aqueles bosques, cujas árvores via dali. Saindo do
quarto, deparou com a mãe, que ia com uma mucama escolher verduras na
horta.}\looseness=-1

--- Tão cedo, mãe, já de pé!

Ela explicou-lhe que se levantava sempre de madrugada, para os arranjos
da casa, e aconselhou-o a tomar qualquer coisa.

--- Vou ao leite e depois irei ter com vossemecê à horta.

\textls[20]{Otávio atravessou o terreiro em direção ao curral, onde um negro alto
mungia uma vaca.}

\textls[15]{--- Eh! Ladina! Sinhô moço qué lete? --- perguntou o
escravo a Otávio, revirando para ele os olhos, cujo branco amarelado
sobressaía do fundo escuro do rosto.}

Otávio esvaziou um copo acabado de encher.

--- Eh! Sinhô moço já não se lembra de mim!

--- Como se chama você?

\textls[-10]{--- Eu? Eu me chamo Teodoro, sim sinhô\ldots{}}

--- Teodoro\ldots{}

\textls[-15]{--- Teodoro pequeno\ldots{} o Teodoro grande já morreu, sim sinhô.}\looseness=-1

--- Ah! O filho de Narcisa!

--- Sim sinhô.

--- Perfeitamente, pois então não me havia de lembrar!

O negro ria, e depois com um modo embaraçado:

--- A gente já maginava que mecê não havéra de vortá\ldots{}

--- De que morreu o Teodoro grande?

--- Foi obra de feitiçaria, sim sinhô.

--- E sua mãe?

--- Essa tá com aíva\footnote{Moléstia incurável.} da cegueira\ldots{} já não
véve aqui não sinhô\ldots{} Sinhô grande vendeu ela pra seu Antune\ldots{}

--- Ah\ldots{}

--- Barganhou ela com uma pareia de besta\ldots{}

\textls[10]{Otávio trocou mais algumas palavras com o Teodoro e seguiu para a
horta, cercada de muros, com uma porta estreita pintada de verde. Logo à
entrada, seguindo em linha reta por toda a grande extensão da rua,
havia uma parreira e aos lados, de vez em quando, uns pés de rosas de
todo o ano, com as suas flores aromáticas, onde esvoaçavam singelas
borboletinhas brancas ou cor de palha. Ao centro do terreno, uns
canteiros estreitos e um ou outro arbusto. Tomando por uma rua de
marmeleiros, Otávio foi ter com a mãe, que estava longe, arrancando,
com as duas mãos, pela rama escura, uma formidável beterraba.}

\textls[10]{Aqui e além, os grandes buquês azuis das hortênsias
punham alegria nos canteiros. Nas fazendas, a horta confunde-se,
geralmente, um pouco com o jardim: plantam-se rosas perto de nabos, e
desabrocham lírios junto de repolhos.}

\textls[10]{A horta de Santa Genoveva era pomar, horta e jardim,
tudo encerrado entre muros e afastado da habitação. O hortelão era um
negro velho e calvo, o Torquato, que, já não podendo trabalhar com
desembaraço no café, era aproveitado ali.}

\textls[10]{Otávio reconhecia tudo: nem mesmo as plantas pareciam ter variado,
somente o hortelão era outro; o velho Tomé de outrora tinha morrido,
naturalmente, pois já não estava em idade de ser vendido ou barganhado
por um animal qualquer\ldots{}}

Otávio aproximou-se da mãe, ajudando-a muito risonho a colher a verdura.

\textls[10]{Sentia-se leve, contente; respirava a plenos pulmões o ar fresco da
manhã, tinha as mãos molhadas do orvalho das plantas em que tocava, via
o rosto sereno da mãe banhado de suavidade, como a fisionomia das santas
dos altares.}

\textls[25]{Despediram-se à porta; ela ia ainda ver a criação, ele caminhar sem
destino.}

\textls[10]{Otávio, ladeando o muro da horta, foi até à porteira, abriu-a e desceu
por uma rampa escalavrada, onde um ou outro pé de erva, meio mastigada
pelos animais, punha manchas de um verde sujo. Na terra avermelhada e
muito seca rolavam crioulinhos nus, de ventres enormes e umbigos
salientes; embaixo, no sopé da colina, estendia-se o tanque, muito
sereno e espelhado; parou aí um pouco, fez com que entrassem na água uns
marrecos que passeavam pachorrentamente na margem; seguiu daí por um
caminho estreito e embrenhou-se na selva.}

\textls[15]{A pouco e pouco foi-se vendo cercado mais de perto por folhagens
espessas de um verde denso.}

\textls[-15]{Sobre a sua cabeça havia uma abóbada cerrada, por onde mal se
descortinava o céu. Tudo verde! Um verde brilhante, um verde uniforme,
que lhe dava a impressão de se achar dentro duma esmeralda imensa! A
própria luz do sol, coada pelas ramagens, caía esmeraldina por aquele
recinto, onde cada árvore tinha mais arrogância e majestade que as mais
imponentes colunas dos templos suntuosos.}\looseness=-1

\textls[-10]{Sobre a terra pegajosa e úmida, estendiam-se aqui e ali tapetes de
musgo veludoso, de um verde tenro, e o aroma da baunilha, suave e
dulcíssimo, voava pela floresta. Ao canto das aves, canto que parecia
gemer uma prece, acompanhava cá em baixo, num ritornelo grave, o
riachinho estreito, que ia em gorgolões sobre\linebreak} 

\pagebreak

\noindent{}\textls[-15]{seixos e troncos
quebrados, atravessando a mata. Nos recantos mais sombrios, a atmosfera
abafada tinha um cheiro forte de folhas apodrecidas, que em camadas
sobrepostas se aglomeravam no chão.}\looseness=-1

\textls[-10]{Subitamente Otávio parou encantado: diante dele ia e vinha, ora num voo
alto, ora num voo baixo, uma borboleta grande, de asas azuis com
arabescos de ouro. E logo atrás veio outra, igualmente formosa. Eram
borboletas assim que na Europa ornamentavam os museus, como
preciosidades raras, espalmadas, imóveis, sob os vidros das
montras, na fixidez e taciturnidade das coisas mortas;
aí estavam agora, diante dos seus olhos, palpitantes, ligeiras,
caprichosas, ora aqui, ora ali, ora acolá, já na sombra, já feridas por
um raio de sol caído através da ramagem opulenta do arvoredo.}\looseness=-1

\textls[20]{À borboleta seguia-se o beija-flor iriado, ave feita de luz, que parava,
sugando o roxo manacá, com o corpo no ar e o fino bico enterrado no
cálice da flor.}

\textls[-10]{Mais adiante, transposto um regato, Otávio viu aninhada na folha
côncava de um tinhorão, uma família de insetos multicores, luminosos,
como um monte de pedrarias deitadas num porta-joias de esmeralda. E por
toda a parte verdura, verdura nas palmas grossas dos coqueiros e nas
rendas moles dos fetos; verdura nas folhas dos cipós intrincados, nas
copas das árvores, no musgo e no limo do chão e na água espelhenta e
parada das poças.}\looseness=-1

\textls[15]{Otávio lembrou-se de um seu amigo, estudante alemão, que tantas vezes
lhe fizera perguntas sobre as coisas do seu país, e lhe abrira a
curiosidade por tudo o que lhe tinha sido outrora quase indiferente, e o
ensinara assim a adorar a natureza. Fora ao bafejo do espírito daquele
inteligente rapaz que o seu se desenvolvera e aprendera a amar
devotamente a terra!}

\textls[-5]{Depois de ter andado muito, Otávio foi sair num terreno espaçoso, acima
de uma colina que se via aos fundos da casa. Dera uma grande volta sem
sentir cansaço; desceu até o vale para de novo subir outra ladeira que o
levasse à escada paterna. Embaixo deslizava o rio que movia o moinho, e
ao pé da ponte estava a casa de banho edificada de novo. Otávio
recordou-se de que havia ali por perto umas pedras, onde ele, em
pequeno, costumava sentar-se a pescar uns mesquinhos lambaris para o
cuscuz da ceia. Vinte passos adiante deparou com elas, meio
limosas, sobrepostas umas nas outras, sombreadas pelos braços de uma
figueira brava, num pequeno espaço muito menos pitoresco do que se lhe
pintava na imaginação. Decididamente as saudades mudam o aspecto às
coisas, pensou o moço, e voltou olhando maquinalmente para uma folha que
arrancara da figueira e que levava entre as mãos. Ao aproximar-se da
ponte ouviu ranger a porta do banheiro e, levantando os olhos, viu a
prima que saía com os cabelos ainda úmidos, desatados, caindo como um
manto pesado e negro até a orla do vestido branco. As águas e as árvores
rumorejavam, as aves cantavam com alegria, e naquele concerto de
harmonias, entre sombras, perfumes e luz, Otávio estremeceu de
surpresa, como o primeiro homem estremeceria no Paraíso, vendo
deslumbrado surgir diante de si, brilhante e bela, a primeira mulher.}\looseness=-1

--- Eva! --- disse ele quase com timidez.

\textls[-30]{A prima sorriu e estendeu-lhe a mão. Seguiam ambos, ao lado um
 do outro, em direção à casa, quando, de repente, Eva parou e disse:}\looseness=-1

\textls[15]{--- É verdade! Você ainda não viu a sua mamã!}\footnote{``Mamã''
é nome que os paulistas dão às pretas que os amamentaram.}
\textls[15]{Olhe, coitadinha, ela mora bem perto, vamos lá?}

--- Obrigado por me lembrar esse dever\ldots{}

Otávio corou por se ter esquecido da pobre mulher; depois,
desculpando-se:

--- Ela devia ter-me procurado ontem mesmo\ldots{}

--- Não lhe faltariam desejos; mas está paralítica.

--- Sim?!

\textls[15]{--- É verdade. Deu-lhe um ataque e desde então não se tornou a levantar.}\looseness=-1

--- Pobre mamã!

\textls[-10]{Continuaram em silêncio até perto de uma casa de
barro,}\looseness=-1\footnote{Taipa.} \textls[-10]{sem janelas, onde a ama, sentada no
chão ao pé da porta, escolhia arroz numa peneira, cantando, com um fio
trêmulo e finíssimo de voz. A doença envelhecera-a muito; estava
mirrada, com a carapinha branca e as faces engelhadas que nem erva
seca. Assim mesmo trabalhava e cantava, ocupando sempre as mãos
emagrecidas e a vozinha débil como a de uma criança. Não que lhe dessem
tarefa; pedia-a ela às outras companheiras para aliviá-las do trabalho. \label{Ref6}}\looseness=-1

--- Mamã --- disse Otávio, chegando-se para ela.

\textls[-30]{A velha estremeceu, fixou no moço os seus olhos fulgurantes, onde toda a
sua vida parecia concentrada, a peneira caiu-lhe das mãos e ela agitou
os braços, sufocada, a chamá-lo para bem perto de si.}\looseness=-1

Otávio aproximou-se, ela abraçou-lhe os joelhos, chorando.

\textls[-10]{Era daquela velhinha paralítica das pernas, imprestável, atirada como um
caco velho para um canto imundo, que saía a maior manifestação de júbilo
pelo seu regresso. Todos os outros o haviam recebido com um sorriso
apenas; ela acolhia-o com lágrimas!}\looseness=-1

\textls[-20]{Eva deixou-os a conversar livremente, e foi sozinha para casa. Otávio
sentou-se na soleira perto da paralítica, que lhe beijava as mãos com
respeito, que o fitava com ternura, num êxtase, num embevecimento! A ama
falava-lhe em coisas da infância, perguntava se não lhe mandavam nas
cartas as saudades que ela pedia e queixava-se de não lhe darem as dele!
Depois elogiou muito Eva, concluindo:}\looseness=-1

--- Ah! Como eu gostava que mecê si casasse cum ela!

\textls[-30]{Quando Otávio voltou para casa, encontrou o pai impaciente pela sua
demora. Devia-se almoçar às nove e eram quase onze horas. Além disso o
comendador tinha prometido na véspera, ao Antunes, ir com o filho no dia
seguinte jantar à cidade, em casa da viúva Teixeira.}\looseness=-1

Mesmo à mesa o informou de que a viúva era irmã do Antunes, possuidora
de uma fortuna invejável e mãe de uma menina que era mesmo um encanto. E
afirmou:

--- É a mulher mais bonita de todo o município!

--- Eu não acho! --- contestou Noêmia.

--- Noêmia é mais bonita --- disse com ironia Nicota.

--- Eu não; mas sempre tenho mais graça; Sinhá é uma pamonha\ldots{} no
colégio nunca sabia as lições!

\textls[15]{O comendador declarou que a Sinhá tinha muito bonitos modos, e era muito
séria e muito concentrada.}

\textls[15]{Findo o almoço, demoraram-se com os preparativos. Otávio ia enfadado.
Não tivera ainda tempo de repousar em família, nem de ver à vontade
todos os recantos de Santa Genoveva, de que tanto se lembrava na Europa.}

\textls[10]{Quando deixaram a fazenda, era uma hora: o sol estava quente, havia pó,
e Otávio sentiu sono. Entretanto, o pai ia a traçar-lhe a biografia da
família Antunes.}

Era boa gente.

--- Mas então por que foi ele feitor?

--- Ora! Isso aconteceu; são épocas; naquele tempo não tinha
recursos\ldots{} hoje está bem.

--- Alguma herança?

\textls[-10]{--- Sim, um filho deixou-lhe uma fazenda, muito maltratada, mas que ele
pôs logo a direito e o caso é que tem enriquecido! Depois, também a irmã
o ajudou; ele é que trata dos negócios dela.}\looseness=-1

--- Ah\ldots{} --- E Otávio bocejou. --- Era viúvo o
Antunes?

--- Não.

--- Então o filho?

--- Era natural.

\textls[-13]{--- Ah! --- Novo bocejo, até que, já sem forças, Otávio se deixou
adormecer, recostando-se no fundo do trole coberto. Acordou à entrada da
cidade, abalado pelas pedras das calçadas, endireitou-se, sacudiu o pó,
e às três horas pararam à porta da viúva Teixeira.}\looseness=-1

\textls[-15]{Entraram, e, recebidos pelo Antunes, foram para uma sala da frente, à
direita, onde já estavam vários homens a tomar cerveja. Ali ficaram,
estupidamente, até que os foram chamar para o jantar. Foram então ter
com as senhoras, reunidas perto da mesa. Entre várias amigas, a irmã do
Antunes parecia radiante, rindo a uns segredos das companheiras,
malgrado a seriedade do seu luto de viúva.}\looseness=-1

\textls[15]{O Antunes apresentou Otávio aos convidados e família. A Sinhá mal
respondeu aos cumprimentos, conservando-se numa atitude impassível. Era
realmente bonita, alta, bem-feita, com os olhos escuros, grandes e
ramalhudos; clara, corada, correta de feições, mas de uma imobilidade
tal de fisionomia, que Otávio desviou o olhar como se tivesse fixado uma
morta. Obsequiaram-no; a dona da casa olhava para ele com ares
maternais; o Antunes falou-lhe da riqueza da irmã e do dote da sobrinha;
um anjo, a Sinhá!}\looseness=-1

\textls[-20]{Ele recebia e ouvia tudo sem ligar sentido a algum plano oculto.}\looseness=-1

\textls[10]{Voltaram tarde para Santa Genoveva; quando chegaram era meia-noite.
Todos dormiam em casa, exceto a mãe, que os esperava, a embalar-se na
rede.}

%\end{linenumbers}

\section{V}

%\begin{linenumbers}

\textls[10]{Passaram-se muitos dias na maior serenidade. Otávio passeava, punha em
ordem os seus livros, interessava-se pela lavoura, influindo o pai para
algumas modificações; planejava já montar uma fábrica nuns terrenos que
lhe pertenciam e calculava estabelecer a sua vida agradavelmente ao lado
da família. Quis tirar a planta da fazenda e nessa tarefa ocupava-se
durante algumas horas; à tarde acompanhava as irmãs, a mestra e a prima
numa volta pelo jardim ou pelo tanque, tendo então ensejo de apreciar a
educação de Eva e o espírito da preceptora.}

\textls[20]{O comendador ligou-se depressa ao rancho, e não permitiu ao filho outra
conversa que não fosse a sua; havia alguma coisa de constrangimento no
seu modo, que foi contagioso para todos os outros. Poucos dias depois,
desmanchou-se o grupo, dando cada qual um pretexto para não ir.}

\textls[20]{Eva resolvera estudar piano a essa hora; a mestra lia; Noêmia
confessava francamente que a aborreciam aqueles passeios, a ouvir falar
em política ou negócios; e Nicota punha-se a fazer o interminável
crochê, sentada ao pé da janela.}

\textls[10]{Corriam assim as coisas, quando uma manhã o fazendeiro, de volta de uma
viagem à cidade, foi bater à porta do filho, que se entretinha a riscar
e aperfeiçoar a planta de Santa Genoveva, ouvindo com prazer a prima
cantar na sala, a distância.}

O comendador vinha satisfeito e explicou-se em poucas palavras. Estivera
com o Antunes. O seu amigo dissera-lhe que a sobrinha estava apaixonada
pelo Otávio, e que toda a família se rejubilava com isso. Entre os
muitos pretendentes fora ele o escolhido, e nisso mesmo dava ela prova
de sensatez e bom gosto:

\textls[15]{--- A moça é bonita, é boa e é rica; a melhor fortuna do município! ---
concluía radiante o Medeiros.}

\textls[-10]{Otávio ouviu atônito aquela declaração imprevista. Depois riu-se.
Tinha graça, o Antunes! Com que então a sobrinha desprezava os
pretendentes e escolhia-o a ele, que não se importava com ela!}\looseness=-1

--- Eles são doidos, meu pai? --- perguntou por fim, zombeteiramente.

\textls[15]{Medeiros, muito sério, respondeu que, ao contrário disso, não havia
ninguém de mais juízo.}

--- Não parece\ldots{} Enfim! Qual foi a sua resposta?

O comendador confessou que não esperava uma recusa do filho, e que
alimentava a esperança de o ver mudar de tenção.

\textls[15]{Otávio afirmou que não se casaria nunca com a Sinhá: achava-a
monótona, sem espírito, mesmo nada atraente; além disso, não compreendia
o amor da rapariga; ela vira-o uma única vez e nem lhe falara sequer!}\looseness=-1

--- Porque tem bom senso; não é leviana como\ldots{} como a
maior parte.

\textls[-10]{--- Então, por ser sensata, revela uma paixão assim de repente, e
escolhe para marido um homem completamente desconhecido?!}\looseness=-1

O fazendeiro explicou:

--- Otávio não era tal um desconhecido. Há muito que a família Antunes
acariciava a ideia de o ver ligado a si. O compadre conheceu-o em
pequeno e sabia quais os seus bons instintos; além disso, voltava da
Europa, formado, tendo feito uma boa carreira e usando um dos mais belos
nomes da província. As cartas de Otávio eram lidas pelo Antunes, que o
elogiava muito a toda a gente. Ora, aí estava porque já o consideravam
noivo da formosa e requestada Sinhá, mesmo antes da sua chegada ao
Brasil.

\textls[-10]{Depois de um grande discurso, em que exalçava as boas qualidades da
sobrinha do amigo, Medeiros afirmou ao filho que não daria uma resposta
decisiva, esperando ainda vê-lo arrependido solicitar a mão daquela por
quem se empenhava tão afincadamente.}\looseness=-1

\textls[15]{Otávio empregou em vão todos os meios os mais persuasivos para o
desviar de tão arraigado desejo. Impossível! Aquilo era uma ideia fixa,
um plano amadurecido durante anos e que se encarava brutalmente no
cérebro daquele homem tenaz. Não havia força nem perícia que o
deslocasse do seu espírito.}

\textls[15]{O fazendeiro saiu do quarto do filho num grande acesso de mau humor,
desabafando a sua cólera sob o mais insignificante pretexto: bateu num
cão, expulsou do corredor os molequinhos, atirou com força as portas da
sala onde se recolheu, e que estrondaram como impelidas pela ventania.}

\textls[20]{Otávio pensou dois minutos na singularidade do caso, acendeu e fumou um
cigarro, abstratamente, e continuou depois muito sereno a planta
interrompida.}

\textls[-15]{Nessa tarde o comendador Medeiros não lhe deu a honra da sua companhia;
montou a cavalo e seguiu estrada fora, sem dizer para onde ia. Otávio
não atribuía o mal-estar do pai à conversa que tivera pouco antes com
ele, julgando-o preocupado por qualquer interesse da lavoura; por isso,
muito satisfeito, convidou as irmãs, a mestra e a prima para um passeio
no campo. Acederam todas, menos Nicota, que se esforçava por acabar
antes da noite a colcha de crochê.}\looseness=-1

\textls[5]{Seguiram os quatro por um dos raros passeios daquela grande fazenda,
onde todo o espaço parecia pouco para pastagens e cafezais. Lembraram-se
do bosque; Otávio indagou se as senhoras não teriam dificuldade em
atravessar o mato, quase cerrado em alguns pontos. Riram-se; medo de
quê? Das cobras?}

Pois sim!

A estrangeira soltava exclamações guturais, corria a grandes passadas
atrás das borboletas, mostrando os pés chatos e os tornozelos finos. Eva
escolhia e colecionava plantas; Noêmia fazia bulha, gozava com a alegria
dos outros, saltava para os lugares mais difíceis, e enfeitava-se de
flores. Otávio desembaraçava o caminho dos cipós e galhos descaídos e
oferecia-lhes a mão para transporem com mais facilidade os obstáculos
que encontravam.

E assim, alegremente deram volta ao bosque, até saírem na esplanada da
colina e descerem ao vale, onde corria docemente o estreito rio; aí,
como ainda fosse cedo, lembraram-se de ir estar um bocado
debaixo da figueira brava, no mesmo sítio onde, em pequeno, Otávio ia
pescar os lambaris para o cuscuz da ceia.

\textls[15]{Sentaram-se, uns na grama, outros nas raízes da árvore; Noêmia
mergulhava as mãos na água e sacudia-as depois sobre a relva, gostando
de ver as gotinhas que aí ficavam luzindo aos últimos raios do sol; não
prestava atenção a mais nada, deliciava-se naquela brincadeira infantil.}

\textls[-20]{Otávio falava da Europa, descrevia paisagens, costumes, tipos, instigado
pelas palavras amáveis da lisonjeada Grüber, e pela curiosidade de Eva.
A prima, muito despretensiosa com a sua toalete clara e
simples, o olhar aveludado fixo nele, nunca lhe parecera tão bonita, tão
atraente; quis ouvi-la e perguntou-lhe se não tinha vontade de ver as
mesmas coisas, de apreciar de perto as belezas europeias.}\looseness=-1

\textls[10]{Que sim, respondeu-lhe ela; embora o seu desejo fosse viver no lugar em
que nascera e que os pais a ensinaram a amar. Minha mãe dizia sempre,
concluía Eva, que eu devia considerar como irmãs até as árvores e as
flores que me rodeavam, dispensando-lhes todos os meus carinhos; por
isso habituei-me a ter-lhes uma amizade de família, ficando com muitas
saudades quando me apartava delas, mesmo que fosse por alguns dias!}

\textls[-30]{Mais adiante, levada a conversa para o culto das flores, Eva contou:}\looseness=-1

--- Uma vez, era eu ainda pequena, ia arrancar desastradamente umas
maravilhas amarelas, quando ouvi um ``ai!'' muito triste; afastei
depressa as mãos, e minha mãe, que me observava de perto, disse: ``Vês?
Foi um gemido da terra, da terra em que nasceste, minha filha, e de onde
querias arrancar uma das tuas irmãs!''. Desde então tive uma verdadeira
idolatria pelas plantas, e chegava às vezes a beijar as flores! Não
soube nunca de onde havia partido aquele gemido, chegado tão a propósito
a meus ouvidos\ldots{} é provável que fosse um suspiro de minha mãe.

\textls[15]{Mme.\,Grüber não aprovava nada que tivesse, mesmo de leve, uns traços
românticos. Meneou a cabeça, sorrindo.}

\textls[20]{Otávio, reparando nisso, afirmou que essas imagens poéticas imprimem,
na sua candidez, muito mais interesse e crença nas crianças do que as
práticas positivas com que muita gente pensa prepará-las melhor.}

\textls[20]{--- Tirar ilusões aos únicos entes capazes de as terem em toda a
plenitude, que barbaridade, meu Deus! Fazer amar a Natureza como uma
mãe, beijar as flores como irmãs, que belo exemplo para a maldade do
homem!}

\textls[-5]{A mestra replicou-lhe, expondo os métodos modernos de educação. Otávio
escutava-a, pensando em fazer Eva repetir vários episódios da sua
meninice, morto por indagar de toda a sua vida passada, aguilhoado
pelo desejo de saber qual a sua impressão ao entrar sozinha e órfã em
casa do tio, e como se pudera afastar do lar paterno, pelo qual mostrava
tão entranhado amor!}\looseness=-1

\textls[15]{Abstinha-se, à espera de uma ocasião em que, sem ser indiscreto, pudesse
saber tudo.}

\textls[-10]{A noite ia caindo. No fundo esmaecido do céu, destacavam-se as árvores
da colina fronteira, em formas rendilhadas. Uma coruja passou num voo
baixo e fofo, indo pousar no telhado de sapé de uma senzala, sobre a
qual abriu as suas grandes asas algodoadas. Ficava bem ali, naquele
teto triste, onde as avezinhas alegres não poderiam cantar sem
remorsos, porque o canto da ave noturna não é, como o delas, um hino de
liberdade, mas um grito de condenação que vai casar-se à voz dos
escravos, nas suas imprecações.}\looseness=-1

\textls[10]{Houve uns momentos de silêncio. Noêmia foi a primeira a cortá-lo,
lembrando a partida: levantaram-se e seguiram caminho de casa.
Atravessaram a ponte, tomaram o carreiro estreito do laranjal, subindo
em zigue-zagues.}

\textls[15]{Mme.\,Grüber respirava com força, deliciando-se com o aroma das plantas.
Noêmia cantava; Eva ia silenciosa, parecia imersa nas saudades
suscitadas pela lembrança da mãe. Otávio seguia atrás dela, a vê-la
sempre, sentindo um inefável e indefinível prazer em pisar sobre os seus
passos, sem refletir verdadeiramente em nada. Ao passarem pela casa da
paralítica, que talvez já dormisse lá dentro o sono leve dos velhos e
dos enfermos, lembrou-se da frase da sua mamã, referindo-se a
Eva:``Como era bom que mecê si casasse cum ela!''}

\textls[-20]{Essa frase soara-lhe ao ouvido como uma música e surgia de vez
em quando no seu espírito, entre as reflexões de todo o gênero; a
propósito, ou fora de propósito, ela vibrava sempre na sua imaginação.}\looseness=-1

Acontecera-lhe muitas vezes não poder dormir, por ter um trecho qualquer
de opereta a repetir-se contra sua vontade na memória; já lhe sucedera o
mesmo com um verso, a bordo, o que o deixara impaciente; com uma frase,
porém, era a primeira vez, e parecia-lhe estranho que essa não o
enfadasse; ao contrário --- sugeria-lhe uma multidão de ideias risonhas
e esperançosas.

\textls[15]{Quando entraram em casa já o comendador Medeiros tinha voltado do
passeio. Mantinha o seu mau modo, um pensamento negro anuviava-lhe o
rosto. Manifestara contrariedade ao saber que o filho acompanhava as
senhoras, e que se demorava tanto. Ralhou com a mulher, que se submetia
paciente à descarga da sua cólera, proibiu que Noêmia tornasse a sair em
companhia da prima, e, quando viu chegar o rancho à sala, voltou-lhe as
costas sem corresponder às ``boas noites'' e foi-se para o quarto, de
onde não tornou a sair naquela noite.}

\textls[-15]{O serão correu insípido, a ceia foi servida uma hora antes da do
costume. Falava-se baixo, como se houvesse em casa um doente.}\looseness=-1

%\end{linenumbers}

\section{VI}

%\begin{linenumbers}

\textls[30]{A borrasca continuou tenebrosa durante todo o dia seguinte, em
que Medeiros não apareceu à família, manifestado o desejo de
que o não fossem ver. ``Aquilo é enxaqueca'', dizia a mulher,
placidamente, ``logo está bom''.}

Descansados com esta ponderação, os filhos não tinham cuidados;
limitaram-se a andar em bicos de pés e a não abrir o piano.

Depois do jantar, mme.\,Grüber e Eva saíram para a
horta, Noêmia e Nicota ficaram na sala, a primeira contrariada, chorosa,
a segunda tranquila. A mãe transmitira à filha o recado do pai
e a sua proibição, sem comentários, sucintamente.
Otávio girou ao acaso e foi ter depois com a
estrangeira e a prima. Quando se recolheram, já as estrelas
tremulavam no céu ainda pálido.

\textls[-15]{As horas do serão decorreram na mesma monotonia da espera. Eva
retirara-se para o quarto da mestra, e na grande varanda, sem conforto,
mal alumiada e fria, rangiam as argolas das redes e perpassavam de vez
em quando, ao fundo, descalços e sem ruído, os escravos como sombras
misteriosas. O enfado trazia depressa o sono.}\looseness=-1

\textls[-15]{Depois de uma noite sossegada, Otávio levantou-se às seis horas,
fez o costumado passeio, demorando-se lá embaixo nos seus terrenos a
calcular o melhor local para a projetada fábrica. Voltando, encontrou o
pai de pé na soleira da porta, com o chapéu enterrado até os olhos e o
chicote seguro na mão direita. A uma exclamação jubilosa do filho,
correspondeu preocupado o fazendeiro, dizendo precisar falar-lhe a sós;
e seguiu para a sua sala, fazendo-lhe um gesto para que o acompanhasse.
Ali chegados, Medeiros estendeu a Otávio uma carta já amarrotada,
dizendo-lhe secamente: ``lede''!}\looseness=-1

Entre muitos termos sem sentido, Otávio leu uma longa denúncia
que do dr.\,Azevedo e de Eva fazia um anônimo. No meio daqueles
caracteres sujos, cuspidos por uma pena grosseira, ressaltava o ódio e a
repugnante preocupação de armar uma intriga.

\textls[15]{Depois de referir-se aos amores da órfã e do magistrado,
afirmava saber que os dois preparavam, coadjuvados pela estrangeira, uma
revolução dos escravos de Santa Genoveva. Dar-se-ia essa revolução na
noite do casamento de Nicota. ``A sua casa é um ninho de abolicionistas.
Previna-se.''}

\textls[20]{--- Então? --- perguntou o fazendeiro, cruzando os braços. --- Que dizeis!}

--- Digo que isto é uma infâmia --- respondeu simplesmente Otávio,
com os olhos ainda fitos na carta.

\textls[15]{--- Infâmia, sim! --- gritou o pai, manchado de cólera. --- Ela há de me pagar, aquela\ldots{}}

\textls[10]{--- Meu pai --- interrompeu Otávio vivamente ---, a infâmia, a
inqualificável maldade é de quem escreveu esta carta. Eva é inocente e,
em vez de acusá-la, devemos defendê-la!}

\textls[20]{O comendador vociferava, batendo raivosamente com o chicote nos
trastes.}

Otávio continuou:

--- Se ela realmente ama o Azevedo, por que não há de casar com
ele? Procure meu pai, que é hoje o seu, auxiliá-a nisso, se é por acaso
o Azevedo, como eu julgo, digno dessa felicidade.

O fazendeiro abriu a boca, atônito; o filho prosseguiu:

\textls[15]{--- Quanto a serem ambos abolicionistas, não vejo motivo para
que os censurem; ao contrário, vejo para que sejam louvados. Não creio
que Eva trabalhe contra um irmão de seu pai, em cuja casa está. Que
poderia ela, pobre moça, fazer para isso? Reflita um momento na
desigualdade que há entre a sua força e a dela, e há de acalmar-se, meu
pai.}

\textls[20]{--- Que mal me pode fazer? Todo! Pode perder a minha família\ldots{}
Tem documentos perigosos, e neste tempo, o que não conseguirá contra
mim? Decididamente é preciso que saia, hei de mandá-la sair.}

Um grande desgosto pintava-se na fisionomia do comendador que,
deixando-se cair no sofá, escondeu o rosto entre as mãos.

\textls[-15]{--- Eu nada sei relativamente a Eva --- tornou Otávio. --- Conheço-a
há poucos dias e do seu passado nunca me falaram. Que deu origem à
desunião de vossemecê e do tio Gabriel? Ignoro-o! Quais os documentos
perigosos possuídos por Eva? Não sei quais sejam! De tudo isto só julgo
compreender uma coisa: é que jogam com a sua reputação, que a caluniam e
que devo defendê-la. Lembre-se, meu pai, de que Eva nos tem a nós
unicamente por conselheiros e guias, pois que ficou a nosso cargo.
Vossemecê não pode nem deve expulsá-la; Eva é-lhe mais sagrada que uma
filha!}\looseness=-1

--- Irra! --- exclamou o fazendeiro, batendo com o pé no chão. --- Eva é
um demônio, e há de ir embora hoje mesmo!

\textls[15]{--- Mas se a teme não a expulse, que isso seria precipitar os
acontecimentos de uma maneira brutal. Conte-me com franqueza em que o
pode prejudicar a pobre moça, o que há de razão em todo este jogo de
crianças!}

\textls[20]{--- Faltais-me ao respeito? Não sois preciso: eu sei o que devo
fazer\ldots{} Saí!}

\textls[-25]{--- Não saio sem lhe repetir que me declaro protetor de Eva e
que espero não delibere repentinamente numa situação que nos pode
comprometer a ambos. Deixe passar o casamento de Nicota e, à vista dos
fatos, vossemecê decidirá fazer o que lhe ditar a consciência.}\looseness=-1

\textls[10]{--- E a atrevida da alemoa, ou sueca, ou lá o que é, que
come o meu dinheiro e trabalha para me arruinar? Essa é que não me
escapa, e há de pôr-se na rua, já, já!}

\textls[-20]{Otávio tentou encaminhar o pai para a prudência, mas o
comendador abriu estrepitosamente a porta e saiu sem lhe dizer mais
nada.}\looseness=-1

\textls[20]{Mme.\,Grüber lia um jornal perto da janela da varanda, onde a
dona da casa, cruzada dentro da rede, se embalava e cosia ao mesmo
tempo.}\looseness=-1

\textls[15]{As filhas e a sobrinha não tinham aparecido, e a seus pés meia
dúzia de negrinhas com timão de baeta, sentadas no chão, uniam a ponto
de serro os panos de um lençol.}

O fazendeiro deu várias voltas, agitadamente, pelo meio da sala;
Grüber, com o seu faro de mulher inteligente, adivinhou estar
próxima uma tempestade e levantou-se respeitosamente, quando o viu ir
direito a ela.

--- Dona --- disse ele ---, preciso falar com a senhora.

\textls[15]{A mestra acompanhou-o até à sala, ao fim do corredor, onde,
ainda de pé, o comendador lhe disse:}

\textls[15]{--- Vamos fazer contas, já não careço dos seus serviços. Nicota
vai-se casar e Noêmia sabe demais!}

--- Mas\ldots{} o contrato\ldots{}

--- O contrato quebra-se.

--- É que\ldots{}

\textls[15]{--- Descanse, que há de receber o seu dinheiro como se ficasse
até o fim do ano.}

Mme.\,Grüber corou, e depois de uma pequena pausa:

--- Incorri em alguma falta? Está por acaso o sr.\,comendador
descontente com os meus serviços?

--- Eu não entendo de livros, não sei se a madama ensinou mal ou
bem.

Esforçando-se por não responder, a professora engoliu em seco e
depois de uma pausa:

--- Tenho a infelicidade de ser antipática a sua família?

\textls[10]{--- Nada disso! Nem eu quero que a minha família saiba que eu a
despeço. Eva, sobretudo, entendeu?! Foi, principalmente, por causa dela ---
disse acentuadamente, sublinhadamente --- que resolvi dar este passo.}

--- Por causa dela!

--- Sim! Sei perfeitamente que a senhora lhe dá conselhos
prejudiciais para mim\ldots{}

--- Eu?!

\textls[10]{--- Portanto, repito, não lhe diga nada, porque se ela me vem
tomar satisfações é que\ldots{} é que eu a mando ao diabo!}

\textls[-30]{--- Descanse --- disse muito pálida e com altivez a preceptora ---,  eu
nada direi, porque não quero desgostá-la; unicamente por isso. Saio com
a consciência tranquila, e certa de que o tempo me fará justiça.}\looseness=-1

\textls[-15]{E, sem procurar explicações que a justificassem, combinou fazer
naquele dia as malas e partir para Santos na madrugada seguinte: ali
receberia do correspondente de Medeiros o ordenado dos meses acumulados
e embarcaria para a Europa, no primeiro paquete.}\looseness=-1

\textls[-10]{Durante o tempo da entrevista, Otávio, preocupado,
passeava no corredor, olhando de vez em quando para a porta da sala. De
que palavras se serviria o pai? Com que pretexto despediria uma senhora
respeitável, de quem o Morton dissera: ``é uma mulher
instruída e severa; foi-lhes recomendada por mim''? Por que
haviam de enredar na intriga o nome da professora? Que interesse teriam
em perdê-la? Qual seria o terrível documento possuído
por Eva?! Decididamente, a doente imaginação do pai criara
visões de uma absurda crueldade! Havia ainda restos de febre naquele
corpo e com certeza era tudo delírio consequente dela. Mas a
carta? Quem a escrevera? Um anônimo, um covarde, um desgraçado,
em quem se não devia pensar. Contudo, sempre era certo que Eva amava o
Azevedo, aquele boneco de retórica balofa!}\looseness=-1

E tornava a pensar na dolorosa expressão do pai: ``Eva possui
documentos perigosos e tudo conseguirá contra mim!''.

\textls[-20]{Um raio de luz feriu a ideia de Otávio; talvez que a prima
tivesse assistido à agonia de algum escravo açoutado\ldots{} talvez que
indignada tivesse proferido uma frase de rancor ou um protesto de
ódio\ldots{}}\looseness=-1

\textls[-20]{O mais leal era ir ter com ela, falar-lhe sem rodeios, pedir-lhe
explicações, instar para que restituísse ao tio a tranquilidade
precisa.}\looseness=-1

\textls[12]{Resolvido a isso, Otávio desceu ao jardim, onde a essa hora Eva
costumava estar cultivando as flores. Não se enganara; a prima
entretinha-se a entrelaçar num gradil as hastes de um jasmineiro. Ao
vê-la tão cuidadosa e serena, Otávio teve remorsos de a ir perturbar.
Depois, que direito tinha para lhe exigir uma confissão sincera? Como a
deveria interrogar? Não se ofenderia ela por a julgarem capaz de uma
denúncia contra o homem que lhe dava abrigo, contra o irmão de seu pai,
cuja memória respeitava tanto? Pô-la ao fato da carta recebida? Uma
indignidade! Preveni-la do desgosto que a ameaçava? Seria antecipá-lo.
Enfim, confessar-lhe tudo seria o mesmo que dizer: duvido de si, Eva, e
por isso, em vez de procurar desarmar o seu inimigo, sem nem por sombras
agitar-lhe o sono, venho miseravelmente dizer-lhe o que se passa,
pondo-a na contingência de, ou se ir embora altivamente, ou ficar
humilhada!}

É necessário também que eu a proteja sem que ela o suspeite ---
refletia Otávio; que eu tenha por ela a solicitude do amigo que lhe
falta e que no-la confiou. Provocarei uma confidência do Azevedo e
concorrerei para que o casamento se faça em breve\ldots{}

\textls[25]{Era com amargura, mas com firmeza, que o moço Medeiros fazia a si
próprio a última promessa. Vim muito tarde, talvez que se amem há muito
tempo\ldots{}}

Eva continuava entre as estrelinhas brancas dos jasmins a
entretecer-lhes as hastes, e o primo olhava-a de longe, lembrando-se do
que lhe dissera um dia o seu querido Adolpho Meyer:

\textls[25]{``Quando vejo uma mulher tratar das flores com amor, tenho
vontade de beijar-lhe as mãos!\ldots{}''}

%\end{linenumbers}

\section{VII}

%\begin{linenumbers}

Otávio tinha trocado algumas palavras com a prima, quando
Noêmia atravessou o jardim, correndo em direção a ambos.

--- Uma carta para Eva, do Mangueiral --- disse.

\textls[-20]{Eva avançou e, desdobrando sem precipitação o papel, leu alto:}\looseness=-1

\begin{quote}
\textls[15]{Peço-lhe que venha imediatamente ao Mangueiral; a família do
Raimundo quer-se ir embora por ter tido uma rixa com a do Salomão. É a
melhor gente da colônia e faz-nos grande falta agora.}

\begin{flushright}
Seu dedicado,\\
\textit{Paulo}
\end{flushright}

\noindent{}\textsc{p.\,s.}\quad \textls[15]{O dr.\,Azevedo apareceu hoje cá e segue à tarde para a
fazenda do Leocádio. Venha sem falta, sim?}
\end{quote}

\textls[10]{A leitura fora feita sem o menor constrangimento. Otávio
afastou-se calado, voltando-se ao ouvir Eva dizer:}

--- Sua mestra estará disposta a acompanhar-me?

--- Ainda não dei lição\ldots{} mas se eu pudesse ir
também!

\textls[20]{Querendo desviar a ideia de convidarem a Grüber, Otávio
ofereceu com a da irmã a sua companhia, olhando fixamente para Eva, que
esperava ficasse contrariada. Mas não! Ela sorriu, aceitando alegremente
a proposta.}

\textls[-25]{--- Vamos a cavalo? Eu cá prefiro ir a cavalo ---
confessou Noêmia.}\looseness=-1

--- Mas o tio Medeiros consentirá? --- perguntou Eva.

--- Que nós vamos a cavalo?

\textls[-20]{--- Não! Que vão comigo ao Mangueiral\ldots{} Sempre fui só
com\ldots{}}\looseness=-1

--- Ora --- interrompeu Noêmia ---, Otávio já disse que eu ia também,
e pai, desde que vá um homem, não põe dúvida!

\textls[-10]{Quando subiram, já Otávio tinha contado ao pai o ocorrido.
Contra a sua expectativa, o fazendeiro aprovou a ida dos filhos ao
Mangueiral, o que era a seu ver uma espionagem. Unicamente lembrou que
seria melhor ir Nicota em vez de Noêmia; a filha mais velha era mais
ajuizada e contava as coisas com mais propósito.}\looseness=-1

--- Como queira --- respondeu-lhe Otávio.

\textls[10]{O fazendeiro fez saber a sua deliberação. Noêmia, muito
influída, chorou quando lhe disseram que, em vez de ir pela
primeira vez fazer um passeio grande a cavalo, tinha de ficar presa
naquele triste casarão. Valeu-lhe pouco depois saber que Nicota passara
a noite em claro, com dor de ouvidos, e que não sairia da cama
durante o dia. Irrefletidamente bateu as palmas, deu uns
saltinhos que provocaram cabriolas do Júpiter e do Vinagre, e foi de
olhos mal enxutos, mas radiante, preparar-se ao seu quarto. Eram nove
horas quando saíram. Ficava na mesa o almoço: o
viradinho}\footnote{Feijão ou ervas, com farinha de milho e torresmos.}
\textls[10]{de feijão, o lombo de porco, as ervas, a
paçoca}\footnote{Carne moída, reduzida a farinha.} \textls[10]{e os
ovos. Iriam almoçar ao Mangueiral, duas horas e meia mais tarde.
Sentaram-se à mesa unicamente o comendador e a esposa. A mestra
sentia dores de cabeça e, ao despedir-se de Eva no ``até
à noite'', abraçou-a com ternura.}

Belo caminho, o de Santa Genoveva ao Mangueiral, aberto de pouco
em matas virgens, cheio de frescura e de sombra.

\textls[-25]{De vez em quando um campo a atravessar, banhado de luz e de
calor; depois, de novo a estrada, protegida pelas árvores das margens.}\looseness=-1

\textls[10]{Noêmia ia na frente, espantando com gritos os pássaros,
acelerando e diminuindo a marcha do animal que montava com imprudente
afoiteza. Advertiram-na em vão do perigo. Otávio, ao lado da prima,
ouvia-a falar com amizade da pessoa que na carta lhe dera o nome de
irmã.}

\textls[-10]{--- É um filho adotivo de meus pais --- dizia ---, órfão desde
pequenino. Criamo-nos juntos, tivemos os mesmos mestres e fomos sempre
muito amigos\ldots{} Paulo é inteligente e modesto. Meu pai, como você
naturalmente já sabe, deixou-o encarregado de olhar pelo Mangueiral. Ao
tio Medeiros é que ficou um legado impertinente, tendo de dar-me os seus
cuidados e a sua proteção\ldots{}}\looseness=-1

Otávio estremeceu ouvindo as últimas palavras de Eva.

\textls[25]{A proteção e o cuidado de Medeiros para com a pupila eram
desgraçadamente irrisórios.}

\textls[-25]{Que terrível legado, tão cheio de responsabilidades! E que bela
reconciliação a proposta pelo moribundo! Como devia ter sido bom e
crente aquele homem, para, ao fechar para sempre os olhos, apontar com
mão trêmula, à única filha, a casa de um inimigo como o único asilo
para a sua orfandade e o consolo único para as suas lágrimas!}\looseness=-1

\textls[20]{Que alma generosa fora a desse trabalhador que, tendo lutado,
sofrido rivalidades, invejas e injustiças, talvez, ainda tinha, já velho
e ao avizinhar-se o momento de vestir a mortalha, a ingênua boa-fé de
uma criança!}

\textls[20]{Imaginara a sua órfã rodeada de afetos, entre umas criaturas
da sua idade que a afagassem e que lhe dessem o doce nome de irmã\ldots{}
Tivera uma visão, um último sonho de ouro, bonito, realizável, mas
irrealizado, desgraçadamente!}

\textls[25]{Otávio trocou mais algumas palavras com a prima, até chegarem
ao Mangueiral.}

Eva, indicando o caminho, pôs-se à frente.

\textls[-10]{Seguiram por uma rua larga de bambus e foram sair num parque
cheio de sombra, encantador, de musgos, água e relva, sobre o qual as
mangueiras estendiam os braços em atitude de benção. Ao fundo, coberta
de silvina e outras plantas, a casa de moradia tinha um aspecto risonho,
bem diverso das habitações rurais da província. Não mostrava aos quatro
ventos, como as outras, a nudez crua das paredes brancas: aninhara-se
em flores, rodeara-se de arvoredos, fazendo mais restrito o seu
horizonte, mais \emph{íntima} a sua posição. Cá fora não brincavam
crianças sujas nem dormiam, cobertos de mosquitos, os cães ao sol;
ouvia-se o murmúrio da folhagem e o correr vagaroso da água numa bacia
de pedra. Pelas janelas, cerradas ao calor, evolava-se o som de um
violoncelo.}\looseness=-1

\textls[20]{Pareceu a Otávio ver diante de si uma \emph{cottage} inglesa,
elegante e discreta.}

\textls[-10]{--- Paulo está estudando; não me espera tão cedo! --- disse Eva.}

E foi bater com o chicotinho nos vidros da janela.

\textls[15]{Soltando uma exclamação de prazer, Paulo veio à porta,
acompanhado do Azevedo, muito risonho.}

\textls[15]{Feitas as apresentações e os primeiros cumprimentos,
dirigiram-se para a sala de música.}

Paulo fazia as honras da casa. Era um rapaz simpático, sem ser
bonito. Trigueiro, olhos pretos, alto, magro, boca rasgada, ensombrada
por um pequeno bigode preto, voz máscula, forte e ao mesmo tempo suave.

\textls[20]{O Azevedo aproximou-se de Noêmia e de Eva, fazendo frases
madrigalescas.}\looseness=-1

\textls[15]{--- Olha, Paulo ---  interrompeu risonha a dona da casa ---, nós ainda
não almoçamos!}

\textls[10]{--- Deveras?! --- E Paulo ia a erguer-se, quando Eva o reteve
com um aceno e levantando-se caminhou para o interior. Noêmia
acompanhou-a, acanhada de ficar ali sem ela.}

\textls[-10]{A conversa entre os rapazes começou animada. Otávio ouvia com
prazer as palavras fáceis e sensatas de Paulo, que expunha ideias
perfeitamente de acordo com as suas, revelando acerto, energia e pleno
conhecimento da lavoura, de que era amigo extremoso.}\looseness=-1

\textls[10]{O Azevedo contestava que fosse a lavoura coisa digna de
entusiasmo. --- Seria a mais rendosa --- dizia ele, mas por
isso mesmo não merecia a adoração de espíritos superiores\ldots{}}

\textls[-15]{Paulo afirmava o contrário: não havia profissão mais bela que a
do lavrador; era certo que mal compreendida até então; mas um dia viria
em que a rotina podre caísse e em que a transformação para o trabalho
livre a elevasse à altura onde era de justiça ser levada.}\looseness=-1

\textls[15]{Otávio apoiava-o, e a discussão seguia animadamente. Depois o
Azevedo principiou uma conversa, cujo assunto parecia interessar
vivamente a Paulo, e Otávio Medeiros, observando tudo o que o cercava,
estabelecia a comparação entre a fazenda de Eva e a do comendador.}

\textls[-10]{Ali tudo era diferente e tinha um cunho original e alegre. A
mobília, de palha com espaldares claros, disposta
artisticamente; o divã de linho de ramagens vivas, a um canto,
perto de uma jardineira de cortiça, tapada nos três
tabuleiros por graciosas hastes de avenca e fetos
rendilhados; o piano alto, coberto de casimira cinzenta com aplicações
de seda; o violoncelo ao lado do piano, junto de uma estante de mosaico
e de um jarrão cheio de rosas frescas; a mesa redonda, sólida, sobre a
qual se viam revistas musicais e retratos de maestros célebres; as
cortinas transparentes barradas a matiz; todos os objetos,
enfim, revelavam que a mão que os dispusera tinha tato
artístico, tão raro nas residências campestres.}\looseness=-1

Otávio admirava, ora a frescura de uma aquarela, onde, sobre o
verde tenro de um pasto novo, babava em fio uma vaca malhada; ora os
tapetes, bordados num desenho chinês e extravagante, esquecendo-se das
apreensões do pai e a refletir na felicidade de quem gozasse naquela
casa as doçuras da vida de família\ldots{}

Voltou depressa a prestar atenção à conversa.

Referindo-se aos escravocratas, exclamava o Azevedo:

\textls[15]{--- Que diabo! Eles põem-me às vezes a cabeça a juros; senão,
vejam: um dia destes apareceu-me lá em casa, cortado de bacalhau e com
ferros ao pescoço, um negro, ainda forte, do Antunes. Paulo, dá-se com
ele?}

--- Com o Antunes?

--- Sim.

--- Conheço-o apenas de vista; nunca lidei com ele.

--- Ora, ainda bem! Pois, como ia dizendo, apareceu-me o negro
queixando-se de maus tratos e expondo à minha compaixão o corpo
emagrecido e retalhado. Mandei tirar-lhe os ferros, curá-lo; dei-lhe
cama, jantar, e, como do legado do sr.\,Gabriel restassem ainda
setecentos mil réis, escrevi ao Antunes propondo por esse preço a
liberdade do escravo. Respondeu-me com uma tremenda descompostura,
exigindo-me a entrega do negro. Nem por um conto o vendo, dizia ele na
carta; eu cá o ensinarei!

--- Interroguei o preto: era africano; exultei de júbilo: a lei
favorecia-me. Sem mais barulho dei liberdade ao desgraçado, pondo-lhe na
mão a cópia da lei de 1831 que proibiu o tráfico dos africanos. Os
setecentos mil réis ficam à espera de qualquer outro escravo que tiver o
infortúnio de ter nascido no Brasil\ldots{}

\textls[15]{--- E o Antunes? --- perguntou Paulo com um modo distraído. ---
Mandou-lhe novas ameaças, já se vê\ldots{}}

\textls[15]{--- Ora! Com isso contava eu. O diabo do homem quis e chegou
mesmo a falar em processar-me! Trabalho inútil. O advogado aconselhou-o
a que metesse a viola no saco, ao que, segundo me consta, ele não se
resigna. Procura por todos os meios fazer-me mal, e, como tem amigos
influentes, talvez o consiga. Nunca fiz alarde de abolicionista; tenho
alforriado, é certo, meia dúzia de escravos, de acordo sempre com os
senhores e na qualidade de testamenteiro. Evito, tanto quanto posso,
manifestar as minhas ideias em semelhante assunto. É este o primeiro
desgosto que tenho tido aqui\ldots{}}

--- E Eva --- perguntou Paulo ---, sabe disso?

--- Certamente, indo lá jantar um dia a Santa Genoveva,
disse-lhe o que havia e, por ser filha do legatário, perguntei-lhe a sua
opinião. Aprovou a minha ideia, prometendo escrever-me no dia seguinte e
enviar uma certa quantia para a libertação de outro qualquer escravo que
reclamasse o meu auxílio. Felizmente não tornou a aparecer nenhum\ldots{} Sua
irmã --- continuou o Azevedo constrangidamente, dirigindo-se a Paulo ---
manifestou a intenção de ceder todos os anos uma quantia para o mesmo
fim\ldots{}

--- Já me falou nisso\ldots{}

--- E não quis despersuadi-la?

\textls[15]{--- Não. Está no direito de o fazer; cede o que lhe é supérfluo.
Eva é simples e gasta pouco consigo. Do que eu a despersuadi foi de o
incumbir dessa tarefa; ela não refletiu que no meio em que vivemos isso
lhe poderia ser prejudicial.}

\textls[20]{O Azevedo não pôde encobrir um movimento de alívio; Paulo
continuou:}

\textls[20]{--- Daqui em diante tomo à minha conta o desempenho
dessa missão.}

\textls[-13]{Contou depois a Otávio que o finado Gabriel Medeiros tinha sido
um homem de vistas largas e grande coração. Ele acreditava que a
redenção e a felicidade do Brasil estava na aplicação de métodos
modernos na sua lavoura; e a sua prosperidade dependente da ambição do
colono livre, bem escolhido e bem mantido. Não era um utopista; era um
patriota humanitário. Se tivesse seguido a rotina, concluía Paulo, teria
deixado maior fortuna; infelizmente os primeiros a revolucionar um
sistema inveterado há anos não tiram grande benefício do seu trabalho. É
isso que faz com que se diga que os mais avançados na teoria são os mais
atrasados na prática. Mas se ele não deixou grande fortuna, deixou um
belo exemplo moral e um testamento em que a fé no futuro da pátria
transparece de um modo eloquente e admirável. Era um espírito justo e
não compreendia que se pudesse enriquecer com alegria sem a colaboração
livre do trabalho com o capital. Entre outros legados interessantes,
deixou trinta contos para alforria de escravos perseguidos. Era a esse
legado que o dr.\,Azevedo aludia há pouco\ldots{}}

%\end{linenumbers}

\section{VIII}

%\begin{linenumbers}

Numa pequena sala quadrada, com portas de vidro para o jardim,
Noêmia e Eva esperavam Paulo e os hóspedes. Pela janela aberta,
emoldurada de mimosas rosinhas de toucar, entrava o aroma das flores e o
alegre chilrear dos pássaros de um viveiro que ficava perto. A mesa,
posta com elegância, despertava o apetite.

\textls[-15]{As garrafas de cristal, com vinho, o abacaxi cortado em espiral,
expondo a sua carne dourada e sumarenta; a fruteira de madeira escura,
das Caldas de Minas, com pêssegos e uvas aninhadas em musgo novo, ainda
cheiroso e úmido, o requeijão feito em casa, a farinheira de coco com
relevos trabalhados na Bahia, a manteiga fresca, fabricada igualmente no
Mangueiral, e os canjirões de barro cheios de leite natoso davam àquela
mesa de campo um aspecto risonho e convidativo, tanto mais que o relógio
marcava já meio-dia.}\looseness=-1

Sentados à mesa, a conversa tomou diversos rumos.

Acabado o almoço, Azevedo partiu, confessando-se com pena de
deixar aquele delicioso retiro e ter de ir à fazenda do Leocádio, a
interesse de umas órfãs menores, que não conhecia, mordido do calor dos
descampados, para se demorar numa casa de incrível aridez, em que o
dono, refratário à civilização, aparecia às visitas em mangas de
camisa, chinelos e chapéu enterrado até às orelhas.

Paulo e Otávio seguiram a pé, numa visita e revista à colônia;
Noêmia e Eva ficaram esperando o Raimundo. Ele não se fez esperar, veio
logo ao primeiro chamado. Era um homem alto, de barba ruiva, olhos azuis
e rosto queimado, carregando muito na sua pronúncia minhota, pospontada
de diminutivos.

\textls[10]{Eva evitava envolver-se na administração do Mangueiral,
acedendo, uma ou outra vez a isso, só a instâncias de Paulo. Nunca,
porém, fora chamada por tão fútil pretexto; manifestando a Paulo a sua
surpresa, ele respondeu-lhe com um sorriso:}

\textls[10]{--- É que eu estava com muitas saudades suas\ldots{} contudo,
Eva, não deixe você de aconselhar o Raimundo e o outro, como coisa sua,
sem que eles suspeitem que eu me meti nisso.}

\textls[-25]{Raimundo entrou embaraçado, e às interrogações de Eva, respondeu
gaguejante, procurando os termos com dificuldade, até que, mais animado
pela atenção com que era ouvido, seguiu sem interrupções:}\looseness=-1

\textls[10]{--- É que\ldots{} minha senhora\ldots{} o Samuel quis bater na
minha pequena\ldots{} a senhora bem sabe que ela é aleijadinha e
fraca\ldots{} vai então, eu perdi a cabeça, e quis dar cabo dele, ora aí
está! A pobrezinha de Cristo ficou estarrecida que nem pinga de
sangue parecia ter! Não, que uma coisa assim!\ldots{} Ainda hoje o
anjinho é ver (com perdão da palavra) o demo do velho que parece logo
de cera, como uma santinha!}

\textls[-15]{O desgraçado tinha medo de que o despedissem. O trabalho dos
colonos é duro, e era em todas as outras fazendas mal
retribuído. Só ali encontrara certas vantagens que lhe permitiam
viver à farta e ainda guardar todos os anos algum dinheirinho.\label{dinheirinho} Tinha
consciência do valor do seu braço de camponês robusto,
e não queria ser explorado. Do esforço que fazia para cavar a terra, via
brotar esperanças de futuro descanso. Antes de entrar no Mangueiral,
trabalhava para os grandes proprietários, suportando injúrias de
feitores boçais; acertara por fim numa fazenda-modelo, estava feliz.}\looseness=-1

\textls[10]{Eva provou que ele tinha sido impetuoso e que, se não refreasse
o gênio, poderia um dia comprometer-se. Que ela advogaria a sua causa se
o visse fazer as pazes com o Samuel. ``Não quero no meu sítio duas
pessoas que se aborreçam'', concluiu.}

O colono saiu cabisbaixo, e foi chamado o Samuel.

\textls[15]{Era um velho baixo, gordo, de cabelos corredios e brancos, a
quem Eva se habituara a chamar, desde pequena, por tio Samuel. Tinha por ele um certo respeito que a impedia de censurá-lo abertamente; deu-lhe um cálice de vinho, que ele esgotou, fê-lo sentar-se e
perguntou-lhe:}

--- Então, que foi isso?! Tio Samuel quis bater numa criança?

\textls[-15]{--- Porque já que o pai não sabe ensiná-la, é preciso que um
estranho se meta nisso! E desanco-a, se ela me torna a fazer caretas!}\looseness=-1

\textls[-10]{--- Não diga isso; quando alguma criança se portar mal com
vosmecê, faça queixa aos pais ou ao sr.\,Paulo. Eu não quero que o bom
tio Samuel seja o papão para os anjinhos. E agora, vamos, confesse que
se visse aqui a filha do Raimundo, dava-lhe um beijo\ldots{}}\looseness=-1

\textls[10]{Mas o tio Samuel era difícil de convencer; pirrônico,
avermelhava-se e gritava contra o Raimundo, a mulher do Raimundo, a
filha do Raimundo, as aves do Raimundo, que lhe saltavam para a horta;
até o cão do Raimundo, um diabólico rateiro, que já lhe rasgara as abas
de um casaco!}

\textls[10]{Eva sabia que o Raimundo estava ansioso por se mudar
para uma casa nova, muito maior do que a que tinha, perto do rio, onde a
mulher poderia ir com mais facilidade lavar. Prometeu ao Samuel
desembaraçá-lo da vizinhança do rival, com a condição de os
ver nesse mesmo dia em boa amizade. Samuel refletia,
mas por fim declarou rudemente que não daria para isso um passo, e saiu
sem ouvir mais nada.}

\textls[-15]{Ia zangado, resmungando baixo. A meio caminho encontrou a filha
do Raimundo; a pequenina, muito desbotada e raquítica, juntava pitangas,
que o vento espalhara no chão; deparando com o tio Samuel, ergueu-se
assustada, deixou cair as frutas que tinha posto na saia arrepanhada em
forma de bolsa, e, depois de vacilar um instante, deitou a correr,
coxeando, muito aflita. Samuel gritou-lhe:}\looseness=-1

--- Não sejas tola! Eu não te faço mal!

\textls[10]{Ouvindo aquela voz e não compreendendo as palavras, a pobre
menina caiu e foi bater com a cabeça no rebordo da pedra de um tanque.
Samuel, que sorrira ao seu primeiro movimento, sentiu grande mágoa, ao
vê-la cair, e apressou-se em socorrê-la. Estava desmaiada, com uma
brecha na testa e inundada de sangue. O velho pôs a criança ao colo e lá
se foi com ela à casa do Raimundo. A mãe da menina gritou contra o velho
maldito; a seu ver, fora ele quem lhe batera, empurrando-a.}

\textls[10]{Samuel esperava aquilo mesmo, e deixou-se insultar. Só depois da
criança voltar do desmaio é que se soube a verdade; o Samuel beijou-a,
deu-lhe dinheiro para doces, fez momices até vê-la
rir; e assim se fizeram as pazes.}

\textls[10]{Percorrendo com Paulo a colônia, Otávio
lastimava a prima mais do que nunca! Tendo uma habitação tão
encantadora, vivia a pobre Eva naquela triste fazenda de Santa Genoveva,
sem conforto para o espírito nem para o corpo, e ameaçada de um
desgosto, talvez irremediável.}

\textls[15]{Quando saíram do Mangueiral, Paulo acompanhou-os até meio do
caminho. Depois, os três seguiram calados, sentindo acabar-se o enlevo
desse dia encantador.}

\textls[15]{Otávio pensava no testamento do tio, no belo sistema
estabelecido por ele, na gentileza de Eva e no cavalheirismo de Paulo;
Eva relembrava a sua meninice; Noêmia meditava em tudo; nas trepadeiras
que vestiam as paredes exteriores do prédio, nos sons do violoncelo, nos
quadros, na mobília apropriada e distinta de cada compartimento, nas
casas dos colonos, com telhados novos e janelas abertas, nas flores,
nas crianças, no parquezinho das mangueiras, em Paulo, em tudo\ldots{}}

\textls[15]{Não imaginara nunca poder haver tanta elegância e graça numa
fazenda!}\looseness=-1

\enlargethispage{1\baselineskip}
\textls[20]{Quando chegaram a Santa Genoveva eram Ave-Marias, caíam as
sombras e perdiam-se no ar as vibrações do sino chamando os escravos
para a revista. No fundo esfumado do céu, destacava-se o batalhão dos
negros suados, doridos de cansaço, com um feixe de lenha e a
enxada ao ombro. Alinhavam-se em frente à casa do senhor. E, ao
aproximarem-se, Otávio e as amazonas ouviram, como um sussurro
de onda triste, o \emph{Sum Cristo!} murmurado ao mesmo tempo por
cem vozes, e o baque da lenha caindo como um fardo ao chão.}

%\end{linenumbers}

\section{IX}

%\begin{linenumbers}

\textls[15]{Era já noite quando Eva, entrando no quarto da mestra, a
encontrou a arrumar em uma grande mala cinzenta os seus muitos livros e
o seu pouco fato.}

--- Que revolução é essa? --- perguntou-lhe a discípula, sem
suspeitar a verdade.

--- Parto amanhã.

--- Para onde?

--- Para a Europa.

\textls[15]{A um gesto de admiração de Eva, a estrangeira arrastou
para perto da mesa uma poltrona azul e fê-la sentar-se nela;
depois, apoiando na mesa as suas mãos esguias e nervosas, fixou demorada
e silenciosamente o rosto da órfã.}

\textls[15]{--- Sim --- prosseguiu ---, o dever obriga-me
a partir amanhã\ldots{} Se fosse possível sairia hoje
mesmo.}

--- O dever! Mas qual é o dever que a faz deixar-nos assim, tão
repentinamente?

--- É justo que eu vá para o lado de minha mãe --- balbuciou a
mestra, como se falasse sozinha.

--- Recebeu carta dela, chamando-a?

Mme.\,Grüber respondeu com certo embaraço:

--- Sim\ldots{}

\textls[25]{--- Faz então muito bem em ir já. Eu pedirei a Deus por ela e
pela senhora\ldots{}}\looseness=-1

\textls[25]{A alemã voltou ao seu trabalho, e os olhos de Eva encheram-se de
lágrimas.}

\textls[-10]{Como passaria agora as noites? A mestra era o seu refúgio; na sua companhia corriam rapidamente as horas, aprendia deleitosamente
nesses serões íntimos a entreter conversações úteis e despretensiosas;
os seus bordados, os seus livros, os seus desenhos parecer-lhe-iam
monótonos e dificultosos desde que lhe faltassem o conselho, a
influência da amiga e o apoio da sua inteligência superior.
Respeitava-a, e nas horas de desalento, enfadada daquela casa sombria,
onde estava condenada a viver, naquela convivência da família que em vão
procurava achar agradável, fora sempre o braço salvador da estrangeira
que a impelira para o trabalho, como o único consolo verdadeiro e a
única distração profícua.}\looseness=-1

\textls[10]{No meio dessas reflexões, Eva ajudava a dobrar os vestidos, a
guardar os objetos esparsos sobre a cama, sobre as cadeiras e
sobre a cômoda. Findo o trabalho, encostaram-se à janela,
olhando sem falar para a escuridão da noite.}

--- A que horas sai? --- disse por fim Eva.

--- Às nove. Venha às oito horas ao meu quarto, para nos
despedirmos. Não esteja mais tempo aqui; vá descansar.

\textls[-15]{Ao retirar-se para o seu quarto, Eva, atravessando o corredor,
ouviu que na sala do fundo o tio discutia alto com o filho; pensando na
preceptora de Noêmia, não prestou atenção ao que diziam.}\looseness=-1

\textls[15]{Otávio lutava, procurava convencer o pai da inocência
da prima e da pureza das suas intenções; descrevia o que vira, com
entusiasmo; exaltava o sistema seguido no Mangueiral e punha em
evidência o critério administrativo de Paulo.}

\textls[15]{O fazendeiro impacientava-se. Não havia nada como a lavoura de
escravos. ``Se o Gabriel fazia num ano vinte contos, eu fazia cem, ora
aí está! E não me importo com isso! Cada qual quebra a cabeça a
sua vontade!''}

\textls[15]{Otávio referia-se ao próximo advento da abolição. Ria-se o
fazendeiro, afirmando vir ainda longe a esperada redenção dos cativos.
``Deixem gritar os jornais!'', concluía.}

\textls[15]{Voltaram a falar de Eva. Otávio procurou em vão demonstrar a
sua bondade e isenção da culpa que se lhe atribuía. Perguntou depois
pela estrangeira.}

\textls[10]{O comendador contou-lhe secamente o que se passara. Desgostoso
com o filho, que via revoltar-se contra as suas ideias e resoluções,
decidira não lhe fazer daí em diante a mínima
confidência relativa à lavoura nem mesmo à família;
contudo, deixou ainda transparecer o desejo de casar Eva quanto antes,
ou com o Azevedo ou com o Paulo, ou com o diabo!}

\textls[-15]{Otávio replicou que ela não era mulher que consentisse em se
casar à escolha de terceiro; era, porém, provável que amasse uma das
duas pessoas citadas, e, nesse caso, aplaudia a intervenção do pai.}\looseness=-1

\textls[15]{Apartando-se do comendador, ele foi pensando com tristeza nas
suas últimas palavras. Eva casar-se com o Azevedo! Que
insensatez! Mas não refletira ele já nisso, resolvido a
protegê-la? É que, nessa ocasião, supunha que eles se amavam;
mas agora, que o juiz lhe tinha aclarado o espírito com a
narração da carta, isso parecia-lhe absurdo. Eva merecia um homem
superior, e ele julgava o Azevedo um medíocre!}

\textls[10]{Eva casar-se com Paulo! Que suposição extravagante! Mas, afinal
de contas, era uma coisa perfeitamente realizável\ldots{} Eram
eles, porventura, filhos dos mesmos pais? Criados juntos quase
desde o berço, amando os mesmos lugares e as mesmas pessoas; ligados
pela mesma educação, pelos mesmos fatos, pelo mesmo passado, era
naturalíssimo amarem-se e quererem continuar a vida em comum.
Era um desfecho lógico\ldots{} esperado talvez pelos pais de Eva, quando
orientavam Paulo na administração da sua casa, elevando-lhe o
espírito a um nível superior, e formando-lhe tão bem o coração!
Nisso viu luz no quarto da estrangeira; a porta estava
entreaberta e ela escrevia em frente; vendo-a, Otávio
dirigiu-lhe um cumprimento, pedindo-lhe a honra de falar-lhe. Grüber
levantou-se e foi ter com ele à porta.}

Otávio começou:

\textls[15]{--- Deploro a sua partida, minha senhora, e rogo-lhe que
disponha de mim como de um amigo.}

\textls[10]{--- Obrigada; estou habituada a lutar sozinha com todas as
dificuldades, e, mesmo que o não estivesse, não desejaria incomodá-lo,
tanto mais que o senhor seu pai levaria isso a mal.}

\enlargethispage{1\baselineskip}
--- Meu pai é vítima de um anônimo perverso, que teve astúcia
suficiente para o desnortear; se ele não é um cavalheiro no trato
social, é um homem honesto e pronto a retratar-se desde que reconheça o
seu erro. Se hoje foi impetuoso e rude, deixando-se levar
irrefletidamente por uma intriga qualquer, amanhã será cordato, bondoso
e humilde; tudo depende de saber a verdade inteira; a senhora não
procurou talvez esclarecê-lo\ldots{}

\textls[-25]{--- Nem procurarei jamais. A minha situação é melindrosa, e já
agora incompatível com a dele. Abreviemos: agradeço-lhe o oferecimento,
mas não devo, nem quero aceitá-lo; parto de madrugada para
Santos. É provável que não nos tornemos a ver; fique, porém,
certo de que, seja para onde for que a sorte me atire, aí terá
uma amiga.}\looseness=-1

\textls[10]{Otávio curvou-se respeitosamente; a professora estendeu-lhe a
mão, que ele beijou, e separaram-se; ele foi para o seu quarto, ela
voltou para a secretária, onde continuou a escrever.}

\textls[10]{No dia seguinte, quando Eva abriu os olhos, o sol entrava pela
janela numa abundância alegre de calor e de luz. Eram
quase sete horas; arranjou-se à pressa e dirigiu-se ao quarto da
mestra. Bateu na porta, não lhe responderam; entrou.}

\textls[10]{Mme.\,Grüber não estava lá. Tinha também desaparecido a grande
mala cinzenta. Correu à janela: já nenhum carro! Só os dois sulcos
paralelos e fundos das rodas de um trole, em direção à estrada. Afastou
o cortinado; a cama estava ainda feita, com a colcha um tanto amarrotada
e uma pequena cova, da pressão da cabeça, na almofada. Por que não se
teria deitado Helena Grüber entre a frescura dos lençóis? Perto da
janela aberta tinha ficado a poltrona azul de ramagens. Passaria a
estrangeira parte da noite a olhar para as luminosas estrelas do céu
brasileiro?}

\textls[-7]{No balde de metal havia cinza de papéis queimados, a
manchá-lo de preto; no lavatório estava ainda a bacia
cheia de água e um copo com resto de elixir para os dentes. As gavetas
abertas e vazias mostravam o fundo claro, de pinho; no chão, a um canto,
fitas velhas, caixas de papelão rasgadas e uma gola servida
misturavam-se com uns malmequeres de pano; sobre a mesa redonda
destacavam-se o tinteiro de porcelana branca e uma caneta de
osso, negligentemente caída no pano de flanela preto e azul. Em
cima da cômoda, ao fundo, estavam um busto em bronze de Goethe
e uma carta em que a letrinha fina de Grüber traçara estas palavras:}\looseness=-1

``Para Eva.''

Eva abriu o sobrescrito, e, sentando-se na beira do leito, leu:

\pagebreak

\begin{quote}
Minha querida,
\smallskip

\noindent{}\textls[-13]{Aproveito o silêncio e a tranquilidade da
noite para lhe dizer adeus. Menti-lhe, Eva, menti-lhe pela primeira vez,
dizendo-lhe que a veria às oito horas, quando eu já tencionava partir de
madrugada! É que me falta a coragem, a mim, que tenho sido tão animosa
sempre, para este último abraço!}\looseness=-1

Geralmente, atribuem às mulheres da minha profissão a
mais profunda indiferença pelas pessoas em cujas casas temporariamente
vivem, cujos segredos muitas vezes surpreendem e cujas virtudes ou
defeitos não desconhecem. A mestra acompanha e dirige a discípula, vendo
só nisso o lucro material. O dinheiro é que a liga à
família estrangeira, a quem, por cálculo,
desde o princípio, pouco se une, para evitar probabilidades de
futuros desgostos\ldots{} Tudo o que há de carinhoso e terno na
sua alma transforma-se com o exílio, com o abandono da família,
com a dura condição mercenária a que se tem de sujeitar, na
consumidora e febril sede do ouro, com que se deseja libertar
depressa. E então não vê, não pensa, não sonha senão com o seu futuro
na pátria, deixando rolar a seu lado as lamentações ou as
alegrias, sem parecer dar por elas! Eva não me fará a injustiça de
acreditar-me assim. Não parto porque tenha a minha
independência feita; parto porque devo partir. Minha
mãe está velha, é justo que me estenda os braços, a mim que sou,
de direito, o seu único apoio. Contudo, deixe-me dizer-lhe que
não me julgo superior ao geral das mestras estrangeiras; vim
resolvida a praticar o mesmo e sairia de olhos enxutos se a não
deixasse, minha filha (permita-me dar-lhe este nome, que é o que
espontaneamente sinto rebentar-me d'alma), se não tivesse encontrado em
si um conjunto de bondade, de inteligência e de aplicação; uma
bela alma, enfim. Não é só a mestra quem faz a discípula, bem
vê; é muitas vezes a doçura da discípula que vence todas as resoluções,
por mais firmes, por mais ásperas que sejam, da mestra.

Creia que, de todas as recordações, será sempre a mais grata
para mim a das horas em que, em sua casa, eu a ouvia e via, sempre
atenciosa e distinta. Seja assim toda a vida, boa e calma, e há de
triunfar de qualquer mal que lhe intentarem fazer, apesar de que na
perversidade do mundo nem sempre os crédulos e os bons são os mais
felizes! Eva não me compreende agora: compreender-me-á mais tarde,
quando tiver com a idade a prática deste tão bem denominado
``vale de lágrimas''! Continue a estudar; leia com cuidado. Um
bom livro é sempre salutar para a alma. Fuja das teorias filosóficas e
das exposições pessimistas dos espíritos doentes. Não se deixe
prender, como tantas outras mulheres inteligentes do nosso tempo e da
sua instrução, pelos assuntos guindados das teses
sociais; deixe tais argumentos à competência e à prática dos
homens; o seu concurso não iria, com certeza, abalar as leis
estabelecidas e, ainda em cima, comprometeria a sua vida íntima! Uma
mulher com pretensões dogmáticas neste meio é, aos olhos dos
outros, uma ridícula, e aos seus próprios olhos uma
infeliz.

Colha, pois, na sua leitura só as flores e os frutos que se lhe
mostrem saborosos e sadios; é com eles que há de nutrir
proficuamente o seu espírito e tornar atraente a sua família
futura. Prossiga na cultura do seu jardim e na bela adoração
das suas flores; contemple as estrelas, como fazia; --- que mal
há nisso? Deixe que a censurem os infelizes que já não encontram
prazer na extática observação da natureza! Eles ignoram que ela é o
melhor dos livros e o mais puro dos mestres. Faça música e
desenhe. Segrede à arte as suas alegrias ou as suas tristezas, que ela
será a mais doce das amigas, pelo menos a mais consoladora. Natureza e
Arte são os dois formosos templos, onde em qualquer ocasião encontramos
um asilo seguro para o nosso espírito. Não se esqueça disto.
Repito, minha querida --- procure ler bons livros. Não lhe recomendo
esta ou aquela escola. Todas elas são boas, quando tratadas por um
escritor de talento e de convicção; todas elas são úteis, quando lidas
por uma pessoa que tenha o seu critério. Cosa, lide,
passeie, dispense às aves o carinho maternal que tão bem revela aos
escravos a sua piedade, e escreva de vez em quando a sua velha
amiga.

Deixo-lhe, como lembrança das nossas palestras e serões, o busto
do poeta cujas obras traduzimos tantas vezes juntas.

\begin{flushright}
Sua dedicada,\\
\emph{Helena Grüber}
\end{flushright}
\end{quote}

%\end{linenumbers}

\section{X}

%\begin{linenumbers}

\textls[10]{A família do comendador Medeiros aceitava, sem discussão e sem
abalo, todas as resoluções do seu chefe. A notícia da partida da mestra
foi recebida sem alarma. Nicota, em vésperas do casamento, sem parecer
preocupada, absolutamente, continuava na mesma frieza e concentração,
tratando com cuidado da sua quase extinta dor de ouvidos. Noêmia
regozijou-se e entristeceu-se a um tempo. Estava livre das lições, de
umas tantas horas de prisão e de enfado, experimentava uma sensação de
alívio, um gosto de indefinível doçura, é certo; mas
simultaneamente pensava que nunca mais tornaria a ver aquela senhora de
espírito e de evangélica paciência, e isso causava-lhe pena.}

A mãe trabalhava, fazia doces desde manhã até à noite, para o
noivado da filha, deliberava costuras, examinava com escrúpulo o
enxoval, recomendando zelo, muito zelo, às lavadeiras e engomadeiras,
que lidavam sem descanso. A sua opinião nunca fora ouvida nem pedida em
assuntos de outra importância.

\textls[10]{Era a governante da casa e isso bastava-lhe. Casara-se aos treze
anos, sem amor, sem simpatia, mas também sem repugnância. Sujeitou-se à
vontade do marido e ao seu mando, no começo por medo, depois por hábito.
De índole bondosa, não se queixava nunca; desculpava sem esforço as
faltas dos outros, mas não advogava perante o marido a causa de ninguém,
justa que fosse, porque ele zangava-se e ela temia-o. Aceitava
os fatos como eles se lhe impunham, sem tentar nem de leve
averiguar-lhes o fundo; e assim deixava correr diante de si,
idiotamente, como se os não visse, todos os acontecimentos domésticos.
De doze filhos, restavam-lhe três. A morte sucessiva dos mais
velhos, já criados todos, já homens dois, tinha acabado de acentuar na
sua alma dolorida a indiferença por todas as paixões,
amesquinhadas pelo grande vulto desses intensos desgostos.}

O marido nunca lhe falara de negócios; dispunha de tudo,
livremente, e, como se temesse indiscrições, evitava comunicar-lhe até
os receios que nutria a respeito de Eva.

O motivo da saída da mestra ficou, portanto, desconhecido para a
maior parte da família.

\textls[5]{Resolvido a proteger a todo custo a prima contra
qualquer intenção do fazendeiro, perdido em conjecturas dolorosas,
suscitadas pela incompreendida animosidade do pai contra a pobre moça,
Otávio decidiu ir, nesse mesmo dia, à casa do velho Morton,
pedir uma explicação que o comendador Medeiros se obstinava em
negar-lhe. Saindo para isso do seu quarto e dirigindo-se ao terreiro,
onde o esperava já pronto o animal, viu Eva na sala de costura armando
uns laços de fita na roupa de linho fino da noiva. Ela não o percebeu, e
ele, aproximando-se para dizer-lhe adeus, sentou-se a seu lado, e
louvou-a banalmente pela paciência e gosto com que enfeitava a
irmã; depois pediu-lhe que encetasse uns serões musicais e que o
ajudasse a ler os seus livros. As noites em Santa Genoveva eram
intermináveis, medonhas! ``Agora que mme.\,Grüber não vive entre nós'',
concluía, ``dispense-nos um pouco a sua atenção''.}\looseness=-1

Eva, sem deixar de trabalhar, respondeu ao primo, prometendo
fazer-lhe a vontade.

\textls[-10]{Noêmia foi interrompê-los, muito queixosa. Não sabia que fazer!
Tinha sono! Que manhã enfadonha e longa! Eva explicou-lhe:}\looseness=-1

\textls[15]{--- É a falta da mestra. Venha trabalhar comigo\ldots{} Olhe,
ajude-me a pregar estas rendas.}

Otávio deixou-as nessa tarefa. Entretanto, Noêmia descrevia a
sua manhã. Que tédio! As horas arrastavam-se como velhas trôpegas! O
calor era mais intenso. Tentara dormir, mas levantara-se depressa,
envergonhada de se deitar àquela hora! Passeara por toda a
casa, tinha lido e abandonado o livro, tocado e aborrecido a
música! Bendizia Eva por lhe compreender a doença!

\textls[-10]{Esta, cuja atividade não diminuía nunca, traçava um programa em
que entrava o estudo de línguas e de música, a cultura das flores e a
criação das aves. Na roça principalmente, insinuava ela, é preciso lutar
com a natureza e reagir contra a melancolia e a indolência que, muitas
vezes, a própria quietação do campo produz!}\looseness=-1

\textls[20]{Entretanto, o primo, batido de sol, percorria a estrada a
galope, até a casa do Morton.}

%\end{linenumbers}

\section{XI}

%\begin{linenumbers}

\textls[-13]{Esperando pelo velho amigo, que dormia regaladamente a
sesta, Otávio sentou-se entre uma estante repleta de
livros e uma mesa coberta de papéis. Na parede branca, sobre a
secretária, balançava-se agitadamente, à viração da janela, a folha de
um calendário, marcando o dia 10 de dezembro de 1887. Mais
acima, numa oleografia, ostentava-se, num fundo muito azul e
envernizado, de céu e de mar, a grande cidade americana de New
York --- da pátria de Morton. Sobre a pele de tigre, já
velha, estendida ao pé do sofá, caíra um número do
\emph{New York Herald}, escorregado para ali das mãos do
leitor sonolento, uma hora antes. Otávio levantou-se e
pôs-se a lê-lo maquinalmente; estava assim havia uma
boa meia hora, quando o dr.\,Morton, de chinelos e casaco de linho,
apareceu risonho.}\looseness=-1

\textls[20]{--- Pode dispensar-me o seu dia, doutor, ou terá algum
compromisso?}\looseness=-1

--- Nenhum\ldots{} Agora o meu único trabalho são as
duas horas de classe ali no colégio. Não tenho forças para
mais; estou gasto.

--- Bem! Nesse caso, poderemos conversar?

\textls[15]{--- Perfeitamente. Mas deixe-me primeiro mandar vir cerveja;
está um calor danado!}

O doutor chamou uma criada branca e sardenta, de avental azul de
riscado, que ia pelo corredor, e ordenou que lhe levasse cerveja e
copos. Momentos depois, saboreando-a, para ele, melhor das bebidas, fez
ao seu ex-discípulo sinal de principiar.

--- Estou disposto a ouvir.

\textls[15]{--- Desejo que se sinta ainda mais disposto a falar --- respondeu
Otávio; e aproximando-se do velho, continuou:}

\enlargethispage{1\baselineskip}
\textls[10]{--- Por estranho que lhe pareça o que venho pedir,
rogo-lhe que me acredite desinteressado e sincero. Como sabe, fui muito
novo para a Europa, desconhecendo detalhes, mesmo os mais íntimos, de
família. Sabia que meu pai tinha um irmão mais novo, Gabriel,
com quem nunca o vi, e de quem raramente ouvi falar. O que era, e onde
morava esse irmão de meu pai, pouco me importava a mim, que tinha sempre
outras coisas em que pensar. Diziam-no um homem ingrato,
frívolo e desafeiçoado aos seus; proibiam-me em pequeno que
repetisse o seu nome, quando acaso o ouvia, e foi talvez daí que nasceu
a minha indiferença. Voltando, porém, da Europa, depois de dez
anos, encontro em casa uma filha desse homem. Indaguei da
maneira por que ali fora admitida; como e por que se
fizeram as pazes; e responderam-me: ``O dr.\,Morton foi quem se meteu
nisso; ele, melhor do que ninguém, poderá explicar tudo''. Ora aí está
porque vim bater a sua porta, meu caro amigo!}

--- E seu pai?

--- Meu pai?!

--- Sim, por que não se dirigiu de preferência a ele?

\textls[20]{--- Temi recordar-lhe cenas dolorosas\ldots{} --- respondeu Otávio,
embaraçado.}

\textls[20]{--- Fez bem. E\ldots{} enfim, eu, como mero espectador,
sou mais imparcial.}\looseness=-1

\textls[10]{E o dr.\,Morton, enchendo de novo o copo, principiou, com o seu
acentuado sotaque inglês, uma história longa, cortada de vez em
quando por pequenas observações de Otávio.}

Uma tarde, havia muitos anos, recebera a visita de Gabriel
Medeiros, que o ia procurar para mestre de sua única filha,
Eva, e de um afilhado, o Paulo. Morton aceitara o encargo, e dois dias
depois seguia num trole até o Mangueiral, nome da
fazenda\ldots{}

--- Sei --- interrompia Otávio ---, já lá estive.

E depois de meia dúzia de frases relativas ao sítio, o velho
continuou o fio da narrativa:

\textls[-5]{Chegando ao Mangueiral, fora apresentado à
dona da casa, uma senhora muito amável, cheia de atenções e
delicadeza. Eva viera-lhe oferecer o rostinho corado, enlaçando-o nos
seus bracinhos redondos; Paulo, magrito e pálido, prestara grande
atenção às suas palavras, e assim correra amavelmente a
primeira lição. Desde esse dia não deixara de ir ao Mangueiral
três vezes por semana, tornando-se a pouco e pouco como uma
pessoa da família. Gabriel Medeiros era um homem empreendedor e
arrojado; a mulher, uma santa; Eva, dócil, e se não tinha lá um grande
talento, era estudiosa e ativa, e Paulo muito inteligente, mesmo
muito!}\looseness=-1

--- Já o conheço\ldots{} --- tornava Otávio.

\textls[-15]{--- Um excelente rapaz --- afirmava o
estrangeiro, fazendo uma pausa para beber uns goles de cerveja;
depois, sempre no mesmo tom, continuava dizendo que ao cabo de dois
anos de lição, ele, Morton, caíra gravemente doente com
uma febre perniciosa. Os médicos aconselharam-lhe mudança de
ares, e o bom Gabriel Medeiros lá o levou para o Mangueiral.
Passara ali um mês delicioso, vendo-se cercado de carinhos, como
se de fato pertencesse à família! Fora então que, uma
vez, enquanto as crianças brincavam no jardim e a dona da casa
cosia noutra sala, Gabriel Medeiros, no seu escritório, vendo
de longe a filha, lhe contara a sua vida toda.}\looseness=-1

\textls[20]{Jamais recebera prova de amizade e de confiança tão cativante
como essa.}\looseness=-1

Otávio redobrou de atenção, fixando sem pestanejar os seus belos
olhos castanhos no rosto pálido e amolecido do velho.

\textls[30]{--- Seu avô --- dizia Morton --- era um homem
ambicioso e\ldots{} vamos lá\ldots{}}

--- Ignorante --- acudiu Otávio.

--- É o termo.

\textls[20]{Esvaziado o último gole de cerveja, a narração seguiu
sem a menor paragem.}

\textls[15]{Era o caso que o avô de Otávio casara as filhas e filhos sem
lhes consultar absolutamente a vontade. Aos quinze anos as meninas
tinham maridos escolhidos a dedo por ele.}

\textls[5]{Aos rapazes poupava ele também o trabalho da escolha;
calculava fortunas, combinava a história, e zás! Era só dizer
ao filho: ``Fulana convém-te; já falei ao pai e está tudo
decidido''. Gabriel Medeiros era o último filho, e,
enquanto crescia a herdeira do homem mais rico do
município, mandou-o estudar a São Paulo. A pequena foi-se
desenvolvendo e, como não era feia\ldots{} ou, para melhor dizer, como
tinha fortuna, choviam-lhe os pretendentes; mas o negócio
estava já tratado entre os pais, e, embora não soubessem disso nem o
Gabriel nem a tal moça, o caso é que eram noivos! Vendo a futura nora de
vestidos compridos, o velho Medeiros escreveu ao filho, ordenando-lhe
que deixasse os estudos e viesse; mas Gabriel tinha tomado gosto pela
história e concluiu os exames do segundo ano. Rogou ao pai que o
deixasse: falou-lhe de futuras glórias e de elogios dos lentes. A nenhum
pai deixa de sorrir a ideia de ter um filho doutor!}

\textls[-25]{Foi concedida a licença, depois de pedida a
transferência para a realização do casamento à família
da noiva, que a deu de má vontade.}\looseness=-1

\textls[-15]{Por esse tempo o comendador Medeiros, pai de Otávio, já
era casado e responsável por uma casa importante, e, tendo sofrido um
grande abalo com as geadas consecutivas, recorreu ao futuro sogro do
irmão, que lhe abriu pronta e gentilmente a carteira\ldots{} com os
largos juros do estilo\ldots{} Correram muitos meses, os pretendentes,
desanimados, foram abandonando o campo, e o pai da moça instava
para que se realizasse em breve prazo o casamento contratado.}\looseness=-1

\textls[-10]{O velho Medeiros chamou seu filho Gabriel à fazenda e
declarou-lhe que resolvera casá-lo nesse mesmo mês, deixando-o depois em
liberdade para continuar os estudos. Gabriel, atônito, disse serenamente
que já tinha feito a sua escolha e prometido o seu coração. O pai,
desapontado, quis convencê-lo de que se devia casar com a tal
moça, e falou-lhe no seu compromisso de há tanto tempo. Em vão,
todas as súplicas; Gabriel negou sempre, e o casamento
desmanchou-se, com grande despeito do pai da noiva.}\looseness=-1

O estudante voltou para São Paulo com o apoio da mãe,
que lhe mandava a mesada, dizendo ser às ocultas do marido.
Esse, meses depois, morria de um aneurisma. Atribuíram a doença e a
morte ao desgosto sofrido com a recusa de Gabriel; puro engano: o velho
sofria há anos. Desesperado com tão tremendo golpe, e apertado
pelo seu comissário e credor, seu pai ficou alucinado e partiu para São
Paulo no intuito de convencer o irmão a voltar à fazenda e
recompor planos desfeitos. Seu tio, porém, não cedeu e zangou-se. Não
foi difícil ao comendador descobrir quem era a escolhida do coração de
Gabriel. Era a filha de um jornalista.

\textls[20]{Nesse ponto, o dr.\,Morton hesitou, um leve rubor tingiu-lhe as
faces amolecidas.}

--- E\ldots{}? --- perguntou Otávio na
ânsia de saber a verdade.

--- E o jornalista recebeu uma carta ameaçando-o de morte caso
não partisse com a filha imediatamente, para longe. Mas o homem não se
intimidou. Ele era querelento e andava na ocasião em polêmica acesa com
adversários políticos. Ora, quis a fatalidade que um
mês depois, ao sair à noite do seu jornal, o homem recebesse um
tiro, que o matou instantaneamente.

Otávio estremeceu fortemente; o professor continuou com
vivacidade:

\textls[20]{--- O crime foi atribuído à política, e para
isso não faltaram razões e poderosíssimas. O caso era desses a que a
conveniência manda pôr pedra em cima.}

Entretanto Gabriel casou-se com a órfã, e um dia, pondo em ordem
os papéis do sogro, deu com a maldita carta, cuja letra reconheceu.
Desesperado, correu a procurar o irmão, que lhe confessou lealmente ser
dele a carta, mas não ser dele o crime, que reprovava de todo o coração.
Uma circunstância fatal guiara o assassino a converter em
realidade uma ameaça que não queria chegar a ser outra coisa. Não são
raras tais coincidências na vida. Gabriel, convencido da
inocência do irmão, voltou tranquilo para casa e estendeu
confiadamente à filha do morto a sua mão.

\textls[15]{Quem assassinou o jornalista? Ninguém o soube. Gabriel
retirou-se com a mulher para o Rio, só voltando a São Paulo quando por
morte da mãe lhe coube, em partilha, a fazenda velha, a que
ele pôs o nome de Mangueiral. Por questões de terras, os irmãos
moveram uma demanda contra ele, e desde então nunca mais se falaram. As
irmãs casadas seguiram diferentes rumos; uma foi para Ribeirão Preto, a
d.\,Eufrosina; outra para o Paraná, a d.\,Maria Clara; morreram-lhe depois
dois irmãos e o único que lhe restava, o comendador, evitava-o.
Desgostoso com tudo isto, Gabriel voltou-se todo para a sua lavoura, que
procurou livrar das rotinas antigas. Libertou escravos, contratou
colonos, lutou como um herói, sacrificando a pequena fortuna no
serviço das suas ideias avançadas, mas conseguiu equilibrar-se
e chegou a gozar a sua independência. A esposa é que pouco
apreciou essa tranquilidade, porque morreu, deixando Eva ainda muito
nova. De várias filhas que tivera, só essa restava a Gabriel,
para consolo da sua viuvez.}\looseness=-1

Dr.\,Morton descansou um momento e depois prosseguiu:

--- Um dia de grande chuvarada, mandaram-me chamar à pressa ao
Mangueiral. Fui.

Medeiros fora atacado de uma congestão pulmonar;
encontrei-o na cama, rodeado de remédios, com o rosto
emagrecido, os cabelos brancos empastados de suor, e um forte cheiro de
febre e de sangue exalando-se da sua larga boca arroxeada, donde não
saíam nunca senão palavras de doçura e de perdão.

Quis animá-lo, esforcei-me por tornar a voz firme; o
enfermo sorriu com tristeza, e, estendendo-me a mão ardente, disse:

\textls[15]{--- Escute: eu não quero deixar a minha Eva só nesta casa com
Paulo, que não lhe é nada; ela não tem noivo nem amigas\ldots{}
lembrei-me de meu irmão, é a única pessoa da família
que me resta; quero reconciliar-me\ldots{} ele tem filhas; há
de ser misericordioso para a minha!}

\textls[-10]{--- Fui imediatamente procurar o comendador --- concluía Morton ---,
que às minhas palavras, honra lhe seja feita, apressou-se em
acompanhar-me. Seguimos para o Mangueiral a galope; quando chegamos, Eva
soluçava. Gabriel Medeiros tinha morrido.}\looseness=-1

\textls[-15]{Otávio, imobilizado, com toda a vida concentrada no olhar,
ouvia a voz lenta de Morton, com tão manifesta emoção, que o velho parou
um momento, como a consultar-se intimamente se deveria ou não continuar
a sua narrativa. Como a consciência o instigasse a prosseguir, ele
relatou que após o enterramento de Gabriel, o comendador apossou-se dos
papéis do irmão, a título de tutor da filha.}\looseness=-1

\textls[-5]{A isso Eva se opôs energicamente. No seu desespero, a
infeliz menina queria conservar tudo como o pai tinha deixado. O tio,
porém, interpretara mal esse movimento; supôs querer a sobrinha
guardar consigo, para futura vingança, a carta ameaçadora,
escrita por ele ao jornalista assassinado. Desse crime não havia outra
prova; compreende-se o desejo do comendador em a fazer
desaparecer\ldots{} Quem lhe diria que, em qualquer ocasião, se não
visse acusado dessa culpa? Se para Gabriel, seu irmão, haviam bastado as
suas afirmações de inocência, seriam elas suficientes perante a
sociedade e a justiça? Não! Perante a sociedade e a justiça, o assassino
do jornalista seria forçosamente o autor da ameaça, o comendador
Medeiros; quanto ao outro, ao assassino incógnito, ao
verdadeiro assassino, esse não poderia ser tido senão como um fantasma,
feito por um covarde e ridiculíssimo disfarce.}\looseness=-1

\textls[20]{Otávio levantou-se, indignado contra o estrangeiro.
Teve ímpetos de o estrangular, de lhe dizer perto do rosto
enrugado: mentes, mentes, mentes! Mas o outro continuou no seu sotaque
dificultoso, com palavras claras:}

\textls[15]{--- Tudo que lhe tenho dito é duro e doloroso de se dizer a um
filho; sou um homem rude, conheço; mas há certos casos em que, mesmo à
força de um grande sacrifício, devemos dizer toda a verdade; tanto mais,
que eu tenho a convicção profunda de que seu pai não foi culpado. Nem o
Gabriel lhe entregaria a filha, se assim o julgasse!}

\textls[15]{Otávio, emocionado fortemente, não respondeu; tinha os olhos
cravados no chão, os pensamentos a chocarem-se-lhe desordenadamente no
cérebro.}

Quem teria assassinado o avô de Eva? Quem teria traído o segredo
do pai? E como poderia ele descobrir o culpado, depois de tantos anos?
Que época iria ele investigar, que recordações e
mistérios revolver? Como prender o fio já quebrado dessa
miserável trama? A vida é pequena, mas se as gerações se sucedem como as
ondas, como fazer voltar as já passadas? Por que se conservara o pai
inerte? Por que não lutou até apresentar aos olhos do irmão, seu
único juiz, o verdadeiro criminoso? A sua dignidade, a sua
honra, exigiam todos os sacrifícios para isso!

Acabrunhado, Otávio escondeu o rosto entre as mãos.

Diante dele, de pé, o dr.\,Morton dizia:

\textls[-20]{--- A única culpa de seu pai foi ter escrito aquela
carta. O resto não foi com ele. E essa mesma culpa encontra atenuante na
inconsciente facilidade com que neste país se escrevem cartas anônimas.}\looseness=-1

%\end{linenumbers}

\section{XII}

%\begin{linenumbers}

\textls[20]{Eram seis horas quando Otávio, tomando as rédeas do
cavalo, seguiu sozinho pela estrada fora.}

\textls[-15]{Havia uma grande tranquilidade em tudo. O céu estava de um azul
suave e igual, sem um traço mais claro ou mais escuro, sem uma ruga,
liso e manso; as árvores pareciam imóveis, nem a mais sutil tremura
agitava a ramaria espessa; as boas-noites campestres abriam sem rumor os
seus cálices dourados, e os copos-de-leite, entre o verde
escuro das folhas, derramavam um aroma capitoso.}\looseness=-1

De longe em longe partia da floresta o grito estrídulo de uma
sentinela solitária, a araponga, e havia no chão, entre as
folhas secas, a bulha de um animal a esconder-se, de rastos,
precipitadamente. Quatro palmos acima da grama,\footnote{Capim.} numa extensão 
enorme, a perder-se de vista, desdobrava-se uma nuvem de formigas
aladas a que na província
chamam bitus, nuvem transparente, parecendo formada por uma infinidade
de malmequeres mimosos, que, tendo-se desprendido da terra, adejassem no
ar. Súbito, uma ave, passando rente de Otávio,
bateu-lhe no rosto com a asa; o moço, desprevenido, deu um salto na
sela.

\textls[-15]{Lembrou-se logo depois de ter, na sua meninice, visto muitas
vezes esse pássaro, que acompanha, ao entardecer, o viajante
paulista, saltitando e voando ao redor do animal. Era o curiango,
cinzento, do tamanho de um sabiá, de pernas muito curtas. Ele ia e
vinha, ora rente ao chão, ora sobre a cabeça de Otávio, ora de
um lado, ora de outro, deixando-se ficar atrás, para saltar de repente e
ir passar adiante. Durante cerca de meia
légua os bitus, abrindo e fechando o duplo par das suas asas
finas, cobriam o campo como uma gaze esbranquiçada e movediça. Semeadas
no solo, as suas habitações erguiam-se em montículos de barro,
amiudadamente. Entretanto, o azul do céu ia escurecendo a pouco e pouco,
e num ponto e noutro desabrochava uma estrela pequenina e trêmula. Nos
ninhos fofos, feitos de grabatos e palhinhas finas,
ajeitavam as aves os seus corpos gentis; e de entre os molhos
roxeados a erva barba-de-bode, que ladeava a estrada, erguiam
o voo grandes vagalumes, levando a sua luz pelo ar,
vagarosamente, como uma bela esmeralda muito iluminada.}\looseness=-1\footnote{\textls[-25]{Estes
  pirilampos têm tão intensa luz, que iluminam um recinto escuro,
  servindo muitas vezes de lamparina, quando reunidos uns três, sob um
  copo emborcado.}\looseness=-1}

E assim serena, sonolentamente, a noite ia-se estendendo sobre a
terra.

\textls[10]{Otávio deixava-se levar pelo cavalo, comprado havia duas semanas
apenas, a um fazendeiro do Córrego. Cansado de corpo e de espírito,
aniquilado, ele entregava-se assim, sem direção, àquela noite
silenciosa e bela.}

\textls[15]{As revelações bruscas do velho Morton tinham-lhe caído
no coração como pontas de punhais. Sentia-lhes a dor e ainda
não as compreendia bem!}

Apertando entre os dentes o charuto apagado, com o olhar fixo
no vácuo, ia como um sonâmbulo, numa grande
abstração.

\textls[15]{Podia gritar rouca ou estridentemente a araponga, à
vontade; podiam farfalhar as folhas à frescura crepuscular, ou luzir as
estrelas, ou bailar ingenuamente o pirilampo; ele não os via, ia
verdadeiramente desligado da natureza e das coisas.}

\textls[15]{Essa tormenta irremediável, esse tremendo golpe inesperado,
produzira-lhe um quebramento de forças, um aniquilamento insuperável e
esmagador.}

\textls[20]{E, no entanto, não penetrara a fundo na exposição do amigo; não
via nada de positivo, nada de claro; de onde extraísse, limpa e
exata, a verdade nítida dos fatos.}

\textls[10]{Em casa de Morton tivera ímpetos de matá-lo,
de sufocar-lhe na garganta aquelas insinuações malignas a respeito de
seu pai; depois, nervoso, doente, num excesso de sensibilidade, chorara;
agora não; sentia um cansaço enorme, a sensação de quem tivesse rolado
uma montanha de alto a baixo.}

\textls[-10]{Ruminando as palavras do mestre, já quase
maquinalmente, à força de pensar nelas, Otávio percorreu a
estrada longo tempo. O animal que ele montava levara-o até à
cancela fechada de um cafezal, e aí, com a cabeça estendida por cima das
traves horizontais da porteira, relinchava com força. Chamado à
vida exterior por essa paragem brusca e pelo relinchar do cavalo,
alarmando no lugar em que nascera os seus velhos companheiros,
já recolhidos, Otávio percebeu estar no Córrego, longe
duas léguas de Santa Genoveva, e ter feito, imprudentemente,
uma viagem inútil.}\looseness=-1

O cavalo não se movia; relinchava sempre, alegre, mas já com
impaciência. De lá de dentro, muito ao longe, veio o som
quase apagado da voz de outro animal, respondendo ao
companheiro.

\textls[10]{Resolvido a seguir para casa, a galope, sem interrupção,
Otávio puxou com força pela rédea, depois de ter feito recuar o
cavalo. Nada conseguiu; colocando as quatro patas fundamente no
solo barrento, com a cabeça sacudida pelos movimentos das
rédeas, a boca espumante, o beiço arregaçado pelo freio de prata, e o
olhar ardente, o animal continuava firme, ouvindo o eco saudoso
da voz do companheiro a chamar por ele.}

\textls[10]{Desesperado, farto de chicotear e jogar as rédeas,
Otávio praguejou e, com um movimento rude, fincou-lhe os
calcanhares nas ilhargas; o cavalo continuou firme; o moço então
apeou-se, tentou, puxando pelo freio, mudar-lhe a posição, e
levá-lo a uns cinquenta metros de distância, para
depois montar e seguir; mas foi tudo em vão. O cavalo escoiceava,
firmava-se ora nas patas traseiras, ora nas dianteiras,
abaixava e levantava a cabeça furiosamente, mostrando os grandes dentes
amarelos, por entre os quais já escorria o sangue. Otávio
viu-se forçado a afastar-se do animal que resfolegava, cobrindo-se de
suor.}

Vendo-se abandonado, o cavalo deu rapidamente meia volta, armou
um salto, transpôs o valo que delimitava a fazenda do
Córrego, e perdeu-se a galope, na escuridão do cafezal.

--- Vai com o diabo! --- gritou-lhe o dono enraivecido.
Que fazer? Caminhar atrás do animal? Tolice. A cancela dos cafezais do
Córrego, lembrava-se ainda, ficava distante da habitação
quase uma légua! Valia mais ir a pé, por um atalho, ao
sítio do Torres; seria mesmo muito agradável; a gente
do Córrego era-lhe antipática; fora-lhe apresentada em
casa da irmã do Antunes. No dia imediato mandaria o pajem
buscar o animal; nessa noite preferia incomodar o Torres, cortando por
uma picada na floresta. Era muito mais perto. E, resolvido, entrou no
mato.

\textls[15]{Havia dez anos que não andava por ali; as suas
reminiscências, porém, não o enganaram; contudo, sentia-se mais
impressionado que outrora.}

\textls[10]{A floresta estava tenebrosa: pareceu-lhe sentir rumores
estranhos, pios funestos e evaporações úmidas de
pequenos lagos solapados em limos e ramos quebrados de arvoredo.}

\textls[-15]{Ao cabo de algum tempo, percebeu que havia errado o
caminho\ldots{} parou indeciso; qualquer lado que tomasse,
tomá-lo-ia ao acaso; continuou, pois, em frente, tateando às
vezes com as mãos estendidas, quando a profusão da folhagem punha mais
trevas em redor. Nos pontos de maior negrura parava, julgando
ver diante de si uma alta parede compacta e
inacessível; avançando os braços, reconhecia o erro e seguia, para de
novo sustar repentinamente os passos, à beira de grandes buracos, que
lhe parecia ver abertos no chão.}\looseness=-1

\textls[15]{Depois de errar algum tempo através de muitos ramos e galhos
acumulados até grande altura, viu luzir fogo. Era uma luz pequenina,
misteriosa, ora oculta, ora descoberta, de uma cor avermelhada,
surgindo de súbito na floresta escura, como num conto de fadas
a lâmpada maravilhosa.}

Otávio suspirou de alívio; encontrava um guia, um
companheiro, no meio daquela solidão; aquilo orientá-lo-ia no caminho de
casa. Procurava uma trilha que o levasse até lá, mas não havia nem um
carreirinho aberto naquela direção! A luz fora feita no recanto
mais ínvio, mais intrincado, e mais difícil do bosque.
Afrontando todos os perigos, Otávio continuou a andar,
esbarrando frequentes vezes nos troncos espinhentos ou
resinosos, com os olhos fixos no único ponto luminoso em toda
aquela vasta escuridão. Por vezes, a grande e pujante floresta
assemelhava-se aos seus sentidos perturbados, a uma estreita
cela de tetos baixos e úmidos. Tudo parecia
cerrá-lo de perto e prestes a esmagá-lo, desabando
sobre ele! À proporção que se aproximava da luz, ela ia-se tornando
maior; chegando perto ouviu vozes, um rumor baixo, medroso, soturno,
mas, evidentemente, humano.

Compreendeu. Fora parar a um quilombo!\label{Ref7}

\textls[15]{Através das paredes esburacadas, improvisadas à pressa,
com sapé, ramos e taquaras, saía, pondo no bosque umas finíssimas e
móveis agulhas douradas, o brilho do fogo em que se aqueciam,
quase nus, magros e famintos, meia dúzia de
pretos. Estavam ali agachados, falando em segredo, roendo os
ossos de umas galinhas roubadas na fazenda mais próxima,
enquanto não viessem dois ou três companheiros, idos à
cata de alimento, sob a proteção benfazeja da noite.}\looseness=-1

\textls[15]{Contrafeitos, apertados, fugindo ao sono, estremecendo ao menor
sopro do vento, ouvindo um passo em cada folha caída, um grito humano em
cada pio de ave, temendo sempre a perseguição do branco, mergulhando nos
rios ao vê-lo de longe, saindo ao sentirem-se morrer, para
rolar na terra ingrata, correndo com o corpo despedaçado de cansaço,
sempre arrepiado com as alucinações do medo, sempre
atrás da quimera-liberdade, e sempre agrilhoados, sob a pressão de um
pesadelo medonho, esses desgraçados tinham um aspecto de fantasmas
lúgubres, carbonizados nos braseiros do inferno.}

\textls[20]{A escravidão selara-lhes nos lábios o sorriso; o medo
pusera-lhes no olhar a desconfiança e o ódio; a fome, o frio e a insônia
descarnaram-lhes os esqueletos, tornando-os numas verdadeiras múmias
aterrorizadoras.}

Sem querer alvoroçá-los, Otávio deixou-os no
seu esconderijo, certo de que o tomariam por um espia, fosse
qual fosse o modo com que lhes falasse. Andou com precaução,
silenciosamente; mas os pretos lá dentro pressentiram-no,
abafaram as vozes e apagaram o lume, entornando-lhe água em
cima. Ficou tudo de novo silencioso e negro. Otávio seguia
às apalpadelas, impressionado, medroso, esperando a todo
instante sentir-se agarrado por um desses desgraçados quilombolas, que,
no entanto, estavam encolhidos, com os dentes apertados e os membros
trêmulos. Depois de errar por muito tempo, sem saber em que direção, se
iria em caminho reto ou se giraria num círculo, começou a
sentir fumaça, que se ia condensando a mais e mais, e a ouvir uns
estalidos secos, miudinhos e consecutivos.

Relampagueou-lhe pelo espírito uma suspeita horrível:
teriam cingido toda a mata de fogo, para caçar o quilombolas?

Tropeçando em tocos do chão, esbarrando em troncos ou
enleando-se em cipós, ele cada vez se sentia mais desorientado e mais
aflito. O cheiro verde do mato redobrava de intensidade como se os
vegetais clamassem todos pelo socorro divino. Toda a atmosfera se
carregava com os vapores quentes dos brejais e das lamaceiras
de húmus que acolchoavam o chão.

\textls[20]{Convencido de que girava sempre sobre os mesmos pontos,
Otávio imaginava já sentir o terror das tonturas e as primeiras
manifestações da asfixia. Vendo avolumar-se o fumo e ouvindo já
distintamente os estalidos das plantas, a arder, admirava-se de não
ver ainda o fogo, que jazia, sufocado por arbustos e ervas
rasteiras. Subitamente, porém, ouviu ruído de vozes humanas e viu uma
labareda cor de ouro, que irrompeu com violência até
uma certa altura, para tornar a descer e ficar como uma onda mole, a
lamber o chão.}

Era agora mais fácil caminhar, tendo aquele farol a
guiá-lo. Urgia ir depressa, antes que as chamas se unissem e não lhe
dessem passagem. Estava fatigado; molhava os pés em
águas estagnadas, batia com o rosto em lianas pendentes, mas
continuou até que se pôde fazer ouvido pelos que sitiavam o fogo.

--- Por aqui! --- gritou-lhe uma voz imperiosa, e
Otávio saltou para o aceiro. Vendo-se livre, respirou com
alegria. Estava em frente do fazendeiro Torres, que lhe fazia perguntas
atropeladas. Otávio comunicou-lhe o seu encontro com os
quilombolas e o medo que nutria de que o fogo os
alcançasse. Urgia salvá-los!

\textls[15]{--- Não há perigo --- respondeu-lhe o Torres
---, eles sabem fugir. A minha intenção não era queimar toda a
floresta. O fogo propagou-se pelo mato dentro, contra a minha vontade.
Você sabe, este mês de setembro é propício às queimadas e eu quis
aproveitá-lo, mas a seca é demais!}

\textls[10]{No entanto, as chamas uniam-se marginando com um listrão
vermelho toda a orla da mata. Num ponto e noutro subiam, enlaçando-se e
ondeando no ar línguas amarelas e azuis, que lambiam as frondes
altas do arvoredo e abrasavam a atmosfera. Pendentes das altas perobas e
dos jequitibás gigantes, os cipós eram os fios condutores de
incêndio, as escadas de assalto, enquanto as labaredas
não tiveram força bastante para se atirarem de chofre,
pesadamente, sobre as grandes como as pequenas plantas. Árvores e
arbustos bracejavam na luta; as aves voavam tresloucadas e os
répteis fugiam das tocas para inutilmente procurarem uma aberta
por onde se pudessem salvar.}

\textls[25]{Em pouco tempo toda aquela região verde parecia convertida num
mar de fogo. No céu, até então escuro, havia agora clarões
rubros, vermelhidões de aurora envoltas em espirais de fumo
sufocante.}

\textls[25]{Os negros gritavam, abrindo o aceiro às foiçadas.
Ouviam-se as crepitações dos vegetais nas chamas e viam-se bailar no ar,
diabolicamente, as faúlhas do incêndio e os fragmentos das
plantas carbonizadas.}

\textls[20]{Invencível e triunfal, o fogo consumia tudo, bebendo sequioso a
seiva da terra.}

Quando o sol rompeu as nuvens, o incêndio empalidecia.

\textls[20]{Otávio, coberto de arranhões no rosto e nas mãos, não esperou
por essa transição.}

\textls[-20]{Lembrando-se das grandes caminhadas a que
se afizera na Suíça e na Alemanha, seguiu a pé para casa, deixando o
Torres no trabalho.}\looseness=-1

\textls[-15]{Eram duas horas quando transpôs a porta do terreiro de
Santa Genoveva. Os galos cantavam e, no tanquezinho de cimento,
a um canto, caía água da bica num rumor ininterrompido e
monótono.}\looseness=-1

%\end{linenumbers}

\section{XIII}

%\begin{linenumbers}

\textls[20]{Dois dias antes do casamento de Nicota, a família Medeiros
fez-se transportar para a cidade.}

\textls[-15]{Na frente tinham ido, a pé, os pajens e as mucamas, em bandos
disseminados, com trouxas à cabeça. Foram também duas carroças cheias de
malas com roupas, grandes latas com doces cristalizados e com
sequilhos,}\looseness=-1\footnote{Biscoitos de polvilho, próprios para chá.}
\textls[-15]{e cestas de frutas, de hortaliça e de ovos. O feitor ficou encarregado
de remeter nas manhãs imediatas todas as flores que pudesse arranjar, e
leite, muito leite. O casamento de Nicota despovoava os jardins e secava
a garganta dos novilhos.}\looseness=-1

\textls[10]{A dois quilômetros do caminho, deparavam com a queimada do
Torres. A floresta que marginava a estrada estava destruída. Em
vez dela estendia-se um campo esbranquiçado, vasto, nu, coberto
de cinzas, triste como um cemitério, onde de longe em longe um tronco
arruinado, baixo, carcomido e negro, semelhava um pobre túmulo
abandonado. No entanto, dentro desses tocos carbonizados, o fogo, lento,
oculto, continuava ainda a sua obra de destruição, sorvendo toda a
pujança da terra, aproveitando gulosamente todas as raízes dos
vegetais extintos.}

\textls[15]{A família Medeiros, acostumada a essas transformações,
pouca importância ligou a isso; só Otávio lamentou
aquela mudança recordando-se, com amargura, da noite ali passada.
Chegados à casa da cidade, fechada havia muito, presidiram
às lavagens do soalho, dos vidros e das portas, ao pregar das
cortinas de renda nas janelas e dos reposteiros vermelhos nas portas;
removeram depois a mesa da grande sala de jantar para outra sala
interior, afim de que aquela servisse para o baile; puseram
então em volta, alinhadas junto à parede de papel envernizado,
cor de creme, as cadeiras alugadas ao clube, e nos intervalos das
janelas os dunquerques de vinhático, com portas de espelho e
tampos de mármore, sobre os quais se ostentavam um alto espelho esguio e
duas jarras de porcelana.}\looseness=-1

\textls[-15]{A sala de visitas, na frente, foi transformada em capela.
Chamaram um armador, homem acreditado, que trabalhava na cidade havia
muito tempo, para igrejas e particulares. O sr.\,João Coelho, muito
sério, grave, com a barba bem escanhoada, o colete e as calças brancas
irrepreensivelmente engomados e o casaco de alpaca preta muito limpo,
acudiu ao apelo sem hesitações. Respondeu por
monossílabos, numa grande economia de palavras, às
perguntas do comendador, e armou ao fundo da sala o mesmo altar que,
havia trinta anos, executava: um alto espaldar chato coberto de
tarlatana prateada, guarnecido de rosinhas de pano cor-de-rosa,
entre folhas verdes espalmadas. Sobre a mesa do altar, linhos finos
orlados de renda larga, o crucifixo, o ritual em \emph{chagrin} preto,
os palmitos de angélicas com espigas prateadas, a salva
pequena, de prata, com os anéis de aliança, grandes castiçais
com velas de cera e ao lado a caldeirinha e o hissope. Aos pés
do altar o tapete, as duas almofadas de cetim para os
nubentes, e flores desfolhadas.}\looseness=-1

Na outra sala da frente, a dos homens, as mesas de jogo com os
baralhos em cima; e numa outra, interior, o bufete.

Nessa do bufete havia ao centro uma grande mesa, em ferradura, e
nela variedade infinita dos saborosos doces paulistas:
pirâmides de fios d'ovos, cachos de caraguatás, de ovos
também, as elegantes espingardas açucaradas postas ao
alto e seguras com um laço de fita; pudins de todos os feitios,
qualidades e tamanhos; os pratos de doce de algodão, de pinha; o afamado
veludo e outros semelhantes; torres ebúrneas, feitas com transparentes e
finas fitas de coco, prendendo-se graciosamente em arrendados
artísticos; taças com frutas cristalizadas; geleias
cor de topázio e geleias cor de rubi; flores
de coco, encomendadas com antecedência, de Itu, e da
d.\,Gabriela, de Campinas, e que se espalhavam por toda a mesa com
profusão, as hastes, em grinaldas, em ramos e em festões aqui arroxadas
como os pendões das bromélias, acolá brancas, acetinadas e
úmidas, como as pétalas dos jasmins.

\textls[-10]{Nas extremidades da mesa viam-se os dois indefectíveis coqueiros
artificiais, cravados em colinas relvadas, onde insetos e
répteis de massa pintalgada se estatelavam regaladamente,
enquanto um macaquinho no alto tirava um coco da planta e
embaixo se enroscava no tronco uma jararaca verde e preta de cabeça
chata.}\looseness=-1

\textls[25]{Era a mesa clássica das grandes solenidades, iluminada
por múltiplas serpentinas.}

\textls[-10]{Em um ângulo da sala havia uma outra mesa carregada de frutas:
mangas perfumadas, melancias abertas, mostrando as sementes escuras
entre a escumilha cor-de-rosa da polpa; abacaxis,
jabuticabas, laranjas, uvas, pêssegos e figos, em peanhas, em
bandejas, em cestas, em montões! Ao lado, um quarto transformado em
botequim; sobre o balcão frutas secas, queijos já em fatias, e nas
prateleiras muitas garrafas de vinhos, cervejas e licores em
abundância.}\looseness=-1

De entre os rótulos luzidios e coloridos das garrafas, para os
carnudos abacaxis, ou os manjares da mesa principal, ia e vinha o sr.\,
João Coelho, armador, numa gravidade de lorde, arranjando simetria onde
via rebeldias de estilo, ou uma falta à estabelecida praxe
sempre rigorosamente mantida nas festas de importância.

Na copa, tudo a postos; e na cozinha quatro cozinheiros
e muitos ajudantes.

\textls[20]{Entravam para os quartos tabuleiros com vestidos,
vindos das modistas; os pajens traziam caixas de luvas, de
leques, de plumas, de sapatos de cetim, até à última
hora eram precisos objetos que os lojistas mandavam em profusão para
escolherem em casa, logo ao primeiro pedido e cujo pagamento só seria
feito no fim do ano, como de costume.}

\textls[10]{Chegara, portanto, o dia marcado para o casamento.
Otávio não se tinha esquecido da carta anônima que acusava Eva e
denunciava para aquele dia uma revolta em Santa Genoveva. Decidido a
falar nisso ao pai, foi procurá-lo à saleta.}

O comendador estava só, enrolando nos dedos um cigarro de palha.
O filho aproximou-se e perguntou-lhe que pretendia fazer: se iria para
Santa Genoveva logo que fosse realizado o casamento, ou se
descansava em alguma ordem que tivesse dado ao feitor.

O fazendeiro, muito calmo, explicou ter contado tudo ao Antunes
e ter-se ele prontificado a ficar com toda a sua gente cercando Santa
Genoveva. Desse modo, concluía, ou os escravos com medo não farão bulha,
logrando a \emph{boa} Eva, ou serão presos e castigados. ``O Antunes
tem jeito; posso ficar tranquilo!''

\textls[15]{Otávio estranhou que o pai desse ao Antunes
uma incumbência tão delicada. Não dera
crédito à carta e por isso não tinha, até esse
instante, pensado em qualquer combinação; vendo, porém, agora, que
admitiam a possibilidade da revolta, resolvera-se ele mesmo a ir para a
fazenda, de prevenção.}

\textls[15]{O fazendeiro abriu a boca, espantado. Depois afiançou ser
Otávio um \emph{europeu} fraquinho, que seria ferido,
maltratado, morto talvez; deixasse o Antunes à vontade.}

--- Aquilo, sim, é mesmo um \emph{homme} de respeito! E
também o nosso feitor não é nenhum sarambé\ldots{}

\textls[-25]{--- Não tente despersuadir-me, porque será em vão ---
respondeu-lhe o filho; --- assistirei ao casamento de Nicota e
partirei logo depois.}\looseness=-1

--- Tolice! E se eu não quiser?!

--- Ao contrário, vossemecê vai mandar-me para lá,
visto acreditar que se dá de fato a revolta; e vossemecê mesmo iria, se
o não prendessem aqui outros deveres! Faço esta noite um passeio
inútil, pouco importa; mas não darei direito a que qualquer
imbecil me ponha na conta de covarde\ldots{} Quero ver para
crer\ldots{}

\textls[15]{--- Fanfarronadas! --- murmurou o velho Medeiros,
encolhendo os ombros; depois calou-se e mordeu com raiva o cigarro sem
lume. Compreendia ser trabalho inútil tentar reter o filho na
cidade e deixou-o sair sem dizer mais nada. Todo o seu ódio se
voltava para a sobrinha, por julgá-la causadora fatal de
grandes males futuros na família.}

Ela surgia no seu espírito doente, como um espectro de
vingança e de ódio; não via nela a filha do irmão, a rapariga
sensata e honesta: via a neta do jornalista, a sombra doce e lacrimosa
da órfã do assassinado, alguma coisa de sobre-humano: como que um grito
de inenarrável dor que tivesse tomado carnes e formas
de mulher e sobre cuja cabeça voasse, como uma borboleta fatídica, uma
certa carta anônima, em má hora escrita\ldots{}

\textls[-20]{O comendador passou o resto do dia numa aflição atroz;
tentou mais uma vez despersuadir o filho; mas, vencido pelos seus
argumentos e pela sua vontade, recolheu-se desalentado ao quarto; não
achava um canto onde estivesse a seu gosto; teve ímpetos de
arrancar as cortinas e os reposteiros novos, de pisar as flores já
dispostas nos vasos e varrer pela porta da rua a pontapés todas as
cadeiras emprestadas pelo clube e muito bem alinhadas no salão de
baile.}\looseness=-1

%\end{linenumbers}

\section{XIV}

%\begin{linenumbers}

\textls[15]{Às oito horas, nas salas iluminadas e floridas movia-se a fina
flor da sociedade da terra.}

\textls[15]{A dona da casa, atrapalhada no seu vestido de gorgorão, com
grande cauda e flores artificiais no peito e no cabelo, sentia-se
vexada, tinha o andar mole, curvava-se para a frente, segurando nas
mãos, calçadas em pelica branca, o seu leque de madrepérola e rendas
verdadeiras.}\looseness=-1

\textls[-10]{Geralmente as senhoras paulistas apresentavam-se nos bailes num
requinte de elegância e de luxo como se não vê facilmente no
Rio de Janeiro. As suas toaletes, importadas quase
sempre diretamente de Paris, tinham, a par da riqueza, graça e
originalidade.}

\textls[-20]{Famílias cujas casas não demonstravam o mínimo
vislumbre de gosto, tendo os interiores sem conforto e sem poesia,
quase vazios de móveis e completamente faltos de
objetos artísticos, apresentavam-se nos bailes de uma maneira
verdadeiramente sedutora. \label{Ref12}}\looseness=-1

\textls[-20]{As representantes da geração passada, senhoras idosas e
ignorantes, não sabendo ler, sabiam vestir-se com a seriedade e a
distinção que a sua idade requeria. Um agente esperto ocupava-se, em ir
e vir de Paris, com encomendas particulares, para as cidades do
interior. Não tendo ainda a doce e bela idolatria do \emph{homme}, o
espírito feminino ocupava-se pertinazmente com os adornos
pessoais.}\looseness=-1

\textls[10]{Era nos bailes onde melhor se podia notar esta
circunstância; neles se reuniam as famílias mais
importantes, com toda a pompa. Pena era que os cavalheiros não
concorressem com igual requinte e esmero para o brilhantismo da sala.}

\textls[-20]{Os primeiros convidados a chegar à casa do comendador foram o
Amaral e a senhora, d.\,Clarinha, mulher formosa, com um par de olhos
singulares, ora esmeraldas, ora turquesas, conforme o raio de luz que os
ferisse, e as pestanas muito negras a sombrear-lhe a alvura doce das
faces. Esbelta sem ser alta, graciosa, pusera na sua toalete
violeta a gentileza de uma parisiense, e na noite negra dos seus cabelos
presos no alto, o luxo de um diadema de brilhantes.}\looseness=-1

\textls[-10]{O marido, um rapaz trigueiro, de bigode farto e escuro e a
cabeça pendida para o peito, por uma tensão dos músculos do pescoço,
deixou-a no salão e foi para a saleta do jogo, onde o comendador
conversava alto com dois amigos e correligionários
políticos.}\looseness=-1

\textls[20]{Os carros paravam à porta constantemente: entrou a
viúva Camargo, a irmã do Antunes, ainda fresca, no seu vestido
de cetim e rendas cor de saudade escura, os olhos
muito brilhantes a inundar de alegria e luz as suas faces rubicundas e
morenas; na frente, a filha, toda de claro, com a encantadora cabeça
engrinaldada de flores e o pescoço branco, roliço e nu, a surgir altivo
dentre a espuma nitente das rendas finas. Otávio acompanhou-as
até à porta da toalete, e d.\,Clarinha, já rodeada de
amigas, apontou o grupo num riso malicioso.}

\textls[10]{Entraram sucessivamente: a condessa da Fontenegra, baixa, gorda,
segurando mal o vestido de veludo, com o cabelo branco salpicado de
pedras preciosas, rodeada de sobrinhas mais ou menos bonitas; depois, as
filhas do Edmundo Queirós, a família do juiz de direito, as dos
drs.\, Lima Soares e Celestino Brandão, a do velho Torres, toda uma
população alegre e festiva; em último lugar, numa
delonga calculada para maior impressão, apareceu, arrastando
majestosamente a sua cauda de pelúcia cor de ouro, a mulher do
rico fazendeiro Borba, velho gordo de grenha ruiva a quem o povo chamava
leão de tabuleta. Entre o brilho louro do seu adereço de
topázios e diamantes, a senhora Borba erguia o colo moreno
polvilhado de \emph{veloutine}, dardejando lume dos seus olhinhos
castanhos e malignos. Uns velhos condecorados apressaram-se em ir lhe
apresentar as suas homenagens.}

Às nove horas uma das portas, até então fechada, abriu-se de par
em par e a noiva apareceu, dando o braço ao padrinho, o velho coronel
Tavares; atrás vinham Eva e Noêmia, toucadas de flores. As
roupagens alvas da noiva, como que trouxeram mais luz à sala; uma
menina, aproximando-se dela, ofereceu-lhe um ramo de rosas brancas.
Trigueirinhos, ao lado do comendador, entrara ao mesmo tempo
pela porta do fundo; parecia mais moreno, mirrado dentro da sua
primeira casaca, com o cabelo em pastinhas lustrosas sobre a testa, e um
sorrisozinho amarelo nos lábios finos. Levantaram-se todos e
aproximaram-se do altar; o padre Rocha já lá estava, paramentado e
risonho; ao pé dele, imperturbável e sério, o armador João Coelho fazia
de sacristão.

\textls[15]{A cerimônia começou e acabou sem lágrimas; ao
ajoelharem-se os noivos para a benção, rompeu a música
num cântico glorioso. Minutos depois recebia a noiva os cumprimentos do
estilo; Otávio abraçou-a e sumiu-se por entre a turbamulta da
sala, fechou-se no quarto, deu ordem ao pajem para que arreasse
o cavalo, mudou à pressa de roupa, saiu pela porta do quintal para não
ser visto pelos convidados, e atravessava direito à cocheira,
quando alguém lhe tocou no braço. Era o pai. Na escuridão da
noite o seu rosto punha uma nódoa branca, de uma palidez marmórea;
sentia-se-lhe a respiração ofegante. Estendeu ao filho, com mão trêmula
e gelada, um revólver de seis tiros; Otávio guardou a
arma silenciosamente. Estiveram um instante calados e indecisos; depois,
abraçaram-se numa efusão de ternura, única em toda a sua vida.}

\textls[15]{Otávio então tentou tranquilizá-lo, dizendo-lhe um
``até amanhã'' esperançoso; mas o velho não respondeu; apontou o
cavalo, que vinha puxado pelo pajem, e ficou-se no mesmo
sítio até Otávio partir, pondo na escuridão da noite a
mancha pálida do seu rosto transtornado.}

\textls[15]{Na sala, a noiva distribuía pelas amigas solteiras botões de
flores de laranjeira. Noêmia, muito risonha, foi mostrar a Eva o ramo
que lhe dera a irmã.}

\textls[25]{--- Veja, foi o do peito; daqui a um ano\ldots{} ---
dizia ela maliciosamente.}

--- Que é um ano, Noêmia, para quem só tem dezesseis!

\textls[20]{--- Mas com dezessete anos parece-me que se está em
muito boa idade\ldots{}}

\textls[15]{--- Você talvez nem tenha de esperar tanto; veja como o
casamento de Nicota se decidiu depressa\ldots{}}

\textls[25]{--- Mas eu não quero que o meu seja assim! Eu é que hei de
escolher o noivo!}

--- E se seu pai resolver casá-la sem a consultar?

--- Digo-lhe que não e não! Olhe, Eva, eu sou da sua opinião; um
dia você disse: prefiro morrer a casar sem amor\ldots{}

--- Estava a brincar\ldots{}

--- Não estava, não! Escute\ldots{}

\textls[-10]{Noêmia parou confusa, apertando e deslaçando a fita do leque.}

--- O quê? Continue.

--- Não, é que\ldots{} já me esqueci\ldots{} Ah! Sim! ---
E tornou de novo a parar, ajeitando desta vez as flores
do corpete.

--- Vamos, Noêmia, acabe, que lá vem o seu par.

--- Tenho vergonha.

--- Não diga tolices\ldots{} Vergonha de quem? Vá lá!

--- Pois sim\ldots{} Sabe? É que\ldots{} por que não
viria o Paulo?

Eva estremeceu; antes, porém, que respondesse, curvava-se
diante de Noêmia o par de valsa e a pobre, ainda confusa,
levantou-se, deixando o seu ramo de flores de laranjeira nas mãos da
prima.

\textls[25]{Eva fixava sem ver aquele buquê esquecido, com as faces
tintas por uma onda de sangue que lhe subira do coração ao
cérebro, impetuosamente. As palavras entrecortadas de Noêmia
tinham-lhe desvendado um segredo que não pudera nunca
suspeitar, e repetia mentalmente, em um doloroso assombro: ``Ela ama-o!
Ela ama-o!''.}

\textls[10]{O dr.\,Azevedo, encostado ao umbral de uma porta, seguia com a
vista o movimento dos pares, prendendo de vez em quando os seus olhos
azuis no formoso busto de Eva. A irmã do Antunes disfarçava mal a
indignação que lhe causava a inexplicada falta de Otávio no
salão; a filha dançava, elegante sem graciosidade, ereta, muda e séria;
aclamaram-na a rainha da noite e caíam sobre o seu vulto de estátua,
como um enxame de abelhas em uma flor, um enxame de olhares
ávidos e cobiçosos.}

O baile durou até às três horas. Nicota partiu com o noivo para a casa do major Trigueiros, onde já tinham preparados os seus aposentos;
despediu-se da família sem lágrimas, serenamente.

\textls[20]{Quando, retirados todos os convidados, Eva entrou no seu quarto,
encontrou nele, de pé, Noêmia, vendo através dos vidros fechados da
janela a sombra do último carro que se afastava; sentindo-a
chegar, estendeu-lhe os braços, apertou-a com força ao peito e desatou a
chorar.}

\textls[15]{Eva enxugou-lhe os olhos e aconselhou-a a que se fosse
deitar, para descansar da fadiga e comoções da noite; levou-a docemente
para a sua pequenina alcova, ajudou-a a despir-se, aconchegou-lhe a
roupa do leito, beijou-a carinhosamente, e saiu triste, sem ter indagado
a causa daquele pranto; adivinhava-a, estava indubitavelmente ligada
à pergunta que lhe ouvira horas antes: ``Por que não
viria Paulo?''.}

Era já muito tarde quando Eva conseguiu adormecer.

\textls[20]{Lá fora, na sala, o comendador passeava agitado,
fumando sucessivamente cigarros, indo de vez em quando à janela,
prestando o ouvido ao mais insignificante rumor, alongando a vista por
toda a extensão da rua solitária, que se ia pouco a pouco
tornando mais clara. Uma faixa de luz amarelada e frouxa batia em
diagonal nas casas fronteiras, todas fechadas e silenciosas; e nas
pedras desiguais das calçadas, aqui e ali, estavam espalhados uns
papeizinhos de cor, franjados nas extremidades,
vestígios das balas servidas em bandejas no baile.}

\textls[15]{Medeiros arrependia-se de ter deixado partir sozinho o
filho. Confiava no Antunes, que tinha boa gente e dispunha de recursos;
além de que, recomendara-lhe que levasse consigo alguns soldados: as
fardas intimidam os negros.}

Pensando nisso descansava um pouco; mas voltava-lhe
depressa o medo de que fossem daí a nada apresentar-lhe o corpo
do filho, coberto de sangue, num esfacelamento horrível!

%\end{linenumbers}

\section{XV}

%\begin{linenumbers}

\textls[15]{Deixando a festa do casamento da irmã, Otávio partiu a
todo o galope para Santa Genoveva. A noite estava escura e quente; de
vez em quando um pequeno relâmpago cortava momentaneamente a
treva. Ele ia na persuasão de encontrar tudo em boa paz; não podia
atribuir a Eva senão sentimentos generosos e puros e dispunha-se a
combater a inverossímil infantilidade do pai no tocante ao terror que
lhe inspirava a sobrinha. Mas combater de que maneira? Provando pela
evidência dos fatos a sua inocência. Anelava por
esclarecer tudo, narrando ao pai o que ouvira de Morton, para
convencê-lo de que não estava em poder de Eva a carta causadora
de tantos sobressaltos.}

\textls[15]{Confiava em que, desaparecido o receio de ver surgir a qualquer
instante uma prova contra a sua honra, desaparecesse também a
desconfiança do seu espírito e o rancor do seu coração. O
respeito que tinha pelo pai dificultava-lhe a ação. Faltava-lhe a
coragem para a luta, temendo sempre ser cedo para ferir, diretamente,
tão grave assunto. Contava estabelecer a harmonia na família,
sem que o pai suspeitasse sequer, que ele, seu filho, conhecia
semelhante segredo.}

\textls[10]{Quando chegou à fazenda eram onze horas; apeou-se
longe, no pasto, prendendo o cavalo ao varal de um carroção; desceu a
pé, atônito por encontrar todas as porteiras abertas.}

\textls[-10]{Deu volta por detrás das senzalas, caminhando com
precaução; assim chegou ao fundo da casa principal; ia rodeá-la
quando, ao aproximar-se de uma esquina, ouviu vozes de homens que
falavam baixo do outro lado. Otávio, encostando-se à parede,
escutou:}\looseness=-1

--- Tá tudo pronto? --- perguntava um deles. \label{Ref8}

--- Tudo\ldots{} --- respondia outro.

\textls[-10]{--- Os negro da casa tão no
curro?}\looseness=-1\footnote{Curro: assim chamam à reunião das senzalas, que também
  denominam ``quadrado'' e ``quartel''.} \textls[-10]{Carece de sabê bem isso\ldots{}}\looseness=-1

--- Decerto estão.

--- Arrepare, Joaquim! É perciso fazê as coisa
cum jeito\ldots{}

\textls[15]{--- Tá dereito, nhô Furtuoso! Eu sou um decidido! Logo pego,
arrombo num baque}\footnote{Num baque: num momento.} \textls[15]{as
porta do quadrado e ponhamo-nos a fazê baruio.
Sai cinza!}

--- Que confusão danada! Mecês tudo não se esqueça de dizê
qui foi d.\,Eva qui mandou nóis cá\ldots{} sinão, os negro nos
arresiste!

--- Entonces, quem si havéra de si esquecê
disso?!

--- Si arguém si quisé fazê de tolo, já se sabe, tiro pra
frente! As almas vieram carregada por mó de isso.

--- Tá bom, gentes\ldots{}

Otávio ouvira palavra por palavra, cheio de espanto.

Era então verdade! Eva descia odiosamente a uma baixeza
daquelas!

\textls[15]{Ruía o pedestal em que a colocara superior a todas as mulheres;
a estátua da compaixão, a mártir, como a supusera, patinhava
descalça e desonesta na lama de uma vingança torpe.}

\textls[15]{Urgia não perder um minuto e evitar a revolução. Otávio
procurava um meio de se reunir aos de casa, sem atrair a atenção dos
malfeitores.}

\textls[10]{Passou-lhe pelo espírito uma multidão de alvitres,
 cada qual mais confuso e mais disparatado. Felizmente, os
 capangas estavam do outro lado da casa e ele poderia voltar à
 frente sem ser visto e chamar em seu auxílio o feitor. Assim
 fez; caminhou com precaução, amparando-se à parede, até perto da torre
 do sino; aí parou. No terreiro da frente caminhava também um
 grupo de homens, falando em segredo; passaram a pouca distância
 de Otávio, num rumor soturno de passos e de vozes que a pouco e
 pouco se foi perdendo à distância\ldots{} Não os
 ouvindo já, Otávio galgou de um salto os seis degraus da escada
 de pedra e chegou ao patamar; bateu com força na janela do feitor,
 dizendo alto: ``Sou eu, Otávio, abra depressa!''. Fora
 sentido fora e dentro da casa ao mesmo tempo; mas, no momento
 em que o feitor, pálido e trêmulo, lhe abria a porta, um tiro disparado
 de entre a escuridão da noite cravou-lhe nas costas uma bala.} \label{Ref9}

\textls[-20]{Otávio caiu de bruços, para dentro, indo bater com o peito no
chão; a mulher do feitor puxou-o, arrastando-o, para o corredor,
enquanto o marido fechava violentamente a porta; depois,
levaram-no para o leito, onde ele ficou estendido, inerte, manchado de
sangue.}\looseness=-1

\textls[20]{A revolta rebentara lá fora; ouviam-se gritos,
imprecações e tiros, estalos de portas arrombadas e vozes aflitas. Sem 
cuidarem de Otávio, na alucinação do medo, sem mesmo saberem se
estava vivo ou morto, correram o feitor e a mulher a colocar móveis
junto às janelas e à porta.}

\textls[20]{Que iria fazer lá fora o feitor? Que era um homem para
uma revolta de cem? Como o tratariam tantos e tão encarniçados inimigos?
Agora não eram os mesmos cordeiros, aos quais ele distribuía a esmo
chicotadas; não eram brutos irracionais, sem dignidade e sem coragem:
eram homens enraivecidos e capazes de tudo para satisfazerem os seus
fins.}

A gritaria cessou depois de uma hora infernal. De vez
em quando batiam na porta e atiravam pedras aos vidros, que se quebravam
em estilhaços. O feitor colava-se, encolhido, a um canto; a mulher
arrastava-se de joelhos, a rezar, batendo no peito grandes pancadas, com
a mão curva. Tinham-se refugiado numa alcova interior, quase
sem ar, fechada hermeticamente, mal alumiada pela luz mortiça e trêmula
de uma lamparina, em frente ao oratoriozinho de madeira que tinham sobre
a cômoda. No seu trono enfeitado de hortênsias, de
lírios e de cravos, a Virgem das Dores, com uma
lágrima cristalizada na face impassível e o
seu manto de veludo orlado de renda dourada, caindo-lhe da cabeça aos
pés numa volta muito redonda, a Virgem das Dores, com um punhal cravado
no peito, a brilhar de vez em quando aos lampejos intermitentes da luz,
era entre as sombras frias daquele quarto mofado e úmido, a
única esperança daquela pobre gente.

O feitor tremia, encolhido e cor de cera; a
mulher, com a fisionomia alterada, levantava para a serena imagem as
mãos súplices. Houve um momento em que o marido, abandonando o canto em
que se colocara, foi ajoelhar-se em frente ao oratório,
acompanhando alto um Padre-nosso, dito nervosamente por ela.

À proporção que diminuía lá fora a gritaria, iam eles
melhorando de ânimo; e quando pelas frinchas das portas entrou a luz do
dia numas estreitas fitas esbranquiçadas, foi que se atreveram a seguir
pelo corredor até o quarto da frente, em que Otávio tinha ficado
abandonado, estendido sobre uma cama sem lençóis.

Eram cinco horas da manhã, quando a voz do compadre Antunes
gritou de fora:

--- Seu Honorato, abra a porta! É gente de paz.

\textls[-30]{A porta abriu-se e o feitor viu os escravos da casa, amarrados e
submissos, numa longa fila silenciosa, cercados pela gente do Antunes.}\looseness=-1

À frente deles, o compadre de Medeiros, com um brilho singular
nos olhinhos pardos, bamboleava o corpo, bufando alto.

--- Mande dar uma pinga a essa gente, homem! --- disse
ele apontando para os seus capangas, logo que viu o feitor.

\textls[15]{--- Tá dereito! --- respondeu-lhe este, ainda muito trêmulo. ---
\emph{Mais} venha primeiro cá dentro\ldots{}}

Entraram ambos para a sala, ainda às escuras. Abriram
as janelas à claridade e ao ar fresco da manhã.

--- Então que há? --- perguntou o Antunes.

--- Há que seu Otávio levou um tiro de garrucha.

--- Seu Otávio! Pois ele não está na cidade?!

--- Não senhor, está aqui.

\textls[15]{Antunes tornou-se lívido, vergaram-se-lhe os joelhos e uma
vertigem escureceu-lhe a vista; pouco depois, voltando à
costumada energia, disse ir ele próprio buscar um
médico.}

--- Médico\ldots{} Para quê? Seu Otávio morreu.

\textls[-15]{Convidou-o para ir ver o cadáver; mas Antunes,
de olhos esgazeados, negou-se a isso, ficando hirto e imóvel no
mesmo lugar.}\looseness=-1

\textls[15]{O feitor deixou-o, e tirando um crioulo da turma, mandou-o
distribuir aguardente aos capangas.}

A emoção tinha vencido toda a atividade do Antunes.
Conhecera Otávio ainda pequenino, sempre tivera
predileção por ele. Entre todos os rapazes distinguia-o sempre, e
acariciava a ideia de o ver casar com a sobrinha. Aquela morte
era um desastre para a sua família, era talvez a ruína
dele, Antunes, que projetava ligar o moço Medeiros aos seus interesses
particulares. Estava todo entregue a essas reflexões, quando ouviu
gritos dolorosos, numa voz tremida de mulher e a bulha de um corpo que
se arrastava no assoalho. \label{Ref1} Voltou-se: era a mamã de Otávio, a
pobre negra velha, que ia como um réptil, com o ventre
encostado no chão, as pernas mirradas pela paralisia, pondo nas mãos
engelhadas toda a sua atividade, estendendo ora um braço, ora outro,
encarquilhando os dedos nas gretas das tábuas, quebrando as unhas,
rasgando as roupas, arranhando as carnes, arfando de cansaço, mas
prosseguindo no esforço até à porta do quarto do ferido, que
ela abriu com a cabeça, num desespero angustioso.

Mas o corpo do moço estava numa cama alta e a desgraçada não se
 podia erguer para beijá-lo e vê-lo; extenuada, pôs-se
 a chamá-lo com uma voz dolorida e débil, levantando o rosto
 lacrimoso para o leito a que não conseguia chegar. \label{ref10}

--- Meu fio!\ldots{} Meu fio!\ldots{}

\textls[15]{Acordara naquela madrugada ouvindo bulha, e pedira a
sua companheira, velha também, a Joaquina, que fosse
indagar de que era proveniente aquele estrondo. A Joaquina saiu, para
voltar com a notícia da morte de Otávio e da prisão do
Jacinto, único filho da pobre paralítica. Ela então, louca de
dor, não quis ouvir detalhes e suplicou à Joaquina que a levasse
para junto deles, a outra fez-lhe a vontade; não podia
carregá-la nos braços com meiguice, puxava-a como quem puxa um
fardo, fazendo na terra uma esteira, varrida e rasa. Pouco tinha andado
assim, quando lhe veio à lembrança o carro de mão com que o
jardineiro trabalhava; foi buscá-lo, sentou nele com
dificuldade a inválida e conseguiu impeli-lo até à
casa-grande.}

\textls[15]{No terreiro viram, logo ao chegar, o Jacinto amarrado a um
companheiro. A mãe quis aproximar-se dele, mas não lho permitiram; ela
ficou então de longe a acenar-lhe, chorando, enquanto ele partia com a
turma para a fazenda do Antunes.}

O tronco de Santa Genoveva não tinha espaço para tanta gente; o
Antunes então deliberara recolher parte dela na sua fazenda, que por
amizade a Medeiros arvorava em prisão.

\textls[20]{Mas não era ao filho preto, das suas próprias entranhas, a quem
 a paralítica queria mais. Entre os dois, preferia o branco, nascido dos
 seus senhores, a quem amamentara e cuja importância e beleza a enchiam
 de orgulho. E nessa preferência não demonstrava Joana nenhuma
 originalidade. Não sentiam todas as outras mamãs do mesmo modo?} \label{ref11}

A razão dessa absurda anomalia residia talvez no
próprio horror da escravidão. Vendo nos filhos entes destinados
a uma condição indefensável, as pretas fugiam inconscientemente de lhes
dedicar muita ternura. Olhavam para os filhos pequenos como para futuros
parceiros, sujeitos a todos os vilipêndios e que o destino
afastaria para longe, quando bem lhe aprouvesse\ldots{}

\textls[-15]{A Joaquina tinha tirado a inválida do carro e subira arfando de
cansaço com ela ao colo os seis degraus da casa-grande, quando recebeu
ordem do feitor de descer. Pousou então a outra no corredor da entrada e
voltou. Vendo-se só, a paralítica não desanimou e conseguiu arrastar-se
até ao quarto do seu adorado Otávio, rasgando os trapos
da saia sob os joelhos, quebrando as unhas nas tábuas, magoando
os ossos e as carnes murchas em sucessivos trambolhões.}\looseness=-1

\textls[10]{Aos gritos lamentosos da velha, acudiu a mulher do feitor; a
preta suplicou-lhe que a pusesse por um momento em cima, que ela ficaria
aos pés do leito, encolhida, imóvel e calada, contanto que
visse seu filho: queria vê-lo a todo transe! Sentaram-na no
cantinho, entre a parede e os pés de Otávio, pés que ela
acariciou e beijou, chorando baixo; depois arrastou-se até pegar-lhe na
mão, sacudindo-a de manso, e repetindo baixinho, como para
acordá-lo sem sobressalto: ``Meu fio\ldots{} meu fio\ldots''.
Ele não se moveu e ela arrastou-se mais; beijou-lhe a testa, as
pálpebras fechadas\ldots{} afagou-o muito e inclinou a cabeça,
escutando-lhe o coração; esteve assim um minuto, com os olhos muito
abertos e a respiração cortada; depois, um sorriso pousou-lhe nos
lábios engelhados, fulgurou-lhe o olhar e soltou um grito de
júbilo, um grito cheio de vida, cheio de amor, vibrante de sentimento,
grito de mãe ao ver tornar a si o filho idolatrado.}

--- Meu fio tá vivo! Tá vivo! Gente!

\enlargethispage{1\baselineskip}
\textls[-15]{Ouvindo estas palavras, Antunes entrou no quarto e aproximou-se
de Otávio. Escutou-lhe também o coração, apalpou-o
nervosa, ansiosamente e, sem dizer nada, saiu apressado, com o olhar
fulgurante e incerto. O seu cavalo estava arreado, montou-o e seguiu a
galope para a fazenda do Navarro, a um quarto de légua de Santa
Genoveva. Sabia estar lá, havia tempo, um dr.\,Castro, médico de
Pernambuco, homem, segundo corria, de reflexão e de estudo.}\looseness=-1

Em poucos minutos batia à porta do quarto do médico.

\textls[15]{O Antunes explicou mesmo do lado de fora o que havia, exigindo
pressa, muita pressa!}

\textls[15]{O médico prescindiu do seu banho de chuva, vestiu-se
sem apuro, respondendo às instâncias do Antunes com um: ``Já vou\ldots{}
já vou'' muito repetido; e aconselhava, lá de dentro, que mandasse
arranjar a condução.}

\textls[-25]{Enquanto o Antunes, muito aflito, via selarem o cavalo do
médico, procurando, agitadamente, ajudar os pajens, a
mulher do Navarro, já de pé, sinceramente interessada pela sorte de
Otávio, mandava ao quarto do doutor, sem demora, o copo de leite
e a xícara de café.}\looseness=-1

\textls[15]{Dr.\,Castro engoliu, sem as saborear, as duas bebidas,
guardou na algibeira o estojo indispensável, cumprimentou a dona da
casa, alisando com as mãos morenas as guias do seu farto bigode negro e
reluzente, e desceu para o terreiro a pensar, satisfeito, na proveitosa
aquisição de tão bom cliente.}

\textls[20]{Depois de o ver partir para Santa Genoveva, Antunes
seguiu até à cidade, esporeando o animal, já cansado.}

\textls[15]{Chegou à casa do comendador às seis horas; achou-o só, no grande
salão iluminado pela luz doce da manhã, cheio de
flores enlanguescidas e murchas, alguns doces
pisados, velas quase gastas e cadeiras em desordem.}

\textls[-15]{Medeiros estava pálido e tinha nos olhos os vestígios
da insônia. Vendo-o, o Antunes recuou indeciso e ficou entre portas. O
fazendeiro, avançando para ele, perguntou, quase sufocado pela
angústia:}\looseness=-1

--- Que houve?!

\textls[20]{Antunes não pôde responder; pálido e trêmulo encostou-se ao
umbral.}

--- Otávio?

--- Está\ldots{}

--- Morto?!

--- Ferido\ldots{} Vá para ao pé dele, compadre; vá quanto
antes; tem aqui o seu trole e o pajem; não se demore,
vá, vá, depressa!

--- E Eva?

--- Depois tratará disso; vá para junto de Otávio.

\textls[-20]{Esta insistência desnorteava o comendador. O Antunes
levou-o consigo ao quintal, bateu à janela do quarto do
pajem, apressou-o a arranjar a condução. O mulato Saturnino,
desembaraçado e obediente, aparelhava os animais; no entanto, o
comendador ralhava, chamando-o moroso. O Antunes não resistiu: supondo
ser mais expedito, afastou o rapaz e pôs-se ele, muito nervoso,
a arrear as bestas.}\looseness=-1

\textls[25]{Momentos depois o fazendeiro partia para Santa Genoveva,
despedindo-se acabrunhado do amigo.}

\textls[15]{Quando, chegado à fazenda, entrou no quarto do filho,
o médico tinha acabado de extrair-lhe a bala de sob a
clavícula direita, onde a procurara logo, em direção ao
ferimento das costas, e onde a achou, saliente como um pequeno tumor.}

\textls[20]{--- Ah! Doutor! Salve o meu filho! --- foram as suas primeiras
palavras.}

\textls[15]{--- Esteja descansado; o sr.\,Otávio viverá largos anos:
por um triz estaria morto há algumas horas\ldots{} agora serei
assíduo e procurarei, por todos os meios ao meu alcance,
pô-lo bom. É preciso cuidado.}\looseness=-1

Amaneirando o tom, continuava:

\textls[-15]{--- A bala deve ter atravessado o ápice do pulmão direito e daí
provirá naturalmente uma pneumonia traumática; terá febre, e,
para evitar as hemoptises, convém guardar o mais absoluto silêncio.}\looseness=-1

Medeiros olhava-o sem compreender e rogava ao médico
que se instalasse em Santa Genoveva e não deixasse o doente.

\textls[15]{--- Virei todos os dias mais de uma vez: estou a dois passos;
além disso tenho mais doentes, e embora não lhes dedique o mesmo
cuidado, não posso abandoná-los, bem vê\ldots mas esteja certo,
que hei de vir uma ou duas vezes por dia\ldots{} Hoje fico, passo com
ele a noite, apesar de outros compromissos\ldots{} Agora o que
lhes peço é cuidado. Sua senhora?}

--- Ignora tudo\ldots{}

\textls[15]{--- É preciso preveni-la; as mulheres sabem lidar com
os doentes, com mais delicadeza que nós.}

\textls[20]{Medeiros deu ordem para que fossem buscar a
família, disposto a aconselhar Eva a que se retirasse nesse
mesmo dia para o Mangueiral. Voltando para o lado de Otávio,
este quis falar-lhe, mas o dr.\,Castro proibiu-o disso e apresentou-lhe a
carteira aberta e um lápis; o doente, trêmulo, escreveu:}

\textls[15]{``Proteja Eva, rogo-lhe por\ldots'' O lápis
resvalou-lhe de entre os dedos e ele ficou de olhos fechados e
silencioso.}

\textls[20]{O comendador leu, franziu as sobrancelhas e, avermelhando-se, ia
responder zangado, quando o médico, notando a sua
cólera, lhe disse baixo:}\looseness=-1

--- Não o desgoste! Uma contrariedade pode matá-lo!

\textls[20]{O fazendeiro disse então alto para o filho, de um modo
visivelmente contrafeito:}

--- Farei a vossa vontade!

Otávio sorriu. Deixando-o em companhia do médico, o
fazendeiro indagou do feitor, fora, na sala, dos sucessos da
noite.

\textls[20]{Honorato, confuso, contou-lhe, repetindo certos pontos, que
às duas horas da noite ouvira Otávio chamá-lo,
batendo com força na porta. Aconselhado pelo Antunes, ele, Honorato, não
tinha saído até aí e estava resolvido a não aparecer: os negros
odiavam-no e eram capazes de matá-lo.}

Medeiros, impaciente, ordenava que prosseguisse.

\textls[20]{Às duas horas em ponto, ouvira a voz de Otávio; e como
estivesse perto e com o ouvido à escuta, acudiu logo ao chamado; ao
tempo em que abria a porta, partia um tiro do terreiro e o moço caía
para a frente. Conduziram-no para o quarto mais próximo;
ouviu-o gemer por algum tempo, depois, como tivesse emudecido, tanto ele
como a mulher supuseram-no morto\ldots{} Às cinco horas chegava
o compadre Antunes e pouco depois Otávio dava sinais de vida.}\looseness=-1

\textls[15]{Quando o doutor chegou, ele já estava de olhos abertos, gemendo
baixinho, afagado pela Joana paralítica.}

\textls[-20]{Referiu-se sem comoção à velha, que mandara recolher à senzala,
por conselho do médico; a pobre mamã, cansada da luta, num
esgoto nervoso, como tinha dito o dr.\,Castro, ficara com os braços
inertes, sem ação, completamente paralisados como as pernas, tanto que a
levaram para baixo ao colo, deitada, como se fosse uma
criança!}\looseness=-1

\textls[15]{Medeiros deu tréguas a outras investigações; já não se ouviria
de manhã o sino, chamando a gente ao trabalho; os campos e os
cafezais pareceriam abandonados; as senzalas fechadas desde a
manhã até à noite e os carreadouros desertos.}

Antes da tarde, chegava toda a família; Trigueirinhos e
Nicota demoraram-se pouco. O comendador, desviando o olhar da sobrinha,
explicou muito por alto a revolta dessa noite.

\textls[-25]{O genro soltava exclamações de rancor e de espanto; Nicota, com
a cabeça pendida, enrolava maquinalmente nas mãos as fitas da sua capa
de viagem; Noêmia chorava; Eva, muito pálida, escutava-o de pé,
com as narinas palpitantes e os olhos fixos; e à cabeceira do
doente a mãe, resignada, suspirava de vez em quando, com
tristeza.}\looseness=-1

\textls[-15]{Acabada a exposição de Medeiros, o médico pediu uma
enfermeira, de entre as três senhoras. Eva ofereceu-se, mas o
fazendeiro replicou secamente ser melhor incumbir desse cargo a
mãe e a irmã do doente. Um gemido de Otávio interrompeu
a discussão, e Eva saiu perturbada. Antes de entrar no seu quarto,
passou pelo de mme.\,Grüber, e, entrando, deixou-se cair sobre a
poltrona azul. O coração pesava-lhe; pressentia qualquer coisa de
terrível que a ameaçava de perto e não podia compreender o que
fosse! Que significaria aquele movimento do tio? Por que a
repeliam, a ela, que não lhes fizera mal e que se tinha sempre submetido
tão placidamente aos seus costumes e vontades? Passou todo o resto do
dia em cogitações dolorosas; à noite voltou de novo ao quarto da alemã,
abriu a janela à solidão do campo e um bom ar fresco encheu o aposento;
arranjou e acendeu o lampião, trouxe para a mesa os preparos de
escrita e começou, com a sua letra firme, uma carta à mestra. Sentia
mais que nunca a falta dessa boa senhora de espírito, com quem trocava
ideias em doce intimidade e por quem se sabia compreendida e
amparada nas suas horas mais desanimadoras. De vez em quando Eva
atentava o ouvido, a escutar os gemidos de Otávio, e tornava
depois a ranger a sua pena no papel acetinado, branco como a
sua consciência e em que ia derramando toda a alma, assustada
qual pássaro solto na amplidão dos mares, sem um rochedo onde
se aninhe ou um mastro onde pouse!}\looseness=-1

%\end{linenumbers}

\section{XVI}

%\begin{linenumbers}

\textls[-10]{No dia seguinte, o dr.\,Castro fez o primeiro curativo ao ferido,
chamando para seu lado as duas enfermeiras apontadas pelo comendador;
mas logo ao desligar das ataduras, ao primeiro ``ai!'' do doente, a
mãe, quase desfalecida, confessou a sua
incompetência e retirou-se com as mãos nos olhos; Noêmia era
nervosa, estremecia sem atinar com as coisas. O médico então,
voltando-se para o fazendeiro, pediu-lhe que mandasse chamar outra
pessoa. Medeiros levantou-se e foi buscar a mulher do feitor, Nhá
Colaca (diminutivo de Escolástica); mas a infeliz era desajeitada e o médico,
impaciente, mandou-a embora, lembrando o nome de Eva. O fazendeiro quis
opor-se; mas recebendo em resposta que de uma boa enfermeira
depende muitas vezes a cura de um doente, resignou-se.}\looseness=-1

\textls[10]{Quando Eva entrou no quarto, Otávio, com o ombro
descoberto, o rosto afogueado pela febre, os cabelos colados em grandes
anéis na fronte, olhava para os que o cercavam sem os conhecer.
Ela, aproximando-se, contemplou-o magoada e com
jeitoso carinho auxiliou o dr.\,Castro a aplicar-lhe sobre a
ferida aberta nas costas pela bala, até à feita na
clavícula para a extração, uma fina pasta de algodão fenicado,
passando depois uma longa atadura encruzada do peito para as costas.}

Era aquela a única missão que lhe permitiam agora! Durante o dia
e a noite zelariam outras pessoas o doente, vedando-lhe a ela a porta o
braço implacável do tio. Desgostava-se com isso: quereria estar ao lado
do primo. À compaixão juntava-se o medo de que morresse aquele em quem
pressentira a sinceridade de um amigo; ele tinha no
gênio pontos de contato com o seu, e muita analogia com o do pai
da pobre Eva; era a única pessoa da família que a
compreendia, o único que se fazia entender por ela,
manifestando as mesmas tendências e os mesmos gostos.

O dr.\,Castro declarara ser muito grave o estado de Otávio, que
expectorava sangue e ardia em febre.

\textls[15]{Sentada à cabeceira do doente e ouvindo-lhe as extravagâncias
do delírio, a moça sentia cravados nela os olhos do tio, que a
vigiava do fundo da alcova. Dominando a sua irritação, ela fingia não
dar por isso, curvava-se aflita para as gargalhadas do doente que falava
em cartas de jogar e num café de Berlim, onde uma mulher caíra com um
ataque e um sujeito gordo lhe despejara em cima um copo de cerveja\ldots{}}

Dr.\,Castro não se arredava para longe.

Volta e meia visitava o enfermo e fazia novas recomendações à
família; até que uma noite a febre baixou e ele dormiu. Quando
acordou, viu a seu lado, dormitando, a mãe e uma
crioula em pé, com os braços cruzados; pediu água; tinha uma secura
horrível, bebeu-a sofregamente e adormeceu de novo. Passava do sono à
vigília continuamente, molhando todas as vezes os lábios
queimados e a garganta seca; de madrugada sentiu-se melhor, e passou
pela memória enfraquecida todo o ocorrido. Anelava por saber de
Eva; tê-la-ia despedido o pai? Quis perguntar por ela, mas a
mãe dormia na cadeira, numa posição contrafeita, e a crioula
cochilava mesmo em pé. Às sete horas da manhã o pai e Noêmia
entravam no quarto, mas Otávio, exausto, adormecera de novo,
profundamente. Às nove horas tornou a acordar, deparando com o dr.\,Castro e Eva, que o fitava condoída e triste; sorriu ao vê-la; depois,
ouvindo um rumor de vozes desconhecidas na sala próxima, perguntou com a
voz sumida pela fraqueza:

--- Quem é?

--- A polícia --- respondeu o doutor.

--- Para fazer corpo de delito, não é assim?

\textls[15]{--- É. Mas não fale. Esteja sossegado; precisa de
silêncio, não diga senão o que for indispensável.}

\textls[15]{Eva ofereceu um caldo ao doente, que pela primeira vez se
sentou, amparado pelo médico.}

\textls[-17]{Lá fora o comendador respondia ao delegado de
 polícia, que viera bater-lhe à porta nessa manhã, com o
 seu escrivão, o Fonseca, moço magrinho e louro. Tinha-se espalhado por
 toda a cidade a notícia do ferimento de Otávio; e
 sabendo do fato, a polícia não deixou de se apresentar, sem
 ter, contudo, tido para isso aviso especial.}\looseness=-1 \label{Ref13}

\textls[20]{O delegado, homem experiente e idoso, não estranhou
essa falta; conhecia de sobejo o sistema preferido dos antigos
fazendeiros para a punição de escravos assassinos\ldots{} Durante toda a
viagem no trole fora ele explicando ao escrivão, ainda novo no
ofício, como se fazia geralmente essa história:}

\textls[15]{--- Há um assassinato numa fazenda --- dizia.
--- Bem! Que se faz do assassino? Entregam-no à prisão
para que siga os processos legais da justiça? Qual! Querem vingança mais
completa. Apoderam-se da presa, escondem-na no quarto lúgubre
do tronco, e, conforme a categoria da vítima, castigam-na
severamente, procurando aproveitar-lhe o trabalho, mas carregando-a de
ferros. Assim, matam-na a fogo lento.}

\textls[20]{Dá-la à decisão de um júri seria mandá-la às galés, e as galés
para um cativo são a liberdade. Não seria punição, bem vê; seria um
prêmio!}

--- Nesse caso não competiria à justiça intervir
seriamente no crime? --- perguntou com ingenuidade o moço
escrivão.

\textls[15]{--- Oh! A justiça é muitas vezes iludida e\ldots{} e muitas
vezes também há uma série de conveniências
que a obrigam a fechar os olhos. Que remédio!}

Como o Fonseca demonstrasse surpresa, o delegado, retorcendo as
suíças grisalhas, continuou:

\textls[15]{--- É bom notar uma coisa, que até certo ponto atenua a
gravidade do delito: muitos negros matam sem ódio,
praticando as mais absurdas vilanias com o único intento de
irem para as galés. Desaparecendo essa esperança\ldots{}}

--- Diminui o número de crimes de tal ordem,
por certo --- interrompeu o escrivão.

--- Homem\ldots{} não sei\ldots{} mas é possível!

O diálogo suspendeu-se à porta de Medeiros.

A polícia foi recebida sem constrangimento.

\enlargethispage{2\baselineskip}
\textls[-10]{Otávio dormia ainda, e o seu médico assistente fez, a pedido do
delegado, por escrito, uma exposição sucinta do ferimento. Foi escusado
o exame no paciente, à vista das autoridades, e mesmo o delegado,
poupando trabalhos incomodativos, dizia mandar assinar por um segundo
médico o relatório feito pelo dr.\,Castro\ldots{}}\looseness=-1

Era preciso prevenir Otávio para um
interrogatório, e, enquanto o médico e Eva o
fortaleciam com o alimento, aconselhando-o a que respondesse com poucas
palavras para se não fatigar, Medeiros expunha ao delegado: que sabia,
por um aviso anônimo, ter de se dar nessa noite, motivado por
uma pessoa que até aí julgara livre de toda a suspeita, um levantamento
em Santa Genoveva; que pedira ao seu amigo Antunes conselho e auxílio,
visto estar nessa ocasião preso na cidade com o casamento da filha; que
dissera tudo a Otávio, sabendo só à última hora da
resolução que este tomara de se apresentar no sítio, sendo por
isso baldados todos os meios para detê-lo em casa, etc.

O delegado pediu-lhe então a carta anônima, que havia de
esclarecê-lo, desde que visse nela o nome do amotinador.

\textls[15]{Medeiros levantou-se e foi direto à
secretária. Um sentimento de vingança punha-lhe lume nos olhos.}

Abateria enfim o orgulho de Eva! Ela seria em breve
desmascarada pela voz insuspeita da autoridade. Vê-la-ia
ajoelhar-se suplicando perdão para a sua falta, num grande temor de que
toda a gente a apontasse depois como intrigante e aliciadora.

\textls[20]{Choraria com certeza muito e, antes da noite, havia de partir
para o Mangueiral, arrependida e humilhada.}

O delegado esperava recostado no sofá, sumindo os dedos nas
fartas suíças grisalhas. Fonseca, o escrivão, diante da pasta
aberta, estava pronto no seu posto para a primeira ordem.

\textls[15]{Medeiros introduziu a chave na gaveta da secretária e abriu-a
com resolução. Guardara ali a carta entre outros papéis de importância e
ia lançar mão dela, quando deparou com a folha da carteira do dr.\,Castro, 
em que a mão enfraquecida de Otávio escrevera: ``Proteja Eva;
rogo-lhe por\ldots{}''.}

\textls[15]{Todo o ódio, todo o desejo de vingança se retraíram
à vista dessas palavras que envolviam uma súplica.}

O comendador voltou confuso, afirmando ter perdido a carta.

O delegado olhou de soslaio para o escrivão e, sob o fino
bigodinho louro deste, despontou, disfarçado e contrafeito, um sorriso
de irônica incredulidade. Dr.\,Castro veio nesse momento dizer-lhe que
Otávio estava pronto para o interrogatório, e seguiram
todos para o quarto. Instalados aí e preenchidas as indispensáveis
formalidades e usuais perguntas, a que o doente respondia por
monossílabos --- sim\ldots{} não\ldots{} --- o delegado perguntou:

--- O senhor viu quem o feriu?

--- Não.

--- Nem o suspeita?

--- Não.

--- Tem algum inimigo?

--- Creio que não\ldots{}

--- De qual dos escravos desconfia?

--- De nenhum.

--- Mas naturalmente terá entre eles algum desafeiçoado\ldots{}

\textls[-30]{--- Não foi um escravo quem me feriu. Olharam-se todos
atônitos.}\looseness=-1

\textls[20]{--- Bem! Nesse caso --- continuou o delegado
---, o sr.\, sabe qual foi a pessoa que disparou o tiro; por que
disse então que a não tinha visto?}

--- Porque a não vi.

\textls[-10]{Mostraram-se admirados; o doente declarou entrecortadamente não
querer ser parte no processo; os negros de Santa Genoveva eram
inocentes; haviam sido incitados por estranhos; a pessoa que o ferira
não o podia ter reconhecido no meio da escuridão. E contou, suprimindo o
nome de Eva, tudo o que tinha ouvido.}\looseness=-1

Falava devagar, parando de vez em quando, encostado em
almofadas.

O médico aconselhava descanso.

\textls[15]{A pena do escrivão traçava as palavras do doente, e acabadas as
declarações leu-as em voz alta. Assinaram como testemunhas o Honorato e
o médico.}

\textls[15]{Medeiros indagou depois, em particular, se iriam proceder ao
interrogatório dos escravos.}

\textls[15]{--- Por ora não --- respondeu o delegado ---;
mas há de dar-me os nomes das testemunhas do fato, para a seu
tempo serem chamadas a juízo, na casa da câmara.}

\textls[15]{Aceitando o almoço, em que os pratos paulistas se sucediam
abundantemente, o delegado inquiria, pela segunda vez, o feitor, de todo
o sucedido.}

O fazendeiro, voltando-se para a esposa, recomendou-lhe:

--- Olha, manda servir a ordenança aqui do sr.\,delegado\ldots{}

\textls[15]{Ela levantou-se, fazendo um sinal afirmativo. Encheram-se os
copos e a conversa animou-se. Até a nhá Colaca confessou ter feito 
uma promessa ao Bom Jesus de Pirapora
para o restabelecimento de Otávio.}

\textls[20]{A promessa constava em ir o moço a pé e descalço, no dia 6 de
agosto, festa do milagroso santo, até o pequeno arraial, daí a muitas
léguas, onde a multidão leva as suas preces e de onde
traz relíquias sagradas e benfazejas.}

\textls[15]{O Bom Jesus acode a todas as tormentas, desde que lhe levem ao
seu rico altar de Pirapora uma oferenda qualquer. No dia 6 de agosto a
povoação transforma-se! É verdadeiramente assaltada por ondas humanas
que afluem de todos os lados da província. Não bastam as
hospedarias; ficam transbordantes as casas particulares e a
própria igreja, onde muitos romeiros passam a noite, à falta de
outra pousada.}

\textls[-15]{A fé, o doce bálsamo para as grandes dores, e o
cumprimento de uma dívida sagrada, impele para aquele cantinho de
milagres os crentes e os aflitos. O mais interessante, porém, o
verdadeiramente característico, é que muita gente vai pagar promessas
que não fez.}\looseness=-1

Acontece frequentes vezes estar um indivíduo qualquer
gravemente doente; então alguma pessoa da família ou de amizade
levanta os olhos ao céu clemente e, num intuito piedoso,
promete que, em estando completamente restituído à vida e à
saúde, levará o enfermo, pisando descalço as pedrinhas da
estrada, um feixe de lenha às costas, ou uma perna de cera, ou uma talha
de água, ou qualquer coisa que o humilhe e o alague em suor, ao
misericordioso, ao poderoso, ao todo justo Bom Jesus de
Pirapora. Pelo caminho encontram-se grupos curiosos: ora umas
mulheres que se arrastam de joelhos, ora outras com pedras à\linebreak

\pagebreak
\noindent{}cabeça.
Homens esfalfados de cansaço, conduzindo pela arreata os animais bem
ajaezados. De longe em longe brilha num ponto e noutro da estrada uma
luzinha: é um círio que uma devota leva.

\textls[15]{Acabada a festa e findo o sacrifício, voltam tranquilos
para as suas casas, na doce alegria, na imperturbável serenidade que dá
a consciência de um dever cumprido. Como preservativo de futuras
doenças, trazem consigo, em cadarço branco, a medida da cabeça, do
ventre ou das pernas da imagem do Senhor Bom Jesus. Desde então não
há pessoa amiga que, sofrendo da garganta, de reumatismo, ou de
qualquer incômodo físico, enfim, não ponha sobre o
lugar afetado a medida correspondente, como a melhor e a mais
benéfica das panaceias.}

\textls[15]{Nhá Colaca não declarou em que consistia o sacrifício
destinado a Otávio. Era um segredo que só ao doente deveria ser
revelado, sob pena de perder a virtude.}

Ela adorava o Senhor Bom Jesus, do qual tinha um \emph{registro}\footnote{Na edição de 1919,
a grafia dessa palavra em itálico, sem nota de rodapé, chama a atenção para o sentido de 
``registro'' como a imagem de um santo de devoção ou culto. [\textsc{n.\,e}.]}
na parede do quarto. Logo de manhã, o seu primeiro olhar era para o
santo, muito resignado, de tanga, capinha curta e mãos atadas. Ela
benzia-se, e agradecia-lhe do fundo da sua boa alma o ter-lhe concedido
a graça de ver ainda a luz desse dia.

\textls[15]{Depois do almoço as autoridades policiais partiram, debaixo de
um sol ardente.}

\textls[15]{Na caixa do trole mandou a dona da casa pôr uma cesta de
ameixas, para as filhas do sr.\,delegado.}

\textls[-15]{Vendo-se à distância da casa de Medeiros, perguntou o escrivão:}\looseness=-1

\textls[20]{--- Que julga de tantas hesitações, e como devemos prosseguir no processo?}

--- Não é preciso esforço para apreender a verdade, e\ldots{}
--- Aí interrompeu-se para acender o charuto, já entalado nos
dentes.

--- E\ldots? --- tornou o escrivão.

\textls[15]{--- E quanto ao processo, é talvez daqueles em que a
conveniência manda pôr pedra em cima.}

%\end{linenumbers}

\section{XVII}

%\begin{linenumbers}

\textls[20]{Trigueirinhos e Nicota seguiram para Casa Branca numa bela manhã
azul, depois de terem feito as suas despedidas à família em
Santa Genoveva e de convidarem Otávio para que fosse
convalescer em sua casa, aproveitando também o bom tempo das
caçadas e a estação das ``fruitas'', como muita gente denomina as
jabuticabas, em São Paulo.}

\textls[20]{Frutas é a designação de espécies variadas, desde a
gabiroba selvagem até o pêssego bem tratado; ``fruitas'', porém,
é termo aplicado unicamente ao fruto da jabuticabeira, árvore
que, nascida espontaneamente nas matas, sem cuidados, sem regas, tem a
forma desigual e o tronco alto, robusto.}

\textls[20]{Têm outro aspecto as jabuticabeiras plantadas nas ruas
dos pomares ou recantos dos jardins; aí endireitam o tronco
acetinado até menor altura, arredondando simetricamente a sua
copa verde-escura, de folhagem miudinha.}

\textls[20]{Trigueirinhos dizia ter no seu sítio um bosque de
jabuticabeiras e convidava com instância a família.}

\textls[20]{Prometeram todos ir passar lá uns dias e fizeram-se as
despedidas sem grande enternecimento.}

\textls[15]{Dr.\,Castro vinha duas vezes por dia ver o seu doente, cujo
estado só no fim de uma semana declarou livre de perigo. Eva continuou a
auxiliá-lo no curativo, até que o aparelho foi levantado
definitivamente; deram os parabéns ao convalescente, que no íntimo
sentia ver extinguidas as únicas horas suaves para si, aquelas em que a
prima, curvada sobre o seu leito, como um anjo de caridade, tratava dele
com desvelado carinho. Tinha às vezes vontade de beijar-lhe a
mão, pedindo que lhe perdoasse ter duvidado dela! Queria
interrogá-la, ou antes, desejava contar-lhe, nessa intimidade de
irmão, tudo o que ouvira na maldita noite em que o haviam ferido. Que
explicação podia dar-lhe? Como e por que envolveriam o seu nome naquela
intriga baixa e tola? Mas Otávio não ficava um minuto
a sós com Eva; ela preenchia o seu dever e saía do quarto antes de mais
ninguém, para ir informar por escrito o dr.\,Morton e Paulo do estado do
doente, por quem ambos se interessavam vivamente.}

\textls[15]{Reinava na fazenda uma quietação profunda, agravando a
tristeza da família.}

\textls[-15]{Medeiros, passados os maiores cuidados com o enfermo, inquiriu
parte dos escravos, no tronco, ouvindo a história singular de
um desafio de capangas, mandados por alguém com o fito de os
revoltar. Desse alguém eles calaram a princípio o
nome; mas, obrigados pelas ameaças e castigos, confessaram por fim ser o
de Eva.}\looseness=-1

\textls[10]{O fazendeiro ouvia-os de pé, e o feitor executava as suas ordens
com desembaraço e calma, escolhendo uns escravos para surrarem os
outros.}\looseness=-1

\textls[10]{Naquela atmosfera carregada, cheirando a sangue, cortada pelas
vibrações sibilantes das pontas do bacalhau}\footnote{Açoite de 
couro cru torcido com que se castigavam os escravos. [\textsc{n.\,e}.]} \textls[10]{e pelos ais dos negros, o
nome de Eva parecia fúnebre, repercutindo-se sempre como um dobre a
finados aos ouvidos do comendador.} \label{Ref14}

\textls[-25]{Às vezes, um grito mais pungente feria-o no coração; mas
lembrava-se do seu pobre Otávio, e a sede de vingança
abafava depressa o sentimento de misericórdia. Houve um momento
em que os ais esmoreceram: era preciso transferir o castigo para outras
vítimas; àquelas, extenuadas, deveriam ser dados os curativos do
costume.}\looseness=-1

\textls[15]{O feitor designou então para o suplício um outro escravo e para
carrasco o próprio pai da vítima. No quarto sombrio,
onde gotejava o sangue dos corpos quentes, estirados e nus,
houve uma cena dolorosa de angústia. O filho avançou
rápido para o lugar da tortura, despiu-se ele mesmo num
movimento febril e, pondo as mãos para trás, encruzou-as para que as
amarrassem; mas o pai continuou no seu canto, com a cabeça baixa, os
olhos reluzentes, as pernas trêmulas e a respiração ofegante; foi
novamente intimado, mas não deu um passo; um silêncio de admiração e de
angústia sucedeu às imprecações e gemidos. Depois de alguns instantes, o
feitor repetiu a ordem outra vez, num tom forte e áspero, e o algoz caiu
de joelhos, elevando para o teto escuro, como a procurar através dele o
Deus clemente, um olhar cheio de lágrimas e as mãos súplices.}

\textls[20]{Medeiros saiu para o terreiro, sufocava; não compreendia bem o
que se debatia dentro do seu coração. Lá no quarto do tronco que se
terminasse a abominável cena, contanto que seus olhos não a
vissem; sentia-se fatigado, mordido de remorsos. Desgraçada
condição a que se sujeitava! Para ele o castigo era a necessidade
absoluta, rígida, indispensável. Sem ele não haveria disciplina. Os
negros são como os soldados, que lá têm ainda o calabouço e os castigos
corporais, para compreensão dos seus deveres! Por isso, muitas vezes,
calcando a piedade, reprimindo o impulso da natureza,
contrafazendo a alma, ordenava a execução das mais absurdas e
intoleráveis penas. Medeiros sentia amargamente essa verdade. Doíam-lhe
agora os golpes que mandava vibrar nos servos e desviava a vista,
impressionado, dos instrumentos de tortura: a grande palmatória, o
relho, a gargalheira de alto gancho, as algemas encorrentadas,
todo o arsenal de instrumentos inventados pela ferocidade humana,
pendurados em uma linha na parede sombria do quarto do tronco. Sentia-se
cansado daquela vida de sustos e de cuidados; não podia conciliar o
sono, sem ter a certeza de que os escravos estavam fechados por fora, no
quadrado, e que o feitor guardava a chave debaixo do travesseiro\ldots{} \label{Ref15}}\looseness=-1

Julgava-se no direito de exercer a justiça pelo mesmo modo por
que a exerciam os outros; mas de vez em quando voavam alarmados na sua
consciência, como grandes corvos negros e pesados, os receios de
uma vingança justificada. Seguia os mesmos processos usados pelo pai,
que por seu turno os seguira do avô; desde criança habituara os olhos e
o espírito àqueles quadros feitos de sombras, dores e
lágrimas, e afizera-se por tal forma a eles, que os achava
naturais e indispensáveis. Caía, contudo, em contradições desde
que Eva fora habitar Santa Genoveva, e principalmente depois da
chegada de Otávio. Os dois moços tinham levado ao seu
espírito endurecido preceitos novos de moral e de compaixão.
Tinha às vezes vergonha diante da serenidade da sobrinha e do
filho, das suas brutalidades e assomos coléricos, que ele explicava como
necessários ao regime da escravidão. Não faziam todos os outros
fazendeiros o mesmo? E deixavam por isso de ser homens bons e bem
considerados pela sociedade?

\textls[15]{Por felicidade ele não tinha levado nunca o seu rigor até além
dos castigos usuais nas propriedades agrícolas. Havia fazendeiros cuja
barbaridade era citada pelos próprios lavradores do
lugar como muito mais criminosa do que disciplinar.}

\textls[-20]{Lembrava-se de se \label{Ref16} sentir arrepiado com a narração de um
fato ocorrido em uma fazenda da vizinhança. O senhor, de imaginação
fértil no ódio, tinha mandado surrar um escravo; até aí nada
de extraordinário\ldots{} mas depois, sobre as grandes bocas
dos golpes e por todo o corpo derramara em camadas grossas mel de
abelhas, e assim ungido fora o infeliz amarrado a um poste, no
terreiro, exposto às grandes nuvens de marimbondos que desciam
zumbindo, deliciados por tão opíparo quão inesperado banquete. O negro,
sem poder defender-se, via descerem os insetos e cobrirem-lhe o corpo,
movendo voluptuosamente as asas transparentes, a sugar com avidez o seu
sangue envenenado pelo desespero e pela raiva.}\looseness=-1

\textls[15]{Ocorriam à memória do fazendeiro outros casos. No
sítio do Gusmão, tinham aparecido esqueletos nos charcos, e
constava serem de escravos supliciados pelo dono. O próprio
compadre Antunes havia abandonado, depois de uma grande surra, uma
negra, que, sem se mexer, nem lavar, criara bichos por todo o corpo e
morrera já podre\ldots{} O Juca Ramalho tivera numa gaiola,
presa, durante quinze anos, uma mulata, sobre quem recaíam suspeitas de
envenenamento; a miserável, na sua jaula estreita e segura, cosia o dia
inteiro em tarefas rigorosas.}

\textls[15]{O Pimenta cerrava com cera as pálpebras de
certas crioulas e mandava-as andar com ligeireza. Medeiros
recordava tais acontecimentos para justificar com eles o seu
procedimento\ldots{} Havia, contudo, fazendeiros mais clementes que ele,
que premiavam os escravos, davam-lhes terrenos para cultura e direito de
venda e ganho; que não usavam pegas nem vira-mundo, e
adormeciam sem sobressaltos nem cuidados na chave do quartel. Lembrando
as insinuações do filho e da sobrinha, Medeiros lamentava não ter sido
um desses, e prometia a si mesmo emendar-se\ldots{} depois de vingado o
ferimento de Otávio!}

\textls[-17]{Voltou à casa do tronco; o pai negro, chorando,
azorragava o filho. Medeiros, num assomo de compaixão e revolta,
suspendeu o suplício. A atmosfera morna, impregnada do cheiro
acre do suor e do sangue, deu-lhe uma náusea; a vista
escureceu-se-lhe, uma vertigem fê-lo tropeçar num corpo úmido,
estirado no chão, e nuns passos precipitados e vacilantes chegou
à porta, tateou-a, abriu-a e tornou a sair. O ar
fez-lhe bem; respirou amplamente e tomou o caminho da casa, moído,
impressionado e ouvindo ainda o nome de Eva, pronunciado entre gemidos,
pelos escravos açoitados. Tudo por causa dela!, ponderava Medeiros,
indignado. Entrando na sala, foi direito à rede e
deitou-se; perguntou por Otávio ao pajem, o Saturnino,
mulato alto, pernóstico, muito estimado na família.
Saturnino voltou com boas notícias: Otávio tinha-se
levantado e lia numa rede, rodeado de almofadas. E o
pajem, depois de o informar da saúde do senhor-moço,
entregou-lhe duas cartas e um maço de jornais. Medeiros pediu café e
rasgou o envelope de uma das cartas. Era de Nicota: falava pouco da
família de Trigueirinhos e pedia à mãe umas receitas de
bolos de espeto e de sequilhos, concluindo
laconicamente: ``quando quiserem \emph{virem},
avisem''.}\looseness=-1

\textls[10]{A segunda carta era de um fazendeiro de São Carlos do Pinhal, o
Siqueira Franco, homem de afamada riqueza, dono de muitas terras, entre
outras a chamada Morro-verde, cuja colheita nunca era inferior
a trinta mil arrobas. Medeiros teve uma surpresa agradável, que
até certo ponto atenuou os desgostos por que tinha passado nessa manhã;
a carta dizia assim:}

\begin{quote}
Meu bom amigo comendador Medeiros,
\smallskip

\noindent{}\textls[-20]{As mulheres devem ser escolhidas como os porcos, pela raça; por
isso, peço-vos para meu filho Júlio, que V. bem
conhece, a mão de uma de suas filhas. Responda com brevidade; sabe quais
as nossas circunstâncias e gênio trabalhador e sério
de meu filho. Se quiser avise logo, para nós irmos.}\looseness=-1

\begin{flushright}
Seu amigo atento obrigado,\\
\emph{Anastácio de Siqueira Franco}
\end{flushright}
\end{quote}

%\end{linenumbers}

\section{XVIII}

%\begin{linenumbers}

\textls[15]{Eva cosia no quarto umas libitas,}\footnote{Camisolas de pala,
  corruptela de ``levitas''.} \textls[15]{para um negrinho seu afilhado, quando
  ouviu bater à porta; era Noêmia; vinha muito pálida e com os olhos
  lacrimosos.}

--- Que é isso? --- interrogou a prima.

--- Ah! Eu sou muito infeliz, Eva!

Noêmia atirou-se de bruços na cama, abafando os soluços nas
almofadas.

--- Que há de novo? Não seja criança!

Noêmia não respondia; chorava sempre.

\textls[20]{Eva sentou-se a seu lado e, ajeitando-lhe o cabelo,
perguntou-lhe com meiguice:}

--- Agora diga-me, que lhe aconteceu?

\textls[-15]{--- Uma coisa muito triste! --- respondeu ela
entrecortadamente.}\looseness=-1

--- Meu Deus! Que será?

--- Pai quer casar-me!

--- Ah! E com quem?

--- Eu sei lá! Com um bruto\ldots{} com um estúpido qualquer!
Mas eu não quero e\ldots{} e não quero!

Uma expressão fugaz de alegria iluminou o rosto de Eva.

--- Não há de ser tanto assim! Ora um bruto! Um
estúpido! Talvez seja um rapaz até bem-educado\ldots{} Como se chama?

--- Júlio Franco.

--- Não conheço\ldots{}

\textls[15]{--- Nem eu, mas imagino! Sabe como e em que termos é feito o
pedido?}\looseness=-1

--- Como hei de eu saber?

\textls[15]{--- Pois é assim, repare bem: \emph{As mulheres escolhem-se como
os porcos, pela raça}, por isso peço para o meu Júlio a mão de \emph{uma
de suas filhas}!}\looseness=-1

\textls[-25]{E, apesar de chorosa, Noêmia não pôde reprimir uma
gargalhada.}\looseness=-1

--- E agora?

--- Agora não sei\ldots{} Aconselhe-me, que hei de fazer?

--- Oh! Filha\ldots{} Por que não é franca para com seu pai?

--- Pai zanga-se\ldots{} e eu fico com medo\ldots{}

\textls[-10]{--- Mas olhe, há certas coisas que a gente diz com muito mais
facilidade a uma mãe; conte à sua o que sente e ela há de protegê-la\ldots{}}\looseness=-1

\textls[20]{--- Mãe não se atreve a contrariar a vontade de pai. Não tenho ninguém por mim!}

\textls[15]{--- Irei eu falar ao tio Medeiros, apesar de não ser a
pessoa mais competente; mas enfim talvez que ele me
atenda\ldots{} espere aqui, eu já volto!}

--- Não, Eva, não vá! --- disse rapidamente Noêmia,
levantando-se e interpondo-se entre ela e a porta.

--- Por quê?!

\textls[15]{--- Porque\ldots{} porque\ldots{} --- E Noêmia, embaraçada,
baixava os olhos, procurando uma resposta.}

--- O tio Medeiros disse-lhe alguma coisa a meu respeito?

Noêmia balbuciou:

--- Pai é seu amigo\ldots{}

\textls[-30]{--- Então? --- tornou Eva, com um ligeiro sorriso de incredulidade.}\looseness=-1

\textls[10]{--- Sim\ldots{} mas você bem sabe\ldots{} ele tem uma maneira de
pensar tão diferente da sua! É capaz de supor que\ldots{} eu
não sei\ldots{} olhe: não seria melhor pedir a Otávio
que lhe falasse?}

\textls[15]{--- Oculta-me qualquer coisa, eu bem percebo isso\ldots{} O tio
Medeiros fez alguma observação contra mim; ah! Escusa negá-lo
--- disse, interrompendo um gesto de Noêmia ---; não me
zango, cada qual tem lá o seu raciocínio e a sua educação, e
estou certo de que o caso não será tão grave que me prive de intervir
num assunto de tanto interesse para todos nós!}

\textls[15]{Noêmia então contou à prima, desviando-a do intento de ir
interceder por ela, que o comendador a chamara à sala e, depois
de um discursozinho preparatório, lhe entregara a
carta do amigo, dizendo: O Luiz Anastácio pede uma das minhas filhas,
esquecido de que Nicota já está casada ou imaginando que eu tenho mais
de duas; ora, portanto, sois vós a noiva, visto eu não ter outra filha
solteira. Tereis um excelente marido; ide dizer isto mesmo a
Otávio. Ela protestara com receio, timidamente: não conhecia o
rapaz\ldots{} e jurara seguir a opinião de Eva\ldots{} A opinião de
Eva?, indagara raivoso o fazendeiro, qual é?! É que o casamento
sem amor é uma imoralidade, uma grande desgraça! Foi então que
o comendador se zangou, dizendo que se Eva interviesse nos seus
negócios de família, ele\ldots{}}

--- Acabe! --- suplicou Eva à prima.

\textls[25]{Mas esta, de novo sufocada em pranto, caiu-lhe nos braços,
trêmula, pedindo-lhe que não a abandonasse, que ela se sentia mal, muito
mal!}

Eva socorreu-a, desapertou-lhe os vestidos, deitou-a, cerrou a
janela à claridade intensa do dia, e encostada ao leito viu-a passar das
convulsões do choro ao sono, nuns suspiros lentos, fatigados.

\textls[20]{A órfã de Gabriel deliberara arrostar com a cólera do
tio, a favor dessa pobre moça meiga e boa, por cujo futuro tanto se
interessava. Provocaria assim uma explicação sincera da parte dele.
Repugnavam-lhe as indiretas; gostava das situações francas.
Compreendia perfeitamente ser ali demais; aborreciam-na e ela tinha
resolvido retirar-se quanto antes para a sua querida casa do
Mangueiral.}

Lembrava-se com amargura da última vontade expressa por
seu pai, queria segui-la e obedecer-lhe em tudo. Ele tinha sido
um homem raro, bom, justo, caritativo, afetuoso e incapaz de
um ato leviano; sempre refletido, sempre sereno,
sempre cheio de boa-fé e aureolado de virtudes. Por isso tudo foi que
ele a pusera em casa do irmão, cuidando abrigá-la das
perfídias da sociedade.

Entretanto, era exatamente aí que Eva encontrava
sorrisos contrafeitos, ditos misteriosos e insinuações malignas.

\textls[20]{A sua dignidade ordenava-lhe que saísse, mas sentia-se
presa, pelas palavras do seu grande amigo morto: ``apesar de toda a
amizade e confiança, Eva não deve morar só com Paulo, porque ele não é
seu irmão''.}

Para se retirar de Santa Genoveva, deveria, pois, casar-se. O
marido seria o seu escudo; ela, inteligente, altiva, honesta, não tinha
direito nem podia assumir a responsabilidade dos seus atos!

Era necessário que um homem qualquer, embora de menos
escrúpulos, ou de espírito inferior ao seu, a tutelasse, lhe
desse um nome, talvez menos limpo, menos honrado e menos digno
que o dela. Sem o amar, sem lhe poder dar uma felicidade perfeita, ela
teria de sujeitar-se à sua vontade, ao seu capricho, ao seu
domínio, sacrificando a alma no exercício de mentirosos
deveres.

\textls[20]{Tudo agora lhe parecia preferível a ter de suportar com
humilhação as desfeitas do tio, a quem o pai tão generosamente perdoara,
oferecendo-lhe o que tinha de mais amado e de maior ternura --- a
filha!}

Onde acharia Eva esse esposo que a arrancasse de uma posição tão
triste? Pensava sempre, obstinadamente, no mesmo homem\ldots{} querendo
desviar dele o espírito, para ele voltava, atraída por uma
força misteriosa. Com ele construía os mais brilhantes sonhos de futuro;
com ele parecia-lhe a vida muito mais formosa e doce; mas esse
alguém não demonstrava amá-la; mas esse alguém
fora sempre para ela quase indiferente e frio. Dele
escondia recatadamente as mais tênues manifestações de amor, num
escrúpulo de requintado orgulho. Se ele não quebrasse o encanto, não
seria ela quem se revelasse; viveriam toda a vida ao lado um do outro, a
olharem-se sem se compreenderem, sem se fazerem felizes. Amar outro
homem? Impossível! Casar com outro?\ldots{}

\textls[20]{Eva procurava esse outro entre o pequeno número dos
seus pretendentes.}\looseness=-1

\textls[15]{Errou-lhe o espírito muito tempo de uns para os outros,
sem se poder fixar em qualquer deles! Mas urgia tomar uma deliberação, e
Eva, sendo altiva, era resoluta. Depois de uma hora de meditação,
sentou-se à mesa e escreveu com mão, que debalde quis tornar
firme:}

\begin{quote}
Meu bom Paulo,
\smallskip

\noindent{}Fez bem em revelar-me o segredo do dr.\,
Azevedo; resolvida a deixar quanto antes esta casa, ofereço-lhe a minha 
mão. Escreva-lhe e proporcione um desenlace rápido a este meu 
projeto.

\begin{flushright}
\textit{Sua, Eva}
\end{flushright}
\end{quote}

\textls[15]{Fechou a carta, saiu em bicos dos pés para não acordar Noêmia e
ordenou ao pajem que a levasse imediatamente ao Mangueiral.
Quando voltou, Noêmia tinha acordado e estava sentada na cama, abotoando
o vestido para levantar-se.}

--- Como está?

--- Melhor\ldots{} É muito tarde?

--- Não: são quase três horas.

--- Ih! Já chamaram para o jantar?

--- Ainda não\ldots{}

--- Teremos tempo de falar a Otávio? Agora me lembro de
que ele me pediu para chamá-la!

--- A mim?

--- Sim; mas fiquei tão atrapalhada que me esqueci! Se eu
pudesse livrar-me do tal noivo!

--- Não pense nisso agora; e espere, que se há de
arranjar tudo!

--- Qual! Pai quando diz as coisas não volta atrás!

--- Veremos. Vamos falar a seu irmão.

\textls[10]{Noêmia levantou-se, beijando a prima com reconhecimento, e
atravessaram juntas a sala de estudo, parte do corredor e da varanda. A
dona da casa balançava-se na rede, cosendo ao mesmo tempo; a
seus pés dormia regaladamente uma negrinha de poucos meses,
muito envolta em trapos. À pequena distância, uma mulata gorda
engomava uma saia de alto folho bordado.}

O Saturnino, sentado à porta do terreiro, polia o metal
de uns arreios; Noêmia chamou-o, ele levantou-se logo, muito respeitoso,
e ouviu a ordem de ir saber de Otávio se lhes poderia falar.

\enlargethispage{2\baselineskip}
\textls[-15]{--- Que sim --- voltou ele dizendo com os seus gestos e
frases adocicadas e pedantes. --- Meu senhor-moço
Otávio manda participar às minhas senhoras que está pronto para
receber tão honrosa visita! --- E curvou-se muito, mostrando
num sorriso a falta de dentes.}\looseness=-1

%\end{linenumbers}

\section{XIX}

%\begin{linenumbers}

\textls[20]{O quarto de Otávio era pouco espaçoso, mas alegrado por duas
janelas sobre um pátio lateral.}

\textls[20]{O convalescente, recostado em almofadas, lia uma revista de
engenharia, recebida poucas horas antes.}

\textls[20]{Barbeara-se, pusera na sua toalete de doente um
apuro elegante. Transpareciam no seu rosto viril uns bons
prenúncios de saúde e de energia. Tinha vestido uma
\emph{robe de chambre} de seda
indiana, escura, com arabescos e cornucópias brilhantes;
enrolara nos joelhos uma manta persa, trazida igualmente de Berlim, da
passageira época de umas entrevistas amorosas com a mulher de um
negociante rico da avenida das Tílias.}

\textls[12]{Quantas vezes a loura cabecinha da amante se encostara perfumada
e lânguida no seu ombro, enquanto os seus lábios, numa
avidez de abelha, lhe procuravam o mel dos beijos! Lembrara-se dela
naquela manhã, quando o Saturnino lhe levara a \emph{robe
de chambre}, e lhe entregara um objeto que
tinha encontrado no bolso, um lenço pequenino, com as iniciais
\textsc{t.\,r.}, bordadas a cor. Eram as suas iniciais. Viera
aquele lencinho ignorado trazer-lhe uma lembrança que o não visitava
havia muito.}

\textls[15]{Como principiara aquilo? Como acabara?}

\textls[10]{Tinha começado num baile da legação brasileira; ela, vestida de
Tonkim branco, fora-lhe apresentada pela mulher do ministro
como uma das suas antigas condiscípulas no convento.}

\textls[-10]{Como tinha acabado? Numa noite de inverno; ela entrara no seu
gabinete de rapaz solteiro e, atirando para o chão a capa de veludo que
a envolvia toda, lançou-lhe os braços ao pescoço, cobriu-o de carícias,
num extravasamento de ternura indômita! Subitamente, retraindo-se,
disse-lhe que se não tornariam a ver, mandou buscar ao
cupê o cofre das cartas, exigindo de Otávio as dela.
Atiraram todas para o fogão, contemplando silenciosos as chamas que
tantas chamas consumiam, e, sem outro adeus, separaram-se!}\looseness=-1

Ele vira-a depois, uma vez, na rua, num carro descoberto, ao
lado do marido, que o cumprimentou tirando rasgadamente o
chapéu; ela moveu ligeiramente a cabeça com orgulhosa
indiferença. E, nem ele pensou mais nela, nem ela pensou mais nele.

\textls[15]{Otávio volteou nos dedos o elegante lencinho, atirando-o depois
numa gaveta aberta por acaso, e começou a polir as unhas, preocupado com
outros pensamentos.}

\textls[-7]{Decidira falar a Eva seriamente nesse mesmo dia, e rogar-lhe que
lhe dissesse de quem deveria desconfiar em todo aquele drama. Ela que o
ajudasse, que lhe esclarecesse o espírito em trevas. Não duvidava, nem
consentia que duvidassem dela. Amava-a!}

\textls[10]{Para ele, Eva era a personificação da pureza e do bom
senso. Revestia-a de todas as virtudes, endeusava-a,
amaldiçoando o momento em que chegara a duvidar dela
também.}

\textls[15]{Julgava-se agora indigno de lhe tocar na mão que tão
benfazeja se estendera sobre o seu leito, a
dar-lhe, com os curativos, o bálsamo para as suas atrozes
dores!}

\textls[11]{Qual seria o infame que trabalhava assim para a perder? Com que
interesse? Qual o motivo para tão inexplicável animosidade? Seria
possível que Eva ignorasse tudo isso?}

\textls[-25]{Otávio refletia desse modo, quando Noêmia entrou,
trazendo, muito influída, um papagaio, a que já ensinara a
dizer a palavra ``tolo''!}\looseness=-1

\textls[15]{--- Quer ver, Otávio? O meu papagaio
já sabe dizer mais uma palavra! Diga, meu louro: tolo!}

\textls[15]{E o papagaio, inflando o pescoço, repetiu com voz gutural e
embrulhada: tolo!}

Ela riu-se muito e depois, reparando no irmão, exclamou:

\textls[15]{--- Bravo! Que luxo! Mal empregada coisa num sítio tão
triste! Que bonita seda! Na Europa usam muito disto?!}

--- Usam; e aqui também.

--- Aqui?! Eu nunca vi.

\textls[15]{--- Certamente; são trajes caseiros, em que só muito na
intimidade da família se pode aparecer.}

\textls[20]{--- Quando eu me casar, hei de dizer a meu marido que compre uma
coisa assim\ldots{}}

\textls[20]{Bateram à porta, e uma mucama disse do lado de
fora:}

--- Nhá Noêmia? Sinhô tá chamando mecê!

\textls[15]{--- Já vou! --- respondeu ela alto; e depois a meia
voz: --- Para que será?! Bem! Até logo, Otávio! E ia a sair.}

\textls[20]{--- Escute --- interrompeu-a o irmão; ela voltou-se,
ele prosseguiu: --- Depois de falar com pai, peça a Eva para
vir conversar um pouco comigo!}

--- Ora essa! O senhor quer uma entrevista com a priminha?

\textls[15]{--- Que tem isso?! Não é natural que me agrade a conversação
dela?}\looseness=-1

--- É; olhe, eu por mim acho-a preferível a todas as outras.

--- Eu também\ldots{} mas vá, que pai não gosta de
esperar!

--- É mesmo; adeus.

Noêmia saiu correndo.

\textls[-15]{Passou-se uma hora, e depois outra, sem que nem a irmã nem Eva
aparecessem a Otávio; ele fumou, leu, planejou a melhor maneira
de atacar o assunto que tanto o preocupava, e voltava de novo
à leitura, quando o Saturnino, inclinando-se diante
dele, perguntou:}\looseness=-1

\textls[10]{--- Meu senhor? As minhas senhoras desejam saber se pode-as
admitir\ldots{}}\looseness=-1

--- Quais senhoras?

--- Minha sinhá moça d.\,Noêmia e\ldots{}

--- Sim, sim, que entrem!

Dois minutos depois entraram Noêmia e Eva.

\textls[15]{--- Desculpe tê-la incomodado --- disse
Otávio à prima, tentando levantar-se.}

\textls[15]{--- Esteja a seu gosto, não se levante! E quanto ao incômodo que
me deu mandando chamar-me, não foi nenhum, tanto mais que eu preciso
mesmo falar-lhe.}

--- É verdade --- murmurou Noêmia ---, Eva tem
muito que dizer!\ldots{} A respeito de uma coisa triste que aconteceu
hoje\ldots{}

--- Sim?! --- perguntou o moço voltando-se para a
prima.

--- Realmente --- tornou-lhe esta ---, o meu
caso é com certeza mais urgente que o seu, e é natural que
também seja mais grave; por isso tenha paciência, mas
hei de ser a primeira a falar\ldots{}

--- Nem podia deixar de ser assim!

\textls[15]{Eva narrou então a Otávio, pedindo a sua intervenção,
a história do pedido de casamento, as lágrimas de Noêmia e o
seu justíssimo pavor pelo futuro.}

\textls[15]{O moço sossegou-as; ele convenceria o pai de não dever
contrariar a vontade da filha, despersuadindo-o de levar a efeito planos
de tal ordem.}

Houve uma pausa. Foi Eva quem cortou o silêncio, dizendo ao
primo:

--- Então, agora é a sua vez!

\textls[-30]{--- Sim\ldots{} --- respondeu ele indeciso; depois voltou-se para a irmã:}\looseness=-1

\textls[15]{--- Deixa-nos um momento, Noêmia; tem paciência!}

Decorreram alguns segundos sem que nem um nem outro articulasse
uma única sílaba. E era já tarde, o relógio de parede lá
fora vibrava uma pancada sonora: a da meia hora depois das
três.

\textls[-10]{Daí a nada viriam interrompê-los, chamando para o jantar.
Otávio refletiu nisso e na inconveniência de uma demora da prima,
sozinha, ali; por isso começou logo, embora muito embaraçado:}\looseness=-1

\textls[15]{--- Ouça-me com atenção, Eva; tenho a pedir-lhe um
esclarecimento, fazendo-lhe ao mesmo tempo uma revelação dolorosa.
Primeiro que tudo, diga-me: não acredita que eu seja muito seu amigo? --- E
olhou-a com ternura, mergulhando nos olhos dela toda a vida dos seus.}

\textls[10]{Eva, atônita, não respondeu logo. A sua cadeira, muito chegada
à de Otávio, punha-a em contato com as dobras mornas da
manta que o envolvia.}

--- Diga-me! --- tornou ele com insistência. 
--- Não acredita na minha sinceridade?

--- Certamente! --- respondeu ela, procurando em mente
o motivo para semelhante pergunta.

--- Então, se eu lhe fizer uma pergunta ou der um conselho, não
o receberá com desconfiança, nem ficará zangada?

\textls[20]{--- Não\ldots{} nem sei que haja razão para desconfiarmos um do
outro.}

--- Bem! Agora outra coisa: Eva tem algum inimigo?

--- Eu? Não.

--- Absolutamente nenhum?!

--- Que eu saiba, pelo menos, nenhum!

--- É singular\ldots{}

--- Por quê?!

--- Faça um esforço de memória!

\textls[15]{--- Não é preciso; a minha vida tem sido tão simples, que eu de
um relance vejo-a toda!}

--- Nunca fez mal a ninguém, nem direta nem
indiretamente?

\textls[20]{--- Que lembrança, Otávio! Nunca! Ou se fiz foi
inconscientemente!}

\textls[15]{--- Está, visto isso, perfeitamente tranquila? Ainda bem\ldots{}
ou, talvez que ainda mal!}

--- Por que diz isso? Há qualquer coisa contra mim? Diga com
franqueza! Eu ando desconfiada e o melhor é\ldots{}

\textls[20]{--- Aclararmos a situação; de acordo. Você não é uma
criança fútil; ao contrário, é ponderada e forte. Mas primeiro
que tudo há de me fazer uma promessa!}

--- Qual?

--- De guardar silêncio do que dissermos aqui!

--- Julga-me leviana?

--- Não, mas nada mais natural que desejar confiar\ldots{} a seu
irmão, por exemplo, tudo o que me ouvir!

--- Na verdade não tenho segredos para ele, e, quanto mais
importante me parece um fato, mais me apresso em comunicar-lho e
pedir-lhe conselho!

\textls[20]{--- Estima-o muito? --- perguntou Otávio,
curvando-se para ela, esquecendo momentaneamente, à dor do
ciúme, o objeto fundamental da entrevista.}

--- Muito! --- respondeu Eva, corando.

--- E ele corresponde-lhe?

--- De igual maneira!

--- Tal e qual, como se realmente fossem irmãos?

\textls[-25]{--- Mas foi para me fazer tais perguntas que me mandou chamar?!}\looseness=-1

--- Vê! Já está zangada!

\textls[15]{--- Não estou! Eu sei que, como convalescente, você goza de
muitos direitos: não o contrariarei\ldots{}}

\textls[15]{Otávio sorriu-se, pegou-lhe na mão e ia falar, quando ouviu a
voz do pai, que se vinha encaminhando pelo corredor.}

\textls[25]{Otávio apontou à prima a porta da sala, murmurando rapidamente:}

--- Saia, Eva, falaremos depois; não convém que a vejam
aqui neste momento. Obrigado pela sua condescendência. Conte
que tem nesta casa um defensor, um servo dedicado e capaz de todos os
sacrifícios para salvá-la, se acaso a vir em perigo
algum dia!

Eva levantou-se e fixou com estranheza os olhos no
primo; depois, respondeu com voz firme:

\textls[15]{--- Eu não saio deste quarto sem que você me explique o
sentido de suas palavras. Aceito e retribuo a amizade que me oferece,
aceito mesmo a proteção, mas não quero sacrifícios nem
consentirei nunca que se exponha por minha causa a situações
perigosas\ldots{} Não há, nem haverá necessidade
disso; não faço mal a ninguém, nem dou crédito a intrigas; o
que peço é a lealdade de me dizerem claramente se os incomodo aqui.}

\textls[15]{--- Eva! Perdoe-me, isto é um resto de febre! Eu não sei
falar-lhe, vê, que quer? Quem não a estimará? Quem não a\ldots{} Olhe
que ofensa me atirou! Como é má\ldots{} --- E esforçava-se por sorrir,
segurando-lhe na mão, olhando-a com amor.}

Eva escutava-o, imóvel, pálida, surpreendida,
enquanto a figura do tio se desenhava entre os umbrais da
porta.

\textls[15]{O comendador estava nesse dia desesperado contra a sobrinha; a
confissão dos escravos e a recusa de Noêmia ao casamento com o Franco,
por influência da prima, acabaram de acentuar-lhe no espírito a
resolução de a afastar de Santa Genoveva, de a mandar embora, clara e
definitivamente. Vendo o filho livre de perigo, já de pé, cessou de
considerar como um dever a concentração do seu ódio e deixou-o
explodir.}

\textls[20]{--- Então --- disse ele, aproximando-se de Eva, com os
braços cruzados e o olhar em fogo ---; não basta que a senhora
vire a cabeça dos meus negros; anda agora também virando a
cabeça dos meus filhos? Que faz aqui?}

Ela fitou-o com altivez e assombro, e Otávio respondeu
logo:

--- Eva veio a meu chamado. Acabei de pedir-lhe a sua mão.

\textls[15]{A estas palavras o comendador, estupefato, arregalou os
olhos e rugiu com força:}

--- Isso é o que ela queria!

\textls[-30]{--- E é o que eu mais ardentemente ambiciono. Diga-me,
Eva --- disse ele, voltando-se para a prima ---,
consente em ser minha mulher?}\looseness=-1

--- Não!

\textls[15]{Ouvindo esta resposta seca e firme, pronunciada num tom
de inapelável resolução, Otávio tornou-se lívido. O
pai riu-se com ironia, abanando a cabeça, e Eva saiu.}

%\end{linenumbers}

\section{XX}

%\begin{linenumbers}

Nessa mesma tarde a órfã de Gabriel Medeiros
despediu-se da tia e de Noêmia, que chorava muito, e desceu a escada de
pedra de Santa Genoveva; embaixo, no terreiro, esperava-a o
trole já pronto e o feitor, o Honorato, a quem tinham ordenado
que a acompanhasse ao Mangueiral. Eva tomou com repugnância
lugar a seu lado, o cocheiro fustigou os animais e partiram.

\enlargethispage{2\baselineskip}
\textls[15]{O caminho era longo, sinuoso e pitoresco. Iam beirando um rio
estreito e sombrio àquela hora; a água marulhava nas pedras
monotonamente e as rãs coaxavam atoladas nos limos das margens. Eva ia
silenciosa, pensando na singularidade da sua vida, nos casos imprevistos
e mesmo romanescos, que lhe iam sucedendo naquele meio tão rotineiro e
tão avesso a aventuras, e invejava a passividade, a doce
tranquilidade das outras mulheres, rodeadas de afetos, de
proteção e de confiança! Como era bom ter pai e ter
mãe! Como era útil ter um gênio sereno,
moldável a todas as circunstâncias! Ela herdara da
família materna aquela rebeldia e independência de
caráter, que a educação do pai, homem de raciocínio,
pouco conseguira atenuar. Ele aplaudia-a algumas vezes, outras chamava-a
precipitada e fazia-lhe pacientemente ver o erro em que havia
incorrido.}

\textls[20]{Eva concentrava-se nestas lembranças, quando o Honorato
principiou a falar, sem olhar para ela:}

--- Os diabo dos negro faz farta! Quando os mardito levá di
reio, aí eles há de se arrependê\ldots{}

Eva continuava muda, ele prosseguia:

\textls[20]{--- Os cafezá tão no mato! Porque os cachorro não
pode trabaiá deréto\ldots{} \emph{mais}
também despois eles fica macio que nem
cetim\ldots{} bem faz o vizinho Simão, aquele é que não
peldoa memo! Sabe aquele Bento véio que costuma i lá in
casa?}

Eva não respondeu, ele continuou:

\textls[-20]{--- Fugiu! \emph{Mais} os capitão do mato agarraram ele e foi
metido no tronco pra sê sovado\ldots{} o
boraco do tronco tava muito apeltado, daí o pé dele inchou
tanto qui foi perciso os médico coltá\ldots{}
--- E riu-se.}\looseness=-1

Eva aconchegou o xale ao corpo: sentia arrepios.

\textls[-15]{--- Negro sem chicote não é gente memo. É à
toa\ldots{} O mato tá cheio de quilombola,
tudo do José Dualte, qui é um lavradô sem corage\ldots{}
muito mole! Si ele castigasse os negro dele não havéra de
assucedê ansim\ldots{} Tá i, seu Antune é qui sabe lidá cum
os preto!}\looseness=-1

\textls[15]{Estas últimas palavras, disse-as com intenção, olhando
fixamente para Eva, que parecia impassível.}

\textls[20]{--- Foi ele qui amarrou os negro lá di
casa\ldots{} aquilo é qui foi baruio, hein sá dona?}

\textls[20]{Eva não tinha ainda ouvido a descrição da revolta de Santa
Genoveva. Não lhe parecia extraordinário o fato; sem imaginar
que o seu nome estivesse enredado nele, fugia de falar nisso, temendo
emitir opinião desagradável aos tios. Agora, que o feitor
parecia disposto a pô-la ao corrente de tudo, sem que para isso
ela desse o mínimo motivo, perguntou-lhe:}

--- Quem foi o cabeça da revolta?

--- Ué, gentes! Mecê tá caçoando?

\textls[-20]{--- Ora essa! Por quê? Ouvi dizer que tinha sido o
Damião\ldots{} foi?}\looseness=-1

--- Ué! Que sperança! Na terra!

Houve uma pausa; Eva, enfadada, voltou o rosto e calou-se. O
Honorato então disse a meia-voz, mas já sem ser percebido:

--- Se mecê tivesse escravo, mecê havéra de sê mais
rigorosa; no que é alhêo não custa fazê
desorde!

Depois destas palavras fez um gesto de arrependimento, mas Eva
já não parecia dar-lhe atenção. Atribuiu aquela referência,
como tinha atribuído a do tio, à sua piedade para com os
escravos.

Honorato continuou:

--- A mardade que os diabo teve de querê matá
seu Otávio! Aquilo merece bem bom tronco!

Eva voltou-se rapidamente e fixou naquele homem nojento, que ia
ali a seu lado, um olhar cheio de curiosidade. Sentia grande
repugnância por ele, e desviava-se do seu contato.

Desde que entrara em Santa Genoveva que o aborrecia; via-o
sempre de chicote em punho, os grossos pés de pele ressequida e
gretada, sem meias, dentro de uns sapatões de couro amarelo, faca à
cinta, camisa de algodão mal abotoada, cabelos ásperos, palavras
injuriosas a ferver-lhe na larga boca sensual e carnuda, ignorante,
carrasco, impondo-se pela força bruta, pela pancadaria;
abominável na sua profissão, intolerável na sua estupidez.
Ouvindo-o, porém, referir-se ao ferimento de Otávio,
Eva não se pôde coibir e dirigiu-lhe pela segunda vez a palavra:

\textls[-15]{--- Mas quem seria que o feriu, e por que seria? Ele é tão
bom\ldots{}}\looseness=-1

\textls[-20]{O feitor arregalou os olhos, pasmado, e depois, para certificar-se:}\looseness=-1

--- Que é qui mecê tá dizendo aí?

Eva repetiu a frase; ele exclamou:

--- Tão vendo só?! Isto é qui eu quiria sabê de
mecê\ldots{}

Eva atribuiu estas palavras a igual ignorância da
verdade e calou-se, pensativa.

Tinham perdido de vista o riozinho que aparecia numa e noutra
curva do caminho, como um brejo meio encoberto de matos e agriões.

\textls[20]{Entraram por fim numa estrada larga, entre cafezais escuros,
prometedores de grandes safras.}

\textls[20]{Escondia-se o sol, incendiando os vidros das janelas de um
casarão branco, isolado num extenso vale à esquerda, muito
verde, cortado pelas listas vermelhas dos caminhos.}

\textls[20]{Na estrada amarelada, lisa, viram ao longe um ponto escuro que,
à proporção que se foram aproximando, verificaram ser um grupo
de três homens, dois capitães do mato, armados e ferozes, a
arrastarem um negro magro, fortemente enleado por cordas grossas,
abatido pelas vigílias e pela fome.}

O negro, vendo-se já perto da casa do dono, recusava-se a andar,
transido de medo, fincando no chão os pés escalavrados pelos espinhos e
estrepes dos caminhos; mas as pernas, bambaleantes, tremiam e
arqueavam-se; por vezes caía de joelhos, e era assim levado, de rastos,
como um animal morto e pesado!

\textls[15]{Através da pele ressequida, via-se-lhe desenhada em
grandes traços salientes toda a carcaça; a carapinha arrepiada,
suja de terra, tinha uns tons ruivos que faziam lembrar as cabeleiras
dos diabos de mágica; os olhos saltados rolavam nas órbitas,
com fatídica expressão de loucura! E assim o desgraçado, sempre
a debater-se, ia rolando entre imprecações e pontapés.}

\textls[25]{Chegando perto do miserável escravo, o Honorato levantou o
chicote, para bater-lhe nas costas e incitá-lo a seguir; mas
àquele movimento brusco, Eva segurou-lhe no braço ainda no ar, com
firmeza, nervosamente, e o trole passou.}

Era demais! O feitor, vermelho de raiva, resmungou baixo:

--- Por estas e outras é qui está tudo peldido! Eu sempre
quero vê de que selviu fazê aquela revorta em Santa
Genoveva\ldots{} irra!

Excitada contra o seu companheiro, Eva gritou para o cocheiro
que pusesse os animais a galope.

\textls[25]{Era já noite quando, muito cansada e febril, parou à
porta da sua habitação. Apeou-se, entrou apressadamente e foi
surpreender Paulo no seu gabinete de trabalho.}

Paulo, depois de ter manifestado o seu espanto, perguntou-lhe
quem a tinha acompanhado.

--- Vim com o feitor --- disse Eva ironicamente
---, e é bom darem de beber a esse animal!

--- Eva!

--- Vai, Paulo, vai!

Ele obedeceu e Eva, vendo-se só, começou a chorar.

\textls[15]{O feitor de Santa Genoveva voltou pela mesma estrada, à baça luz
do luar, assobiando alto, repoltreado à vontade, com os braços
abertos sobre o encosto do trole.}

\textls[-10]{O gabinete de Paulo era adornado com severidade:
mobília de jacarandá, mesa espaçosa, com carrancas a
guarnecerem-lhe os cantos, estantes envidraçadas e cadeiras antigas de
espaldar alto.}\looseness=-1

Alguns dos seus amigos riam-se francamente daquele desusado
luxo, dando-o como a mais irrecusável prova de pouco juízo. O
enjeitado, o filho adotivo de Gabriel Medeiros enterrava as suas
economias em coisas para eles supérfluas, nos deliciantes
objetos artísticos, tão raros na província! Entretanto, os
agentes e comissários dos grandes armazéns de Paris recebiam listas
enormes de toaletes caras; as senhoras encomendavam sedas e
veludos, guarneciam-se de rendas, envolviam-se em perfumes e habitavam
casas sem conforto, quase nuas e
aborrecíveis.

\textls[20]{Os rapazes esbanjavam os seus lucros de lavradores ricos lá
fora, no jogo, com mulheres, em extravagâncias de toda a ordem,
e recolhiam-se sem alegria para os seus quartos cheios de apetrechos de
caça, com roupas pelas paredes e garrafas de conhaque e bolsas de
fumo sobre as mesas.}

A pouco e pouco abriram-se algumas exceções; já não se pediam
só, exclusivamente, os vaidosos vestidos desta ou daquela modista;
reclamavam-se também alguns aparelhos de jantar, com o seu
monograma, uns estofos para cortinas ou uma mobília a gosto do
fabricante ou do expedidor. Contudo, prevaleciam
nesses pedidos os adornos de toalete. E a persistência
de Paulo em mandar vir para o sítio trastes elegantes, concentrando
nisso todo o seu vício, era tida como espantosa e absurda.

--- Que se enfeite uma casa da cidade --- diziam eles
---, vá! Mas a do sítio!

No entanto é na fazenda que o lavrador passa a maior
parte da sua vida, é nela que lhe deslizam, bons ou maus, os dias; é
nela que tem os seus interesses e o seu orgulho de fortuna.

\textls[15]{Quando Paulo voltou, Eva, sentada perto da mesa, com os
cotovelos firmados numa larga pasta de couro e o rosto entre as duas
mãos, tinha os olhos ainda úmidos e brilhantes.}

\textls[15]{Ele sentou-se em frente dela, fez girar o \emph{écran} para que
a luz do lampião caísse em cheio sobre as feições da moça, e,
inclinando-se um pouco, perguntou-lhe carinhosamente o que se tinha
passado. Parecia calmo.}

Eva estremeceu à sua voz aveludada e grave; levantou-se, deu
algumas voltas na sala sem lhe responder; depois, aproximando-se dele,
ainda de pé, com os lábios trêmulos e o olhar indignado,
começou nervosa e ininterruptamente a narração de tudo: as frases do
tio, o pedido generoso de Otávio, a sua recusa e saída
precipitada, a desconfiança de que a cercavam e a maneira
esquisita por que a tratavam desde algum tempo.

Paulo escutava-a em silêncio, com o rosto carregado por uma
grande sombra; deixou-a expandir todo o seu ressentimento; depois,
fê-la sentar-se e disse-lhe pausada e moderadamente:

\textls[15]{--- Você fez mal! De um movimento instantâneo e
irrefletido depende muitas vezes a felicidade da nossa vida; e
você foi precipitada; devia antes pedir uma explicação com altivez, mas
serenidade. Nunca se chega sem causa a tais extremos. E é essa causa que
precisamos conhecer.}

--- Quem se poderá conservar serena diante de um
insulto?! Eu não o compreendo, Paulo. Sempre o tive na conta de
orgulhoso e de justo.

--- Não o tenho sido, porventura?

--- E então?!

\textls[20]{--- Escute: há uma grande diferença entre nós dois; eu
sou homem, independente, responsável pelos meus atos, pronto a
descer à luta, a rolar com o inimigo na terra, na
própria lama se for mister. Você é mulher, sem
pai, sem marido, sem um braço forte que a defenda, que apare os golpes
que lhe forem dirigidos, que esbofeteie, enfim, quem
se atrever a dizer-lhe uma palavra menos cortês; provocar um sujeito
pouco delicado e impetuoso como é seu tio é, portanto, um erro
imperdoável numa pessoa do seu sexo e do seu critério.}

\textls[15]{--- E é o filho adotivo de meus pais, o \emph{meu irmão} quem me
diz isto! --- exclamou Eva nervosamente.}

\textls[-20]{--- Sim! Sou eu, o filho adotivo de seus pais, a quem prezei
sempre, a quem devo posição, nome, fortuna, tudo, tudo; mas não sou seu
irmão, Eva! A lei não me dá os mesmos direitos que o coração me confere;
e, se assim fosse, se a sociedade me julgasse digno de a
proteger, de a amparar, de tomar a responsabilidade de suas palavras e
das suas ações, como um verdadeiro irmão, por que não teria
você ficado aqui, no Mangueiral, a meu lado, na doce, na santa
intimidade em que nos criaram seus pais?! Oxalá que eu pudesse ter aos
olhos de toda a gente o dever de a aconselhar e de vingá-la!}\looseness=-1

--- Mas esse é o dever de todo o homem de honra para com uma
mulher honesta\ldots{}

--- Quando essa mulher é ultrajada em sua presença.

--- Eu não fui em sua presença, mas negará também que
eu tivesse sido ultrajada?

--- Nego.

--- Oh!

\textls[15]{--- Você colocou-se numa posição falsíssima em casa de
seu tio; mais de uma vez eu lhe disse que as suas manifestações
repetidas de piedade para com os escravos poderiam ali ser tidas como um
desafio. A sua atitude enérgica e bondosa era com certeza mal
interpretada, à vista dos tristes precedentes da
família\ldots{} velhos rancores, coisas que a gente moça
esquece e que a gente antiga conserva na memória!}

\textls[-25]{--- Velhos rancores! Mas que tenho eu com isso? Serei porventura
a causa desse ódio misterioso que tantas amarguras deu a meus
pais?}\looseness=-1

--- Não sei; conheço tanto como você essa história.
Algum motivo grave deveria tê-la determinado, apesar de que,
nesse tempo, as rixas nas famílias eram frequentes\ldots{} mas
não se trata agora disso; a verdade é que você foi imprudente.

--- Não me arrependo!

E a verdade é também que terá em breve um
protetor natural, um marido que a resguarde de todas as calúnias
e traições. Tive escrúpulos em mandar hoje a sua carta ao dr.\,Azevedo;
mandá-la-ei amanhã e o seu desejo será cumprido!

\textls[25]{Paulo tirou a carta do bolso, releu-a alto e perguntou, muito
pálido e comovido:}

--- Não está arrependida, Eva?

A moça continuou silenciosa.

--- Persiste em oferecer a sua mão ao Azevedo?

--- Persisto.

Paulo estremeceu; depois, com um sorriso contrafeito:

--- Farei o que manda.

\textls[15]{Dobrou outra vez o papel e ia guardá-lo quando Eva lho
pediu, com um gesto, estendendo-lhe a mão. Paulo obedeceu, e ela, muito
pálida, rasgou em muitos pedaços o papel e atirou-os ao chão. Sentou-se,
depois, num canto do divã.}

\textls[20]{--- Decididamente, Eva, você está demasiadamente nervosa! ---
disse Paulo, mal contendo uma expressão de alegria.}

--- Sim, estou.

\textls[15]{Paulo aconselhou-lhe prudência e um esforço para se
tranquilizar. Sem força de vontade, qualquer perigo é um abismo,
afiançava ele. Vá deitar-se e procure dormir. Isto de se entregar a
gente a um desgosto é o mesmo que entregar-se a um inimigo.
Descanse e espere, que há de ser feliz\ldots{}}

Eva levantou-se, estendeu-lhe a mão, que ele apertou comovido, e
voltando-lhe as costas seguiu para o seu quarto sem lhe dizer mais nada.
Paulo abriu a janela, sentou-se, e começou a fumar, refletindo
em tudo que ouvira e perdendo-se em conjecturas, com o olhar fixo na
treva do campo e o coração sobressaltado.

%\end{linenumbers}

\section{XXI}

%\begin{linenumbers}

\textls[-20]{O dia seguinte amanheceu chuvoso e fresco, o céu, cinzento, sem
a mais pequenina nesga azul; as gotas de chuva caíam compassadamente,
num tan-tan-tan ritmado, escorrendo nos vidros das janelas.}\looseness=-1

\textls[15]{Eva levantou-se tarde; a fadiga, o grande desgosto da
véspera, tinham-na extenuado; dormira toda a noite um sono
profundo, produzido pelo violento choque que sofrera.}

Vendo-se no seu antigo quarto, dormindo na mesma cama estreita,
para onde tantas vezes vira curvar-se solicitamente a sua querida
mãe, contemplando aquelas paredes claras, sem adornos, a
mobília elegante e leve, o alto espelho em que pintara a um
canto um bando de andorinhas\ldots{} revendo assim, ao acordar, aqueles
objetos queridos a que dera outrora tantos dos seus cuidados,
Eva sentia-se triste e alegre alternadamente e alheia à sua
situação, como se tivesse ido para ali na melhor paz do mundo.

Depois de erguida, remexeu nas gavetas; logo na primeira deparou
com o seu livro antigo de orações, de marroquim granada e fechos de
prata, presente dos pais no dia da sua primeira comunhão. Que
recordações ele lhe despertou! Como se tinha sentido feliz, nesse dia,
no seu lindo vestido de cambraia branca e véu de filó sobre a grinalda
de rosas\ldots{} fora naquele mesmo quarto que ela se vestira
assim! Palpitante de alegria, como uma pombinha alva ensaiando o seu
primeiro voo! E da gaveta iam saindo vários
objetos perfumados pela saudade de uma época distante e feliz.
Agora era um leque de sândalo já quebrado e muito pequeno, que
o dr.\,Morton lhe dera num dia de Natal, quando, ainda de vestidos
curtos, ela acreditava que as fadas andassem por este mundo a semear
pastilhas de chocolate e bonecas bonitas nas camas das crianças! Daí a
nada era uma pasta com os seus primeiros desenhos, coisas monstruosas,
igrejas mais baixas que os homens, homens mais baixos que os
bois, bois mais baixos que as ovelhas! Seguiu-se o seu segundo livro de
leitura, o \emph{Tesouro dos meninos,} já roto, amarelado, cheio de
flores secas; depois umas rendas de crochê, trabalhadas sem
capricho, na pressa que preside ao trabalho das crianças! Depois uns
retratos seus e de Paulo, cujo nome aparecia ligado a todas as
recordações; ora era a sua letra de colegial em exercícios de francês
feitos em comum, ora a certidão de batismo de uma boneca, assinada por
ele; ora uma flor dentro de um envelope com dedicatória. Passou
assim pelas mãos de Eva uma multidão de coisas fúteis, velhas, guardadas
como relíquias e que lhe ressuscitavam na memória
trechos da sua venturosa meninice. Daí foi abrir a janela e demorou-se a
olhar para as flores do jardim, todas aljofradas da chuva.

\textls[15]{O céu pardacento tinha já uns tênues laivos azuis.
Estava assim encostada a um dos umbrais, aspirando o doce aroma da
madressilva, quando ouviu dar dez horas; dirigiu-se então para a sala de
jantar.}

\textls[15]{Paulo, como bom lavrador que era, tinha-se levantado cedo, e,
apesar do mau tempo, andara já pela roça, envolto na sua larga capa de
borracha.}

Toda a narração de Eva havia-lhe causado uma grande surpresa e
uma dolorosa impressão. Julgara ter percebido nela um sentimento oculto,
forte, dominado por um orgulho de ferro.

Esse sentimento era, a seu ver, o amor pelo primo.

Indubitavelmente Eva amava Otávio e, para fugir-lhe,
procurava à pressa um marido, por mais imbecil que fosse, pronta a todos
os sacrifícios, menos ao de curvar a cabeça diante do
tio.

\textls[-10]{Era isso, era! Eles haviam de ser felizes algum dia; e ele,
Paulo, que tinha por Eva uma paixão sem limites, uma paixão
vastíssima, nascida na adolescência, avigorada na
mocidade; ele, que sonhara a sua posse como o ideal da ventura na terra,
e que em cada dia parecia sentir maior e ainda mais sólido o seu amor;
ele mesmo trabalharia para que Otávio desposasse a prima, e,
depois de os ver estabelecidos no Mangueiral, em pleno gozo do seu amor,
fugiria para bem longe, para a Europa, ou para os Estados Unidos.}\looseness=-1

\textls[20]{Decidido a intervir e a esclarecer todo o obscuro daquela
situação, Paulo armou-se de ânimo e resolveu abreviar o casamento de Eva
com Otávio.}

A má vontade do comendador Medeiros seria vencida com pequeno
esforço; para ele era ponto de fé que o velho fazendeiro não tinha, nem
poderia conservar ódio à sobrinha, e que a sua
última e incompreensível atitude para com ela não tinha
outra causa além do receio e percepção do mútuo amor de ambos.

\textls[-20]{Medeiros era ambicioso e a fortuna de Eva, conquanto
fosse boa, estava longe de ser comparada às de algumas moças
do município.}\looseness=-1

\textls[15]{Falava-se, havia muito, no casamento de Otávio com a
Sinhá, sobrinha do Antunes. Essa sim poderia satisfazer todos os
projetos gananciosos do comendador.}

\textls[15]{Paulo dispôs-se a convencê-lo, cedendo a maior
parte da sua pequena fortuna em dote a Eva.}

\textls[15]{A posição de Eva era melindrosa; queria, sem agravá-la
com desabafos de indignação, pô-la a coberto de qualquer
suspeita. A solidão e o estudo tinham-lhe apurado a reflexão; dedicado e
honesto, era entretanto capaz de um crime se dele dependesse a honra e a
felicidade de Eva; mas, como não se deixava cegar pelas primeiras
impressões, e não era violento, estava sempre certo do triunfo. Calmo,
animoso e inteligente, ia ao encontro do perigo sem blasonar
valentias, discutia com firmeza e serenidade e em quase
todas as questões saía vencedor.}

\textls[15]{Quando Eva entrou na sala de jantar, Paulo rufava com os dedos
na porta do jardim, olhando, através dos vidros, para fora,
para as árvores muito lavadas da chuva.}

\textls[-15]{O almoço, já pronto, esperava-os: sentaram-se à mesa;
pareciam ambos constrangidos; comeram e falaram pouco; servia-os uma
criada da colônia, rapariga ativa, trigueira e gorda, de dentes
muito brancos e o cabelo negro e liso, enrolado em tranças apertadas.}\looseness=-1

\textls[-10]{Acabado o almoço foram para a sala de música. Paulo
instou com Eva para que tocasse; e renovou conselhos. Que ela se devia
fingir alegre, até que nisso se tornasse pelo hábito. Que se
fizesse sentir em casa, dando ordens, tomando conta do \emph{ménage},
exercendo toda a sua atividade de corpo e de espírito em
preocupações diversas e absorventes: que afastasse da lembrança o
desgosto da véspera, tudo se arranjaria sem amofinações; a paz
e a alegria haviam de ligá-la em breve à
família Medeiros; e afirmava isso com um bom sorriso, conduzindo
a moça para o piano. Depois disse-lhe que tinha de sair e que não o
esperasse senão à tarde. Eva, sem responder a nada, automaticamente,
principiou \emph{En pleurant}, de Godard. Paulo ainda ouviu uns compassos e
depois saiu.}\looseness=-1

\textls[-10]{Deixando em meio a música, Eva chegou à janela; na
terra molhada, as rodas do carro em que Paulo partira tinham imprimido
dois sulcos fundos. Já não chovia; o céu estava agora todo
azul, o sol de brilhos refrangíveis, as folhas úmidas
das plantas, os campos batidos de luz e empapados de água
tinham uma frescura encantadora, e as cabecitas airosas das pombas,
assomando às portas do pombal a consultar curiosamente o tempo,
fizeram-na sorrir.}\looseness=-1

\textls[15]{Eva calculara que Paulo tivesse ido a Santa Genoveva; esperava-o
com impaciência, não tendo ao mesmo tempo ânimo de lhe fazer
qualquer pergunta a respeito; temia alguma coisa que não sabia definir
qual fosse e vinha-lhe impertinentemente à lembrança a frase de
Paulo, referindo-se ao acontecimento da véspera: ``Eva foi
precipitada e imprudente\ldots{}''.}

\textls[15]{Que deveria ela ter feito? Humilhar-se, triste e
resignadamente, diante do tio? Justificar-lhe as suspeitas de
sedução, aceitando o generoso oferecimento e pedido de Otávio?
E amá-la-ia ele, realmente? Voltou para o interior da saleta, sentou-se
perto da mesa e pôs-se a folhear um livro, resolvida a falar
com firmeza a Paulo sobre a conveniência de arranjar um marido.
Esteve assim pouco tempo; depois, seguindo o conselho de Paulo,
percorreu toda a casa, renovando ordens, alterando a posição de alguns
trastes e feitorando o serviço. Os cuidados domésticos conseguiram
muito; Eva distraiu-se.}

%\end{linenumbers}

\section{XXII}

%\begin{linenumbers}

\textls[15]{Eram duas horas da tarde quando Paulo chegou a Santa Genoveva;
apeando-se do trole fechado em que fizera a viagem,
entrou no corredor da casa de Medeiros. Uns moleques de camisolas de
algodão grosso, barrigudos e ranhosos, correram ao vê-lo, para
o interior, e de uma sala próxima saiu Noêmia desprevenida; ao deparar
com Paulo, recuou admirada, fazendo-se vermelha; mas, esquecendo
depressa a sua própria comoção, aproximou-se e perguntou-lhe
com os olhos rasos de lágrimas:}

\textls[15]{--- Como está Eva? Por que não volta?! Tenho sentido tanta
falta\ldots{} nem imagina! Ralhou com ela, não?}

--- Eu não tenho o direito de ralhar com Eva, minha
senhora\ldots{}

--- Mas se tivesse, ralharia, não é assim?

--- Não\ldots{}

--- Não?!

--- Eva fez o seu dever!

--- Meu Deus do céu! Eu não compreendo nada! --- disse Noêmia,
erguendo para o moço os seus ingênuos olhos castanhos.

--- Nem eu.

--- Nem o senhor?! Então\ldots{}

--- Foi para me esclarecerem que eu aqui vim.

--- Ah! Quer falar com pai?

--- Exatamente!

\textls[-30]{--- Que pena! Ele não está aqui! Foi ao sítio
do compadre Antunes.}\looseness=-1

--- Posso esperar\ldots{}

--- No quarto de Otávio?

--- Onde determinar!

\textls[15]{--- É mesmo melhor ir para lá! Coitado, ele está só\ldots{}
Olhe, faça o favor de entrar\ldots{} por aqui.}

\textls[15]{Paulo obedeceu, seguindo o vulto airoso e ligeiro da moça e ia
a compará-la com uma avezinha, com uma rosa perfumada e fresca
entre sarças secas e sem odor.}

\textls[-5]{Depois de introduzir Paulo no quarto do irmão, Noêmia fugiu para
a antiga saleta de estudo, fechou-se por dentro e, toda contente e
trêmula, sentou-se num canto, apertando o coração com as mãos; fazia
aquilo inconscientemente; desejava estar ao pé de Paulo, ouvir-lhe a
voz, vê-lo, sorrir-lhe; e, no entanto, afastava-se dele, dele,
que vinha pela primeira vez a Santa Genoveva, e com quem desejava estar
todos os dias! Tinha ímpetos de voltar para o seu lado, mas
abstinha-se medrosa e pensava: ``Ele deve achar-me tola e
insípida, há de dizer que eu não sei falar\ldots{} que sou uma
criança!''. E repassava na memória as coisas que
dissera e que ouvira, analisando-as atrapalhadamente, num
delicioso embaraço.}\looseness=-1

Entretanto, Paulo e Otávio conferenciavam
acerca do ocorrido entre Eva e o comendador.

\textls[20]{O primeiro instava por uma satisfação peremptória, exigindo a
exposição límpida do caso. O segundo, deixando transparecer o desgosto
que a recusa da prima lhe causara, relatava, com minudências de auto,
tudo o que ouvira na noite da revolta e o inabalável propósito
em que estava de descobrir o forjador da intriga. Demonstrava uma tênue
esperança de que, trabalhando para demonstrar a inocência de
Eva, ela compreendesse o seu sentimento e cedesse ao seu pedido.}

\textls[20]{Essa confissão fez com que Paulo recuasse do seu
propósito. Julgando Eva apaixonada pelo primo, não a poderia
aproximar dele sem que dele tivesse partido esforço para isso.
Deixar-lhe-ia nessa campanha toda a parte simpática, embora
exigisse, por sua honra, uma explicação imediata do fato.}

\textls[20]{Otávio recapitulava as cenas desde o dia da sua
chegada; ligava ideias a coisas antes confusas e despercebidas;
procurava com aflição a ponta da meada terrível.}

\textls[15]{--- Tudo isso --- concluía ele --- parece-nos
de um romanesco falso e completamente fora do nosso tempo;
entretanto, a verdadeira culpa é exatamente da época que
atravessamos. Os escravocratas estão sempre prevenidos contra os
abolicionistas e prontos para acusá-los. Veem eles, na
sua frase mais simples e sincera, uma trama oculta ou uma
insinuação maligna.}

\textls[20]{Os proprietários de escravos têm geralmente uma compreensão
muito errada do seu tempo; não tratam de averiguar de onde parte a razão
nem em que se baseia a moral.}

Não achando argumentos seguros para a defesa das suas
ideias, não procuram batalhar, mas esmagar os que lhes são
contrários. Desta maneira atiram sobre os abolicionistas os
mais vexatórios insultos. Não é já preciso trabalhar pela causa
da liberdade, basta manifestar qualquer pessoa um sentimento
piedoso por algum escravo para ser imediatamente tido como suspeito.

\textls[20]{A mulher brasileira, bem sabe, não tem mostrado coração
neste sentido.}

É triste, mas é assim.

\textls[15]{Eva, apesar de não estender além da família a sua
propaganda a bem dos escravos, é para mim um assombro! Todas as outras,
que fazem? Onde escondem a lágrima da compaixão, que ninguém a vê?
Decididamente, à vista disto tudo eu vou descrendo da tão
apregoada bondade da mulher.}

\textls[-10]{Foi exatamente essa piedade para com os cativos que prejudicou
Eva. Fosse ela uma indiferente, como todas as outras, e
absolutamente ninguém se lembraria de envolver nisso o seu nome.}\looseness=-1

Influenciados pelo meio em que vivem, estremecidos na sua
fortuna e dominados pelos hábitos antigos, os escravocratas não medem o
mal que fazem à sua causa com os grandes alaridos.

Em vez de transigirem, ao menos num ou noutro ponto, redobram de
severidade, querendo pôr freios à liberdade que principia a agitar-se,
sem perceberem que, quanto mais apertado for o freio, mais
indômito é o desejo de correr! Parece que à proporção que vai entrando a
luz no espírito do preto, se vão condensando as trevas no
espírito do branco, porque o medo é negro!

\textls[-3]{Meu pai, terminava Otávio, nasceu entre escravos; foi
educado a ouvir e a ver, desde os mais tenros anos,
cenas da escravidão; afez-se a isso. Foi como senhor de escravos
que adquiriu fortuna, sem cogitar nunca se a lei que lhe garantia essa
propriedade era injusta ou não. Agora, há poucos anos,
levantou-se uma voz a protestar contra uma coisa, para ele natural
à força do hábito: atrás dessa voz veio outra, e logo outras
mais que o desvairaram completamente. Os lamentos e as súplicas de Eva
acabaram de irritá-lo. Sem compreender a imprudência
em que incorria, levada unicamente pelos impulsos do coração, ela
intervinha com frequência em favor dos cativos, de quem captou
simpatias. Essa intervenção foi desastrosa. Contando com o seu
apadrinhamento, os negros amiudavam faltas a tal ponto que meu pai se
viu obrigado a negar-lhe as graças pedidas. Nasceu talvez daí um pouco
de ressentimento, e desse ressentimento a mútua
desconfiança. Foi isso que, em parte, concorreu para o receio pueril que
meu pai nutre acerca do mal que Eva lhe queira causar.}\looseness=-1

\textls[-5]{Deve-se-lhe, porém, fazer a justiça de não ter partido dele
semelhante suspeita; aceitou-a de outra pessoa que, sabendo
naturalmente qual o estado do seu espírito, lançou-a nele. Com que fim?
É o que não posso adivinhar! O que afirmo é que há quinze anos
não teria com certeza sucedido assim. Certas coisas fúteis tornam-se
às vezes graves, conforme a maneira de serem tratadas ou a
ocasião em que acontecem. Ora todo o mal do nosso caso está na época que
atravessamos, repito; pelo menos é a minha convicção. A
província vive da lavoura e entende que esta depende
exclusivamente do escravo; faltando o escravo, arruína-se o
país!}\looseness=-1

\textls[15]{Daí o terror e, por conseguinte, o medo que em geral os
lavradores têm a tudo que lhes pareça reforma, e a todos que apregoem
alto a necessidade dela!}

\textls[-15]{Na perseguição do abolicionismo veem não a indignação nem a
piedade por uma raça submetida e infeliz, mas a inveja dos seus bens e o
desejo feroz de os arruinar! Hoje não se raciocina; aceitam-se todas as
ideias, por mais absurdas, desde que sejam contrárias às dos inimigos.
Os lavradores têm o espírito obcecado e doente, e nessas condições não é
difícil germinar neles um sentimento errôneo.}\looseness=-1

\textls[15]{Paulo discutiu com Otávio longamente; por fim,
resolvido a ceder-lhe toda a parte simpática na defesa de Eva,
levantou-se, instando com ele para que abreviasse a reconciliação da
família e definisse a situação em que se achavam. Por sua vez
prometeu auxiliá-los despendendo toda a sua influência
com Eva; mas, como o caso não comportava adiações, urgia deliberar e
agir; e mostrava a inconveniência dos comentários
prováveis, quando na cidade soubessem ou desconfiassem do fato!}

\textls[15]{--- Ninguém lá fora saberá do motivo por que
Eva saiu daqui --- afirmava Otávio ---, e nem
é para estranhar que ela passe algum tempo em uma propriedade
sua! Sejam discretos, que eu me encarrego de uma explicação que os
há de satisfazer.}

--- E até lá?

--- É preciso paciência\ldots{}

\textls[15]{--- Impossível, meu amigo! A posição de Eva é falsíssima; não é
justo que cruzemos os braços à espera de uma satisfação que pode chegar
tarde\ldots{}}

--- Que não! --- afirmava Otávio; tudo seria
resolvido em poucos dias; ele já se sentia com forças e sairia nessa
mesma tarde, em um pequeno exercício, para no dia seguinte poder ir
à cidade\ldots{} e pedia instantemente a Paulo que não falasse
com o comendador, aconselhando-lhe e a Eva uma atitude reservada.

Quando se separaram eram quatro horas. Paulo atravessou sozinho
a sala e o corredor, subindo para o trole, sem reparar em uma
janela cerrada, de onde o espreitavam os olhos de Noêmia.

\textls[20]{Não ia satisfeito. Desagradara-lhe a retórica doentia de
Otávio e a sua perturbação de namorado piegas; ou
fosse a voz do ciúme que falasse nele, ou indignação
por não achar mais energia em Otávio para a defesa de
Eva, o caso é que ia desconsolado, ponderando as palavras que ouvira.}

\textls[20]{Era evidente que Otávio e Eva se amavam. Para abreviar
um desenlace feliz àquele afeto recíproco, ele, Paulo, devia
exatamente fazer o que prometera: deixar toda a ação a
Otávio e encolher-se no seu desgosto.}

\textls[15]{Refletiu durante muito tempo nisso; depois convenceu-se de que
houvera precipitação de parte a parte, e que talvez o caso não tivesse a
gravidade que lhe queria dar.}

\textls[10]{Distraiu-se, logo que atravessou as terras do Mangueiral.
Agora, à esquerda, via os telhados vermelhos das casas da
colônia, com os terrenos em frente bem cultivados; daí a nada,
os cafezais muito extensos, a perderem-se de vista de um e
outro lado, em longas ruas simétricas; depois um rancho de
trabalhadores, homens e mulheres, rijos, alegres, com os pés enterrados
na lama, e a vibrarem a enxada com destreza e coragem; umas
crianças risonhas deram-lhe vivas ao passar; deixando os
cafezais, ladeou o pasto de um verde de cetim, onde o
gado punha manchas brancas e pretas; depois, o grande muro baixo do
pomar, os caramanchões do jardim, as ruas das magnólias, os
canteiros de flores e a parede lateral da casa, toda coberta pela
folhagem espessa da silvina. Que diferença entre o Mangueiral e as
outras fazendas dos arredores! Comparava-a com a de Medeiros! Santa
Genoveva era o que são geralmente todos os sítios paulistas: grande
casarão, cercado de terrenos sem cultura, nem flores nas
janelas, nem uma árvore a sombrear-lhe a porta! Isolada como uma
sentinela medrosa, a olhar com desconfiança para todos os lados! A
horta, o jardim e o pomar, acumulados numa só área, separada da
habitação por um longo terreiro nu, cheio de calor. No interior da casa,
o mesmo desconforto; raros móveis, nenhuma elegância. No
corredor, negrinhos em camisa; na sala de jantar, comprida e
sombria, algumas redes, uma mesa e uns bancos toscos. Em
frente, o terreiro de café, atijolado; muito sol batendo na terra seca,
a refletir uma cor amarelada das paredes das senzalas; quartos
sem luz, deitando para um corredor onde as janelas têm grades como as
prisões!}

\textls[-10]{No eito os cativos, nus da cintura para cima, luzidios de suor,
levantando penosamente a enxada, a olharem de revés para o feitor, até
virem numa gamela o feijão e o angu do jantar. Era então a hora
do repouso; sentavam-se no chão e devoravam aquele alimento, enchendo
muito as colheres de ferro ou de estanho.}\looseness=-1

\textls[20]{Como se destacava de todas elas o Mangueiral, onde os
trabalhadores eram livres, a cultura feita sem a rega das lágrimas, as
habitações dos colonos claras, arejadas e limpas, a casa da proprietária
atufada em verduras, coberta de perfumes e de sombras doces!
Seria menos rendosa, talvez, mas era com certeza muito mais
agradável. Era a fazenda futura, salientando-se entre todas as
outras, rotineiras e estúpidas. Tinha, como poucas, abundância
de frutas, de hortaliça e de água, e, sobretudo, muita paz e muito
contentamento! \label{Ref17}}\looseness=-1

\textls[-17]{Era discutida pelos vizinhos, uns carranças, que a lamentavam,
censurando-lhe a administração; afirmavam que daria o triplo se a
soubessem dirigir; faltava-lhe o elemento principal: o escravo, que
trabalha indubitavelmente mais que o branco, quando debaixo do respeito
de um feitor severo; faltava-lhe economia, rotina, e sobejavam-lhe os
seus ramos de rosas modernas, as suas árvores frutíferas e
ornamentais, o conforto do lar bem adornado, ``à moda da cidade'', a
hortaliça e o leite que distribuía à farta pelos trabalhadores; todas as
regalias, enfim, que permitem forçosamente propriedades dessa
ordem. No entanto, o Mangueiral continuava a enfeitar-se e a
prodigalizar vantagens aos seus empregados.}\looseness=-1

\textls[15]{Os vizinhos riam-se a cada novo melhoramento posto em
prática, e levantavam desdenhosamente os ombros; não sentiam,
infelizmente, inveja, porque se a experimentassem procurariam talvez
seguir o mesmo sistema, fazendo assim um grande benefício à sua
educação. Paulo tinha pena de se ver isolado em meio de tantas
terras fertilíssimas e belas, escravas ainda de uma direção
pervertida, produzindo espantosamente, mas sacrificando milhares de
homens à sua produção.}

As grandes propriedades comparadas com o Mangueiral faziam-no
sorrir; nelas, fortunas acumuladas, safras enormes, mas
extrema miséria nos degradantes quadros da escravidão; no
Mangueiral, a mediania farta, o bem-estar espalhado desde a habitação
principal até à ínfima; em tudo o direito, a razão, a
justiça! Para o lavrador de coração, que se prende com amor à terra que
cultiva, não será essa felicidade, porventura, a mais consoladora?

Paulo pensava assim, quando o trole parou à
porta da casa.

Eva veio-lhe ao encontro.

--- Vem de Santa Genoveva?

--- Sim, venho de lá\ldots{}

--- Otávio está melhor?

--- Está quase bom\ldots{}

--- Falaram, já se vê, a respeito do que ontem aconteceu?

--- Certamente.

--- E então?

--- Em breve estarão as pazes feitas e você tornará para
lá\ldots{}

--- Nunca; quero viver sempre aqui.

--- Ou isso; mas não desta maneira\ldots{}

--- De qual?!

--- Casada.

--- Mudei de ideias.

\textls[15]{Paulo replicou: --- Deixe-se disso! Os ressentimentos passam.
Ainda havemos de ver o comendador muito seu amigo\ldots{}
Otávio é um rapaz de coração e ama-a profundamente!}

Eva levantou para Paulo um olhar cheio de espanto; este, mudando
de tom e procurando fugir, disse com um sorriso forçado:

--- A senhora dona da casa mandou preparar para hoje um
banquete, não é verdade?

\textls[15]{Eva não respondeu e caminhou para dentro, e ele, precipitando-se
para o seu escritório, deixou-se cair em uma poltrona e
escondeu o rosto entre as mãos.}

%\end{linenumbers}

\section{XXIII}

%\begin{linenumbers}

\textls[5]{Depois de muito tempo de reclusão, Otávio dirigiu-se
para a casa do seu velho amigo Morton, numa bela manhã de sol. Foi
encontrá-lo a escrever na sua larga secretária coberta de papéis.
Conferenciou longamente com ele; precisava orientar-se com as suas
informações, fê-lo repetir a mesma história que já lhe
ouvira, escutando-o então com muita serenidade, a pesar as palavras uma
por uma. Mas o velho exigiu-lhe também, com todo o seu direito
de amigo, a narração do sucedido em Santa Genoveva, e Otávio
referiu minuciosamente o caso, não ocultando mesmo a parte referente a
Eva e expondo as suas intenções.}

\textls[20]{Morton meditava, recolhido, com as mãos nas algibeiras, o olhar
fixo num ponto, o corpo recostado na cadeira redonda e gradeada. Depois
de um pequeno silêncio:}

--- E que conta fazer?

--- Descobrir os capangas e arrancar-lhes à força a verdade.

--- Como espera encontrá-los?

--- Isso é que ainda não sei!

--- Fixou bem as feições de algum deles?

--- Não; era noite, como sabe, e de mais a mais muito
escura; mas ouvi as vozes.

\textls[25]{--- Isso não é suficiente: pode falhar\ldots{} Lembrei-me de uma coisa\ldots{}}

--- Qual?

\textls[20]{--- Não é muito verossímil, mas enfim, é
possível\ldots{} Lembrei-me, digo, que andasse envolvido nisso o
Antunes!}

--- O Antunes?!

--- Sim, o Antunes; de que se admira?

--- Conheço-o há muitos anos: sei que é
incapaz disso! --- respondeu Otávio.

--- Eu ainda o conheço há mais tempo! Fique sabendo.

--- Pelo amor de Deus, doutor! Pois não vê que isso é absurdo!

--- Não é tal absurdo\ldots{}

--- Que interesse teria o pobre homem em intrigar Eva?

\textls[15]{--- Vocês estão muito enganados com ele, é o que é! O Antunes é
mau, é ignorante e é covarde! Toda esta história está-lhe
perfeitamente no gênero! Creia, meu amigo, que eu não falaria assim sem
ter motivos. Já o pilhei numa coisa semelhante, eu! Ora aí está! Além
disso ele é a única pessoa que pode ter empenho em
retirar Eva de Santa Genoveva\ldots{}}

--- Por quê?!

--- Por quê! Olhe que ingenuidade! Consta por aí que o
homem cobiça-o para marido da sobrinha\ldots{}

--- Parece que sim\ldots{}

--- Então percebe?

--- Não percebo nada!

\textls[-20]{--- É boa! Pois não vê que a estada de sua prima na fazenda
pode prejudicar a causa da outra? Eva é inteligente, é bonita,
tem uma educação muito pouco vulgar; é portanto perigosa! Nada mais
natural do que você apaixonar-se por ela, e isso é que ele não
quer!}\looseness=-1

--- Isso seria fútil! Perfeitamente insensato! Desculpe-me,
doutor, mas não aceito a sua hipótese!

\textls[-20]{--- Pois faz mal! Conheço o Antunes há uns trinta anos;
tenho razões sérias para desconfiar dele! Era inimigo encarniçado de
Gabriel Medeiros e estendeu o seu rancor até à filha. Teve há
pouco tempo, aí há coisa de um mês, se tanto, uma
pendência com o Azevedo a propósito de um escravo que esse moço
remiu com dinheiro de Eva. Um filho natural de Antunes, que morreu
às mãos dos escravos e de quem ele herdou o sítio que hoje
administra, teve a veleidade de pedir a mão de Eva. Olhos no dote, está
visto; ela era ainda uma criança! A família opôs-se e
ele engoliu a tábua!}\looseness=-1

\textls[20]{Parece-me que tudo isto junto deve ter influído para a
concepção e realização da história que acabou de contar.}

--- Sempre tive o Antunes em muito boa conta!

\textls[-20]{--- Pois se eu estivesse em seu lugar procuraria saber
quais são os capangas dele, e então talvez não fosse difícil
descobrir a verdade.}\looseness=-1

\textls[-10]{--- Nesse caso seria mais leal ir diretamente a ele e
interrogá-lo.}\looseness=-1

\textls[20]{--- Seria uma imprudência, meu caro! Enfim,
faça lá o que entender; contudo, fique certo que com
aquele sujeito é melhor lidar-se com manha do que com lealdade.}

A conversa prolongou-se por mais de uma hora.

Saindo da casa de Morton, Otávio retirou do correio a
correspondência e seguiu para o Monjolo, fazenda do Antunes.

Fazia calor; a estrada sem sombra cortava grandes campos
esbranquiçados e maninhos, cobertos de barba-de-bode.

Otávio, moído do sol e da viagem, fechava os olhos
quase adormecido, quando o trole, dando uma volta,
desceu aos solavancos até um milharal, que atravessou depois, devagar,
esmigalhando sob as rodas, num ruído áspero e surdo, a folhagem
seca que atapetava o chão. De um lado e do outro pendiam dos
arbustos, numa atitude de cansaço, como espadas vencidas, as
folhas do milho, e em cima as espigas erguiam-se envoltas nas suas capas
de palha já amarelecidas, com as espessas borlas dos corutos soltas,
como penachos. Passada a roça de milho entrou de novo o trole
numa estrada, sombreada a espaços por umas árvores de copa achatada e
larga. Ao fim de uns quatrocentos metros, adiante de umas piteiras
inclinadas dum barranco, aparecia uma casa rústica, com duas portas na
frente, paredes mal caiadas, teto baixo, um terreiro ao lado cercado de
pedaços grosseiros de pau enleados de cipó, com duas laranjeiras da
terra, uma bananeira a um canto e uma goiabeira raquítica com os galhos
estrangulados na cerca. Era uma venda, e cantavam lá dentro ao
violão. Otávio julgou reconhecer aquela voz e mandou parar o
trole. Entrou.

\textls[-10]{O taverneiro, com as mangas arregaçadas, fazia em letras
garrafais uma conta do café roubado pelos escravos da vizinhança aos
senhores, e vendido a ele por uma ninharia: era o seu comércio e o que o
enriquecia à sombra de meia dúzia de garrafas de
cerveja nacional, que vendia a custo a um ou outro viandante.
Otávio pediu-lhe aguardente, água e açúcar, e
enquanto o vendeiro, debruçando-se sobre o balcão
ensebado em que pregara duas moedas de cobre, o servia, ele
passeava curiosamente o olhar pelo recinto.}\looseness=-1

\textls[20]{Numas três prateleiras toscas, laterais, havia muitas
botijas e garrafas vazias, uma ou outra com cerveja ou conhaque;
juntamente peneiras de diversos tamanhos, chapéus de palha grossa, uns
rolos de fumo e, num caixote, muito sabão de cinza, preto, envolvido
às bolas em palha de milho.}

\textls[-20]{Ao fundo, sentado num barril de décimo, estava um caboclo alto,
magro, de grande cabeleira desgrenhada a cair-lhe nos ombros, barba
falhada e rente, olhos lânguidos, camisa aberta mostrando o peito
cabeludo, lenço mal atado ao pescoço, calças de algodão mineiro, e
grande faca ao lado, na bainha de couro, entalada na cinta.}\looseness=-1

\textls[10]{Ao pé, sentada no primeiro degrau de uma escadinha que descia
para o interior, uma caipira moça, de quem não se via senão o busto,
alisava com um pente de búfalo os cabelos corredios e negros\ldots{} Ela
sorria para o violeiro; ele prosseguia cantando, a bambolear o
corpo sobre o barril de décimo:}

\begin{verse}
\footnotesize
Santo Antônio amarra negro\\*
na beirada da capoeira:\\*
Todos santos 'stão quieto,\\*
Santo Antônio está com asneira!\\!

Madama de Campina\\
me mandou chamá,\\
madama de São Paulo \\
não quer que eu vá!

Tenho meu chapéu de palha,\\*
Me custou mil e quinhento;\\*
Quando boto na cabeça \\*
Não me farta casamento!\\*
\end{verse}

\textls[15]{--- Cante a outra, a do patchuli\ldots{} --- pediu a caipira;
ele acedeu logo, e mudando de tom, recomeçou:}

\begin{verse}
\footnotesize
Chinelo de tapete \\*
Forrado de marroquim,\\*
Relógio de ouro,\\*
Corrente de trancelim,\\!

Lenço branco n'algibeira,\\
Cheirando patchuli\footnote{Na grafia ``patchoulim'', do começo 
do século \textsc{xx}, ficava mais evidente a rima com ``assim'', 
do último verso. [\textsc{n.\,e}.]}\ldots{}\\
Ai, ai, meu bem,\\
Se eu pudesse andar assim!
\end{verse}

\textls[15]{Otávio deu tempo a que acabasse a canção; o caipira tangeu
ainda alguns compassos até o estalido do acordo final, que abafou logo
com a mão espalmada sobre as cordas.}

\textls[15]{--- Cante outra moda, Nhó Quim! --- pediu a moça num tom arrastado e
langoroso.}

--- Mais logo, Nhá Tuca!

--- Eia! Mecê hoje tá muito sarambé.\footnote{\emph{Sarambé}:
  tolo, parvo.}

\textls[15]{--- Não tô não! Di noite nóis vimo cá tudo junto pra i fazê
um cateretê rasgado! Eta nóis!}

--- Que não esqueça a sanfona! --- observou o taverneiro.

--- Eu empresto ela pra Cacuta e venho
tocá\ldots{}\footnote{Os paulistas dizem geralmente \emph{emprestar}
  por pedir emprestado; assim é frequente ouvirem-se frases como estas:
  ``F. mandou-me emprestar o meu cavalo''; ``Eu não tinha casa, mas
  emprestei a do meu amigo'', etc.}

\textls[15]{--- É perciso arranjá as bugia}\ldots{}\footnote{Os
  caipiras chamam ``bugia'' ao lampião.} \textls[15]{--- observou a
  moça.}

Pondo de lado a viola, o caipira ergueu-se e fez um cigarro,
enquanto Otávio pagava as despesas. Saíram ao mesmo
tempo. Ladearam o cercado do quintal da venda, onde algumas galinhas
cacarejavam soltas, depenicando os seixinhos e as ervas
rasteiras.

\textls[15]{O caboclo parecia ter adivinhado a intenção do moço: seguia-o de
perto, esperando ser chamado.}

Assim foi; Otávio, voltando-se para ele, disse:

--- Vai fazer-me um favor.

--- Diga.

\textls[15]{--- Quero saber se foi vossemecê quem me atirou
há vinte dias um tiro num ombro.}

--- Pode sê\ldots{}

--- Eu quero a certeza.

--- Se explique meió\ldots{}

--- Já me expliquei: quero saber quem me feriu!

--- Ah! Foi mecê o atirado?

Otávio, enchendo-se de paciência, relatou as suas
intenções ao caboclo, propondo-lhe boa paga.

\textls[-10]{O outro, depois de um curto momento de reflexão, murmurou:}\looseness=-1

--- Tá dereito! Ele me enganou, eu não encubro ele!

--- E quem foi esse \emph{ele}?

--- Pois foi memo nhô Furtuoso!

--- E quem é nhô Frutuoso?

--- Ué! Mecê não conhece ele não?!

--- Não.

\textls[-10]{--- É um capanga destemido, e tem fama por todo esse
sertão\ldots{}}\looseness=-1

\textls[20]{Contou então que o Frutuoso morava em terras do Antunes, indo
muitas vezes pousar lá no sítio e que o chamara com outros para o
serviço em Santa Genoveva.}

Otávio seguiu esperançado para a fazenda do compadre e do pai.
Depois de meia hora de caminho abriu a porteira do Monjolo.
Tudo vulgar; pastos cheios de vassoura, cafezal maltratado e habitação
pobre, baixa, suja, isolada num terreno manchado e árido. A pequena
distância o tanque, brilhando ao sol, como uma grande placa de aço, e
na cerca da horta muita roupa a secar.

\textls[-10]{Otávio entrou para uma sala térrea com duas janelas sem
caixilhos, onde, por toda mobília, havia uma mesa de pinho e
quatro cadeiras. A um canto estava um cacho de bananas a amadurecer; no
batente da porta uma gaiola com um sabiá; e embaixo da mesa, encostada à
parede, uma caixa de folha com a tampa já amolgada. Sobre uma prateleira
uma garrafa de mel de fumo, para golpes e mordeduras, uma caixinha de
pau com mercúrio-doce de Lisboa, para curativo das bicheiras, e
uma lata de unguento, para feridas. Ao lado do cabide, um rabo de tatu e
um chapéu do Chile. Num outro canto três perobas brancas, para
cabos de relho.}\looseness=-1

Nada mais.

\textls[15]{--- Sinhô tá na roça --- informou uma preta, nova
e asseada, a Otávio. Este decidiu-se a esperá-lo.}

A caseira espreitava-o, passando, sem o cumprimentar, na varanda
próxima, com um prato na mão, a caminho da cozinha; de volta, tornou a
observá-lo, atrevendo-se a entrar na sala em que Otávio
estava, não tendo pejo de lhe mostrar os pés sem meias, nuns chinelos de
trança, e o corpo sem colete num vestido de chita; mas arrependendo-se
logo, voltou depressa para dentro e pôs-se a
espreitá-lo pela greta da porta lateral. Reinava o maior
silêncio. O sol refletia-se abrasadoramente nas paredes da
casa, os galos cantavam de vez em quando, ao longe, e o sabiá depenicava
a metade da laranja, posta no chão da gaiola.

\textls[-10]{A mucama tornou a entrar com a bandeja do café; Otávio
bebeu maquinalmente; depois, voltando-se para a preta, perguntou:}\looseness=-1

--- Sabe dizer-me se o Frutuoso está hoje aqui?

--- Seu Furtuoso? Ele foi fazê viage\ldots{}

--- Quando partiu?

--- Nhô?

--- Quando se foi embora?

--- Di já hoje\ldots{} di menhãzinha.

\textls[15]{A mucama saiu; os galos cantavam ao longe, o sabiá, muito
sonolento, subiu para o poleiro; e Otávio, contrariado pela
ausência do Frutuoso, sentia que o espreitavam ainda pela porta
da esquerda, semicerrada.}

\textls[20]{Veio quebrar aquela monotonia a rude voz do Antunes, que se
aproximava, dizendo alto a outra pessoa:}

\textls[15]{--- É assim mesmo. O Frutuoso lá se foi para Casa
Branca\ldots{} pelos modos não vorta tão cedo!}

--- E o Navarrinho?

--- Também foi!

--- Pra Casa Branca?

--- Sim senhor.

--- Cuns diabos! O negócio então
parece que está feio\ldots{}

--- É exato.

Entraram.

\textls[15]{O Antunes deitou uma alegre exclamação ao deparar com
Otávio e apresentou-o ao amigo, um lavrador gordo e trigueiro.
Este, depois de trocar meia dúzia de palavras, tornou a referir-se
à ida do capanga Frutuoso para Casa Branca,
mostrando-se contrariado com isso; precisava do homem, afirmava.}

\textls[15]{Serviram de novo café. A conversa tomou diversos rumos.
Otávio levou jeitosamente o assunto para a revolução havida em
Santa Genoveva, das suas causas e efeitos, sem perder de vista a
fisionomia do dono da casa, em cujas feições procurou em vão estudar a
culpabilidade.}

No largo rosto do amigo não transparecia a mínima
comoção. Otávio, entristecido, recolheu-se a Santa Genoveva.
Logo ao entrar em casa soube pelo pai que o Trigueirinhos escrevera
convidando-o para, com toda a família, assistir no seu sítio à
inauguração da sua máquina Paulistana, das oficinas
Lidgerwood.

O comendador estava contente e Otávio resolveu no mesmo
instante ir também. O acaso favorecia-o.


%\end{linenumbers}

\section{XXIV}

%\begin{linenumbers}

\textls[15]{Marcaram a partida para dali a dois dias, enquanto a
família não tinha prontos os seus arranjos de viagem.}

\textls[-20]{Noêmia alegrou-se com a ideia do passeio e de ver a
irmã; não lhe tinham tornado a falar no projetado casamento com
o filho do Cerqueira Franco, e ela supunha estar tudo definitivamente
acabado com a intervenção de Otávio. No entanto, não era assim.
Nessa mesma tarde, a mãe foi dizer-lhe, a mandado do marido,
que os Francos, pai e filho, iriam ter a Casa Branca, convidados pelo
Trigueirinhos, a pedido do comendador; Nicota daria um jantar aos noivos
e marcariam aí a data para as bodas. Falou logo em encomendar
roupas brancas, vestidos, joias, todo o enxoval bonito e rico, para não
haver delongas nem aborrecimentos! Era a vontade do pai.}\looseness=-1

\textls[-15]{Ouvindo tais palavras, a pobre menina chorou; a mãe
demorou-se um pouco a seu lado, procurando talvez uma palavra de
consolo, que não lhe chegou, nem como um fraco murmúrio, aos
lábios.}\looseness=-1

\textls[15]{Durante a noite Noêmia sentiu-se febril e levantou-se na manhã
seguinte abatida e pálida.}

O irmão procurou animá-la, prometendo-lhe novo
auxílio.

\textls[-10]{Depois do almoço Otávio saiu; sabia que se retirava o
dr.\,Castro para Pernambuco e quis agradecer-lhe os seus cuidados;
projetava ir também ao Mangueiral, julgando da sua
lealdade confiar a Paulo os seus planos, explicando assim a sua ida a
Casa Branca.}\looseness=-1

Quando chegou à fazenda do vizinho, era meio-dia.
Mandaram-no entrar para a varanda. O dr.\,Castro, entre a dona da casa e
meia dúzia de crianças tagarelas, chupava laranjas, amontoadas
numa bandeja sobre uma cadeira. Ele descascava-as com um canivete bem
afiado; a casca da fruta desenrolava-se numa longa fita encaracolada e
estreita, que ia caindo a seus pés. A laranja girava nos seus
dedos como um pequeno globo cor de leite, sem
a mais pequena arranhadura na película; ele partia-a em duas metades, à
moda paulista, e oferecia-a à dona da casa ou às crianças, que o
cercavam de perto, com sentido na fruta.

\textls[-25]{Vendo entrar Otávio Medeiros, puxaram uma cadeira para
a roda, fazendo-o sentar-se e felicitando-o pelo seu milagroso
salvamento.}\looseness=-1

A família Carvalho era amável; Otávio
sentia-se bem.

Ouvindo de seu cliente o motivo daquela visita, o dr.\,Castro
confirmou a notícia de partir em breve para Pernambuco, mas
para voltar com toda a família.

A hospitaleira província de São Paulo prendia-o, dizia
ele, e resolvera mudar para ela a sua residência.

E falaram em viagens, progressos materiais do país, em diversas
coisas, ligadas entre si. Ao cabo de uma hora, Otávio
dispunha-se a sair quando entrou o dono da casa, o sr.\,Hipólito de
Carvalho, tipo de lavrador, acentuadamente bondoso.

\textls[-5]{O fazendeiro vinha impressionado com umas desgraças sucedidas a
um colono alemão da vizinhança. O patrão tratava-o como escravo, não lhe
poupando mesmo o tronco. Isso indignara-o. Não o devendo recolher em
casa, ruminava a melhor maneira de o proteger. Subitamente,
voltando-se para Otávio, disse:}\looseness=-1

--- O senhor é que poderá fazer alguma coisa pelo infeliz.

--- Da melhor vontade! Mas de que maneira?

\textls[20]{--- Arranjando-lhe um lugar no Mangueiral. Sua prima
estará por isso?}

--- Talvez\ldots{} --- respondeu Otávio, de um
modo indeciso.

\textls[15]{--- Falei-lhe no Mangueiral --- continuou o fazendeiro
---, porque julgo preferível a todos os outros esse cantinho do
paraíso; nele melhor que em outra qualquer parte teria o pobre
diabo compensações para o que sofreu. E depois é mesmo provável que d.\,Eva, 
agora que está para casar-se, não rejeite fazer um
benefício desta ordem; fale-lhe, meu amigo!}

\textls[-20]{Otávio ouviu imóvel as últimas palavras,
sentindo uma angústia silenciosa e profunda. Entretanto, a
esposa de Carvalho perguntava:}\looseness=-1

--- Quê? A Eva Medeiros vai casar! Com quem?!

--- Com o dr.\,Azevedo --- respondeu-lhe o marido
---; pelo menos é o que ele próprio fez constar ainda
ontem à noite no clube, em uma roda de amigos.

\textls[15]{--- Ora, quem tal diria! --- exclamou a fazendeira,
olhando de frente para Otávio; e depois, com um sorriso:}

--- Vejam como são as coisas! Já se murmurou por aí que o senhor
ia-se casar com ela!

\textls[-20]{--- Nunca pensamos nisso --- respondeu Otávio,
ainda um pouco trêmulo ---; estimamo-nos como irmãos\ldots{}
Eva é muito boa, mas nunca se lembrou nem da probabilidade de
eu vir a ser seu marido\ldots{}}\looseness=-1

--- Em todo o caso, o senhor deve estar bem-informado. É verdade
o que diz o Azevedo?

\textls[-20]{--- Não sei, minha senhora; Eva está há alguns dias no
Mangueiral e poderia ter deliberado qualquer coisa sem nosso
conhecimento.}\looseness=-1

Entreolharam-se todos um tanto surpreendidos.

--- Correm certos boatos esquisitos --- tornou o
Carvalho ---; dizem, isto é, disse-me o Antunes, que ela tem
sido ingrata para com seu pai. É exato?

\textls[-20]{--- Não! Ao contrário; Eva é muito sensata e muito nossa amiga.}\looseness=-1

--- Parece um pouco soberba\ldots{} --- aventurou-se a
dizer a dona da casa ---; mas isto é tão comum, que a gente já
não estranha!

\textls[15]{--- Realmente, ela é um pouco altiva --- murmurou
Otávio; e, morto por fugir da questão, voltou a
conversa para a história do colono, prometendo ir imediatamente
falar à prima.}

\textls[15]{Acompanharam-no até à porta e em poucos minutos ele
desaparecia entre os cafeeiros escuros.}

\textls[15]{O Azevedo vai casar com Eva!, repetia de instante a instante
Otávio, num sentimento de dolorosa surpresa. É pois verdade,
amam-se\ldots{} mas como eu fui cego e como eles são perversos!
Amam-se, vão ser felizes\ldots{} e eu?}

\textls[15]{Perguntava-se se deveria continuar a trabalhar para pôr em
evidência a pureza e isenção da prima no negócio de Santa
Genoveva, ou se seria melhor afastar-se para sempre dela, deixando as
coisas no pé em que estavam. Evitaria assim uma convivência,
que muitas vezes o constrangeria amargamente.}

\textls[20]{Não queria assistir a esse casamento: demorar-se-ia em Casa
Branca. Ela que fosse feliz, contanto que ele a não visse na
sua felicidade!}

Convencia-se depois de que tudo aquilo era falso.

\textls[20]{A sua Eva adorada não pertenceria nunca a outro homem; iria
encontrá-la triste, tranquila e só; falar-lhe-ia outra
vez do seu amor, até vê-la persuadida da sua
sinceridade; trabalharia por ela, venerá-la-ia como outrora e sempre, e
havia de alcançar um dia a realização dos seus sonhos!}

Chegando ao Mangueiral viu à porta da casa um
trole desconhecido. Vão sair, pensou; mas nisso chegou a uma
janela o Azevedo, a olhar radiosamente através dos vidros da luneta.

Otávio sentiu como que uma grande pedrada que lhe tivessem
arremessado ao peito; mas, disfarçando bem a sua contrariedade, fez um
gesto amigável ao juiz de órfãos, que lhe correspondeu
com um ``bem-vindo seja'', em que pôs toda a hospitalidade de
um amigo vendo chegar à sua casa um companheiro
íntimo.

\textls[15]{Que tolo!, pensou Otávio de si para si; e
entrou. Contra a sua expectativa, Azevedo estava só com uma
senhora idosa, que fora morar no Mangueiral a pedido de Paulo,
para acompanhar Eva. Foi ainda o Azevedo quem, com ares de quem está no
que é seu, a apresentou, dizendo:}\looseness=-1

\textls[15]{--- D.\,Miquelina, viúva do capitão Rodrigues, senhora
da mais fina educação e hoje, pode-se dizer, mãe de
Eva.}

\textls[-15]{Otávio sorriu com desdém ouvindo o Azevedo pronunciar tão
familiarmente o nome da prima. D.\,Miquelina franziu as sobrancelhas,
evidentemente contrariada, e voltando-se para Otávio disse:}\looseness=-1

--- Eva foi visitar umas crianças doentes na colônia; e
Paulo não deve tardar: foi ao cafezal com um engenheiro, seu
amigo.

\textls[15]{--- Ah! O Custódio? --- perguntou sacudidamente
Azevedo --- Sempre veio cá? Coitado, é um pobre idiota, mas não
tem maus sentimentos, afinal\ldots{}}

--- Não foi ele --- respondeu a governante
---; a pessoa que veio é muito distinta e de nossa amizade, é o
dr.\,Morton.

--- Ah! O velhote do colégio?

--- Esse mesmo.

--- Um belo homem --- observou Otávio.

\textls[15]{--- Sim, não digo que não\ldots{} efetivamente não faz
mal a ninguém --- replicou o Azevedo ---; mas é
maçante e antipático como todo o ianque!}

Otávio ia responder azedamente, quando Paulo entrou.

\textls[15]{O dr.\,Morton seguira do cafezal para a cidade, indo encantado
com a lavoura do Mangueiral. Paulo, surpreendido com a presença de
Otávio, cumprimentou à pressa o Azevedo, que principiou a
cantarolar, dissimulando o seu despeito.}

\textls[20]{Como todas as atenções de Paulo caíam sobre Otávio, o
juiz, voltando-se para d.\,Miquelina, convidou-a a ir com ele à
colônia, onde, haviam dito, estava Eva exercendo a sua caridosa
missão de visitar doentes.}

\textls[15]{Miquelina vacilou titubeante, mas Paulo aprovou essa resolução,
aproveitando o ensejo para conversar a sós com Otávio. Este
expôs-lhe então os seus projetos, justificando a
partida e talvez mesmo a sua demora em Casa Branca.}

\textls[15]{Entretanto d.\,Miquelina e o Azevedo seguiam pela extensa rua
dos bambus, em direção à colônia; ali havia
sombra e frescor, cantavam as cigarras estrídula e prolongadamente e
perseguiam-se em voos circulares borboletinhas amarelas.}

Iam assim a um longo espaço, silenciosos e pensativos, quando
Eva assomou na extremidade da rua, com um galho de flores na mão e uma
sombrinha de linho escarlate a protegê-la do sol, colorindo-a
com os seus reflexos sanguinolentos e vívidos.

\textls[20]{--- Vamos\ldots{} coragem! --- disse a meia-voz o
Azevedo, sem reparar no espanto que essas palavras produziram em d.\,
Miquelina. Pouco depois reuniram-se os três.}

Eva informou a governante de que tinha achado em plena
convalescença a filha do Salomão, e pediu-lhe que arranjasse numa
cestinha uns doces para mandar à pequena; ela ainda tinha que fazer: ia
dali à casa do Jerônimo, que tinha levado uma chifrada de um
boi bravo e estava a deitar sangue pela boca.

--- Soube agora mesmo que Paulo mandou chamar o médico;
estou impaciente por vê-lo chegar\ldots{} pobre Jerônimo! Então
vá, d.\,Miquelina, faça favor, arranje os docinhos para a menina!

\textls[15]{Eva estendeu a mão ao Azevedo, num gesto de despedida, quando
este, curvando-se, pediu permissão para acompanhá-la até
à porta do doente.}

\textls[25]{--- Como quiser --- respondeu a moça com um modo
distraído e frio.}

\textls[20]{Seguiram, ora sombreados pelas ramas, ora batidos pelo sol que,
por entre as touceiras de bambus, punha arabescos e listrões
dourados no chão.}

\textls[20]{O silêncio não podia prolongar-se por muito tempo; num assomo de
ânimo o juiz rompeu:}

--- Nutro há muito tempo um desejo doído de lhe
falar\ldots{}

\textls[-10]{Eva voltou para ele o rosto, serenamente; o Azevedo continuou:}

\textls[20]{--- Não lhe deve ter passado despercebido o sentimento que me
inspira\ldots{} A senhora é que é de uma crueldade tamanha, que, não
demonstrando por mim a mínima simpatia, finge ignorar o grande,
o extraordinário afeto que lhe consagro! Agora, porém,
seja franca, suplico-lhe! E perdoe-me a ousadia de lhe ter
confessado o meu amor!}

--- Pede-me para ser franca, como se houvesse alguém
que em semelhante posição o não fosse! O senhor poderia
poupar-me o desgosto de o desenganar, se tivesse \emph{querido} perceber
que eu, \emph{fingindo}, como disse, ignorar o afeto
que me consagrava, era porque não podia corresponder-lhe! Disse
também que não lhe demonstrei nunca a mínima simpatia!
Nesse ponto é injusto: creio ter-lhe dado suficientes provas de
confiança e amizade.

Calaram-se; o Azevedo procurava, cabisbaixo, uma frase qualquer
com que pusesse fim ao diálogo. Deixaram a rua dos
bambus e atravessaram um pequeno campo, estrelado pelas
florinhas dos juazeiros; o sol batia de chapa, numa irradiação ardente e
luminosíssima; viam-se já perto, enfileiradas em linha, as casas dos
colonos com as suas paredes claras e os seus telhados
vermelhos.

\textls[20]{--- Deixe-me na persuasão, d.\,Eva, de que ao menos não se zangou
comigo, não achando insolente a minha aspiração\ldots{} ---
balbuciou o Azevedo.}

Eva parou, estendendo-lhe com lealdade a mão, e disse:

\textls[15]{--- Creia que tem em mim uma amiga sincera. Esqueça tudo o que
me disse e tudo o que me ouviu, exceto esta última
promessa: a de ser para o senhor o mesmo que uma irmã!}

\textls[-10]{O juiz apertou molemente a mão que ela estendia, e, 
levantando o seu olhar cor de aço, demorou-se um pouco a contemplá-la.}\looseness=-1

\textls[20]{Nunca a órfã de Gabriel Medeiros lhe pareceu tão
bonita. A sua sombrinha vermelha, penetrada de luz, tingia-a de uns tons
quentes, rubros, pondo-lhe reflexos de aurora nos cabelos negros, no
rosto amorenado e no vestido claro, franzido na cinta e salpicado de
botões de rosa.}

\textls[15]{Nos olhos aveludados de Eva, nadava uma doce umidade,
um sentimento de compaixão e de desgosto que os enlanguescia um
pouco, tornando-os talvez mais formosos.}

\textls[-15]{Azevedo murmurou meia dúzia de palavras banais num
agradecimento murcho e chato e afastou-se, depois da moça entrar na casa
do colono doente. Ia por todo o caminho a revê-la em
pensamento, alta, esbelta, pisando a grama e as florinhas rasteiras do
campo, sob o guarda-sol que, como uma grande papoula aberta, deixava
cair sobre ela, diluída e mole, a sua brilhante e viva
cor de púrpura.}\looseness=-1

\textls[15]{Quando Eva voltou para casa, já lá não estavam nem Azevedo nem
Otávio. D.\,Miquelina cosia junto à porta do jardim, e,
pela janela aberta do escritório, ela viu Paulo de pé, ao lado
da estante, a folhear um livro.}

%\end{linenumbers}

\section{XXV}

%\begin{linenumbers}

\textls[10]{Às nove horas de uma manhã de fevereiro, a família do
comendador Medeiros partiu para Casa Branca.}

\textls[-25]{Entrando no vagão, depararam todos com a irmã do Antunes e a
filha, a Sinhá, que, sentada a um canto, cruzava as mãos enluvadas sobre
uma rica bolsa de couro da Rússia com fechos de metal
reluzente.}\looseness=-1

Além deles iam no compartimento uma família de quatro
pessoas, mulher, marido, filha e ama, já vindas de outras terras e
empoeiradas da viagem, e um padre, que dormia a sono solto.

\textls[-20]{Com a entrada dos Medeiros, houve um remeximento de
cestas e de embrulhos que os viajantes, já aboletados, tinham
espalhado por toda a parte e que se viram então obrigados a acumular
perto de si.}\looseness=-1

\textls[20]{Foi um transtorno! Eles tinham até ali vindo à vontade,
suspirando por que não entrasse gente nas estações.}

\textls[15]{Chocaram-se latas e arrastaram-se sacos à pressa. A negra
entregou a criança à senhora e foi recolher fraldas e laranjas que
tinham espalhado pelo vagão.}

O chefe da família, depois de uma pequena hesitação,
dignou-se desenrolar as pernas compridas e moles de cima do banco para
afastar do caminho as suas malas e caixas de papelão.

\textls[15]{Tudo arranjado, voltou sereno para o seu canto, tirou com um
suspiro de alívio uma das botinas e pôs-se a afagar o
pé, por sobre a meia de algodão branco.}

\textls[-10]{Entretanto a mulher, esbugalhando contrariada os seus grandes
olhos verdes ocultava dos recém-chegados o cigarro aceso, que
viera a fumar em frente à papada bamboleante do padre
adormecido.}\looseness=-1

\textls[-15]{Tanto o comendador Medeiros como a mãe de Sinhá estavam
satisfeitos com o encontro e agradeciam mentalmente ao bom Deus do acaso
o tê-los guiado na mesma ocasião para aquela viagenzinha.}\looseness=-1

\textls[20]{Infelizmente não se dirigiam para o mesmo ponto. A
viúva ia passar uns oito dias em Campinas; agradava-lhe aquela
cidade e manifestava um certo desejo de fixar nela a sua
residência, logo que a filha se casasse.}\looseness=-1

\textls[20]{Ouvindo tais projetos, o comendador olhava de soslaio
para o filho, que junto à janelinha do vagão, lia serenamente
uma folha da capital.}

O comboio seguia, silvando ao aproximar-se das estações,
vencendo rapidamente as distâncias, sem que Otávio dirigisse
uma amabilidade, um olhar, uma palavra, à moça que tão evidentemente lhe
ofereciam; repugnava-lhe a condescendente passividade daquela formosa e
elegante criaturinha, que assim deixava jogar o seu destino, sem
intervir sequer com uma reflexão, uma pergunta, um vislumbre de
espírito, de dignidade ou de ação.

\textls[15]{Como nunca, a sobrinha do Antunes lhe pareceu bonita! Era
realmente a mais bela mulher de todo o município! De entre a gaze branca
que lhe envolvia o chapéu atando-se num grande laço embaixo do queixo,
emergia o pescoço roliço, alvo, e o rosto de linhas puras,
harmônicas, de estátua.}

\textls[15]{Em frente dela, Noêmia, com o seu perfil incorreto e
gracioso, chegava a parecer feia; a pobre menina tinha os olhos
avermelhados pelas lágrimas, que a resolução do pai a obrigava a verter.
A seu lado, a mãe, numa serenidade invejável, ouvia as palavras
da irmã do Antunes e oferecia os sequilhos trazidos na sua cestinha de
vime escuro, cingida por duas voltas de correia.}

\textls[-10]{A viúva mastigava os biscoitos voluptuosamente,
voltando para um e outro lado os seus grandes olhos negros. Variava o
assunto, passando do suspirado casamento da sua bonita Sinhá, para as
questões de lavoura e fortuna. Afirmava ser a sua fazenda uma das mais
produtivas; que o dr.\,Azevedo, que tinha jantado na véspera com
ela, afirmara não saber de sítio mais valioso e vasto; que o dote da
filha era, por conseguinte, um dos melhores da província. Evidentemente
contrariada com a indiferença de Otávio, chamou-lhe a atenção
perguntando-lhe maliciosamente se era verdadeiro o boato, que corria
com insistência, de ser ele noivo de Eva\ldots{}}\looseness=-1

\textls[20]{Tudo isso fora muito repetido e gritado, porque a bulha
do trem e o choro da criança, que pedia leite, não permitiam
ouvir nitidamente as coisas.}

\textls[10]{Percebendo na viúva uma tática sutil para
enredá-lo, Otávio resolveu ser franco, cortando pela
raiz a esperança que porventura ela ainda tivesse de o vir a ter por
genro; por isso, com toda a calma, enquanto o comendador, contrariado,
resmungava entre dentes, respondeu dobrando o jornal:}

\textls[15]{--- Não é exato que eu seja noivo de Eva;
sê-lo-ia se ela não tivesse rejeitado o pedido que lhe fiz da
sua mão\ldots{} Foi implacável! Paciência\ldots{}}\looseness=-1

\textls[20]{A viúva, muito desapontada, olhou para o comendador
Medeiros como a pedir-lhe explicações; mas o fazendeiro guinchava
estrondosamente num grande ataque de tosse, como se se tivesse engasgado
com as palavras do filho.}

\textls[15]{Sinhá compôs o laço do véu, enquanto Noêmia
lhe dizia, confirmando a declaração do irmão:}

\textls[-20]{--- É verdade, Otávio pediu a mão de Eva e ela não
quis; mas, ainda tenho esperança que mude de resolução! Gosto tanto
dela\ldots{}}\looseness=-1

\textls[-10]{Sinhá sorriu palidamente e principiou a sacudir com a ponta do
lenço as migalhas dos biscoitos que lhe tinham caído no vestido.}

\textls[15]{Minutos depois chegavam a Campinas. As despedidas foram
rápidas; a irmã e a sobrinha do Antunes perderam-se depressa no
meio da multidão que enchia a gare.}

A família Medeiros baldeou-se para o trem da
Mogiana. O comendador ia contrariado, fumando sucessivamente
cigarros e mais cigarros. A mulher observava-o com olhar medroso, e,
ouvindo Noêmia falar-lhe quase em segredo do desengano de
Sinhá, limitou-se a exclamar pausada e langorosamente:

--- A pobre!

A outra família tomou o mesmo carro.

O resto da viagem correu bem. A pequenita, depois de beber leite
e besuntar-se com doces, adormeceu. A ama cochilava também, a
mãe pasmava para o ar os seus grandes olhos verdes e o pai
continuava alisando no pé a meia de algodão branco.

A família Medeiros não conversava tampouco. Dois
passageiros entrados na estação de Campinas é que falavam muito,
discutindo as últimas eleições. Eram dois políticos,
um liberal, outro republicano, que argumentavam com todo o fogo, num
tiroteio de frases com que se procuravam esmagar reciprocamente.

Como o partido republicano tivesse feito uma bonita entrada nas
urnas, o passageiro liberal escondia o seu despeito num
desprezo altivo e afetada condescendência, que
mais fazia exasperar o outro. Cada vez que o republicano exclamava
jubiloso:

\textls[-20]{--- Estamos em maioria! E viva a República! ---
o liberal sorria com piedade, murmurando com voz que a raiva
tornava mal firme:}\looseness=-1

--- Vocês verão para o ano como ficam rodados!

\enlargethispage{1\baselineskip}
\textls[-10]{Aquela questão exasperava ainda mais o comendador Medeiros!
Conservador antigo e aferrado às suas ideias, ele sentia
freimas de intervir e jogar também aos seus
adversários políticos todas as pedradas que a sua
enferrujada eloquência lhe pudesse fornecer; mas continha-se,
mordia raivosamente o cigarro, mandando a todos os diabos a política e
franqueza quase grosseira do filho perante a irmã do Antunes e
a própria Sinhá! Estúpido!, pensava ele consigo. Vá
ver que quer mesmo casar com a tola da Eva!}\looseness=-1

\textls[15]{Quando chegaram a Casa Branca, chuviscava. As nuvens
aglomeravam-se em grandes massas escuras e uma suave viração agitava
tenuemente as folhas das árvores.}

\textls[-15]{O Trigueirinhos, avisado por telegrama, esperava a
família na gare: estava num grupo de rapazes, ouvindo o
que os outros diziam, com os beiços delgados e pálidos abertos num
sorriso inexpressivo.}\looseness=-1

--- Nicota não veio? --- perguntou o comendador ao apertar a mão
do genro.

--- Ela espera mesmo lá no sítio.

--- Está boa?

--- Hum\ldots{} tem andado esquisita\ldots{}

--- Isso é novidade\ldots{}

--- É exato.

\textls[-10]{Depois de uma resposta tão categórica, não podia haver
dúvida; Nicota ia fazê-lo avô. Medeiros recebeu sem alvoroço a
notícia, talvez com um certo desgosto. Parecia ir-lhe fugindo
toda a afeição da família! Otávio magoava-o
profundamente, desmoronando um projeto que havia já tantos anos
ele acariciava com amor! Noêmia rebelava-se também contra a sua
vontade, embora se fingisse resignada e humilde; a mulher tinha por ele
um respeito muito próximo do medo; havia na sua
obediência alguma coisa que lhe fazia lembrar o animal
domesticado diante do seu domador.}\looseness=-1

O comendador acusava de todos esses males ``o tempo''!

\textls[-15]{Ah, os chefes de família de há vinte anos
ainda, quanto mais felizes eram! Dirigiam à sua vontade o barco para a
direita, para a esquerda, para a frente ou para a retaguarda, sem que
partisse de dentro uma observação, um dito, um queixume ou uma súplica!
Um pai casava à sua escolha as filhas e os filhos, encaixava-os na
própria família entre os primos mais ricos; aos treze
anos as meninas seguiam para as suas novas casas, julgando-se muito
felizes, condescendentes e passivas; o mesmo pouco mais ou menos
acontecia com os rapazes, que enfim esses sempre gozavam de um
pouco mais de liberdade! Amor! Essa palavra só arrepiava os
velhos paulistas, como se ela fosse um sinônimo de degradação e
de torpeza.}\looseness=-1

Os anos passaram e, rapidamente, numa evolução quase
incompreensível, os espíritos dos moços rebelaram-se contra as
leis estabelecidas, a ponto de quererem agir por si! E hoje\ldots{}

\textls[15]{O comendador meditava nas loucuras de agora, quando
Trigueirinhos o empurrou para dentro de uma grande e pesada
sege, onde já havia acomodado a família.}

\asterisc

\textls[15]{A fazenda do Trigueirinhos era perto e a estrada magnífica. De
passagem ele ia informando a família dos nomes dos
sítios; de repente, voltando-se para o sogro,
exclamou:}

\textls[20]{--- A gente do Franco chega hoje à tarde\ldots{} ainda
vão cear lá em casa.}

\textls[15]{Noêmia estremeceu e colou o rosto ao vidro do carro
para encobrir as lágrimas que lhe assomavam aos olhos.}

A chuva caía peneirada e miúda; atravessavam um vastíssimo
campo, onde se destacavam ao longe oito ou dez barraquinhas de pano
grosso, umas brancas, outras pardas, algumas cobertas de
remendos e colocadas a pequenas distâncias umas das outras.

Eram as habitações dos morféticos.

\textls[20]{Aquela triste aldeia ambulante punha uma nota de desconsoladora
tristeza na extensa planície, frouxamente alumiada por um dia
cinzento.}

Afigurava-se à pobre Noêmia que em torno daquelas barracas não
brincaria nunca um raio de sol, que a relva daquelas campinas estaria
sempre assim, alagada, sem viço e sem cheiro, que cobriria sempre aquela
paisagem a grande melancolia desse instante.

\textls[-15]{Lá, dentro dessas miseráveis barracas, cuidadosamente fechadas
à umidade do tempo, revolviam-se, nas mais dolorosas
contorções da agonia, corpos imundos, mutilados, disformes,
cobertos de chagas, de podridão e de pus! Pais e filhos viviam
a contemplar-se, vendo as feições desaparecerem, dia a dia, ficarem
carcomidas e medonhas, os cabelos tornarem-se ásperos, hirtos e ralos,
pelos estragos que instante a instante ia cavando a invencível
morfeia!}\looseness=-1

O carro passava, mas o campo era longo, descoberto, e Noêmia não
perdia de vista as barraquinhas. Olhava para aquilo abstratamente.
Apesar de meiga e bondosa, não pensava então na desgraça dessa gente,
segregada da família, do amor, da sociedade e da alegria.
Pensava só e egoisticamente no seu desgosto e no encontro,
inevitável, com o noivo que lhe propunham.

\textls[15]{Através dos vidros da sege, e das gotas da
chuva, que se amiudavam, viu, na mesma indiferença, um dos habitantes
das barracas, que se recolhia à pressa, vindo da cidade, onde
fora pedir esmolas. Montava um cavalo magro e velho, que ia num
passo trôpego, apesar das repetidas vergastadas com que, agitando
nervosamente na varinha verde, o cavaleiro o zurzia.}

\textls[7]{O desgraçado morfético tinha um aspecto hediondo! Faltava-lhe o
nariz, os lábios estavam arrepanhados pelas cicatrizes, as mãos
envoltas em trapos, os pés igualmente enrolados em umas longas tiras de
pano sujo, o cabelo em falripas, aparecendo-lhe na nuca pela abertura de
um lenço de chita vermelha, amarrado na cabeça, por baixo do chapéu de
feltro muito ruço.}

Atrás, na garupa, levava o sapicuá\footnote{Sapicuá: alforge.}
com as suas duas bolsas cheias de mantimentos, ali despejados
pelas mãos caridosas das pessoas da cidade; pendurada pela asa ao atilho
do saco bailava uma caneca de folha de um lado para o outro,
aos movimentos desconjuntados do animal, enquanto dentro do sapicuá se
entrechocavam, tinindo, uma colher e um prato de metal.

Nenhum lázaro deixa de levar consigo, quando aos
sábados desce às povoações, os seus \emph{trens}, como intitulam
na província os objetos caseiros e familiares.

Pedindo um gole de água para mitigar a sede ou
uma sopa para matar a fome, essas repelentes criaturas, para
que os seus beiços nojentos não rocem nos copos nem nos pratos da gente
feliz, que tem saúde, apresentam o prato e a caneca que trazem consigo.
Vão sempre prevenidos com isso nessas peregrinações tristes, em
que, provocando a piedade, exibem os seus corpos mutilados e os seus
rostos repulsivos! A consciência do mal que os
corrói, afastando-os, muitas vezes, das próprias
famílias, fá-los sombrios e geniosos. São os galés sem crime,
sem remorsos, que arrastam numa expiação dolorosa e perpétua um erro da
natureza, tão boa para uns quão bárbara para outros!

Esses infelizes fogem das cidades, onde são perseguidos e
expulsos pelas municipalidades, e têm um verdadeiro pavor aos
hospitais, por não encontrarem neles as condições que os confortem e
animem! Por isso, armam as suas frágeis habitações de pano na solidão
dos campos. Ali podem viver, ao menos alguns dias, no isolamento,
ignorados por todos; ali podem, sem humilhação, beber da fonte
a água límpida, pura, que escorre entre o musgo aveludado das pedras
numa corrente que não para ao contato dos seus lábios
entumecidos, nem lhes nega frescura; ali não os observam olhos
compassivos ou enojados, as flores perfumam-lhes os pés
chagados, e as estrelas brilham como uma doce bênção sobre as
suas cabeças leprosas. Mas\ldots{} chega o sábado, o dia
consagrado às esmolas, e lá descem eles então em turmas
à cidade, onde os não recrutem para os asilos, afrontando todas
as humilhações, desde a esmola pedida em nome do bom Deus, que,
entretanto, lhes nega a felicidade, até o cobre que lhe atiram de longe,
para dentro do chapéu, com medo do contato.

O carro seguiu vagarosamente, enterrando as rodas muito fundo na
terra mole. O mísero cavaleiro ficou para trás, encolhendo o pescoço sob
a gola do seu casaco ensebado. Mais adiante, a sege passou por
um grupo de quatro desses desventurados, que se encolhiam
também com medo da chuva, apressando os animais, quase
tão gafados como os donos, incitando-os nas suas vozes enrouquecidas e
fanhosas a prosseguirem no caminho; mas o campo era vasto, as
barraquinhas estavam ainda longe, meio encobertas pelo tom cinzento do
dia nebuloso. Pelos vidros das portinholas corriam as gotas de
chuva como lágrimas grossas, e a sege ia-se arrastando
penosamente sobre a terra fofa e molhada.

Noêmia, sempre com os olhos fitos nas barraquinhas distantes e
no campo alagado, ia pensando\ldots{} que apertaria nesse mesmo dia a
mão do homem que lhe destinavam e que ela aborrecia!

%\end{linenumbers}

\section{XXVI}

%\begin{linenumbers}

\textls[20]{A fazenda do Trigueirinhos era bem situada, no alto de uma
colina, varrida de ar puro, cercada por grandes matas escuras, que lhe
davam um aroma sadio e agreste. O dono tinha-a batizado com o
despretensioso diminutivo Sertãozinho. Quando a sege
com a família Medeiros parou à porta da residência,
uma casa amarela com pretório}\footnote{Pretório: varanda na frente da
  casa.} \textls[20]{de madeira pintada da mesma cor e frisos
  escarlates, Nicota veio à escada, mostrando num dos seus raros sorrisos
  uma satisfação sincera.}

\textls[20]{Entraram e percorreram toda a casa, o alpendre, a sala dos
homens, a varanda, os quartos e a vasta cozinha térrea, onde,
além do grande fogão, havia a um canto a tacuruba, três pedras
cercando um brasido crepitante e rubro, onde assentava um enorme tacho
de cobre cheio de goiabada ainda mole e saltitante, que uma
negra gorda mexia a distância, segurando pela ponta a longa
vara da cuipeva.}\looseness=-1\footnote{Cuipeva: pá de cabo longo.}

\textls[15]{Nicota preparava uma ceia lauta. Além dos Francos,
esperavam outros hóspedes; tinham já tudo em ordem ---
dera uma boa dona de casa. Previdente e ativa,
enquanto a família descansava da viagem, ela ia e vinha da
despensa, despejava os doces dos arimbás}\footnote{Arimbás:
  boiões.} \textls[15]{para as compoteiras, mandava que ordenhassem o leite
  das vacas para ir à sobremesa numa terrina bojuda; recomendava
  que a canjica ficasse bem cozida e os ovos queimados
  bem-feitos. Estava nesses preparativos quando recebeu aviso de que os
  Francos só iriam no dia imediato.}

\textls[20]{Sabendo disso, Noêmia suspirou de alívio; adiava-se o
momento da tortura.}

\textls[20]{Nicota não pareceu dar importância à tristeza nem às
queixas da irmã; tratou de dar ordens para o almoço do dia seguinte,
pedindo ao marido que mandasse matar um boi: era indispensável, queria
pôr um banquete para os escravos, solenizando a inauguração da máquina e
deliberou logo levantar-se muito cedo, para aprontar
vários doces.}

\textls[15]{Entretanto, o comendador e Otávio desciam com
Trigueirinhos à nova casa da máquina, que se estreava.}

\textls[15]{Trigueirinhos tinha os seus planos e explicava-os ao
sogro e ao cunhado.}\looseness=-1

\textls[20]{À espera de bom preço no mercado, ele tinha ainda na tulha a
colheita do ano anterior; o café estava em coco, por
beneficiar. Decidiu dar começo ao trabalho da máquina quando estivessem
os Francos; seria completa a festa.}

Conversando, seguiram os três até o fim do terreiro,
indo encostar-se a uma cancela baixa que dava para o pasto. Tinha
cessado a chuva; o sol rompia as nuvens e dourava a grama muito verde e
fresca, aljofarada de grossas gotas de água. O gado
corria desenfreadamente, açodado pelos gritos de dois negros que, a
distância um do outro, giravam no ar com força e perícia a corda
de couro laçada para a pega do boi. O animal condenado escapara
duas vezes com os chifres ao laço, e corria raivoso, corcoveando,
arremetendo ora para a direita ora para a esquerda, espumante e lesto,
com os olhos em fogo, a respiração aflante e o dorso saliente ou curvo,
segundo os movimentos da cabeça que, ou parecia tocar o chão ou se
erguia desesperada numa provocação altiva!

Trigueirinhos gritava da cancela a um dos pretos:

--- Agora, Zezinho! Atira! Não seja lerdo!

Mas o Zezinho recuava com o corpo numa volta rápida
às investidas do boi.

Aquela cena de luta durou alguns instantes, até que o
animal, escorregando no terreno molhado, se deixou prender. Ouviu-se
então o ruído seco da corda caindo e apertando-lhe os chifres.

O Zezinho, moleque ágil, correu a abrir a porteira,
enquanto o parceiro, suado de cansaço, mas com um bom ar de triunfo na
fisionomia, vinha puxando o animal pela corda até o terreiro, onde o
amarrou a um mourão, deixando-o a descansar.

Entretanto Zezinho corria a buscar o machado e a grande
faca pontuda. O boi esperava, quedo, reluzente de suor.

\textls[20]{--- É pena matarem um animal tão bonito! --- dizia
Otávio ao cunhado.}\looseness=-1

\textls[20]{--- É que ele era meio bravo, manhoso como o diabo! ---
informava o dono da fazenda.}

Durante esse tempo o preto que laçara o boi punha-se em frente
dele e, calculando com os braços erguidos uma pancada certeira, deixou
cair por fim vigorosamente o olho do machado entre os chifres do animal.
A pancada ressoou forte e surda, e o boi entontecido caiu para
o lado.

\textls[-20]{Zezinho de um salto desamarrou-lhe a laçada e o outro escravo,
agachando-se, levantou-lhe com uma das mãos a cabeça e com a outra
fincou-lhe a faca sob a papada; o sangue, muito vermelho e quente,
espirrou célere, alastrando-se depois em poças pelo chão.}\looseness=-1

\textls[15]{Enquanto os dois negros rasgavam de alto a baixo o
animal e lhe arrancavam os intestinos fumegantes, o comendador Medeiros
seguia com os seus dois companheiros em direção do açude, obra
a que o Trigueirinhos dava grande importância.}

\textls[-20]{No lago, muitos marrecos bravos, de cores variegadas,
fizeram recair a conversa sobre assuntos de caça; ainda na
véspera o dono da fazendo tinha trazido do mato uma bonita paca
e algumas perdizes.}\looseness=-1

--- Por aqui há onças?

--- Há\ldots{} mas \emph{menas}\footnote{Menas: este
  comparativo, fazem-no sempre concordar com o gênero: menos frutos,
  \emph{menas} folhas.} que no sertão. No Jaú, no sítio de meu
tio, elas aparecem com frequência e atacam as reses.

--- Eu já assisti duas vezes --- contava o comendador
--- ao assalto da onça; uma vez aos cavalos e outra vez aos
bois.

\textls[20]{E lembrava-se. Tinha sido em solteiro; ele era então uma
espécie de capitão de mato do pai; andava às vezes dias
e dias em busca dos escravos fugidos. Uma tarde, de cima de uma colina,
presenciara um espetáculo estranho e para ele completamente
novo. Numa vasta planície, embaixo, onde a luz do sol no ocaso
punha uma grande vermelhidão, movia-se convulsivamente uma enorme massa
redonda manchada de claro e de negro. Ao redor dessa massa, aos saltos,
cabriolando em contínuas investidas e recuos, uma grande onça
arreganhava os dentes, acendendo os olhos de cobiça e de raiva.}

\textls[2]{De repente tudo se imobilizava\ldots{} a onça, a certa
distância, agachava-se, cravando as unhas na terra e armando o
bote; a massa preta e branca aquietava-se também. Ouvia-se
então o resfolegar medonho, de cinquenta bocas respirando ao
mesmo tempo, com igual ansiedade e igual esforço. Todas as coisas em
volta pareciam tomar parte na ação. O campo, as
próprias árvores negras e mudas, como que sofriam vendo aquele
drama da natureza desenrolar-se ali, sobre os filamentos nervudos de
suas raízes! Foi num desses momentos de ansiosa
expectativa que ele percebeu tudo. Aquela massa movediça era
formada por muitos cavalos que, instintivamente ordenados em
círculo, com as cabeças unidas no centro, formando com as ancas
a circunferência, defendiam-se aos coices dos
terríveis assaltos da onça. Ele fugira aterrorizado, deixando no
vasto campo silencioso e avermelhado pelo sol moribundo aqueles furiosos
contendores.}

\textls[-10]{O outro assalto deixara-lhe menor impressão, tinha dois
companheiros consigo que o animavam muito; depois, a hora era
diversa, fora de manhã, entre o chilrear dos pássaros
e o aroma fresco das ervas úmidas. Os atacados pela
onça eram então bois que, formando igualmente círculo, mas em
posição inversa, isto é, com as cabeças voltadas para a periferia,
defendiam-se às chifradas.}\looseness=-1

\textls[15]{A luta fora majestosa. De lado a lado havia
astúcia e força; os olhares trocavam-se flamejantes num ódio
agressivo.}

\textls[15]{Em investidas e recuos, a onça ia e vinha cada vez mais
assanhada e feroz, até que, de repente, apanhada pelos chifres de um
boi, foi arremessada muito alto, indo cair no chão com o dorso para
baixo, numa bulha surda e fofa. Viram-na por alguns instantes agitar
para o ar as pernas, procurando erguer-se, e então um dos amigos de
Medeiros matou-a a tiro. Os bois, espantados, disseminaram-se, correndo
desabridamente sobre a relva ainda úmida do sereno da noite.}

Quando voltaram para casa encontraram visitas: dois lavradores
vizinhos, o major Seabra e o Porto, que iam prosear um pouco com o
Trigueirinhos; este, porém, não se afastou da varanda, onde a sogra
descascava marmelos, Noêmia acariciava a filha do feitor e Nicota enchia
duas pichorras com melado e mel.

As senhoras mostravam-se cansadas da viagem, mas acompanhavam
Nicota nos preparativos da festa.

\textls[20]{A conversa entre os homens caiu depressa no assunto magno do
tempo, os abolicionistas!}

\textls[15]{O Seabra informava o Trigueirinhos, com uma indignação
extremada, de que ouvira do próprio juiz a franca declaração de
não consentir na captura de negros fugidos!\ldots{}}

\textls[25]{E cuspiam injúrias sobre o magistrado, em extravasamentos de
bílis.}

Otávio afastou-se, indo fumar perto de uma janela.

\textls[-25]{Era a hora do crepúsculo, e o sol a esconder-se iluminava a
paisagem ainda fresca da chuva. Vinha do pomar o aroma leitoso e agreste
das figueiras. As rosas de todo ano, desfolhadas pelo vento, deixavam
cair as suas pétalas viçosas que atapetavam alegremente o chão.}\looseness=-1

\textls[10]{No terreiro, o gado que se recolhia do pasto para o curral
agrupava-se em torno ao sangue já frio do animal matado horas antes, e
desenhando no espaço os seus grandes chifres arqueados, com os focinhos
erguidos, magoados pela perda do antigo companheiro, mugiam em
coro, lamentosamente\ldots{}}

%\end{linenumbers}

\section{XXVII}

%\begin{linenumbers}

\textls[15]{Eram seis horas da manhã quando Otávio, acordando,
ouviu dois caipiras dialogarem sob a janela do seu quarto:}

--- Seu Romão já disse qui o mais fólte di
nóis tudo é o Brás\ldots{}

--- Tchê! Na terra! O mais fólte de nóis tudo sou eu!

\textls[20]{--- Seu Zé Riba mandou chamá um capanga di
fora, um tá Furtuoso!}

--- Eh! Sai cinza!

--- Os coitado\ldots{}

--- Que coitado?

--- Da gente do dotô.

--- Foi nhô Trigueiro que mandou chamá mecê?

\textls[15]{--- Não foi não; eu vim trazê um recado de nhô Juca, pro mó
de ele i lá combiná as coisa; o diabo é que fica talde e eu perciso i mi
embora.}

--- Mecê foi convidado pra festa?

--- Fui.

--- Quem foi que convidou mecê?

--- Nhô Carro de Lima, \emph{mais} não vou não!

--- Ué! Isso é o mesmo que pinchá dinhero
fora!

\textls[15]{--- É\ldots{} \emph{mais} não vou. Eu quando vejo sangue fico tão
patife}\footnote{Patife: desanimado, covarde.} \textls[15]{que até
chego a sê vil!}

\textls[20]{Ouvindo o nome de Frutuoso ligado àquelas palavras misteriosas,
Otávio ergueu-se e abriu a janela, mas já os dois caipiras se
tinham afastado; um deles conversava com o Trigueirinhos, que o ouvia de
rosto carregado; o outro desaparecia na porta do pátio com um
feixe de palmitos ao ombro.}

\textls[-10]{Meditando nas palavras que acabara de ouvir, Otávio
julgou perceber a intenção oculta de um crime. Essa suspeita
desvaneceu-se depressa. Que diabo!, pensou ele consigo, se houvesse
realmente tal propósito, os capangas não teriam tanta facilidade em
falar nisso tão alto e junto a uma janela. O que é preciso é ir ter com
o Frutuoso, segurá-lo bem e arrancar-lhe da goela toda
a verdade.}\looseness=-1

Otávio resolveu procurá-lo nesse mesmo dia. Quando
entrou na varanda deparou com a irmã, que fazia um requeijão, sentada
entre dois barris, um de leite coalhado, outro de leite fresco.

\textls[15]{Otávio pôs-se a ler os jornais, pensando ao mesmo tempo
na entrevista com o Frutuoso.}

Que verdade negra lhe iria revelar esse homem infame, que se
vendia para o crime, como uma prostituta para o amor?

A pouco e pouco foi-se reunindo toda a família.

\textls[15]{Estabeleceu-se a prosa: entrou o Trigueirinhos, preocupado, com
os beiços esbranquiçados e as faces terrosas. Entretanto Nicota,
rubicunda e serena, espremia num guardanapo a coalhada de leite com
ligeireza e força, e tirava-a depois dali, já bem enxuta, para
o leite fresco de um tacho.}

É a mais feliz da família, pensava Otávio,
observando-a na sua ocupação de \emph{ménagère} distraída.

\textls[15]{Estavam assim, quando ouviram o rodar de um trole:
chegavam os Francos. Noêmia estremeceu e aproximou-se instintivamente
da mãe, como a pedir-lhe misericórdia e apoio.}

Momentos depois, entravam na sala o coronel Cerqueira Franco e o
filho.

\textls[20]{O coronel era um velho robusto, alto, corpulento, de barba farta
a encobrir-lhe todo o peito, andar pesado e grandes mãos cabeludas. O
filho, igualmente alto e forte, tinha o aspecto agradável, o
olhar sereno, um sorriso bondoso e feliz de criança posto nos grossos
lábios de um homem.}

O comendador apresentou-o a Noêmia, dizendo-lhe:

--- Esta é que é vossa noiva!

Ele corou; ela, sem levantar os olhos, estendeu-lhe a mão,
gelada e trêmula.

Houve um instante de silêncio, que o coronel Cerqueira Franco se
apressou em cortar, falando num tom alto, com certa intimidade e
alegria.

Voltando-se para o filho exclamou:

--- Você tem uma noiva muito bonita; carece agora de
tratá-la bem, ouviu? Bonita e\ldots{} cômoda! Porque é
tão delicada que há de ser por força muito leviana!\footnote{Leviana:
  leve.} Cabe num canto do bolso do Júlio.

Pouco depois, uma mucama chamou para o almoço.

\textls[15]{Enchendo seu prato de quirera}\footnote{Quirera: milho
  pisado.} \textls[15]{e lombo de porco, o coronel Franco
  prosseguia:}

\textls[17]{--- D.\,Nicota? A senhora descurpe a franqueza! Mas quando eu
passei na cachoeira de Pirassununga, ontem de talde, tavam
caçando}\footnote{Empregam o verbo ``caçar'' por ``pescar''.}
\textls[17]{peixe, então apeguei de trazê um dourado
pra sua ceia\ldots{} E dirigindo-se imediatamente ao
Trigueirinhos:}

--- Seu sítio é bem bom\ldots{} eu já pousei
aqui na vizinhança uma noite, em casa do\ldots{} de um
ximbeva\footnote{Ximbeva: designam por este nome na
  província as pessoas de nariz chato.} moreno\ldots{}

--- O Prates?

--- Isso!\ldots{} Não me alembrava o nome, era casado com
uma moça da Limeira\ldots{} oh! Que mulher fogueta!

Os outros riram-se; ele continuou:

\textls[-15]{--- Estou ficando com os pés fria!}\looseness=-1\footnote{Pés fria: é assim
  que geralmente dizem na província.} \textls[-15]{Acho que é porque a
  varanda é de tijolo! Trigueirinhos? Vocês não aquentam fogo, não?}\looseness=-1

  --- Que sim --- responderam. De vez em quando, no
inverno, faziam fogo na sala de jantar, entre as redes; agora,
porém, em fevereiro, nem se lembravam disso!

\textls[15]{O coronel Franco prosseguiu falando sempre, ora das
suas propriedades em diversos locais da província, ora do
casamento do filho, etc. À sobremesa disse ter ouvido de dois
passageiros, no trem, grandes injúrias às autoridades de Casa
Branca, e também que haveria nessa cidade uma revolução de
escravos. Isso é que era o diabo! Deveria dar crédito a
semelhante boato?, indagava ele, mastigando com a completa dentadura
postiça o doce de abóbora moranga, servido às talhadas.}

\textls[15]{--- Que não desse crédito --- respondeu o
Trigueirinhos; tudo aquilo eram histórias de gente
casteleira!}\footnote{Exagerada, mentirosa.}

--- Homem\ldots{} será?

--- É, sim!

--- Bom!

\textls[20]{Durante todo o dia reinou alegria entre os escravos do
Sertãozinho.}

\textls[20]{Otávio não se pôde ausentar da família e desejou mesmo
consagrar o seu tempo em favor de Noêmia, por quem até então nada pudera
fazer.}

Para isso conduziu o noivo da irmã para a sala dos homens, vazia
nesse instante. Júlio sentou-se num canto do sofá.
Otávio, arrastando uma cadeira, sentou-se em frente dele e
começou:

--- Vamos tratar de um assunto muito delicado! Principio
pedindo-lhe permissão para lhe fazer uma pergunta.

--- Pois não\ldots{}

--- Reparou bem nos olhos da sua noiva?

\textls[20]{Júlio Franco olhou atônito para o futuro cunhado, e respondeu
depois, muito candidamente, sem atinar com o motivo de semelhante
pergunta.}

--- Não\ldots{}

\textls[-10]{--- Se tivesse reparado, poupar-me-ia talvez o dissabor
de lhe dizer estas palavras: Noêmia tem os olhos vermelhos e inchados
por passar horas inteiras a chorar! O casamento amedronta-a; ainda muito
moça e de caráter um tanto volúvel, receia não fazer feliz o
homem que lhe impõem para marido. É natural; o casamento sem amor
parece-lhe uma coisa bárbara, uma coisa assim como um patíbulo moral,
onde tenha de sacrificar toda a sua vida!}\looseness=-1

\textls[10]{Nunca o tinha visto; o senhor ainda é para ela um
estranho. Algum tempo de convivência, estou certo,
bastaria para fazê-la aceitar com imenso júbilo tal resolução.
Se meu pai consultasse a sua vontade, não sucederia nada do que se está
passando. Responderia à honrosa deferência do coronel Cerqueira
com duas palavras de recusa perfeitamente explicável. Creio que as
antigas relações entre ambos, homens sensatos e experientes, ficariam
como de antes. O contrário seria pueril; não lhe parece?}

\textls[25]{Júlio Franco limitou-se a acenar que sim, com a cabeça. No seu
olhar havia alguma coisa de indeciso e de errante, como a buscar um
ponto em que se fixasse.}

Otávio concluiu:

--- Talvez lhe pareçam tardias estas minhas observações; mas,
não o conhecendo pessoalmente, não me atrevi a escrever-lhe; trabalhei
contudo para o mesmo fim em família, sem conseguir abalar a
resolução de meu pai. Agora nada mais tenho a dizer-lhe; se fui
compreendido ou não, não sei; o senhor o dirá.

--- Perfeitamente --- respondeu Júlio,
levantando-se ---; parto amanhã cedo para São Carlos e de lá
escreverei ao comendador. --- E estendeu lealmente a mão a
Otávio, que, apertando-a, sentiu com júbilo que tinha entre as
suas a mão de um homem de bem.

\textls[-5]{Já se ouvia lá fora uma grande algazarra; as crianças davam
vivas, e a família do Trigueirinhos, pais, irmãos, tios,
primos, uma infinidade de pessoas, acabava de chegar para assistir
à inauguração da máquina. Nicota e Noêmia foram ao
encontro das senhoras, que abaixavam os olhos, acanhadas; não tinham
nunca saído do interior, e torciam caladas os lenços, olhando para o
chão. Não houve conversa, ideia, vontade que se lhes comunicasse. Noêmia
ia e vinha trefegamente de umas para as outras: debalde todas as
tentativas de distração! Mostrou álbuns de fotografias, de
gravuras; foi buscar flores, frutas, trabalhos de agulha, perdendo
comentários e secando inutilmente a língua; por fim,
não tendo mais que dizer de alegre nem que mostrar, sentou-se,
queixando-se num bocejo de que os mosquitos lhe davam cabo da pele, e
estendeu para uma das visitas a mãozinha muito alva, assinalada por uns
pontinhos rubros. Ia a fechar os olhos, sonolenta, teve de os arregalar
de espanto: a senhora a quem indolentemente mostrara a mão,
desenroscou-se, dizendo com timidez:}\looseness=-1

--- Decerto é piririca.

\textls[10]{Quebrou-se o encanto; daí em diante uma ou outra frase
semelhante vinha boiar à tona daquela pasmaceira, até que
desceram todas para a casa da máquina.}

\textls[20]{Nela estavam reunidos alegremente os brancos e os escravos.
Muitos galhos de cafeeiros adornavam as paredes, e ao rumor da máquina
unia-se o rumor das vozes.}

\textls[15]{Um estudante de São Paulo, parente de Trigueirinhos, fez um
discurso; ouvindo-o, Otávio lembrava-se do Azevedo, que, se ali
estivesse, não perderia ocasião de botar retórica; mas, infelizmente,
estava longe, ao lado de Eva, a falar-lhe talvez na doce intimidade dos
noivos.}

\textls[-15]{Finda a cerimônia da inauguração, voltaram para a casa
da residência, onde Nicota, ajudada pela mãe, fez
servir um jantar opíparo.}\looseness=-1

À noite, no terreiro, rebentou o samba. Um negro, reforçado e
coxo, vibrava no tambor grandes pancadas ressoantes; um velhote
repenicava o caracaxá, revirando alegremente os olhos, e no meio do
barulho do batuque, uma grande roda de negros e negras cantava num
ritmo monótono, em que a melodia parecia nublar-se e
esconder-se de vez em quando, para surgir depois mais doce, mais
sentida, mas igualmente fugitiva. As palavras repetiam-se sempre, sem
significação, como mero pretexto para as modulações musicais.
Os pajens puxavam fieiras e as mucamas saíam para a dança, de
braços caídos, saracoteando os quadris.

Alguns escravos antigos, e já inválidos, assistiam de lado,
contando entre si os deslumbramentos das Congadas, dança complicada em
que figurava um rei do manto roçagante, feito de paninho, doze ministros
de blusa cor-de-rosa e azul, e os doze conguinhos, de
saiote e corpinho, todos com capacetes de papelão, enfeitados de fitas e
plumas velhas, ou gorros de metim.

--- E o generá?! --- lembrava um.

--- Ah! O generá! --- suspirava outro.

O general entrava fardado na dança, cheio de galões, numa casaca
extravagante!

\textls[15]{Da Congada, passavam a relembrar o vigor da famosa dança dos
caiapós! Nessa sim! Os negros trazem por cabeleira madeixas de crina de
cavalo, e, por sobre camisas de meia esfrangalhadas, muitas penas e
adornos vistosos.}

\textls[15]{Os velhotes relembravam essas coisas, enquanto as
moças, num passo miúdo, iam e vinham, ora para trás, ora para
diante, com os meneios vagarosos e arredondados,
repetindo sempre num tom cheio de melancolia o}

\begin{verse}
\footnotesize
Petinga-a-á, petinga-a-á, \\*
petinga-a-á, meu bem!\\*
\end{verse}

\textls[15]{Otávio passeava no pretório, observando às vezes a
dança dos negros, indo outras vezes encostar-se à janela da sala do
jogo, ficando aí a olhar para dentro, vendo jogar.}

\textls[20]{Numa dessas ocasiões ouviu uma grande discussão; tratava-se de
alguma coisa grave: acusavam um juiz do lugar, em termos injuriosos.}

\textls[15]{--- O cachorro --- exclamava um fazendeiro da
vizinhança --- não só não pega os negro
fugido, mas ainda por cima é acoitadô dos escravo!}\looseness=-1

Ele tá tão sossegado e é de tanta corage, que
deixa eles andá de dia na rua!

--- Está mesmo desafiando a gente! --- dizia um outro.

\textls[15]{--- Assim o quer\ldots{} assim o tenha --- sentenciava
um velho magro, de olhar amortecido.}

--- Olhe; a demissão do seu cargo já está pedida.

--- Isso é que já se deveria ter feito há muito tempo!

\textls[20]{--- Bem discutida tem sido a tal demissão! --- insinuava
maldosamente, num sotaque estrangeiro, o mesmo velho magro de
olhar amortecido.}\looseness=-1

\textls[15]{--- Ele acoita negros para ter quem o sirva --- dizia o
dr.\,Azambuja, pondo na mesa um trunfo.}

--- Isso é que é! Você deu no vinte! --- respondeu o parceiro,
recolhendo o trunfo com outro maior.

\textls[-10]{--- Agora mesmo lá tem ele dois; um ali do Moscoso. --- e
mostrava com o beiço um sujeito gordo, sem barba, que enrolava um
cigarro, descansando no ventre arredondado os braços muito curtos.}\looseness=-1

--- Com o meu é que ele não fica! --- rosnou o Moscoso,
balançando a papeira.

--- Ora! É o que você diz! Mas queira ele! ---
incitava o coronel Trigueiros, batendo-lhe no ombro --- e então
que remédio terá?

\textls[20]{--- Que remédio terei? Ora essa! O dito, dito! Que foi
que se combinou?}\looseness=-1

--- Na verdade --- respondeu com um sorriso o coronel
---, dizem que você tem uns planos\ldots{}

--- Eu?!\ldots{} Ora! Isso é lá com os Pereira!\footnote{Locução
  popular que significa: ``não tenho nada com isso''.} --- e olhou
de soslaio, maliciosamente, para o dr.\,Azambuja; este acudiu:

--- Bem\ldots{} bem! Será mais prudente variar de conversa.

--- Deixem sossegar o juiz!

\textls[20]{--- Mas se o diabo está pedindo forca! --- exclamava um
mais exaltado.}\looseness=-1

--- Que tempo de abusos!

--- Não! Que conosco ele não pinta! Há de ser ensinado!

--- Com a demissão --- concluía o velho.

--- E com\ldots{}

\textls[20]{O Moscoso acabaria a frase se os outros não a abafassem, falando
e tossindo alto. Evidentemente, pensava Otávio, esta gente está
doida! E saiu da janela, indo deitar-se na rede a um canto do
pretório, onde começou a fumar um havana, olhando para o céu
estrelado, a ouvir a cantiga dos negros, cheia de tristeza e de
selvageria.}\looseness=-1

%\end{linenumbers}

\section{XXVIII}

%\begin{linenumbers}

\textls[15]{Júlio Franco saiu do Sertãozinho às cinco horas da manhã,
deixando o coronel incumbido de o desculpar para com a família
Medeiros e o Trigueirinhos.}

\textls[15]{Passara parte da noite a discutir com o pai, que ao
ouvi-lo dizer: ``Já não me quero casar; resolvi ficar
solteiro'', pôs as mãos na cabeça, perguntando:}

--- Ó diabo! Por quê?

\textls[-30]{--- Porque vejo perfeitamente que não agrado a Noêmia Medeiros!}\looseness=-1

--- Ora, que tolice! Mulher lá sabe o que lhe convém! O pai
quer, acabou-se!

--- Eu não penso assim.

--- Deixai-vos de asneiras!\ldots{}

--- Isto é até muito sério\ldots{}

--- Qual muito sério! Sois um idiota.

\textls[20]{O diálogo foi longo. Júlio logrou acalmar o
velho depois de muitas horas.}

O coronel não cessava de dizer:

--- Com que cara hei de eu ficar diante desta gente?

O filho, porém, não se demoveu do seu intento e o velho
teve de resignar-se ao desconsolador papel que lhe estava
destinado.

\textls[15]{À hora do almoço, chamou de parte o comendador, e, muito
titubeante e nervoso, disse-lhe tudo.}

\textls[-15]{O comendador não respondeu; indignado e pálido, voltou-lhe as
costas e foi encerrar-se no seu quarto. Otávio redobrou de
atenções para com o coronel, acompanhando-o ao meio-dia à estação.}\looseness=-1

\textls[15]{Voltando da gare, Otávio indagou numa loja de
ferragens, armarinho e louça, onde àquela hora se reunia meia
dúzia de pessoas, que fazia ali o seu cenáculo discutindo
política e tomando café, se o sr.\,José Ribas estava presente, ou se lhe
poderiam indicar onde o encontraria.}

O dono da loja, um sujeito gordo, moreno e risonho, veio
à porta, explicando não se achar o José Ribas na cidade; mas
que, se o sr.\,Otávio Medeiros quisesse, poder-lhe-ia
transmitir algum recado, visto estar nesse momento empacotando machados,
enxadas, facões e garruchas que ele lhe tinha mandado pedir.

--- Entre\ldots{} entre, escreva à sua vontade ---
insistia o lojista.

Otávio, aproveitando o convite, foi sentar-se à
secretária.

A pequena distância, um caixeiro em mangas de camisa ia
encaixotando os objetos, mencionando-os em voz alta.

--- Uma garrucha\ldots{} três machados\ldots{} cinco
enxadas\ldots{} dois revólveres\ldots{} No balcão, um outro caixeiro,
muito asseado, ia escrevendo e repetindo: uma garrucha\ldots{}
três machados\ldots{} etc.

\textls[-30]{--- Para que diabo quererá o Ribas tanta coisa? ---
perguntou de um canto um sujeito magro, de chapéu de palha, que
estava na prosa.}\looseness=-1

\textls[20]{--- Sei lá!\ldots{} O que afirmo é que tenho feito muito
negócio ultimamente.}

--- Então sempre será certo?

--- O quê?

--- A revolução dos negros\ldots{}

--- Pelos modos\ldots{} em todo o caso é bom estar prevenido.

\textls[15]{--- É sim; faz o José Ribas muito bem; que se os negros pilham
os senhores sem força\ldots{}}

--- Deus nos acuda!

Nesse momento entrou ousadamente um caipira na loja e,
dirigindo-se ao dono da casa, disse:

--- Eu venho emprestá um facão pra mecê\ldots{}

--- Eu não empresto; vendo.

O caipira ficou um momento pensativo; depois, tomando uma
resolução:

--- Pois entonces pode ponhá na conta de seu Zé
Riba\ldots{}

--- Pois sim. Como é o seu nome?

--- Meu nome?

--- Sim.

--- Ah! Eu me chamo Furtuoso da Cunha.

\textls[15]{Enquanto serviam o Frutuoso, Otávio agradeceu ao
lojista o seu obséquio e foi esperar fora o caipira, que um bom acaso
lhe havia apresentado.}\looseness=-1

\textls[15]{O capanga Frutuoso era chuchado, baixo, de olhos pequenos,
penetrantes, lábios finos, narinas dilatadas.}

Arrojado, apesar da aparência insignificante, ele era o primeiro
a atirar-se num perigo; desconhecia o temor, deleitava-se com o cheiro
do sangue, provocando as ocasiões de o derramar para molhar nele as
mãos, voluptuosamente. Ágil e astucioso, escapava sempre às
garras da polícia, e, enquanto os soldados o procuravam nas
tavernas, estradas e povoações, ele ria-se deitado sobre o ventre num
quintalzinho ignorado, onde improvisava uma rinha, vendo e incitando os
galos brigadores que se arrancavam as penas e a pele, sangrando-se e
cegando-se mutuamente.

\textls[-10]{--- Eta, carijó, ataca! --- gritava ele, apostando com
outros caipiras que se agrupavam, acocorados, contando o dinheiro
amarrado na ponta dos lenços, e dando guinchos de raiva ou de prazer!}\looseness=-1

\textls[-25]{Era da boca desse homem que Otávio esperava ouvir as
palavras da verdade, a confissão sincera de um erro, a revelação de um
nome execrável e a salvação de Eva. Mas Frutuoso era leal aos seus
patrões.}\looseness=-1

\textls[20]{Vendo-o sair da loja de ferragens, Otávio chamou-o.
Frutuoso aproximou-se, disse tê-lo reconhecido e seguiu a seu
lado, sem revelar a mínima perturbação. Chegados a um ponto
isolado da rua, pararam.}

\textls[15]{Estavam entre dois muros de quintais. O sol ardentíssimo batia
de chapa nas pedras irregulares da calçada. Num dos muros debruçavam-se
curiosamente para a rua as hastes intrincadas de uma trepadeira sem
flores; no outro nem um vestígio de vegetação; viam-se-lhe
unicamente em cima, espetados e ameaçadores, muitos cacos de vidro
brilhando ao sol.}

--- Vamos --- começou Otávio ---; já
sei que foi você quem tentou matar-me, e olhe que não me zango
por isso; o que desejo saber é o nome da pessoa que o assalariou para
tal fim. Quem foi?

Frutuoso sorriu, e depois, passando vagarosamente a mão pela
cara:

--- Não sei\ldots{} pra que serve dizê?

--- Isso é comigo. Diga-me, quanto quer pelo seu segredo?

--- Nada. Eu não recebo dinheiro quando não trabaio.

--- Pois bem, trabalhará depois por minha conta.

\textls[15]{O caipira mirou o seu interlocutor da cabeça aos pés, e sorriu
de novo.}\looseness=-1

--- Hum\ldots{} o senhô? Quá!

\textls[20]{--- Eu, sim! Olhe\ldots{} --- e Otávio falou
largamente, tentando por todos os meios, dos mais brandos aos mais
violentos, obrigar o capanga a dizer a verdade.}

\textls[-16]{Frutuoso, porém, fechava-se na sua obstinação e nada o arrancava
dela; até que por fim, ouvindo de Otávio como suspeito o nome
do Antunes, deu uma gargalhada franca, dizendo logo depois:}\looseness=-1

\textls[15]{--- Tá enganado. Seu Antune é um tolo
pra estas coisa\ldots{} só digo que, quem arranjou e pagou a
gente pra esse selviço foi muié, não foi home!}

--- Mentira! --- exclamou Otávio, indignado,
percebendo a que mulher se referia o caipira.

--- Ah pois --- respondeu o outro pachorrentamente
---; si quisé eu lhe amostro uma porva\ldots{}

\textls[-10]{--- Que venha e já! --- respondeu Otávio, estendendo-lhe a mão.}\looseness=-1

--- Não tá qui! Aminhã de taldinha mi espere
nesse lugá memo, que eu venho e entrego ela pra mecê.

\textls[-20]{--- Não! Eu prefiro ir buscá-la hoje mesmo em casa do
José Ribas.}\looseness=-1

--- Como é qui mecê sabe qui eu pouso lá?!

--- Sei.

--- Tá dereito; \emph{mais} hoje eu não vou pro
sítio.

--- Então amanhã\ldots{}

--- A que hora?

--- Às sete da manhã.

--- Cuns diabo!\ldots Tá dito! Eu espero mecê.

E separaram-se.

\textls[15]{Otávio seguiu perplexo, pensando nas palavras do capanga,
revendo-lhe a fisionomia móbil e impenetrável, o piscar nervoso dos seus
olhinhos malignos e os repetidos movimentos da sua boca movediça, como a
dos coelhos. Decidiu ir nessa mesma tarde à fazenda do Ribas.}

O Frutuoso poderia tê-lo enganado\ldots{} Num homem
daqueles não se podia fiar\ldots{} Desejava apoderar-se nessa mesma
tarde do tal documento oferecido espontaneamente pelo caipira.

\textls[-15]{Antes do pôr do sol, chegava à fazendo do José Ribas, e
aí foi informado pela mulher do feitor, de uma maneira terminante e
seca, de que não havia homem branco em casa; ficariam todos na
cidade.}\looseness=-1

--- Mas o Frutuoso?

\textls[15]{--- Qual Frutuoso! Hoje não vem cá ninguém! É meió o senhô
vortá amanhã\ldots{} --- e voltou-lhe as costas.}

\textls[15]{--- Malcriada! --- resmungou entre dentes
Otávio, descendo a pequena escada.}\looseness=-1

\textls[-25]{Eram oito horas da noite quando tomou a estrada do Sertãozinho.}\looseness=-1

%\end{linenumbers}

\section{XXIX}

%\begin{linenumbers}

\textls[-15]{--- Posso entrar? --- perguntou o Trigueirinhos com a voz
transtornada, batendo na porta do quarto de Otávio, onde havia ainda
luz.}\looseness=-1

\textls[20]{--- Pois não! --- respondeu o moço, atônito, indo ao encontro do
cunhado.}

\textls[15]{Era mais de uma hora da madrugada; pela janela aberta entrava o
ar úmido da noite, uma noite pesada, sem estrelas. Sobre a
mesa, ao lado da vela e do tinteiro aberto, estava uma carta acabada de
sobrescritar ao dr.\,Morton.}

\textls[20]{Trigueirinhos, muito amarelo, entrou cambaleante, deixando-se
cair numa cadeira.}

--- Que tem você? --- perguntou-lhe Otávio
assustado.

\textls[15]{Trigueirinhos tentou responder, mas a palavra sumiu-se-lhe nos
lábios, num som mal articulado e incompreensível.}

\textls[15]{--- Sente-se doente? --- tornou Otávio,
curvando-se com interesse para ele.}

--- Que não --- respondeu-lhe o outro, com um gesto e
um sorriso de louco.

Decorreram alguns minutos no maior silêncio. A luz da vela
tremulava; uma ave noturna passou rente à janela; o
Trigueirinhos, estremecendo, chegou-se mais para o cunhado e disse, num
arrastar de língua que lhe tornava a palavra confusa:

\textls[20]{--- Vou contar-lhe tudo\ldots{} mas peço-lhe segredo\ldots{} Vim
bater à sua porta porque estou com medo\ldots{}}

--- Medo? Medo de quê?!

Trigueirinhos relanceou os olhos pelo quarto e, passando pelos
cabelos a mão morena, pequena e seca, murmurou:

--- Escute\ldots{}

\textls[15]{Suspendeu de novo a palavra, como se alguém lhe
estivesse a apertar o pescoço.}

Otávio procurou acalmá-lo, assegurando-lhe que ali
podia estar tranquilo, mas que falasse!

\textls[20]{O cunhado refletiu, e continuou depois num tom apenas
perceptível:}

--- Muitos fazendeiros aqui do município estavam
indignados com o juiz, por causa dos escravos que ele protegia e
acoitava.

--- Já ouvi isso\ldots{}

\textls[15]{--- Pois é!\ldots{} E, como ele não quis nunca aceder nem a
rogos nem a ameaças, deliberaram\ldots{}}

\textls[15]{Aqui o Trigueirinhos engoliu em seco e baixou
a cabeça, num movimento sombrio.}

--- Deliberaram o quê? --- indagou Otávio.

\textls[15]{--- Matá-lo! --- respondeu Trigueirinhos com a
voz estrangulada e os cabelos hirtos.}

\textls[-20]{Otávio endireitou o busto, deixando cair pesadamente a mão sobre
o ombro do cunhado e dizendo-lhe num tom áspero e vibrante:}\looseness=-1

\textls[30]{--- Mas você não entrou na combinação, diga-me, você não
entrou!}

--- Entrei\ldots{} --- murmurou o outro quase de
joelhos.

\textls[-20]{Otávio, num pasmo terrível, ficou silencioso;
entretanto, Trigueirinhos ia murmurando na grande humildade de quem se
confessa:}\looseness=-1

--- Ficamos de nos ajuntar todos hoje, e de irmos à
casa do juiz tirar os escravos que ele tem lá e\ldots{} \emph{mais} eu
não tive coragem. Cheguei lá embaixo às onze horas, vi já muita gente
reunida no pasto do Simões Rodrigues, onde combináramos deixar os
animais\ldots{} então, não sei o que foi que eu senti\ldots{}
\emph{mais} puxei das rédeas e voltei depressa para casa\ldots{} Todo o
caminho parecia-me sentir vir atrás de mim aquela gente toda, a arrastar
o juiz\ldots{} Entrei pelo corredor como um bêbedo, vi luz no seu quarto
e bati.

\textls[15]{Trigueirinhos, muito amarelo e nervoso, dava já livre curso
às suas palavras, que se precipitavam, comprometendo-o,
revelando os nomes dos outros numa catadupa impetuosa.}

\textls[15]{Querendo retroceder, ele avançava sempre, gesticulando como um
náufrago a lutar com a onda.}

\enlargethispage{1\baselineskip}
\textls[15]{Acentuara-se a sua cor terrosa; as suas mãos magras e
finas crispavam-se, salientando nuns cordões escuros o ramo das veias, e
as sobrancelhas apertavam-se ou subiam e desciam, enrugando-lhe a testa
acanhada. Parecia um doido no momento de crise. De vez em quando parava,
julgando ouvir um tiro, e punha-se na atitude de quem escuta.}

\textls[15]{Otávio, à vista daquela excitação, já lhe não dava
crédito, julgando-o doente, e aconselhou-o a que repousasse.}

\textls[-15]{Para não assustar Nicota, disse-lhe que ficasse ali mesmo, na
sua cama, e ajudou-o a despir-se. Trigueirinhos caíra numa docilidade de
criança e deixava-se mover sem repelir nem auxiliar. Tirando-lhe o
casaco, Otávio viu-lhe na cinta um revólver e uma faca de mato
ainda nova, além da outra do costume, na cava do colete.}\looseness=-1

\textls[-30]{--- Que é isso!? Você estava realmente preparado para o
ataque!\ldots{}}\looseness=-1

\textls[15]{E, de repente, lembrando-se do que ouvira na loja de
ferragens, percebeu que não tinha diante dos olhos um doido
perseguido por uma visão, mas sim um culpado atormentado pelo remorso; e
então, sem dó, indignado e trêmulo de raiva, atirou sobre o cunhado uma
tempestade de censuras.}

\textls[15]{Trigueirinhos esbugalhava os olhos e no seu terror não
encontrava réplica.}\looseness=-1

--- Precisamos avisar o homem e salvá-lo! ---
disse Otávio.

\textls[20]{--- Como? --- articulou o outro com um raio de
esperança a luzir-lhe no olhar.}

\textls[15]{--- Como\ldots{} não sei, hei de pensar nisso em caminho! Onde é
que ele mora?}

Trigueirinhos indicou a casa do juiz.

\textls[-15]{O cavalo em que viera estava ainda selado, com as rédeas presas
a um argolão da porta. Otávio desamarrou os nós que a
mão nervosa de Trigueirinhos fizera, e montando o animal partiu
a galope.}\looseness=-1

%\end{linenumbers}

\section{XXX}

%\begin{linenumbers}

\textls[-15]{Trigueirinhos ficou só. Sentia percorrer-lhe a espinha dorsal
como que uma cobra de neve, e encolhia-se todo. Arregalava os olhos,
fixando-os na porta aberta para o vazio negro do corredor. Queria fechar
aquela porta, que se abria para a treva, e não se podia mover.}\looseness=-1

\textls[15]{Desejava iluminar bem o quarto, ouvir vozes amigas, sentir o
rumor da casa, que o tranquilizasse, e ver a claridade do sol. Aquele
silêncio e aquela escuridão apavoravam-no. Procurava com a vista,
ansiosamente, ora a janela, ora a porta, esperando e temendo ver numa e
noutra surgir, sinistra e aterradora, a cabeça do morto. Como estivesse
no meio do quarto, temeu que alguém, à traição, se colocasse
atrás dele, e recuou espavorido até unir as costas à parede
caiada e fria; mas de um e outro lado, os buracos escuros da janela e da
porta enchiam-no de terror, e engatilhou o revólver, disposto a
defender-se dos fantasmas que a sua imaginação excitada lhe sugeria.}

Houve um curto instante de alívio! No interior da casa
uma criança chorou! Enquanto durou o som débil dessa voz infantil, um
grande conforto revestiu a sua alma de homem. Desgraçadamente a criança
foi depressa acalentada e tudo recaiu na grande quietação anterior. A
viração da noite tornava-se cada vez mais úmida, e trazia dos
campos o aroma fresco do capim melado e os sons esparsos e indecifráveis
dos animais e da vegetação.

\textls[-15]{Pelo espírito desordenado de Trigueirinhos passavam e
repassavam numa vertigem todas as cenas que presenciara, desde
o dia em que um grupo de amigos o foi convidar para a \emph{grande
desforra}!}\looseness=-1

\textls[10]{Tinha sido ali mesmo, no Sertãozinho! O dr.\,Azambuja e mais
cinco amigos diziam todo o mal do juiz, jurando que o haviam de
amordaçar de uma vez. E discutiam baixo a política e o
caráter do homem, pedindo a opinião e o auxílio do
Trigueirinhos, que fugia com evasivas, sem saber que partido tomasse.}

\textls[-15]{Repugnava-lhe a ação proposta; chegou mesmo a avançar um
``mas\ldots'', os outros porém caíram-lhe em cima nuns termos enérgicos,
ressumantes de indignação, torcendo a lei, esmagando a moral, abafando o
que houvesse de bom nos seus corações, abrindo unicamente a
válvula ao ódio, que explodia deles em borbotões.}\looseness=-1

\textls[15]{--- Todos os lavradores estão contra ele; veja se quer ser só
você o piegas! --- exclamou o Azambuja.}

--- Venha daí! --- insistia outro. --- Um homem é um
homem!

\textls[-15]{Trigueirinhos tentou ainda esquivar-se, mas, não sabendo
discutir com o Azambuja e receando principalmente que o chamassem
``poltrão'', anuiu a tudo, embora muito constrangidamente. Desde
esse momento envolveram-no numa rede de combinações.
Enterravam-no na questão, comprometiam-no. O caso era fazer
número, estendendo a toda a gente a responsabilidade do ato.
Assim, faziam reuniões, forjavam intrigas e espalhavam pela cidade
boatos aterradores, designando astuciosamente uma data certa para uma
revolta de escravos, afastando toda a suspeita de que fossem eles, os
brancos, os ricos, os senhores, os proprietários, quem andasse pelas
ruas de revólver em punho, para despedaçar a felicidade de uma
família honesta e pobre. Os burgueses medrosos encolher-se-iam
entre os lençóis, deixando lá fora berrarem os negros
à vontade.}\looseness=-1

Trigueirinhos seguia todos os planos, boquiaberto, assustado,
tentando fugir àquelas combinações, mas obedecendo sempre.

Chegara o dia designado para o assalto. Trigueirinhos passou-o
em jejum; tinha fastio, amargos na boca e arrepiava-se-lhe o corpo com
um frio nervoso e úmido. Ao aproximar-se a noite sentiu
recrudescer o seu mal-estar; teve vontade de dizer tudo ao cunhado e ao
sogro; chegou a procurá-los para isso, mas a coragem fugiu-lhe
e, abatido pelo medo de uma censura, transigiu com a
consciência. Deliberou não comparecer e estava nesse propósito
quando um capanga do Azambuja o foi avisar de novo da parte do patrão,
com a instante recomendação de não faltar.

``Seja homem!'', dizia-lhe no seu estribilho o amigo.
``Acompanhe-nos no perigo como nos tem acompanhado nos dias bons!''

\textls[20]{Trigueirinhos julgou de seu dever e lealdade assim fazer. Às
onze horas da noite montou a cavalo e partiu. O céu sem estrelas parecia
mais profundo e mais vasto; as árvores, mal sacudidas pelo vento sutil
do verão, agitavam preguiçosamente as suas ramas escuras; a natureza,
bela e serena, suscitava pensamentos calmos, ideias de tranquilidade e
doçura. Era meia-noite quando Trigueirinhos chegou ao pasto do Simões
Rodrigues. Assaltaram-no com mais vigor ideias desencontradas: via o
condenado coberto de sangue, ouvia a opinião pública depois,
açoitando o seu nome salpicado de lama, e vinha em cortejo a prisão, o
interrogatório\ldots{} o desprezo da família e
a impossibilidade de continuar a viver ali!}\looseness=-1

No pasto estavam já cerca de duzentas pessoas; ouvia-se
um rumor surdo de vozes e a bulha seca, de vez em quando, de
uma arma que se aprontava. Trigueirinhos então, sem saber por
quê, como um autômato, deu volta ao animal e
pôs-se a galope para casa. Parecia-lhe que o perseguiam, que o
vaiavam, chamando-o repetidas vezes: ``Pusilânime! Poltrão!''. Tentava
parar, ir reunir-se aos companheiros, mas esbarrava com a visão do
morto; e a ideia do sangue punha-o fora de si. Sentia
necessidade de se refugiar em alguém. Iria acordar a mulher,
pedir-lhe que o defendesse, rogar-lhe uma proteção que não supusera
nunca dever-lhe. Entrou à pressa em casa, caiu nos degraus; levantou-se
arrepiado, transido de medo, imaginando sentir atrás de si o Azambuja a
dizer-lhe com ironia: ``Fraco! Medroso! Desleal!''. Viu então luz no
quarto de Otávio, apoiou-se à porta quase
desmaiado e bateu.

%\end{linenumbers}

\section{XXXI}

%\begin{linenumbers}

Quando Otávio chegou à cidade, amanhecia.
Pelas janelas entreabertas apareciam cabeças curiosas. Aqui e acolá, um
mais afoito atrevia-se a atravessar a rua ou a seguir mesmo até o lugar
em que se passara pouco antes uma medonha cena de sangue.

Em um instante, Otávio parava em frente à casa
do juiz.

\textls[15]{Na calçada fronteira algumas pessoas comentavam o triste caso,
olhando para as janelas sem vidros, as paredes esburacadas pelas balas e
pelas pedras dos assaltantes; a porta arrombada, caída para trás,
mostrando o corredor escuro por onde, duas horas antes, se precipitara,
sedenta e bárbara, a multidão dos assassinos. Entre os espectadores da
rua, achava-se o dono da loja de ferragens, que, vendo Otávio
apear-se do animal, atravessou pressurosamente a rua, indo dizer-lhe:}\looseness=-1

\textls[-15]{--- Que desgraça, meu amigo, e que vergonha! Que hão de dizer
lá fora, quando souberem disto! E lembrar-me que fui eu, eu!,
quem forneceu a toda esta gente os instrumentos para o crime!}\looseness=-1

--- Está então tudo acabado, não é verdade?

--- Tudo!

Otávio estremeceu; o ferragista continuou:

--- Acordei às quatro horas, ouvindo um barulho de
tiros e gritaria. Chamei o moleque e mandei-o saber o que era\ldots{} o
moleque viu tudo\ldots{} foi-me dizer, eu vim e encontrei uns amigos que
já estavam cá e me disseram o mesmo que me tinha dito o
pajem\ldots{} uma desgraça! Que hão de dizer de nós lá
fora! Quer entrar? Vamos\ldots{} eu já lá estive; entre, olhe,
venha comigo!

--- Mas\ldots{} poderemos entrar assim?\ldots{} E a autoridade?

--- Ora! Se é a própria \emph{autoridade} que está
morta!

--- Pois sim, entremos.

\textls[15]{Atravessaram juntos toda a casa, por entre trastes quebrados e
caídos, nódoas de sangue e estilhaços de vidro.}\looseness=-1

Estendido no quintal, sob uma das janelas, estava o
cadáver do juiz, todo mutilado.

\textls[-20]{Grupos de curiosos conversavam no quintal, em frente ao
assassinado; alguns de condição inferior espreitavam por cima do muro.}\looseness=-1

O ferragista explicava:

\textls[15]{--- Veja o senhor! Quanta pancada levou este infeliz! E algumas
vibradas pelos próprios parentes.}

A um gesto de incredulidade de Otávio:

\textls[-15]{--- Pois então! Na turma estavam muitos lavradores da família\ldots{}}

E enumerou uns poucos.

\textls[20]{--- Olhe, quem presenciou tudo foi ali aquele moço\ldots{} Ó seu
Juca? Faça favor: venha cá!}\looseness=-1

\textls[15]{Seu Juca aproximou-se. Era um rapaz ainda novo, inteligente, de
olhar experto, filho de um fazendeiro de Mogi das Cruzes, e que
andava à escolha de sítio por ali.}

Um instante depois ouvia Otávio a narração sucinta dos
fatos. A seus pés, coberto de nódoas e sangue coagulado, estava o corpo
do juiz numa tira de sombra projetada pelo telhado. O grupo dos
comentadores engrossou: já não era só um a contar a história,
eram dois, quatro, cinco, dez! Afirmavam todos o mesmo:

\textls[-15]{Às quatro horas da manhã subiram, reunidas e armadas, do pasto
do Simões Rodrigues, cerca de cento e tantas a duzentas
pessoas. E assim, arrebanhados ainda alguns amigos pelo caminho,}

\pagebreak

\noindent{}\textls[-15]{chegaram à casa do juiz; cercaram-na, entrando muita da gente à
força, pelos fundos, dando tiros, arrombando portas e gritando: ``Deita
cá para fora os nossos escravos! Apareçam os abolicionistas!''.}\looseness=-1

\textls[-5]{Vendo-se ameaçado, o juiz tentou fugir e aconselhou à mulher que
se salvasse com os filhos\ldots{} A pobre senhora atirou-se por uma
janela; o marido seguiu-lhe o exemplo, mas, desgraçadamente,
esperavam-no já embaixo, no quintal, os seus assassinos. Principiou
então com toda a ferocidade a negra ação. Espancaram, mutilaram,
estrangularam a vítima, sem atenderem aos rogos da esposa, que se
arrastava de joelhos, implorando misericórdia.}\looseness=-1

À sua voz dolorida, transtornada pelo pavor e pelas lágrimas,
unia-se a voz chorosa e doce de uma filhinha inocente; mas os
sicários respondiam-lhes com termos injuriosos, redobrando os
golpes, com sanguinolenta fúria. Só quando viram tudo em
destroços foi que eles saíram, felizes, satisfeitos da sua obra.

\textls[15]{Raiava a madrugada. Otávio ouvira tudo isto comovido,
com os olhos fixos no morto.}

\textls[20]{--- O que é extraordinário --- dizia um --- é
que a polícia não apareceu!}\looseness=-1

--- Pudera! --- exclamou ironicamente um outro. --- Se
o primeiro suplente do delegado, logo aos primeiros tiros, foi
à cadeia proibir a guarda que saísse!

--- Oh!

--- Como?

\textls[15]{--- Dizendo ao sargento comandante do destacamento que o grupo
era muito grande, a cadeia tinha muitos presos, e que dizia aquilo por
ter sido ele, o suplente, nomeado delegado!}

--- Oh! Diabo!

--- Que me diz, homem?

--- A verdade.

\textls[15]{Soltaram exclamações de indignação e a biografia do suplente foi
exposta e comentada por várias pessoas do lugar.}

\textls[25]{A claridade do sol aproximava-se, lambendo já os pés do
cadáver. Otávio estendeu maquinalmente a mão ao lojista
e ia retirar-se quando ouviu a um dizer:}\looseness=-1

\textls[20]{--- Aquela pancada na cabeça foi dada pelo capanga do José
Ribas\ldots{} um tal Frutuoso\ldots{}}\looseness=-1

\textls[20]{--- É exato! --- confirmou o outro. --- E que
demônio de força, hein?}\looseness=-1

\textls[20]{--- Ele não se contentou de matar; roubou também; que eu o vi
saltar a janela\ldots{}}

--- Eu também vi\ldots{} com uns papéis\ldots{}

--- Que vergonha! --- murmurava o lojista, abanando
melancolicamente a cabeça. --- Que hão de dizer de nós lá
fora!

\textls[15]{Aquela preocupação acabou de impacientar Otávio. O nome
do Frutuoso abalara-o. Lançando um derradeiro olhar de compaixão ao
morto, saiu.}

\textls[15]{Ia terrivelmente impressionado; tudo lhe parecia ignóbil ---
homens e coisas; olhava para as casas, como se todas fossem
antros de assassinos, e em qualquer recanto de paisagem, por mais
sereno, parecia-lhe distinguir um negro vestígio de conspiração
e de intriga! Montando a cavalo, seguiu a galope pela estrada do
Sertãozinho; mas, logo no começo, teve de parar; vinha o pajem
do cunhado, que lhe disse ir chamar um médico.}

\textls[-15]{--- Nhonhô tá qui nem morto! Tá memo sem
talento}\looseness=-1\footnote{Talento: sinônimo de força.} \textls[-15]{nenhum!}\looseness=-1

--- Meu pai está em casa?

--- Tá, sim sinhô.

--- E o coronel Ribeiro?

--- Também tá lá!

--- Bem.

\textls[15]{Em vez de seguir para casa, Otávio tomou o caminho do
sítio do José Ribas. A sua presença agravaria talvez o estado do
marido da irmã; além disso urgia arrancar do maldito Frutuoso a
prometida prova. Repugnava-lhe ter de pedir àquele homem uma palavra, em
que repousaria toda a sua felicidade futura. E que confiança poderia ele
depositar nessa palavra? Deixar que aqueles lábios mentirosos e
impuros pronunciassem o querido nome de Eva era já uma profanação! Mas
era preciso. E se acaso o Frutuoso a apontasse como autora daquela
baixeza, dar-lhe-ia ele crédito? Não, tapar-lhe-ia a boca, e havia de
chamá-lo infame. Desceria à violência, e por mais que ouvisse
repetir ``foi Eva, Eva!'', ele diria ``não, não e não!''. Pensava em
tudo isso e continuava, contudo, a galopar para a fazenda do José
Ribas.}

Há na vida de toda a gente, mesmo na dos mais sensatos, uns
momentos de loucura, em que fatalmente se cai nas mais absurdas
contradições! A razão diz-nos:

``É inútil! Não faças isso!''. Mas lá desponta no
íntimo uma dúvida misteriosa, um ``quem sabe?'' a impelir à
ação, mesmo contra a vontade!

\textls[-20]{Era o que sucedia a Otávio. Chegando à casa do
José Ribas, viu a uma janela a mulher com quem falara na
véspera, caseira do sítio.}\looseness=-1

--- Poderei hoje ver o Frutuoso? --- perguntou-lhe o
moço.

--- Tchê, poá! Que sperança! Mecê é desinfeliz.\footnote{Infeliz.}

--- Por quê?

--- Porque tá campeando\footnote{Procurando.}
ele à toa!

--- Já se foi embora?!

--- Eia! Gente! Desde já hoje de menhãzinha.

--- E quando voltará?

--- Não vorta mais; aquele só qui qué é andá vira-virando!

--- Mas ele disse que não voltava? Tem certeza disso?

\textls[20]{--- Não disse, \emph{mais} nóis viu; arranjou os trem dele mal e
má, falou com nhô Zé e saiu!}

\textls[15]{Otávio, perplexo, maldisse a sua ingenuidade. O Frutuoso
enganara-o duas vezes; afirmando ter uma prova que não possuía
e emprazando-o a procurá-lo numa hora em que já tencionava
estar longe! Para onde teria ido o caipira?}

E para que procurá-lo? Tornaria a ser enganado, duas,
dez, vinte vezes, e escarnecido outras tantas! Era uma lição. Pobre Eva!
Por que mundo de baixezas arrastavam o seu nome!

\textls[-10]{Quando Otávio entrou em casa, perguntou à irmã pelo doente.}

--- Ele está muito desinsofrido;\footnote{Insofrido.}
o médico disse que é um ataque de cabeça; venha ver\ldots{}

\textls[20]{O Trigueirinhos coçava as pernas queimadas dos
sinapismos, fazendo trejeitos com a boca e
revolvendo-se todo. Ao deparar com o cunhado, pôs-se a dizer
embrulhadamente:}

\textls[20]{--- Eu não sou cúmplice! Eu não matei ninguém! Juro! Juro! Sou
incapaz de semelhante infâmia!}

\textls[20]{O médico, à cabeceira, procurava acalmá-lo.
Ouvia-se pelo corredor o rumor das saias das senhoras, que iam e vinham
em bicos de pés.}\looseness=-1

À tarde, Trigueirinhos caiu num sono reparador.

\textls[-20]{Dias depois o médico aconselhava que tirassem o doente
dali, que o transportassem para outra terra, ao menos temporariamente.}\looseness=-1

\textls[20]{A família Medeiros deliberou então voltar para Santa
Genoveva, com a Nicota e o enfermo. Ele, como um idiota, não se opunha;
deixava que o movessem, sem protestos.}

O terror espalhara-se por toda a cidade. Só se falava no crime;
via-se sangue em todas as mãos, mentira em todas as caras. As
famílias fugiam medrosas e horrorizadas; o lugar ficava
deserto.

\textls[15]{Enquanto os pais e irmãs seguiam diretamente para a estação,
Otávio foi despedir-se do lojista que o tinha sempre
distinguido. Achou-o cercado por meia dúzia de sujeitos, que
lhe iam bebendo o café e perguntando coisas.}

--- Em todo o caso foi uma lição --- dizia um.

--- Não acha que era perciso um exemplo
destes? --- interrogava outro, voltando-se para o dono da
casa; e ele, com energia:

--- Certamente! De agora em diante, os tais pescadores
de águas turvas hão de ter mais cautela\ldots{}

Otávio saiu enojado; o lojista compreendeu e foi
alcançá-lo na primeira esquina; aí, depois de olhar
precatadamente para um e outro lado, disse, numa voz baixa e assoprada:

\textls[-15]{--- O senhor compreende\ldots{} se eu contrariar a opinião desta
gente terei de fechar a casa! Os lavradores são os que nos sustentam!}\looseness=-1

Otávio respondeu-lhe evasivamente e continuou o seu caminho,
repetindo mentalmente a frase, que tantas vezes ouvira àquele mesmo
homem, em tom indignado e enfático:

--- Vergonha! Que hão de dizer de nós lá fora!

%\end{linenumbers}

\section{XXXII}

%\begin{linenumbers}

Quando a família Medeiros chegou a Santa Genoveva, o
feitor disse, muito amarelo e embaraçado:

--- Tamo sem gente!

O comendador não o compreendeu e fê-lo repetir a
frase:

--- Que está você pra aí a dizer, homem?

\textls[15]{--- Não tive curpa\ldots{} ---
murmurou o feitor envergonhado ---; os
demônio dos negro são memo o diabo!}

--- Explique-se! --- gritou colérico o comendador.

\textls[-15]{O empregado contou-lhe, muito humilde, que os escravos,
aproveitando a ausência dos donos da casa, tinham fugido em
massa.}\looseness=-1

--- Para onde?!

O feitor não pôde responder, baixou a cabeça, encolhendo
quase imperceptivelmente os ombros.

Dois dias antes, protegidos pela escuridão da noite,
puseram-se todos os escravos de Santa Genoveva a caminho para a
Serra do Cubatão, o misterioso refúgio dos pretos fugidos. O caminho era
longo, penoso, e as primeiras horas de marcha precipitada eram contadas
com as aceleradíssimas palpitações do coração!

\textls[15]{O batalhão esfarrapado, taciturno, seguia mudo, cortando a treva
da noite a largas passadas desiguais. Ouvia-se num grande anseio o
resfolegar cansado de cento e tantas criaturas que respiravam alto. Nem
uma palavra, nem uma queixa, nem um gemido! Esmagando a relva
seca, sob os pés calosos e chatos, os negros iam de olhar aceso,
bocas entreabertas, numa grande expressão de dor e de
ódio, como se em vez de caminharem para a liberdade
fossem em direção ao patíbulo!}

\textls[-10]{As mães, envolvendo nos xales rotos os filhos pequenos,
deixavam expostos à aragem fria da noite os ombros nus, marcados de
chicote, com o peito na boca das crianças, para que elas não chorassem;
essas obscuras heroínas despendiam as suas forças sem se
lamentarem, andando sempre, apesar do fardo, apesar da treva, apesar do
medo, com o ouvido à escuta, os lábios secos, os
soluços retidos na garganta! Os pais levavam pela mão os filhinhos mais
velhos, que foram às primeiras horas aos saltos, e a desmaiar
depois\ldots{} Alguns velhos que não quiseram deixar partir sem
eles os companheiros, arrastavam-se, abriam as mãos trêmulas, deixavam
rolar no chão o pau a que se arrimavam, e caíam de joelhos, com os olhos
cheios de lágrimas e os lábios cheios de preces.}\looseness=-1

\textls[10]{Procurando os desvios mais ensombrados, sem atender à
fome, nem atender ao cansaço, atravessou assim aquela gente os caminhos
pedregosos ou os caminhos alfombrados, sangrando a carne já tão pisada
do trabalho e martirizando a alma nas alucinações do pavor. A mais de
meio do caminho, quantos faltavam! Velhos e crianças, muitos haviam
ficado esparsos e mortos pela floresta, de pasto aos animais bravios e
aos bandos dos corvos!}\looseness=-1

\textls[15]{Levantava-se o dia e descaía a noite, sem que a alegria viesse
bater as suas asas sobre essas frontes. Sempre o medo da perseguição, o
susto do amanhã, do regresso à casa do senhor, do castigo, da
vergonha e da morte!}

\textls[15]{Chegando ao Tietê, largo e profundo, e julgando ouvir a
bulha dos soldados, num movimento incompreensível, duas mães atiraram os
filhinhos ao rio, e com os braços para o ar correram vertiginosamente
pela margem, até caírem de bruços, batendo com o peito e a cara
nas urzes que as feriram. E à sombra daquelas formosas árvores,
ficavam cadáveres dos vencidos, e sobre as folhinhas tenras das plantas
rasteiras o sangue dos outros que arfavam, sofriam e passavam!}

Entretanto, o feitor de Santa Genoveva aguentava por isso a
formidável descarga de injúrias que lhe lançava o patrão. O comendador
Medeiros, exasperado, exclamou por fim:

\textls[15]{--- Tudo isso é obra do tal Antônio Bento! Arre! O que
fizeram àquele demônio do juiz é o que deviam fazer a toda essa
canalha que nos desencaminha os escravos!}\footnote{O nome glorioso do
  dr.\,Antônio Bento, o denodado, o grande abolicionista de São Paulo,
  foi alvo de todas as injúrias de que tem agora a recompensa na
  gratidão popular. {[}Nota da 1ª edição.{]}}

E depois de uma pausa:

--- Então não ficou ninguém?!

--- Ficou só a Joaquina, e a Joana que está
pra morrê.

--- Só os cacos das negras velhas!

\textls[15]{--- Foi bom mecê levá o Saturnino, sinão
ficava sem ele também\ldots{} --- aventurou medrosamente o
empregado.}

\textls[20]{Algum tempo depois, mais animado, o feitor dizia em ar de
consolo que o Gusmão ficara igualmente sem todos os escravos; esses
haviam seguido para Campo Limpo e aí o chefe da estação, combinado com o
chefe do trem, fê-los entrar precipitadamente nos carros de
segunda classe.}

\textls[20]{--- Corja! Canalha do diabo! --- vociferava, rubro de
cólera, o comendador.}\looseness=-1

%\end{linenumbers}

\section{XXXIII}

%\begin{linenumbers}

Otávio desceu a ver a ama. A tia Joaquina, fazedeira de
sabão, estava sentada na soleira da porta da senzala, com os magros
cotovelos fincados nas pernas, e o rosto escondido nas mãos. Sentindo
passos, levantou os olhos, avermelhados e úmidos.

--- Que é isso, Joaquina, você estava chorando?

--- Nhô sim\ldots{} --- respondeu com um suspiro a velha.

--- A mamã está pior, já sei.

\textls[-10]{--- Ah! Nhô moço! Tia Joana vai indo mal e má\ldots{} eu tou
chorando porque o Jacinto foi também com os outro tudo na rodada\ldots{}
i agora cumo há de sê, quando tia Joana
preguntá por zere?}\looseness=-1

--- Tudo se há de arranjar, descanse. --- E
Otávio, comovido com a bondade da pobre Joaquina, entrou na
senzala da ama.

Ao fundo de um quarto amplo e escuro, ressumando
umidade das paredes e do chão, estava a pobre mamã deitada em
uma esteira sobre as tábuas do catre, entre trapos velhos de baeta e
algodão grosso. Espalhava-se um cheiro de sujidade e de mofo.
As moscas zumbiam, voejando sobre a moribunda, que já não as sentia. A
carapinha branca aureolava-lhe a estreita fronte enrugada; a
cor preta do seu rosto tornara-se numa cor de castanha
amarelada; os lábios murchos metiam-se-lhe para dentro da boca,
salientando o queixo muito agudo; o nariz afilado tomara a
forma correta do nariz dos brancos, e tinha uma
transparência de cera; os olhos, abertos e parados, iam-se
cobrindo pouco a pouco por uma névoa esbranquiçada que lhes empanava o
brilho.

\textls[20]{--- Vigie só! Tia Joana ficou tão pequena, que
inté chega dentro de uma xicaca}\footnote{Cesta
  quadrada.} \textls[20]{pequenina. Tá memo qui nem qui
  fosse criança! --- E a Joaquina chorava, arredando
  de sobre o corpo da companheira uns restos de cobertor cinzento.}

\textls[-10]{Realmente, o corpo da paralítica fora-se mirrando a tal
ponto que não parecia de um adulto. Os braços pendiam
finíssimos sobre a esteira, as mãos semelhavam aranhas
encarquilhadas e secas.}\looseness=-1

\textls[15]{Pela abertura do casaco de chita viam-se-lhe os seios
engelhados, moles, pendentes como dois saquinhos vazios, esses
mesmos dois peitos que ela pusera outrora tantas vezes,
férteis e túmidos, na sequiosa boca do pequeno Otávio.}

\textls[20]{--- Mamã\ldots{} --- murmurou ele, curvando-se para
afagá-la. ---Mamã! Sou eu, Otávio, quem está
aqui\ldots{} --- E pegou-lhe na mão, carinhosamente.}\looseness=-1

\textls[15]{A velha não se pôde mover, mas desfranziu os lábios
contraídos; ouvira a voz do filho. Ele repetia:}

--- Mamã\ldots{} mamã!

\textls[15]{E a névoa dos olhos dela ia passando, ia-se dissolvendo como uma
nuvem aos raios do sol; não podia falar, mas o olhar aparecia agora
claro, doce, feliz; um olhar agradecido, um olhar saudoso e meigo, que o
inundava de ternura e de amor.}

--- Mamã\ldots{} pobre mamã!\ldots{} ---
repetia Otávio, comovido e trêmulo. E o olhar dela
respondia-lhe: ``obrigada pela tua piedade! Eu esperava por ti para
morrer!''.

Foi como se a sua alma, a sua vida, lhe tivesse saído pelos
olhos. Joana expirou.

\textls[-10]{O filho branco, que tanto orgulho lhe tinha dado sempre,
adoçou-lhe o passamento. A falta do Jacinto, seu próprio filho,
nascido das suas entranhas, foi bem compensada. Nem um relance de vista,
à procura dele! Esqueceu-o como se lhe tivesse adivinhado o abandono.
Otávio era o seu deus, Otávio era o seu amor.}\looseness=-1

Desde pequeno fora Jacinto o preterido; o filho preto
que rolasse nu, esfomeado e sujo, contanto que o filho branco gozasse a
fartura do bom leite, a maciez da cambraia e o aroma da limpeza.

\textls[15]{Entre as escravas era isso frequente. Iam da roça calejadas da
enxada para a alcova da parturiente, recebendo nas mãos
magoadas pelo castigo e pelo trabalho o mimoso corpinho
acetinado e róseo da criança que haviam de amamentar.
Desde então, a sua atenção, o seu desvelo, o seu carinho convergiam para
o menino branco; o preto que morresse\ldots{} era um desgraçado de
menos. E assim acontecia a maior parte das vezes.}

\textls[-20]{Por mais extraordinário e inverossímil que o caso
pareça, o certo é que era assim. A doçura do amor materno, amor tanto
mais cheio de sacrifícios quanto mais intenso, não lhes vibrava
na alma o sentimento, ou era abafado pela voz do orgulho. É de crer,
porém, que o principal motivo dessa anomalia fosse a
miséria da sua condição.}\looseness=-1

Era até bom que os filhos morressem em pequeninos, desconhecendo
os amargores a que estavam destinados\ldots{}

A pobre Joana fora dessas. Tivera muitos filhos,
criara várias crianças brancas; de tempos a tempos ia
emprestada pelo senhor para uma ou outra família, como se
fosse uma coisa ou um animal inferior; não se revoltava\ldots{}
se nascera cativa! E dava o seu sangue aos filhos alheios, enquanto os
seus morriam à fome.

\textls[15]{Otávio fora o último e era entre todos o
único que até os dez anos a beijava, pendurando-se ao seu
pescoço. Os mais olhavam-na de cima, com a sobranceria do branco, e só
muito raramente a procuravam. Além de tudo, como se conservara em casa,
o Jacinto não teve o destino dos irmãos; uma ou outra parceira lá o
fazia tomar de vez em quando um caldo ou lhe mudava a camisola, pondo-o
consoladamente no enxuto.}

\textls[20]{Como ia longe esse tempo! Aí estava agora mirrada, fria,
mumificada entre panos rasgados, sobre uma esteira úmida, a
gorda mamã de outrora, a ativa e enérgica Joana!}\looseness=-1

\textls[15]{A Joaquina acendeu um círio, guardado por prevenção
para esse momento, e colocou-o aos pés do cadáver. A chama da
vela, que nenhuma brisa agitava, destacava-se amarela e fixa do fundo
escuro da parede, pondo uma estreita réstia pálida no esfarrapado
cobertor cinzento.}

--- Eu não maginava que tia Joana havera de morrê hoje!
--- dizia chorosa a companheira da morta, cruzando-lhe com
dificuldade as mãos no peito. --- A povre!

\textls[-25]{Pelo quarto escuro e úmido espalhava-se o cheiro frio
da morte\ldots{}}\looseness=-1

Otávio, com os olhos rasos de lágrimas, olhava silencioso para a
pobre mamã, enxotando-lhe as moscas.

%\end{linenumbers}

\section{XXXIV}

%\begin{linenumbers}

\textls[15]{Durante um mês as coisas correram monotonamente em Santa
Genoveva. Trigueirinhos convalescia com lentidão, mostrando-se sempre
nervoso e fraco. O comendador indagava do paradeiro dos escravos,
amiudava as visitas ao Antunes e vociferava contra a inépcia das
autoridades.}

\textls[15]{As senhoras cosiam, andavam mais atarefadas com os serviços
caseiros, ensinando criadas italianas há pouco emigradas do
seu país, com quem não se entendiam absolutamente. Noêmia
ria-se dos quiproquós dados entre a mãe e as servas, e na
despreocupação dos seus quinze anos, ia deixando apagarem-se sem esforço
as doces impressões do seu primeiro amor.}

\enlargethispage{1\baselineskip}
\textls[-20]{Otávio procurava no trabalho uma distração que não encontrava;
riscava mapas, fazia planos, media terrenos, calculava somas, pulava de
um projeto para outro, achando todos inúteis, a convencer-se de que
perdia tempo, com o espírito obcecado pelo amor de Eva.}\looseness=-1

Voltou à fazenda do Antunes, a perguntar pelo Frutuoso;
não lhe souberam dar notícias dele. Indagou na venda da
estrada; o taverneiro estava só, dormitando sobre o balcão: não conhecia
o Frutuoso, disse. Tornou lá uma, outra e outra vez, até encontrar o
caipira que ali o tinha informado do capanga do Antunes.

\textls[-10]{Ele estava de pé, encostado a uma ombreira da porta, enrolando
um cigarro entre os dedos. Falava para o interior, à mesma mulher da
outra vez, que vinha arrastando as saias muito engomadas:}\looseness=-1

--- Eh! Nhá Tuca! Tá só serenando!\ldots{} ---
dizia numa modulação arredondada, meneando a cabeça e piscando
os olhos.

\textls[25]{--- Oie que jeitinho! É memo um macaco! ---
respondia ela, arremedando-lhe o gesto.}

\textls[15]{Otávio interrompeu bruscamente o diálogo, chamou o
caipira, deu-lhe dinheiro e perguntou pelo Frutuoso.}

O capanga do Antunes andava fugido à polícia, respondeu
o outro; e não havia quem lhe pusesse a vista em cima. E
concluiu:

--- Mecê não descorçoe não,
quarqué dia ele há de aparecê\ldots{}

Mas não aparecia; e Otávio, para iludir saudades de
Eva, fazia longos passeios a cavalo pelos sítios e matas vizinhas. Numa
ocasião foi até bem perto do Mangueiral. Pensou mesmo em visitar a
prima; mas retraiu-se. A última vez que lá tinha ido
ela não lhe aparecera; aquilo fora de propósito;
estava persuadido disso; e, desgostoso, voltou para trás, galopando para
a cidade.

\textls[-15]{De longe em longe lembrava-se de caçar; passava todo o dia no
mato e voltava à noite para casa, fatigado e aborrecido. Estava
perto de Eva e não a via nunca! Esperava a todo o momento ouvir falar no
casamento dela com o Azevedo e, receando isso, evitava a
convivência dos amigos. A situação parecia-lhe cada vez mais
embaraçosa; e não lograva, sem provas, convencer o pai da
inocência da prima.}\looseness=-1

Escreveu a Paulo, expondo-lhe a inutilidade dos seus esforços e
esperou a resposta; não a recebendo, contou indignado o caso ao dr.\,
Morton. O velho sorriu-se contrafeito, murmurando:

\textls[25]{--- Paulo é delicado: ou não recebeu a sua carta, ou estará
doente\ldots{}}

\textls[-10]{Doente! Otávio preferia que o silêncio do amigo fosse
motivado pela descortesia! Imaginava os mil cuidados que Eva
teria pelo enfermo; via-a sentada à sua cabeceira, dando-lhe os
remédios, o caldo, aconchegando-lhe as roupas na penumbra do
quarto, sempre consoladora, meiga, atenta ao relógio,
debruçando-se solícita sobre a bela cabeça do doente para velar-lhe o
sono!\ldots{}}\looseness=-1

\textls[20]{Querendo afastar do seu espírito essa ideia, indagou se
tinha ouvido falar do casamento do Azevedo, e não teve coragem de dizer
com quem.}

\textls[-20]{--- Sim; diz-se por aí que ele se vai casar\ldots{} não sei nada
ao certo.}\looseness=-1

\textls[15]{Otávio saiu impressionado. Procurava com frequência o velho
Morton, porque o ouvia sempre falar em Eva; entretanto, esse mesmo
parecia evitar referir-se a ela.}

\textls[-15]{``Vamos!'', pensava Otávio, pondo o cavalo a passo,
``Morton sabe de tudo e oculta-me a verdade, para poupar-me um
desgosto.''}\looseness=-1

Então já não o atormentava a ideia de ser Eva a cuidadosa
enfermeira de Paulo. Via-a agora, risonha na sua saleta do Mangueiral,
com as faces rosadas, os olhos luminosos, o anel de núpcias no
dedo, trabalhando no seu enxoval ao lado do Azevedo.

\textls[20]{Assim, variando de quadros, cada qual mais atormentador, mais
vivo, tomou por um atalho escabroso, cheio de pedras resvaladiças e de
troncos de árvores. Sobre a sua cabeça cruzavam-se os galhos
compactamente, e por vezes tão baixos que ele, para passar, deitava-se
quase em cima do animal.}

As dificuldades do caminho conseguiram distraí-lo um
pouco. Chegando a Santa Genoveva, encontrou Nicota a emalar roupas.
Trigueirinhos tinha sido chamado a depor no processo do crime,
e não parecia agora tão impressionado; o Azambuja escrevera-lhe
longamente, afirmando que sairiam vitoriosos de tudo.

\textls[15]{Otávio entrou para o quarto, deitou-se cansado, e ia adormecer
quando o Saturnino lhe entregou duas cartas; abriu a primeira, cuja
letra lhe pareceu conhecida. Era um convite para o casamento do dr.\,
Azevedo com a formosa Sinhá! Otávio releu espantado os nomes
dos noivos. Teve um sorriso de alegria, um momento de alívio e
de júbilo!}

Abriu por fim a segunda carta, escrita em caracteres francos e
enérgicos, e leu:

\begin{quote}
Caro Otávio,
\smallskip

\noindent{}\textls[30]{Cheguei hoje de uma viagem; eis o motivo por que ainda não
respondi à sua carta\ldots{}}
\end{quote}

\textls[15]{Otávio voltou a folha, procurando a assinatura curiosamente;
lendo o nome de Paulo, levantou-se, dirigiu-se para a claridade da
janela e começou muito atento a leitura.}

%\end{linenumbers}

\section{XXXV}

%\begin{linenumbers}

\textls[20]{O casamento de Sinhá devia realizar-se em Campinas, na Matriz Nova.}\looseness=-1

\textls[20]{A irmã do Antunes tinha fixado nessa cidade a sua
residência. Além do convite oficial, escreveu ela
particularmente ao comendador, pedindo-lhe que não faltasse; mas o velho
fazendeiro, preocupado com a perda dos escravos, encarregou o filho de
representá-lo e de acompanhar Noêmia.}

\textls[-15]{Os dois irmãos partiram para Campinas no próprio dia da
cerimônia, de manhã. Noêmia ia radiante; não assistia, havia
muito tempo, a um baile, e a sua mocidade reclamava alegrias e festas.}\looseness=-1

\textls[15]{A irmã do Antunes esperava-os na estação e levou-os para o seu
chalé no Guanabara.}

\textls[-20]{Pelo caminho, dentro do carro fechado que os conduzia, contava
que a Sinhá amava apaixonadamente o noivo; que ele, o Azevedo,
fazia-lhe, havia muito tempo, a corte, praticando por ela
verdadeiras loucuras, dando-lhe, enfim, evidentíssimas
provas de amor!\ldots{}}\looseness=-1

Noêmia abria muito os olhos, atônita pelo que ia ouvindo.

\textls[15]{Otávio esforçava-se por não sorrir, compreendendo o sentimento
oculto que ditava aquelas palavras; sentia, à medida que os
elogios ao Azevedo iam crescendo, desejos de perguntar se ele não
indagara nada acerca do dote\ldots{} contudo abstinha-se da
mínima manifestação de ironia, respondendo às expansões
maternais da viúva com frases lisonjeiras aos noivos.}

\textls[15]{Atravessaram assim toda a cidade, descendo a colina da
estação para as ruas de maior atividade comercial e delas para o bairro
moderno, em que morariam durante dois dias.}

\textls[20]{Após a cerimônia na igreja, às oito horas da
noite, seguiriam noivos e convidados para o Club Semanal, onde a
mãe da Sinhá ofereceria um baile.}

Era o costume da terra.

Assim se fez.

Às oito horas da noite, a porta principal do templo abria-se de
par em par. Dentro, a luz forte do gás espadanava os seus
brilhos sobre as paredes brancas, de que se destacavam, imponentes e
graciosos, os belos altares, de obra de talha. Entre os genuflexórios de
\emph{chagrin}, muitas pessoas curiosas esperavam o cortejo.

À hora marcada, a noiva entrou na igreja, deslumbrante de alvura
e de beleza. Do coro derramaram sobre a sua cabeça velada uma chuva de 
flores. Ela seguiu, amparada ao braço do padrinho, e
atravessou pausadamente o cruzeiro, arrastando a longa cauda do seu
vestido branco.

\textls[-15]{Aos pés do altar-mor parou, e o Azevedo, muito pálido,
colocou-se a seu lado; o sacerdote disse as primeiras palavras
do ritual, e o órgão, lá em cima, rompeu uma suave melodia. A
luz punha reflexos dourados e fugitivos na capa de asperges do vigário,
penetrando por entre os festões de rosas de madeira dos altares. De
cima, das tribunas, algumas famílias olhavam indiferentemente
para aquele espetáculo; a mãe da noiva sorria orgulhosa, com o
seu largo rosto trigueiro iluminado por uma dulcíssima
alegria.}\looseness=-1

Otávio observava curiosamente o interior do templo, quando
deparou com a prima, que rezava de joelhos num genuflexório. Atrás dela
Paulo assistia, de pé, à solenidade do casamento.

\enlargethispage{1\baselineskip}
\textls[-10]{Era a primeira vez que ele a via assim, decotada, mostrando a
carnação aveludada e leitosa do seu formoso colo roliço e dos seus
braços bem-feitos. A luz batia-lhe em cheio sobre as rendas e a seda
cor de marfim do vestido, onde não brilhava a mais pequena joia;
os cabelos negros, presos no alto, sem flor, sem fita, sem plumas ou
diadema, davam-lhe, na sua simplicidade, um aspecto de
majestade e de graça. Eva rezava, num grande recolhimento.}\looseness=-1

\textls[15]{O padre deitava a benção aos noivos e, dando-lhes a mão,
ajudava-os a erguerem-se, felicitando-os. O som do órgão
expirou lentamente e o cortejo voltou, pisando sobre as folhas
espalhadas, que rangiam no chão.}

\textls[15]{Otávio, puxando devagar o braço da irmã, mostrou-lhe Eva, que,
já de pé, cumprimentava a noiva.}

\textls[15]{Noêmia deu um gritinho de prazer, e, afastando pressurosamente
as pessoas que estavam adiante, foi abraçar a prima, exclamando ao mesmo
tempo:}

--- Que bom! Que bom! Como eu estou contente! Tinha
tantas saudades suas! Sabe? Eu tenho muito que lhe dizer\ldots{} muito!
Não é verdade, Otávio? --- dizia, voltando-se para o
irmão, que estendeu timidamente a mão à prima. Eva sorriu, e
puxando Noêmia beijou-a com ternura.

--- Então, faço-lhe falta? --- perguntou.

\textls[15]{--- Muita! Não imagina! Santa Genoveva está que parece um
cemitério! É verdade, morreu a mamã de Otávio!}

--- Pobre Joana!

\textls[15]{--- Coitada mesmo\ldots{} eu tive muita pena! Agora temos
criadas italianas\ldots{} Que graça! Nem a gente as entende, nem elas
nos entendem a nós! Falamos por mímica! Escuso dizer que a mais
expressiva é a minha!}

--- Acredito. Agora uma outra coisa: por que é que não me foi
ainda ver?

--- Desejos não me têm faltado, mas\ldots{} Que bom dia aquele
que eu passei lá! Assim é que vale a pena ter sítio!
Nunca mais passeei a cavalo\ldots{}

--- E tem estudado?

--- Não\ldots{}

--- Que faz, então?

--- Se quer que lhe fale com franqueza\ldots{} nem eu sei!

\textls[-20]{Na ansiedade de vazar no coração da prima os sentimentos do seu,
Noêmia ia relatando tudo o que tinha ocorrido no sítio, depois da sua
separação, desde as mais importantes até às mínimas coisas.}\looseness=-1

\textls[20]{Entretanto, os noivos recebiam os últimos abraços, e todos saíam
da igreja.}\looseness=-1

\textls[20]{--- Por que não vem você no meu carro? --- perguntou
Noêmia à prima.}

\textls[20]{--- Porque estou com a família do Siqueira\ldots{} É
verdade! Fui apresentada ao Júlio; ele falou muito em
você\ldots{}}

--- Ah! O Júlio Siqueira Franco?

--- Sim.

--- Achou-o simpático?

--- Muito. É um rapaz distinto.

--- É parente da família Siqueira?

\textls[15]{--- É. E agora adeus, Noêmia; vá ver-me, se porventura
o tio Medeiros consentir\ldots{}}

--- Quando volta para o Mangueiral?

--- Amanhã.

--- Não assiste ao baile?

\textls[-20]{--- Não. D.\,Olímpia Siqueira está doente e volto já para seu
lado.}\looseness=-1

\textls[20]{--- Adeus, Eva! --- disse Noêmia num tom ressentido,
com os olhos úmidos. Eva abraçou-a e estendeu a mão a
Otávio, que falava a pequena distância com Paulo. Ele corou,
embaraçado; ela sorriu com prazer.}

\textls[-15]{Uma hora depois a igreja estava fechada e escura; o largo lá
fora silencioso, e longe, na esquina de uma rua
solitária, rompia atroadoramente, no salão do Club Semanal, a
primeira quadrilha do baile.}\looseness=-1

%\end{linenumbers}

\section{XXXVI}

%\begin{linenumbers}

\textls[15]{--- Bem boa cerveja! Quer outro copo? --- dizia, no
botequim do clube, um velhote gordo e calvo, a Otávio
Medeiros.}

--- Não, obrigado.

--- Oh! O senhor não parece educado na Alemanha!

\textls[15]{--- Joãozinho! --- continuava ele dirigindo-se a um
amigo, homem magro, pálido, de longas barbas pretas e olhos vivíssimos,
que distribuía amavelmente refrescos a um e a outro. --- Eu aceito
mais\ldots{}}\looseness=-1

--- Barão! Olhe que é a décima vez que me diz isso!

--- Mas como não é a quinta que você me serve\ldots{} vá
botando. --- E o barão riu-se, piscando os olhos. Esgotado o
copo, o velhote saiu, convidando Otávio a ir fumar com ele na
sala dos homens.

\textls[15]{--- Ao mesmo tempo a gente vai proseando\ldots{} eu dou o cavaco
por uma prosa.}

No salão do baile dançavam nesse momento uma valsa, e
como o dos fumantes era contíguo e o pavilhão da música comum
de ambos, nem o barão nem Otávio se compreendiam bem.

\textls[15]{Os sons dos instrumentos abafavam-lhes as vozes. Contudo, o
barão ia falando sempre; Otávio olhava para ele muito atento,
mas com o pensamento completamente alheio ao que se passava; de vez em
quando sorria, movia a cabeça afirmativamente, quando lhe parecia ouvir
uma pergunta, atirando uma palavra ao acaso, um ``sim\ldots{} é
verdade\ldots{} talvez\ldots{} tem razão!'' Com esforço apanhava de
longe o fio da narração, mas, malgrado seu, o espírito
fugia-lhe depressa para outro lugar.}

\textls[15]{Era Eva que o preocupava assim. Eva, que ele revia de joelhos,
com os cotovelos nus pousados no marroquim do genuflexório, toda de
claro, numa singeleza ideal. Achara-a indiferente. Queria tê-la
visto perturbada, trêmula, e ter percebido no seu olhar de veludo a
longa carícia do amor, que, por mais concentrado e silencioso que seja,
brilha nos olhos num momento de alegria ou de angústia, de modo
iniludível.}

Mas não! Eva era sempre a mesma, impassível! Tinha para ele a
doce expressão de uma simpatia sem extremos, a manifestação da amizade a
mais simples, a mais suave, a mais natural. Esse mesmo sentimento o
impacientava; parecia-lhe filho da comiseração; era a seus olhos como
que um laivo de piedade feminina, reflexo vago, sutil, da compaixão
inspirada pelas suas dores.

No entanto, o barão ia dizendo:

\textls[-5]{--- Pois era assim, meu caro senhor, Campinas, há vinte anos! À
noite não se via uma janela com luz; as ruas eram tenebrosas,
esburacadas e cheias de perigos. Quando havia espetáculo, era necessário
levar, de casa para o seu camarote, cadeiras, tantas quantas
fossem as pessoas; uma moringa de água e um copo, o
indefectível cuscuz, que se comia no intervalo, no corredor, e
a lanterna para a volta! Eu achava então muito mais graça em tudo do que
acho hoje. O que é a mocidade! Faça ideia: para conduzir tanta coisa era
forçoso irem mucamas e pajens, que ficavam de pé, ao fundo do
camarote\ldots{} O cheiro que se exalava daquela gente não era
positivamente de rosas\ldots{} No palco, à luz fumosa do
querosene, deslizavam choramingadamente dramas como:
\emph{O cego e o corcunda}, \emph{As
mulheres de mármore}, etc. Mas não sei! Homens e senhoras choravam,
tanto na plateia como nos camarotes ou galerias. Hoje já não se vê tanto
disso; estamos com o sentimento embotado e, por mais comovente que seja
a peça, a maioria dos espectadores tem os olhos enxutos. É o que eu digo
e sustento: o progresso vai nos arruinando o coração! E o carnaval,
então? Ah! Pelo carnaval é que eu gozava à grande! Que quer? Achava
bonita e pitoresca aquela farsada toda. Constituíamo-nos, nós,
os mascarados, num bando; levava cada um uma cesta de flores e
plumas, mas coisa rica, bem-acabada, que nos custava rios de dinheiro,
e íamos de janela em janela falando às moças e trocando com
elas as rosas de pano, finas, que levávamos, por outras que
elas nos davam, e que, valha a verdade, eram sempre inferiores!}\looseness=-1

\textls[-15]{E o entrudo?! Oh! Isso é que era bom! Punham-se tinas na rua,
seu Otávio! Tinas cheias de água fria! As senhoras
vinham para a esquina, agarravam quem bem lhes aprouvesse e\ldots{}
fogo! Fogo não, água! Era mergulho certo, quisessem ou
não quisessem. Tiroteio de laranjinhas de cera, nem se
fala! Basta dizer que havia aí famílias cujo emprego durante
todo o ano era esse: fazer limões de cheiro!}\looseness=-1

Dávamos assaltos às casas, de manhã, durante o dia, à noite, a
qualquer hora!\ldots{} Pintávamos! E aquilo é que era riso\ldots{} e
aquilo é que era mocidade! Acabada a brincadeira do entrudo, começava a
da Semana Santa. Outra folia. Morria a gente de calor nos apertos da
igreja, mas não faltava à festa. Havia luxo: os vestidos eram de boa
seda, as mantilhas de ótima renda, e viam-se ali brilhantes, que
faiscavam! Naquele tempo as mulheres, quer fosse Semana Santa ou não, só
iam de preto à igreja; a primeira família que apareceu
vestida de cor para assistir à missa foi censurada por
toda a gente; mas, afinal, todas as pessoas que a criticavam
foram-na imitando, e hoje há nas igrejas, seja qual
for a cerimônia, uma miscelânea de
cores que tiram toda a solenidade ao templo! Acredite! Fazem-me
saudades desse tempo as raras \emph{baratas} que encontro hoje nas manhãs de
domingo pela cidade! São poucas as senhoras que se recatam agora embaixo
do puçá\footnote{Puçá: renda de nó que guarnece a parte superior das
  antigas mantilhas de pano e que desce até a barra do vestido, cobrindo
  completamente o rosto.} ou por detrás das rótulas! Se
há vinte anos uma família saía sem o acompanhamento da
mucama predileta, ou se as moças apareciam à gente que passava, assim
como hoje, numa janela, francamente! Qual! Espiavam pelos buraquinhos
das rótulas, que se entreabriam numa pequeníssima fresta. E um homem
qualquer casava sem medo, porque tinha a certeza de encontrar
na esposa mulher modesta e trabalhadora. É ou não é assim?

Otávio teve um sorriso vago, moveu a cabeça numa afirmação
indecisa. Estava num dos períodos de abstração; o outro
interpretou de modo diferente aquele gesto e replicou:

\textls[15]{--- Ainda o senhor hesita! Pois há lá ponto de
comparação entre a moça de hoje, que sabe francês, alemão,
inglês, que toca em concertos e mostra toaletes, com as do outro
tempo, que viviam cosidas às saias da mãe e não se atreviam a
falar com estranhos, mormente se eram homens! Hoje, sim! Botam
as manguinhas de fora, têm o seu clube, e tagarelam, e apostam
nas corridas, e depois da quadrilha passeiam meia hora conversando com o
par. Esta geração oferecerá as garantias da outra?}

\textls[20]{E como Otávio não respondesse, continuou ele com mais
veemência:}\looseness=-1

--- Não oferece! É uma geração estragada e macaca. Talvez me
julgue, pelo que está ouvindo, inimigo do progresso. Não sou tal. Do que
eu sou inimigo é da imitação. Campinas hoje não tem caráter
próprio.

\textls[2]{A estrada de ferro trouxe-lhe estrangeiros, oficinas,
gás; espalhou a indústria e enriqueceu o comércio; de
acordo. Mas também veio tirar-nos muitas das nossas ilusões e,
portanto, da nossa felicidade. Tornamo-nos exigentes e insaciáveis. As
companhias dramáticas provincianas, que outrora nos arrancavam
lágrimas de entusiasmo e comoção, hoje nem se atrevem a pisar o nosso
palco; que diabo viriam cá fazer, depois de Rossi, Tessero, Pezana,
Gemma, Brazão, Rosas e outros que tais? E, se viessem, seriam pateadas, ou
o público, levantando com desprezo os ombros, não iria
vê-las. E ora, como os grandes artistas só aparecem de longe em
longe, estamos por largos intervalos sem divertimento nem gozo de
espírito. Fazemos hoje o que todas as cidades
civilizadas fazem. Nos bailes já se não dança o alegre
caranguejo: são valsas, quadrilhas e polcas, só; é rara a fazenda que
tem o alegre e pitoresco mumbava}\footnote{Mumbava: sujeito que vive
  pelas fazendas e tem ordenado só para conversar e fazer rir o
  fazendeiro.} que tanto nos fazia rir antigamente. O nosso
  teatro é como todos, a mesma ordem, a mesma monotonia, o mesmo
  botequim, as mesmas guloseimas e bebidas; tudo o que era
  original, extravagante e absurdo morreu, ou fugiu com espanto aos silvos
  do trem de ferro. Presentemente nem o medo de bexigas é como era! Não se
  teme a transfiguração ou a morte, tão desvirtuado está isto! Pois
  há vinte anos, tempo de epidemia era tempo de fuga e de
  abandono! À porta das casas flutuavam bandeirinhas brancas, se dentro o
  bexiguento era atacado de simples cataporas; negras, se era de pele de
  lixa. Decididamente as coisas eram bem outras! O que eu, sobretudo,
  confesso, lamento mais, são as ilusões! Oh! Naquela época nós
  tínhamos a doce persuasão de ter dinheiro.
  Julgávamo-nos ricos!

--- Sr.\,barão! --- disse, chegando-se para perto, uma
menina de treze anos, clara e loura, que vinha puxando pelo braço um
rapazinho, quase da sua idade:

--- Mande Joãozinho dançar com Vica! Porque ela não tem par, e
eu não tenho \emph{vis-à-vis}.

\textls[15]{O barão sorriu, convenceu o Joãozinho, seu neto, de que devia
dançar, e, voltando-se depois para Otávio, continuou apontando
o gracioso par que se afastava:}

\textls[-15]{--- É isto! Sabe que mais, senhor Medeiros? Eu vou-me embora!}\looseness=-1

E afastou-se.

\textls[-5]{--- São assim todos os velhos; para eles não há tempo
melhor do que o tempo passado! --- disse zombeteiramente um
rapaz que fumava encostado a um umbral e ouvira parte do discurso do
titular.}\looseness=-1

\textls[15]{Otávio, poucos minutos depois, entrava no salão. Noêmia dançava
com Júlio Siqueira, risonha e corada, com os seus anéis louros
coroando-lhe a fronte estreita, e o vestido azul desenhando-lhe o corpo
delicado. O Azevedo estava ao lado da mulher; ela serena, ele muito
pálido.}

\textls[15]{Sentindo uma pancada de leque num ombro, Otávio
voltou-se. Era d.\,Clarinha. O seu colo emergia airoso e branco do veludo
granada do corpinho.}

\textls[20]{--- Que inveja, hein? --- disse-lhe ela mostrando o
Azevedo. --- Quando assistirei ao seu casamento?}

Ele respondeu qualquer banalidade e foram ambos
valsar.

%\end{linenumbers}

\section{XXXVII}

%\begin{linenumbers}

\textls[20]{No primeiro dia de maio o comendador Medeiros acordou
sobressaltado, ouvindo de um amigo comum a notícia de ter desaparecido o
Antunes.}\looseness=-1

\textls[-15]{--- Ora essa! --- exclamou ele muito espantado. ---
Então um homem é coisa assim que se suma, sem se saber como nem por
quê?}\looseness=-1

--- Seria assassinado?

--- Hum\ldots{} é pouco provável, o Antunes não tem inimigos,
que me conste\ldots{}

--- Ter-se-ia suicidado?

\textls[25]{--- Qual! É verdade que os negócios dele estavam
atrapalhados\ldots{} mas não dava para isso.}

\textls[15]{--- Homem, ele gostava do seu joguinho e voltava a que horas da
noite para casa; pode ser que na estrada\ldots{}}

--- Também não me parece.

--- Sim, é verdade que ele era muito caipora no jogo; não levava
nunca dinheiro para casa\ldots{}

--- Teriam escondido o Antunes?!

--- Para que fim?

--- Na verdade, para quê?

--- Ora! Vocês estão mal-informados. O compadre a estas
horas está a regalar-se em casa da irmã, em Campinas.

\textls[20]{--- Não está, já lhe perguntamos; ela não sabe do irmão desde o
casamento da filha.}

--- Oh! Diabo, o negócio então é sério.

E o comendador foi com o amigo ao sítio do Antunes.
Encontraram a caseira chorosa. O Antunes saíra sem dizer para onde,
havia uns vinte dias, e o sítio passara do poder dele para o de
um outro sujeito, que se dizia administrador de um banco.

\textls[25]{--- Seu Antunes sempre teve mau rejume}\footnote{Regime.}
\textls[25]{--- disse ela com um suspiro; e depois de uma pequena pausa:}

\textls[15]{---É verdade! --- acrescentou, quando os visitantes se
dispunham a sair. --- Seu Antunes me deu uma carta pra eu entregar a seu
comendador, \emph{mais} eu não tive portador e guardei ela comigo
pra quando nos encontrasse\ldots{}}

\textls[-20]{E levou um tempo imenso a procurar a tal carta na cesta
de costura, indo por fim achá-la entre o travesseiro e a fronha
da cama.}\looseness=-1

Medeiros rasgou o envelope, mas, lendo logo no começo a palavra
``Reservada'' em grossos caracteres, sofreou a
impaciência e guardou-a no bolso, desculpando-se com o
companheiro.

Saíram e foram juntos até à cancela de Santa Genoveva;
aí o outro seguiu para a cidade e o comendador entrou. Um quarto de hora
depois, fechado por dentro no seu quarto, ele abria curiosamente a carta
do Antunes e lia entre a indignação e o pasmo:

\begin{quote}
Meu bom amigo Medeiros,
\smallskip

\noindent{}Fujo, e peço a Deus e a você que me
perdoem. Como nunca mais há de tornar a me pôr a vista em cima, e como
as coisas já tomaram o caminho que haviam de tomar, deixe eu lhe fazer
as minhas confissões; vou mudar de vida e não quero carregar remorsos.
Meu compadre e amigo! São histórias tristes as que eu tenho para vos
dizer! O primeiro caso de que me arrependo e peço perdão, foi passado há
muitos anos. Lembra-se, meu amigo, da morte do jornalista, avô de sua
sobrinha?\ldots{}
\end{quote}

\textls[20]{Neste ponto o comendador Medeiros sentiu perturbação na vista e
um suor aflitivo umedeceu-lhe a testa; teve um momento de pausa
e prosseguiu depois na leitura:}

\begin{quote}
Pois fui eu que matei ele, meu compadre, fui eu!
\smallskip

\noindent{}Deveis estar lembrado que pousamos no mesmo hotel em São Paulo;
ora, o seu pajem, o Antoninho, já tinha sido meu escravo e me
respeitava muito; ordem que eu desse a ele era coisa cumprida!
Eu então ordenei que me levasse todas as cartas que
escrevêsseis, era para ver se vós me comprometíeis com
o vosso irmão Gabriel, a respeito da hipoteca do sítio e da
berganha da casa de vosso pai com a minha\ldots{} Desejava
também ver se lhe queria persuadir de casar com minha
irmã. Foi assim que eu li o bilhete em que o senhor ameaçava de morte o
jornalista, \emph{mais} não pus logo no sentido o que depois
sucedeu!

O diabo foi o demônio do jogo, que foi sempre a minha
peldição. Na mesma noite em que li o seu bilhete fui para o clube e
calhou de jogar com o avô de Eva; ele ganhou, eu perdi tudo!
Fiquei desesperado, me lembrei do seu bilhete, esperei na esquina e
matei o homem; mas quando ia tirar meu dinheiro das algibeiras do casaco
dele, não sei o que senti: entrei a tremer, e voltei para o hotel, como
um idiota. Encontrei no meu quarto o seu pajem, zanguei-me com
ele \emph{mais} fiquei sempre com medo que vos dissesse da minha
indiscrição, e pra ter ele sempre embaixo das mãos foi que vos
pedi o lugar de feitor no vosso sítio. Aí estive; quando tive
de me retirar, o seu pajem morreu\ldots{} Deus me perdoe!
Depois disso passaram-se muitos anos, e fui sempre bem recebido por toda
a vossa família, exceto o Gabriel, que parecia desconfiado de
mim. E mesmo eu não tinha coragem de encarar com ele! Julgava-me livre
de cometer outro crime, infelizmente, porém, não foi assim! Agora,
há poucos meses, tornei a cair; eis o caso: desejava com grande
empenho que a minha sobrinha Sinhá casasse com o seu filho
Otávio, não só por simpatizar muito com ele, como porque eu,
como tutor dela, tinha disposto do seu dote em transações com amigos;
sendo você o meu maior credor, estava claro que seria o mais indulgente,
passando depois a sua dívida a ser liquidada com seus filhos,
isto é, estava já na família parte importante do dote de Sinhá,
e eu não sofreria nada com isso! Supunha que Otávio se
agradasse da menina, mas desconfiei de Eva, que bonita e
instruída acabaria, se não a pusessem à margem, por
apossar-se do coração de Otávio. Fui espiando o
negócio e quando vi que as minhas prevenções se realizavam, fiz
com os meus capangas o barulho em Santa Genoveva, atirando depois com a
responsabilidade para as costas da órfã do Gabriel! A coisa foi
bem-feita. E vi com alegria que V. despedia de casa aquela que eu temia
vivesse ao lado de Otávio; o ferimento de seu filho é
que eu não previ e me entristeceu, felizmente ele está bom e não levo
por isso cuidados. Do que me arrependo também é de não ter feito a
intriga a tempo, foi talde e já nada aproveitou. O casamento do Azevedo
com a minha sobrinha me pôs em embaraços, houve
exigência de dinheiro, etc., rompi com minha irmã; o
meu sítio, já hipotecado há anos, foi-me tirado ontem!
Vejo-me sem recursos e para não viver humilhado onde já aparentei tanto,
fujo para outro país, onde não saibam os meus credores. Escrevo esta
carta como quem se confessa! E o único alívio que
posso ter no fim de minha vida é saber que me perdoais. Rasgai esta
carta, esqueça-se do desgraçado que a escreveu!\ldots{} Mais uma vez,
perdão!

\begin{flushright}
Seu amigo\\
\emph{Antunes}
\end{flushright}
\end{quote}

O comendador, lívido e trêmulo, tornou a ler, do
princípio ao fim, uma, duas, quatro vezes, aquele papel maldito,
que parecia queimar-lhe as mãos. O homem em quem tanta confiança tinha
depositado, a quem sempre recorria numa ocasião de desgosto ou de
dúvida, era um depósito de veneno, um coração falso. Buscava-o como um
consolo, e era só ele, única e exclusivamente ele, a causa da
sua desgraça! As lágrimas caíam-lhe em fio pelas faces manchadas de
indignação. Sentia raiva e dor ao mesmo tempo: uma
ilusão de muitos anos não cai sem que o golpe cause
dor intensíssima. Era como se fosse uma árvore
copada e viridente, a cuja sombra se tivesse afeito e que arrancada de
repente deixasse tudo árido e queimado pela soalheira.

\textls[-20]{Nessa mesma tarde, o comendador, mostrando ao filho a carta do
Antunes, confessava-se arrependido do que fizera à sobrinha.
Otávio, radiante de alegria, lembrou-lhe irem todos ao
Mangueiral.}\looseness=-1

\textls[20]{--- Para quê? Estou velho e não me devo abaixar a pedir perdão a
uma criança! Ela que fique sabendo que já não a odeio e é quanto basta!}

Otávio quis insistir: o pai voltou-lhe as costas.

%\end{linenumbers}

\section{XXXVIII}

%\begin{linenumbers}

A viração fria de maio enrugava a superfície
do lago do Mangueiral, desfolhando as camélias do parque
contíguo à habitação.

O sol de inverno, dourado e morno, tingia de cores
prismáticas a areia luzidia das ruas e as relvas dos canteiros. Voavam
borboletas, e o aroma do resedá penetrava docemente em tudo.

\textls[-10]{Junto à janela da saleta de música, numa
cadeira de espaldar alto e esguio, Eva entretinha-se fazendo uns
casaquinhos de tricô para as crianças da colônia. Sobre
a lã cinzenta do seu vestido de xadrezinho caíam-lhe baralhadamente os
novelos encarnados, pretos e brancos, enquanto as suas mãos
ágeis e pálidas moviam com rapidez as longas agulhas de marfim.
Estava só na sala quando Paulo entrou; ela não levantou a cabeça; com o
queixo unido à gola alta do vestido, guarnecido por um estreito friso
branco, continuou atenta no seu trabalho de caridade; ele é que,
encostando-se ao peitoril da janela, disse com a voz ligeiramente
trêmula:}\looseness=-1

\textls[20]{--- Acabei de ler uma carta de seu primo; está radiante! O pai
reconhece, finalmente, que foi injusto com você e está disposto a
abraçá-la! Para a grande felicidade dessa reconciliação é justo
que se faça algum sacrifício, desde que circunstâncias como as
atuais o exijam. Ora, a minha boa Eva não se negará por certo a ceder a
essa necessidade, visto ser a mais nova e ter o espírito
superior às mesquinharias vulgares\ldots{}}

\textls[20]{--- Não o entendo bem; o tio Medeiros manifestou desejos de
ver-me?}\looseness=-1

--- Sim\ldots{} está arrependido do que lhe fez, e é
natural que, para repouso de consciência, o homem queira
saber-se perdoado\ldots{}

--- Qual é então o sacrifício que me pedem?

--- Ir lá.

--- A Santa Genoveva?

--- Sim; e que há de extraordinário nisso?

--- Tudo! Eu não tornarei a entrar naquela casa!

--- Isso é um capricho que pode comprometer a sua felicidade!
Trate melhor o seu coração! É tempo de pensar no futuro\ldots{}

--- Você está hoje enigmático!

--- Diz isso com sinceridade?

\textls[20]{--- Certamente. Bem sabe que não dei nunca para decifrar
charadas.}\looseness=-1

\textls[10]{--- Deixe-se de ironias e escute-me: Otávio Medeiros
ama-a; não lhe digo uma novidade; da sua própria boca ouvi isto
mesmo, na noite em que saiu da casa de seu tio; não foi assim?}

--- Foi\ldots{}

\textls[5]{--- Bem; nessa ocasião ele solicitou cavalheirosamente a sua
mão, e você, para satisfazer o seu amor próprio, unicamente por
isso, respondeu-lhe que não. Hoje, porém, o caso é outro; o obstáculo
que se antepunha à realização desse enlace caiu; não há já
motivo para ocultar sentimentos nem assumir atitudes de mal-entendido
orgulho. A verdade aparece em toda a sua pureza; não lhe ficará mal,
portanto, mostrar ao pobre Otávio, que tantíssimas provas de
amor lhe tem dado, quantas o seu coração lhe reserva também.
Vamos, minha boa Eva! Não hesite em confessar-se; os momentos de
ser feliz são tão raros na vida!}

\textls[-25]{Estas últimas palavras foram pronunciadas com mal
disfarçada tristeza e amargura. Houve depois um momento de pausa, em que
se ouvia unicamente o atrito seco das duas longas agulhas de
marfim.}\looseness=-1

--- Então? --- tornou Paulo com insistência,
curvando-se um pouco. --- Otávio pede-me que lhe dê um conselho
e uma esperança; uma e outra coisa devem ser ditadas por você!

--- Acha que Otávio me fará feliz? ---
perguntou Eva com a voz quase sumida, como se alguma
dor a estrangulasse.

\textls[-20]{--- Acho; é um rapaz muito digno, inteligente, bondoso, e,
sobretudo, muito apaixonado\ldots{} Não será fácil encontrar
outro nas mesmas condições. Os nossos rapazes visam, hoje em dia,
unicamente uma coisa: o dote! Os Azevedos constituem, por assim dizer,
toda a caterva dos pretendentes provincianos. Uma desgraça! Senão,
vejam-se os exemplos: as filhas do Betencourt, instruídas,
bonitas, como são pobres\ldots{} estão solteiras!\ldots{} As do Lemos
igualmente; e outras! Com meia dúzia de contos estariam todas
casadas, porque bem sabe que até por causa dos \emph{bons partidos} vêm
rapazes de fora para cá.}\looseness=-1

--- Você quer dizer com isso que eu não devo vacilar em dar a
minha mão a Otávio?

\textls[15]{Paulo não respondeu; passou os dedos pelo cabelo, e
voltando-se para fora respirou com força, como se se quisesse
retemperar; depois, retomando o seu tom plácido e usual:}

--- Sim, não deve vacilar, visto que o ama.

--- Quem lhe disse isso?!

\textls[-15]{--- A sua comoção, Eva; ah! Eu estudo-a há muito. Na
noite em que vi você chegar de Santa Genoveva, pálida, transtornada,
doente, pude melhor do que nunca observá-la, e adquirir a
certeza do seu afeto por ele. Meu Deus! Não há nada mais
natural e que menos surpresa me causasse. Deixe-me até dizer
que esperava isso mesmo\ldots{}}\looseness=-1

\textls[10]{Eva cessara de trabalhar. As agulhas e a lã caíram-lhe
das mãos: com a cabeça inclinada, os olhos baixos, ela ouvia Paulo, num
recolhimento misterioso e profundo. Ele continuou:}

\textls[-10]{--- Otávio foi mais franco para comigo: confirmando o
que você já me tinha dito a respeito do pedido de casamento, ele
declarou-me simples e lealmente que a adorava, suplicando-me uma
proteção que, não sei por quê, julgava que eu lhe pudesse
dispensar! Nessas coisas eu fugi sempre de intervir, como bem sabe;
contudo, tratando-se de pessoas a quem tanto prezo e por quem me
interesso tão vivamente, o caso mudou\ldots{} Há muito tempo que eu
desejava antecipar esta hora, para obrigá-la a ser feliz.}\looseness=-1

Eva sorriu com ironia; Paulo prosseguiu:

\textls[-10]{--- O seu orgulho venceu muito, mas não venceu tudo; você teve
força para sair da casa de seu tio com altivez e sozinha; pois bem,
tenha agora a coragem de enfrentar com a dificuldade do momento e ir
oferecer-lhe a sua fronte para o beijo reconciliador; por mais embotado
que aquele homem tenha o sentimento, eu vou jurar que ele há de
chorar de alegria e chamá-la sua filha. Vamos, um
sacrifício, Eva! Rogo-lhe em nome de Otávio e do seu
próprio coração\ldots{}}\looseness=-1

--- Do meu coração?\ldots{} --- balbuciou ela a meia-voz.

--- Sim, do seu coração, que você teima em tratar tão mal.

\textls[15]{--- Meu Deus!\ldots{} --- Eva ia continuar, mas conteve-se e
baixou ainda mais a cabeça, para que não fossem vistas duas lágrimas que
iam rolando pesadamente nas suas faces pálidas.}

Paulo, encostado ao peitoril da janela, olhava para o jardim,
onde as pétalas das camélias tombavam como borboletas mortas,
tapetando o chão. Sem desfitar os olhos daquilo, perguntou:

--- Que me responde?

--- Que hei de eu dizer? Que julgava você mais meu
amigo\ldots{}

\textls[-10]{Desta vez Eva levantou os olhos, e Paulo, enternecido e
surpreso, viu as lágrimas, a expressão dolorosamente
ressentida, a dor profunda, indescritível, que boiavam naquelas
pupilas veludosas, até então impenetráveis, cristalinamente
transparentes agora!}\looseness=-1

Ele estremeceu, sentindo a vertigem dos que se debruçam num
abismo; foi Eva que, com a voz mal segura, principiou:

\textls[-12]{--- Acostumei-me a obedecer-lhe e a seguir de olhos fechados
todos os seus conselhos. Você quer que eu me case com
Otávio? Pois faça-se a sua vontade; serei mulher dele, visto que
a sociedade exige que eu tenha um marido, censurando-me por viver à
minha vontade com um irmão\ldots{} ah! Eu já sei o que você vai dizer!
Que eu, \emph{de fato}, não sou sua irmã. Não sou sua irmã! Se precisam
lembrar-me isso os estranhos! Recorda-se, Paulo, do dia em que você
chegou? Eu estava acolá, embaixo daquela mangueira! Minha mãe foi lá
ter, levando-o pela mão, e disse-me: ``Eva! Aqui tens um irmão! Abraça-o
e ama-o!''. Abraçamo-nos; e desde então\ldots{}}\looseness=-1

\enlargethispage{1\baselineskip}
Eva parou, mas os seus olhos falaram; falaram, não; gritaram
alto tudo o que lhe estava no coração. Paulo curvou-se, embebido naquele
olhar, esquecido de tudo o mais, silencioso, recebendo na alma aquela
rajada inesperada. E assim estiveram, mudos os lábios, mas a
trocar nos olhares um mundo de promessas e de paixão que se revelava
enorme, enraizada, amadurecida pela reflexão e pela
convivência, que os tomara a ambos no decorrer do tempo, desde a
meninice, sem que eles o percebessem talvez\ldots{}

\textls[20]{Transportado de júbilo, Paulo tomou a cabeça de Eva entre as
mãos, e beijou-a nas pálpebras, repetidas vezes.}

\textls[15]{E, assim, sem pronunciar a palavra definitiva, eles
declararam-se apaixonadamente o seu amor.}

%\end{linenumbers}

\section{XXXIX}

%\begin{linenumbers}

Quando a mão patrícia da princesa Isabel
decretou no Rio a abolição dos escravos, davam-se na província
cenas do mais grotesco e irresistível cômico.

\textls[15]{Otávio, para quem tudo parecia agora risonho e belo, envolto na
doce esperança de desposar a prima, fazia passeios longos, tendo ocasião
de verificar até que ponto o egoísmo dos velhos lavradores
consumia e estragava tudo. Passou dois dias de tranquilidade de
espírito, apesar de não ter a definitiva certeza do perdão e da
aquiescência de Eva. Estava na expectativa; mas,
vencida a má vontade do pai, tudo o mais lhe parecia fácil e
ameno. E enquanto não recebia a resposta de Paulo e o consentimento da
prima, ele ia animado e sereno observando com curiosidade a atitude dos
antigos amigos da família.}

Quando vibrou, como um toque de clarim, a notícia da
redenção dos escravos, o comendador Medeiros foi o primeiro a oferecer
à observação do filho um fato singular.

\textls[-15]{Ele, que até então apoiara o governo, aplaudindo, por convicção
e sistema, todos os atos do ministério conservador, partido em
que militara desde a mocidade; ele, que fora sempre fiel
à sua ideia, agarrando-se a ela como uma ostra à rocha, e que
era nas eleições um dos influentes e poderosos; ele, o velho Medeiros,
tradicional na política da terra, como um dos mais ferozes e acérrimos
defensores do trono; ele, na tarde do glorioso dia 13 de maio, bufava de
indignação, dispersando aos quatro ventos todas as suas velhas opiniões,
e proclamando, como única salvação para esta terra, a
república.}\looseness=-1

\textls[25]{Descontente, ferido nos seus interesses particulares, era
sincero; não mudava de política por acinte: mudava porque o impelia a
isso a sua decepção.}

\textls[15]{Otávio presenciou, pasmado, aquela mudança. No dia seguinte
observou uma cena mais extraordinária em casa de
Gusmão, lavrador tido por implacável e duro.}

\textls[15]{A fazenda do Gusmão ficava a meio caminho do Mangueiral, e
Otávio acercava-se da casa da prima, procurando sempre um ou
outro pretexto, a ver se a encontrava na estrada, ou se tinha
ânimo de lhe ir bater à porta; mas nem uma nem outra coisa
sucedia nunca. Eva encerrava-se na sua deliciosa vivenda, e
Otávio perdia positivamente o seu tempo. Nesse dia ele tomara o
caminho da fazenda do Gusmão, como teria tomado qualquer outro daquele
lado; foi encontrar o rancoroso fazendeiro discursando aos escravos, aos
quais dizia conceder a liberdade! E pavoneava-se na sua ação generosa,
querendo persuadir os servos de que partia dele tão grande
benefício!}

\textls[-10]{Otávio não pôde deixar de rir. Dali, depois do indefectível
café, e um pouco de prosa, seguiu de novo pela estrada do Mangueiral,
onde as cássias punham grandes manchas amarelas; ia pensativo: tinha
escrito a Paulo relatando-lhe tudo, falava do seu amor e das esperanças
que acalentava. Era já tempo de ter uma resposta\ldots{} E por que não
iria ele próprio sabê-la? Estava perto; em dez minutos
poderia adquirir a certeza de um futuro feliz; por que havia de hesitar?
Todavia, a cada passo, surgia no seu pensamento uma dúvida, um medo
instintivo, vago, pueril, de ser mal recebido e sofrer uma desilusão. E
varria logo do espírito aquelas suposições; julgava adivinhar
em Eva um sentimento oculto de paixão abafada, e por isso mesmo mais
intensa. E revia-a nas belas horas da sua convivência, desde o
dia da chegada, quando a vira pedir misericórdia para o negro
escravo, até o dia da partida, em que a tinha visto sair de cabeça
erguida e olhar faiscante e altivo! Pensando nela, a viagem tornou-se
longa; deixara o animal seguir à vontade, placidamente, até que
se encontrou, já tarde, à luz das primeiras estrelas, no
terreiro de Santa Genoveva. Na grande varanda sombria rangiam, como
sempre, as redes e ouviam-se bocejos. Noêmia já não brincava
com os cães; perdera essa mania; agora tinha a de ensinar uns
pombos-correios, mas já se ia mostrando desanimada com a estupidez dos
pobres bichinhos e andava à cata de outra distração. A
lembrança de Paulo tinha-se apagado. Com o seu temperamento e os seus
quinze anos, as sensações iam-lhe e vinham-lhe sem que ela realmente se
sentisse muito perturbada.}\looseness=-1

--- Veio alguma carta para mim? --- perguntou
Otávio ao Saturnino, logo ao chegar.

\textls[15]{--- Não senhor\ldots{} não veio nada, que me conste! ---
respondeu o mulato com os seus gestos pretensiosos.}

%\end{linenumbers}

\section{XL}

%\begin{linenumbers}

\textls[20]{``Decididamente é melhor esperar em casa a resposta\ldots'',
pensava Otávio consigo.}

\textls[20]{E não saiu durante todo o dia seguinte: as horas passavam lentas
e ele perdia-se em conjecturas. Que teria sucedido? Que faria Eva?
Chegava de vez em quando à janela, observava a estrada que via ao longe
serpeando entre a verdura do pasto, isolada e batida de sol; voltava
para o interior, abria um livro, tentava escrever, estudar, distrair-se;
mas surgia-lhe obstinadamente a mesma ideia e erguia-se de novo, para
contemplar, já sem esperança, a longa estrada, além.}

\textls[20]{Por fim, às cinco horas da tarde, entrou-lhe o
Saturnino no quarto com uma carta do Mangueiral.}

\textls[25]{Otávio abriu-a nervosamente. Em frente dele o pajem,
que, apesar de liberto pela grande lei, não quisera abandonar a
casa onde tanta afeição o prendia, contemplava-o com curiosidade e
tristeza, destacando, no fundo claro da janela aberta, a sua figura alta
e esguia. Tal era a comoção de Otávio que não reparou na
atitude do mulato; no entanto ia lendo, e à proporção que avançava na
leitura empalidecia.}

Paulo escrevia constrangido; contava a sua história com
forçada serenidade; pedia quase perdão a Otávio de ser
preferido por Eva, revelando-se de uma sutileza de sentimentos
quase doentia; havia alguma coisa de doloroso e castigado no seu
modo de dizer que era feliz, como se a sua ventura lhe causasse
remorsos. De vez em quando, porém, ressaltava luminosa e contente uma ou
outra expressão que ele não pudera conter; mas abafava depressa,
caridosamente, todas as manifestações de gozo sob uma linguagem serena.
A carta era longa, torturada --- carta escrita com a dor, a
extraordinária dor de dizer a um desgraçado: ``Eu sou feliz! O que te
magoa e prostra é o que me encanta e rejubila!''. Por maior que fosse o
disfarce, Paulo não conseguiu ocultar a Otávio a sua ventura.
Aquele estilo delicado e pungente lembrava um roseiral em flor, a que
tivessem, para que o não vissem, abafado com sebes espinhosas e secas;
as flores, de fato, ficavam escondidas, mas o seu aroma inebriante
rompia a coberta piedosa e ia denunciando pelo espaço em fora a sua
existência oculta.

\textls[15]{Acabando a leitura, Otávio levantou os olhos. O
Saturnino continuava imóvel, como uma estátua. Otávio então
descarregou nele a sua cólera, mandando-o sair. O mulato obedeceu, sem
revolta. Otávio ficou só.}

Amarrotou a carta, atirou-a ao chão.

\textls[17]{--- Hipócrita!\ldots{} Hipócrita! --- repetia ele
furioso, pisando o papel com raiva. Fechou-se por dentro, sentou-se
perto da mesa, apoiou nela os cotovelos, escondendo o rosto entre as
mãos. Pelo seu espírito ia passando a prima, multiplicando-se, como uma
procissão singular; era sempre Eva, ora radiante, ora altiva, ora
concentrada, ora expansiva; vi-a de costas, à porta do galinheiro,
atirando milho às aves, singela no seu vestido de percal caseiro, como
no dia da chegada; via-a de cetim branco, decotada, sob a luz crua do
gás, de joelhos no genuflexório da matriz de Campinas; via-a entretecer
as hastes do jasmineiro num cuidado de \emph{ménagère} delicada; via-a
galopando ao lado de Paulo com a amazona e o véu flutuantes; via-a
debruçada no seu leito, tratando-lhe a ferida do ombro, doce e
maternalmente; via-a repelindo a afronta do pai; via-a sempre, ora
alegre, ora triste, ora desdenhosa e ora meiga, sucedendo-se,
repetindo-se, escarnecendo, chorando, repelindo, atraindo.}

\textls[20]{A visão continuava: agora falava-lhe ao pé do moinho e
ia levado piedosamente por ela a ver a pobre mamã paralítica; e
lá no fundo do seu coração ecoava com tristeza o dito da velha:
``Como eu gostava qui mecê casasse cum ela!''. Depois, guiava-a
através do bosque, dava-lhe flores, ouvia-a encantado,
sentava-se a seu lado embaixo da figueira brava, conversando ao som da
água que corria soluçando a seus pés\ldots{}}

Tinha passado tudo; a doce Eva de outrora deveria dissipar-se no
seu pensamento, como se fosse uma visão de mero sonho\ldots{}

A noite chegou, escura e calma. Otávio desceu ao
terreiro; evitava a convivência da família; mas aí
mesmo, na solidão e na escuridade, perseguia-o uma recordação ainda de
Eva; lembrava-se, sob a janela do quarto da alemã, da branda
luz do lampião de vidro fosco, junto do qual
a prima recitara, numa entonação grave e séria, uma poesia de amor.
Então tivera dúvidas; despontava o ciúme de um outro,
do Azevedo! Como ia tudo longe! E que saudades sentia daquele estado
dúbio e indeciso da alma!

O vento frio de maio sibilava nos bambus do
tanque e agitava as duas casuarinas do pasto, numa bulha triste como a
do mar.

Otávio passeava de um lado para outro, envolto na negrura da
noite e na do seu desgosto.

\textls[20]{Que havia de fazer? Como reagir contra aquela decepção? Pouco se
importava com isso! Sentia como que uma amarga voluptuosidade em sofrer
por ela, assim!}

\textls[-20]{--- Tive ao menos a fortuna de amar --- concluiu ele
com tristeza.}\looseness=-1

%\end{linenumbers}

\section{XLI}

%\begin{linenumbers}

O casamento de Eva foi celebrado na própria fazenda.

\textls[10]{Poucos convidados: o dr.\,Morton, uma filha casada da d.\,
Miquelina e a família Medeiros. O comendador não faltou; era a
segunda vez que entrava naquela casa e, lembrando-se comovido da
primeira, em que fora dizer um último adeus ao irmão,
o infortunado Gabriel, ele sentia apertar-se-lhe o coração num
sentimento que envolvia o arrependimento e a saudade.}

Abraçou a sobrinha com ternura, como se pusesse naquele
abraço toda a amizade que tão injustamente lhe retirara sempre. Não se
referiu ao passado, e fugia do olhar de Paulo como um réu foge da
investigação do juiz.

\textls[-15]{Otávio mostrava-se sereno; cumprimentou os noivos como um
vencido resignado, conservando a sua linha elegante e a sua maneira
lhana de conversar. Estava pálido, mas aparentemente alegre, alegria
motivada pela superexcitação nervosa. Desejando fugir à vista
dos noivos, levou o pai para fora, obrigando-o a observar as belezas do
Mangueiral como efeito de uma direção inteligente.}\looseness=-1

\textls[20]{Tudo ali era risonho e prometedor; em tudo um ar de modernismo e
de asseio, uma brancura lavada nas casas, abundância de frutas
no pomar, de flores no jardim, conforto na habitação, e os cafezais
lindamente tratados.}\looseness=-1

--- Não há profissão mais bela do que a do lavrador
--- dizia Otávio ao pai ---, quando ele
consegue fazer do canto em que vive um paraíso como este! Cavar a terra
brutalmente, sem tirar dela senão o lucro material, é um egoísmo torpe.
Lamento ter fugido do meio em que nasci; com a educação de Paulo, talvez
que também tivesse feito de Santa Genoveva uma coisa
assim\ldots{}

\textls[20]{O comendador olhou com estranheza para o filho e, com um suspiro
em que transpirava ainda a saudade dos seus antigos escravos, murmurou:}\looseness=-1

\textls[-15]{--- Que remédio terei eu agora senão transformar tudo
aquilo\ldots{}}\looseness=-1

Entretanto, Noêmia borboleteava em torno de Eva, dando-lhe a
consolação de se mostrar esquecida do seu primeiro amor.

\textls[-20]{O comendador e as filhas regressaram à tarde a Santa Genoveva,
atravessaram os cafezais e os pastos, onde as ervas daninhas
começavam a aparecer. No largo terreiro latiam os cães em correrias
desenfreadas, e ao de lá da cancela que dava para o tanque focinhavam os
porcos nos sabugos e palhas de milho espalhados em quantidade no chão.
Dentro, na varanda, a mãe balançava-se na rede,
peneirando arroz, e uma criada italiana punha a mesa para o jantar.}\looseness=-1

%\end{linenumbers}

\section{XLII}

%\begin{linenumbers}

Poucos dias passados, Otávio batia à janela do dr.\,Morton. O
velho lia, segundo o seu costume, em voz alta, um livro de ciência.

\textls[15]{--- Que é isso?! Está de viagem! --- exclamou, vendo o moço de
guarda-pó e bolsa a tiracolo.}

--- É verdade! Vou viver na capital.

--- Segue para São Paulo?

--- Sim, senhor; e venho despedir-me e agradecer-lhe\ldots{}

\textls[25]{--- Agradecer o quê? Ah, eu não sabia de nada, e logo que
desconfiei\ldots{} Com que então vai para São Paulo? ---
prosseguiu ele, mudando de tom. --- Mas para sempre? A modo que
lhe ouvi que ia lá viver!}\looseness=-1

--- Vou matricular-me na Faculdade de Direito\ldots{}

--- Que ideia! Já é engenheiro\ldots{} Enfim, são
glórias!

\textls[-15]{--- Não; são manias: como provinciano e moço, sigo fatalmente a
carreira preferida pelos meus patrícios\ldots{} Bem sabe,
ficaria incompleto se não fosse bacharel! Mas é tarde, adeus, e
até breve, doutor.}\looseness=-1

--- Adeus, Otávio; boa viagem!

%\end{linenumbers}

% \begin{center}
% \textsc{fim}
% \end{center}
