\chapter*{Introdução à edição\\ de 1919}
\addcontentsline{toc}{chapter}{Introdução à edição de 1919, \textit{por Alfredo Sousa}}

\hyphenation{Dickens}

\begin{flushright}
\textsc{alfredo sousa}
%\textit{Rio de Janeiro, janeiro de 1919}
\end{flushright}

\noindent{}\textls[20]{A Empresa Nacional de Publicidade surge 
em hora propícia. É a hora fundamentalmente construtiva 
de uma Humanidade diferente da conhecida, sobre as ruínas
calcinadas e ainda fumegantes da Europa dilacerada pela 
maior guerra da história, envolvendo na sanha destruidora 
o mundo inteiro.}\looseness=-1

O pensamento da sua fundação é o de colaborar eficazmente, de um lado,
na renovação da cultura da justiça e da beleza, enfim vitoriosa no
conflito da civilização com a barbaria; e de outro, na constituição
visceralmente nacional de uma consciência coletiva imbuída da firme
intenção de intensificar todas as energias, no sentido de tornar mais
fecundas e concretas as forças vivas do país, valorizando o patrimônio
moral e econômico da nação para uma profícua concorrência universal de
capacidade produtiva.

\textls[5]{Triunfante, plenamente, em toda a linha, o espírito democrático que
decidiu da sorte do mundo, no momento mais periclitante da sua evolução
para a conquista do direito, na sua mais ampla expressão individual e
coletiva, nacional e internacional, a sua obra quer ser virtualmente
democrática, e sob tal modalidade ética, etnicamente americana,
persuadida como está de que é neste hemisfério que uma nova civilização,
florescente no futuro em todo o globo, se acha em adiantada gestação,
certamente dominadora, à força da exuberância vital que a encuba.}

Tirando das lições da guerra os seus corolários naturais,
corporificando-se ao alvorecer da paz num aparelho consciente de ação
doutrinadora e prática de divulgação de quanto interesse individualmente
apreender para formar o espírito e adestrar o corpo no sadio e
tonificante exercício do trabalho proteico --- fonte de saúde e de
riqueza ---, é seu propósito capital despertar as energias latentes de
todos para a obra comum do maior desenvolvimento da nação, a fim de que
esta realize no mais próximo futuro os altos destinos a que a fadou a
sua grandeza e situação, a sua fecundidade inesgotável e as virtudes
ancestrais da sua raça.

\textls[10]{Para concretizar esse pensamento, abrangendo uma larga série de atos
especulativos no domínio da atividade eficiente, o seu primeiro
movimento foi a criação da \emph{Biblioteca Brazilia}, o órgão por
excelência preparador da mentalidade para a aquisição dos conhecimentos
indispensáveis à formação da consciência e à cristalização do caráter
que hão de aperceber-se das necessidades do tempo para as grandes
resoluções dos problemas nacionais de instante solução, como reação da
cultura sobre a incapacidade e a inércia dominantes no maior âmbito do
país.}

Preconcebida e assentada segundo um plano estrutural de generalização
das ideias que mais importa seguir para chegar ao terreno das
realizações positivas, comporta, com a mera ilustração do espírito, a
mais eficaz instrução da inteligência e a sistemática educação da
vontade para executá-las deliberadamente.

\textls[20]{Para iniciar esta \emph{Biblioteca}, a empresa lembrou-se de um nome
aureolado nas belas letras, que é, de há muito já, um patrimônio
nacional, tal o fúlgido relevo que ele assume e a solidez da fecunda
obra que há realizado, enriquecendo a literatura pátria --- dona Júlia
Lopes de Almeida.}

\textls[5]{A insigne escritora acolheu, generosa, o convite que para isso lhe foi
feito, oferecendo-nos a escolha entre um inédito em elaboração e o seu
primeiro livro publicado, e que, esgotado há mais de 20 anos, está
desvanecido da memória dos leitores desse tempo, sendo original para os
da nova geração. Não hesitamos, preferindo dar novo curso a este último,
e por óbvias razões: primeira obra de fôlego da autora, desde logo
triunfante, pois que a 1ª edição se esgotou em três meses, êxito
excepcional de livraria nossa, é, como romance de costumes, dos mais
acentuadamente brasileiros que possuímos. E por ser um modelo de
literatura popular, é o espelho sem jaça de uma época muito agitada da
vida nacional, aquele em que pela evolução das ideias humanitárias do
nosso liberalismo, que resultou em fautor da República, marca a
transição do trabalho escravo para o trabalho livre.}\looseness=-1

\emph{A família Medeiros}, assim se intitula essa obra, foi começada em
1886, em pleno regime da escravidão no Brasil, mas, interrompida por
motivos de longas viagens da autora, só foi concluída em 1888, vindo à
publicidade, em 1\textsuperscript{a} edição, em 1892.

\textls[10]{Livro admirável pela singeleza da linguagem, é hoje duplamente estimável
pelo cunho histórico autêntico das emocionantes narrativas, que lhe
formam o fundo moralizador. Obra de pintura e de doutrina,
simultaneamente, o espírito do leitor tem nas suas páginas, ora a
encantá-lo de cor, a paisagem paulista em vários aspectos e contrastes,
numa gama discreta de tons e vibrações; ora a lição de moral,
profligando as desumanidades da cruel legislação dominante e incutindo
na alma ingênua das suas personagens centrais, deformada pelo bárbaro
costume, ideias generosas e compassivas de regeneração e altruísmo.}

\textls[10]{Há uma tocante sinceridade de desenho em todas as suas variadas
descrições, realçadas de notável sobriedade de tintas, em cromos
esbatidos e flagrantes de verdade regional, ao lado de retratos vividos
sem violências de contornos ou colorido, acusando um sentimento exato da
natureza através de um temperamento todo delicadeza e emoção na
interpretação das gentes e das coisas que lhe servem de modelo pictural.}\looseness=-1

\textls[-3]{Todas estas qualidades mestras na obra de arte, privilégio de
raros engenhos criadores, que pela simplicidade de processos artísticos
e pelo dom interpretativo fazem compreendida e estimada do maior número
a sua produção, por eminentemente verídica, natural, local e temporânea, 
assinalam \textit{A família Medeiros} como um documento 
permanente e imorredouro na pintura de costumes, resultando continuamente, 
a despeito da distância, no tempo e no espaço, sempre novo e atual. Foi
pelos mesmos requisitos de técnica e de emoção que se celebrizaram e
vivem, ainda hoje, a mesma vida de curiosidade e alvoroço da sua aparição
a um público fiel ao gosto sem artifícios da sã literatura ---
\textit{Inocência}, de Taunay, e toda a obra tão humana e honesta de Dickens e
Júlio Dinis, de que tanto se aproxima o primeiro romance da mais
ilustre das escritoras patrícias de todos os tempos.}

\textls[10]{Por isso o preferimos, reeditando-o, para estreia das nossas
edições, tão seguros estamos de com ele proporcionar, aos nossos
assinantes e leitores, um regalo único, concorrendo ao mesmo tempo para
que um tão vivo documento da nossa história literária se perpetue como
merece, prosseguindo a sua carreira vitoriosa a comunicar às novas
gerações e a renovar nas anteriores o amor cada vez mais acrisolado da sua terra 
e da sua gente, através da cândida beleza da sua arte nativa.}\looseness=-1

\textls[10]{O que levamos dito não é uma apresentação, 
que dela prescinde a gloriosa escritora de tantas obras consagradas
pela crítica e pelo sufrágio do público legente, esgotadas em
sucessivas edições, marcando a justificada predileção pela sua
leitura, sempre instrutiva e deleitosa. Não é uma crítica, função e
autoridade que não temos, e muito menos um preconício para ela ou para
nós, meio de êxito forçado que escapa às nossas cogitações e é excluído
pelo valor intrínseco da obra escolhida para a presente edição, que por
si mesma se impõe como um sucesso a mais e justificadíssimo, do seu
peregrino talento.}

\textls[-10]{Como novos introdutores de obra sua no mercado bibliográfico,
reputamos oportuno oferecer aos nossos leitores, com a sucinta
notícia da índole e estrutura de \textit{A família Medeiros}, um rápido esboço
da biografia da autora, um atrativo a mais a juntar ao da atual
edição, imprimindo-lhe maior curiosidade e valor mais alto.}\looseness=-1

\textls[15]{D.\,Júlia Lopes de Almeida, em solteira Júlia Valentina da Silveira
Lopes, filha dos Viscondes de São Valentim, portugueses de origem,
nasceu nesta capital em 24 de setembro de 1862,}\linebreak 

\noindent{}\textls[20]{num palacete da rua do
Lavradio, onde seu pai, médico e literato muito culto, dr.\,Valentim
José da Silveira Lopes, mantinha um internato intitulado Colégio de
Humanidades.}

\textls[15]{Fez os primeiros estudos na casa paterna, com sua mãe, senhora
ilustrada e musicista de merecimento, e com sua irmã mais velha, a
festejada poetisa, d.\,Adelina Lopes Vieira.}

\textls[-30]{Estreou-se nas letras, escrevendo na \textit{Gazeta de Campinas}, folha de
tradições republicanas, ao tempo dirigida pelo poeta Carlos Ferreira.}\looseness=-1

\textls[15]{Publicou ainda em solteira os \textit{Contos infantis}, de colaboração com
sua irmã Adelina, e \textit{Traços e iluminuras}, contos, esgotado,
festivamente recebidos pela crítica.}

Em 1887, casou-se com o poeta e jornalista Filinto de Almeida, hoje
membro dos mais autorizados da Academia Brasileira de Letras, de cuja
prole o primogênito é o jovem, e já considerado poeta e jornalista,
Affonso Lopes de Almeida.

\textls[-5]{Fazendo várias viagens pela Europa, foi da última vez que aí esteve
aclamadíssima pela intelectualidade de Lisboa, que organizou em sua
honra festas artísticas e literárias às quais compareceram os maiores
vultos portugueses; e em Paris, foi-lhe oferecido, em 16 de fevereiro de
1914, um suntuoso banquete de 200 talheres, seguido de imponente sarau
de arte, como simpatia e solidariedade literária das suas colegas Mmes.
Jane Catulle Mendès, Daniel Lesueur, Condessa de Martel, Alphonse
Daudet, Adolphe Brisson, Georges de Peyrebrune, Séverine, Edmond
Rostand, Marcelle Tinayre, Jean Richepin, Jean Bertheroy, Rachilde,
Aurel, Gabrielle Réval, Annie de Pène, Myriam Harry, Camille du Gast,
Fernand Gregh, Rosita Matsa, C.\,de Broutelles e Berthe Dangennes.}\looseness=-1

Nessa memorável festa de confraternização artística, proferiram
eloquentes discursos Mmes. Catulle Mendès e Lesueur, em nome da
\emph{Société des Gens de Lettres}; Ernest Gaubert, como representante
da \emph{Critique Littéraire}; Medeiros e Albuquerque, da Academia
Brasileira de Letras; Georges Bourdon, chefe da seção \emph{Amérique
Latine}, do \emph{Figaro}, e a festejada, agradecendo.

\textls[-15]{O ``menu'' e o programa do sarau eram ilustrados de finíssimas gravuras
do Rio de Janeiro, tendo por legendas versos entusiásticos de Mme.
Catulle Mendès, exaltando as maravilhas da nossa capital.}\looseness=-1

\textls[10]{Também no país lhe não têm sido regateados os aplausos e as homenagens
mais sentidas à sua proeminente individualidade e à sua magnífica obra.
Ainda em 1915, por ocasião do seu natalício, um grupo de escritores
patrícios organizou em sua intenção, nesta cidade, uma formosa festa
literária, em que tomaram parte, proclamando os méritos e as glórias da
excelsa escritora, os nossos mais representativos intelectuais.}\looseness=-1

Idênticas homenagens lhe prestaram os Estados do Espírito Santo e do Rio
Grande do Sul, quando da sua visita a eles.

\textls[20]{Mato Grosso tem um grêmio literário com o seu nome, que, há dois anos,
mantém uma revista, e agora acaba de sugerir a criação de uma Escola
Doméstica, em Cuiabá, prestes a converter-se em realidade.}

\textls[5]{Além dos seus livros de literatura, contos, novelas, romances, teatro,
feitos com independência de ideias e de escola; além das suas
conferências, das suas crônicas, das suas obras didáticas e das suas
obras de ironia --- \textit{Eles e elas} etc. ---, tem d.\,Júlia Lopes de Almeida
livros intencionalmente dedicados ao anseio de auxiliar e dilatar o
aperfeiçoamento moral da família brasileira, promovendo o maior
embelezamento e riqueza da sua terra, como: \textit{O livro das noivas}, \textit{Livro
das donas e donzelas}, \textit{Correio da roça} e \textit{A árvore}, livros que têm feito
o seu caminho, semeando bênçãos e produzindo benefícios reais. De vários
desses benefícios tem a autora sido ciente direta ou indiretamente,
por cartas de leitores, encontros casuais ou comunicações de
terceiros.}

\textls[-20]{É toda uma grande e luminosa obra de construção e patriotismo, sincera
no mais alto grau e, por isso mesmo, sempre carinhosamente acolhida, de
alma e coração, pelo nosso povo. Nenhum dos nossos escritores se mostrou
mais extremoso pelo seu país do que esta infatigável semeadora de boas
ideias de grandeza e progresso nacionais.}\looseness=-1

\textls[15]{Os seus artigos, as suas descrições de viagens pelo interior, são
disso uma prova inconcussa. 
Honra lhe seja.}

% \begin{flushright}
% \textit{Rio de Janeiro, janeiro de 1919}\\
% \medskip
% Pela empresa editora,\\
% \textsc{alfredo sousa}
% \end{flushright}
